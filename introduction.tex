\chapter{Introduction}
\section{Background}
Some problems in Computer Science are known to be NP-complete,
meaning that assuming P$\neq$NP there is no polynomial-time
algorithm that can solve these problems.
Even so, they still can be amenable to different approaches,
such as approximation or parameterization.

\begin{defi}
In the \textbf{$\SetCover$} problem we are given a set of elements (universe)
$\points$ and~a~family of sets $\sets$ that are subsets of the universe $\points$
and sum up to the whole $\points$.
Our~task is to find a set $\sol \subseteq \sets$
such that $\bigcup \sol = \points$ and the size of $\sol$ is minimum possible.
\end{defi}

$\SetCover$ is a classical example of an NP-complete problem,
which has been proven
in \cite{set_cover_inapproximation} to be
inapproximable with factor $(1-o(1))\ln n$ assuming P~$\neq$~NP
(which is a stronger result than APX-hardness),
and W[2]-complete with the natural parameterization,
see Theorem 13.21 in \cite{platypus_book}.
However, restricting the problem to various specialized settings
can lead to more tractable special cases.
In this thesis we take a closer look at the $\GeometricSetCover$ problem
in the plane, where elements to cover are~points in the plane
and sets to cover them with are geometric objects.

\begin{defi}
\textbf{$\SegmentSetCover$} is $\GeometricSetCover$ where
objects that we cover the points with are segments in the plane.
\end{defi}

\paragraph{Approximation}
Over the years there has been a lot of work related to approximation
algorithms for $\GeometricSetCover$. Notably,
$\GeometricSetCover$ with unweighted unit disks admits a PTAS (see
Corollary 1.1 in \cite{unit_disks}). When we consider the same problem
with weighted unit disks (or unit squares), the problem admits a QPTAS
\cite{settling_apx_hardness}, see also \cite{voronoi_true}.
On the other hand, \cite{rectangles_apx_hard} 
proved that $\GeometricSetCover$ with unweighted axis-parallel fat rectangles
is APX-hard; they also show similar hardness
for $\GeometricSetCover$ with many other standard geometric objects.

\paragraph{Parameterization}
We consider $\GeometricSetCover$ 
parameterized by the size of solution.
$\GeometricSetCover$ with unit squares was first proven to be W[1]-hard 
in \cite{marx05} (Theorem 5). A later follow-up work \cite{voronoi}
shows that there is an~algorithm running in time $n^{\mathcal{O}(\sqrt{k})}$
that solves $\GeometricSetCover$ with unit squares or disks
and that there is no algorithm running in time $f(k) \cdot n^{o(\sqrt{k})}$
for any computable $f$ under the~Exponential-Time Hypothesis,
so this is a tight bound for this problem.

We also consider parameterization of weighted problems.
There does not seem to be a~consensus of what parameterization
in the weighted setting is exactly; there
was an attempt to introduce a quite complicated general
framework of weighted parameterized setting in \cite{weighted_framework}.
Kernels for several well-known weighted problems
such as \textsc{Weighted Subset Sum} or~\textsc{Weighted Knapsack} are presented in \cite{kernel_weighted}.
Another work \cite{weighted_flow} considers weighted
parameterization of \textsc{Weighted Directed Feedback Set} and \textsc{Weighted $st$-Cut}.

\paragraph{$\delta$-extension}
In this paper, we focus on $\SegmentSetCover$ with $\delta$-extension.
$\delta$-extension is a problem relaxation method based on the
$\delta$-shrinking model which was introduced in \cite{shrinking_original}
to provide interesting results for
the \textsc{Maximum Weight Independent Set of Rectangles} problem.
In this problem one is given a family of weighted rectangles
and needs to find a set of non-overlapping rectangles
with the largest possible total weight.
In the $\delta$-shrinking relaxed problem
the returned set of rectangles must be non-overlapping
after all the rectangles are shrunk by a tiny fraction $\delta$
towards the centre of symmetry.
This problem is easier, because we compare the
weight of the obtained solution
to the optimum result before the shrinking. It might
even lead to finding a set with result better than the optimum
for the original problem.
The authors in~\cite{shrinking_original} present a PTAS
for \textsc{Maximum Weight Independent Set of Rectangles} with $\delta$-shrinking,
which was later improved to an EPTAS in \cite{shrinking1}, alongside
with presenting a new FPT algorithm
for this problem with the natural parameterization.
A similar $\delta$-shrinking model was used in \cite{shrinking2}
to present a PTAS for
\textsc{Maximum Weight Independent Set of Polygons} with $\delta$-shrinking.

\begin{defi}
\label{definition:delta_extension}
For any $\delta > 0$ and a centre-symmetric convex object $L$ with
centre of symmetry $S = (x_s, y_s)$,
the \textbf{$\delta$-extension} of $L$ is the open set of points:
$$L^{+\delta} = \{(1 + \epsilon)\cdot(x - x_s, y - y_s) + (x_s, y_s) : (x, y) \in L, 0 \le \epsilon < \delta\}.$$
That is, $L^{+\delta}$ is the image of $L$ under homothety centred
at $S$ with scale $(1+\delta)$ but with the extreme points excluded.
In particular, $\delta$-extension turns a segment
into a segment without endpoints
and a rectangle into an interior of a rectangle.
\end{defi}

Analogous to $\delta$-shrinking,
$\delta$-extension provides a framework for relaxing
\textsc{Geometric} \textsc{Set} \textsc{Cover} problems, where we allow the returned set of
objects $\sol$ to \textit{almost} cover the points in the universe
by requiring that they are covered by $\sol$ after $\delta$-extension,
i.e. by the set $\sol^{+\delta}$.
The same concept could be used for \textsc{Geometric Hitting Set} problems.
 
For a longer discussion of this concept see Section
\ref{section:def:delta_extension}.

Similar model is used to prove that $\GeometricSetCover$ with fat polygons
relaxed with $\delta$-extension admits an EPTAS \cite{harpeled12}.
The $\delta$-extension model presented there is well-defined only
for fat polygons. An object P is extended by all the points that
are at distance to the closest point in the object $P$
no larger than $\delta\cdot \mathsf{rad}(P)$, where $\mathsf{rad}(P)$
is the largest radius of a circle inscribed into $P$.
Since segments do not have any circle inscribed into them,
the definition presented there cannot be utilized
for the setting of segments considered here.
Polygon extended by $\delta$-extension
defined in Definition \ref{definition:delta_extension}
covers a superset of points that the polygon extended
by $\delta$-extension defined in \cite{harpeled12} covers.
Since our definition is more permissive for any polygon,
the EPTAS from \cite{harpeled12}
also works for polygons extended
according to our definition of $\delta$-extension.

\section{Our contribution}
In this thesis we make the following contributions.

We show that $\SegmentSetCover$ is APX-hard,
even if segments are axis-parallel and we relax the problem with $\frac{1}{2}$-extension,
(Theorem \ref{segment_cover_apx_hard}).

\begin{restatable}{tw}{segmentCoverApxHard}{
\label{segment_cover_apx_hard}
	\textbf{($\SegmentSetCover$ is APX-hard)}.	
	$\SegmentSetCover$
	is APX-hard even when 
	relaxed with $\frac{1}{2}$-extension
	and segments are axis-parallel.
	That is, assuming $P\neq NP$, there does not exist a PTAS
	for this problem.
}\end{restatable}

Theorem \ref{segment_cover_apx_hard} implies the following.
Note that segments are just degenerated rectangles.

\begin{corollary}{
\label{rectangle_cover_apx_hard}
	\textbf{($\GeometricSetCover$ with rectangles is APX-hard)}.	
	\textsc{Geometric} \textsc{Set} \textsc{Cover}
	with axis-parallel rectangles is APX-hard
	even when relaxed with $\frac{1}{2}$-extension.
}\end{corollary}

This expands the previous result of \cite{rectangles_apx_hard} 
that $\GeometricSetCover$
with axis-parallel fat rectangles is APX-hard,
we improved the result that rectangles no longer
have to be fat (Corollary \ref{rectangle_cover_apx_hard})
and it holds when the problem is relaxed with $\frac{1}{2}$-extension.
It also proves that the assumption in \cite{harpeled12}
about polygons being fat is necessary, because
covering with arbitrary polygons with $\frac{1}{2}$-extension is APX-hard.

We also provide two FPT algorithms for parameterized $\SegmentSetCover$ 	
(Theorem~\ref{segment_cover_fpt})
and $\WeightedSegmentSetCover$ relaxed with $\delta$-extension
(Theorem~\ref{fpt_weighted_segment}).

\begin{restatable}{tw}{segmentCoverFpt}{
	\label{segment_cover_fpt}
	\textbf{(FPT for $\SegmentSetCover$).}
	There exists an algorithm that given a family $\sets$ of
	segments (in any direction),
	a set of points $\points$
	and a parameter $k$,
	runs in time ${k^{\mathcal{O}(k)} (|\points|\cdot|\sets|)^2}$,
	and outputs a solution $\sol \subseteq \sets$
	such that $|\sol| \le k$ and $\sol$ covers all points in~$\points$,
	or determines that such a set $\sol$ does not exist.
}\end{restatable}

\begin{restatable}{tw}{fptWeightedSegment}{
	\label{fpt_weighted_segment}
	\textbf{(FPT for $\WeightedSegmentSetCover$ with $\delta$-extension).}
	There exists an algorithm that given a family $\sets$ of
	$n$ weighted segments (in any direction),
	a set of $m$ points $\points$, and parameters $k$ and $\delta > 0$,
	runs in time $f(k, \delta) \cdot (nm)^c$ for some computable function $f$ and a constant $c$ and
	outputs a set $\sol$ such that:
	\begin{itemize}
	\item $\sol \subseteq \sets$,
	\item $|\sol| \le k$,
	\item $\sol^{+\delta}$ covers all points in $\points$,
	\item the weight of $\sol$ is not greater than the weight
	of an optimum solution of size at most $k$
	for this problem without $\delta$-extension,
	\end{itemize}
	or determines that there is no set $\sol$ with $|\sol| \le k$
	such that $\sol$ covers all points in $\points$.
}\end{restatable}

On the other hand, we prove that $\WeightedSegmentSetCover$ 
is W[1]-hard even when segments are limited to 3 directions (Theorem~\ref{w1_hard})
and assuming ETH there does not exist an algorithm for this problem
that runs in time ${f(k)(|\points| + |\sets|)^{o(\sqrt{k})}}$.
See Figure \ref{tab:weighted_fpt} for a summary of parameterized
results for $\SegmentSetCover$ and
\textsc{Weighted} \textsc{Segment} \textsc{Set} \textsc{Cover}.

\begin{restatable}{tw}{wOneHard}
\label{w1_hard}
	\textbf{($\WeightedSegmentSetCover$ is W[1]-hard).}
	Consider the problem of covering a set $\points$ of points
	by selecting at most $k$ segments
	from a set of segments $\sets$ 
	with non-negative weights $w : \sets \rightarrow \mathbb{R^+}$
	so that the weight of the cover is minimal.
	Then this problem is W[1]-hard when parameterized by $k$ and
	assuming ETH, there is no algorithm for this
	problem with running time
	$f(k)\cdot(|\points| + |\sets|)^{o(\sqrt{k})}$
	for any computable function $f$.
	Moreover, this holds even if all segments in $\sets$
	are axis-parallel or right-diagonal.
\end{restatable}

See Section \ref{section:def:geometric__set_cover}
for exact definitions of axis-parallel and right-diagonal segments.

This result is particularly interesting,
because the problem without weights is FPT,
while the weighted variant is W[1]-hard.
Moreover, $\delta$-extension allowed us to provide an FPT algorithm
for the problem which is W[1]-hard otherwise.

Note that the result of Theorem \ref{w1_hard} is not tight:
there exists a simple algorithm 
running in time ${f(k)(|\points| + |\sets|)^k}$.
So the question whether there exists an algorithm
for this problem running in time ${f(k)\cdot(|\points| + |\sets|)^{o(k)}}$
is still open.

Permissive FPT is a relaxed FPT problem, where 
we need to find a solution of \textit{any} size in FPT-time,
but we compare it to the optimum solution of size at most $k$.
Idea for permissive FPT in local search was presented
in \cite{permissive_problem1}, \cite{permissive_problem2}.
Theorem \ref{w1_hard} can be improved to show that a permissive FPT
algorithm does not exist.
This is formulated precisely in Theorem \ref{permissive_w1_hard}.

\begin{figure}[h]
\begin{center}
\begin{tabular}{ | c | c | c | c | }
\hline
                & exact weighted    & $\delta$-extension weighted  & exact unweighted \\ 
\hline                
 axis-parallel   & ? & FPT* & FPT*\\  
\hline                
 3 directions    & W[1]-hard~ & FPT* & FPT*\\  
\hline                
 any direction   & W[1]-hard* & FPT~ & FPT~ \\
\hline                
\end{tabular}
\caption{Our results for $\WeightedSegmentSetCover$ and $\SegmentSetCover$
parameterized by the size of a solution.
Results marked with * are not explicitly given in this thesis,
but they trivially follow from stronger results shown in the other cells of the table.}
\label{tab:weighted_fpt}
\end{center}
\end{figure}

\paragraph{Future work.} There are two aforementioned problems
that relate to Theorem \ref{w1_hard} and were not solved in this thesis.
We have given a W[1]-hardness proof
for \textsc{Weighted} \textsc{Segment} \textsc{Set} \textsc{Cover}
where segments are limited to 3 directions,
but the segments in the construction may be also right-diagonal.
However, it may be possible to improve this construction to use segments
in 2 directions instead of 3 directions. 
The other question is what is the tight bound for this problem.
The simple algorithm solving
this problem is running in time ${f(k)(|\points| + |\sets|)^{\mathcal{O}(k)}}$,
while our lower bound refutes running time ${f(k)(|\points| + |\sets|)^{o(\sqrt{k})}}$.

Another problem to consider is whether
\textsc{Geometric Hitting Set} relaxed with $\delta$-extension
can yield some better results.

\paragraph{Acknowledgments.} Words cannot express my gratitude
to my supervisor, dr hab Michał Pilipczuk, for his patience and eager help
with research and editing. I am also grateful to Krzysztof Maziarz
for countless hours of proofreading this thesis.
