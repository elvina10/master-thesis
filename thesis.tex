%
% Niniejszy plik stanowi przykład formatowania pracy magisterskiej na
% Wydziale MIM UW.  Szkielet użytych poleceń można wykorzystywać do
% woli, np. formatujac wlasna prace.
%
% Zawartosc merytoryczna stanowi oryginalnosiagniecie
% naukowosciowe Marcina Wolinskiego.  Wszelkie prawa zastrzeżone.
%
% Copyright (c) 2001 by Marcin Woliński <M.Wolinski@gust.org.pl>
% Poprawki spowodowane zmianami przepisów - Marcin Szczuka, 1.10.2004
% Poprawki spowodowane zmianami przepisow i ujednolicenie 
% - Seweryn Karłowicz, 05.05.2006
% Dodanie wielu autorów i tłumaczenia na angielski - Kuba Pochrybniak, 29.11.2016

% dodaj opcję [licencjacka] dla pracy licencjackiej
% dodaj opcję [en] dla wersji angielskiej (mogą być obie: [licencjacka,en])
\documentclass[en]{pracamgr}

% Dane magistranta:
\autor{Katarzyna Kowalska}{371053}

% Dane magistrantów:
%\autor{Autor Zerowy}{342007}
%\autori{Autor Pierwszy}{342013}
%\autorii{Drugi Autor-Z-Rzędu}{231023}
%\autoriii{Trzeci z Autorów}{777321}
%\autoriv{Autor nr Cztery}{432145}
%\autorv{Autor nr Pięć}{342011}

\title{Approximation and Parameterized Algorithms for Segment Set Cover}
\titlepl{Algorytmy aproksymacyjne i parametryzowane dla problemu
pokrywania punktów odcinkami na płaszczyźnie}

%\tytulang{An implementation of a difference blabalizer based on the theory of $\sigma$ -- $\rho$ phetors}

%kierunek: 
% - matematyka, informacyka, ...
% - Mathematics, Computer Science, ...
\kierunek{Computer Science}

% informatyka - nie okreslamy zakresu (opcja zakomentowana)
% matematyka - zakres moze pozostac nieokreslony,
% a jesli ma byc okreslony dla pracy mgr,
% to przyjmuje jedna z wartosci:
% {metod matematycznych w finansach}
% {metod matematycznych w ubezpieczeniach}
% {matematyki stosowanej}
% {nauczania matematyki}
% Dla pracy licencjackiej mamy natomiast
% mozliwosc wpisania takiej wartosci zakresu:
% {Jednoczesnych Studiow Ekonomiczno--Matematycznych}

% \zakres{Tu wpisac, jesli trzeba, jedna z opcji podanych wyzej}

% Praca wykonana pod kierunkiem:
% (podać tytuł/stopień imię i nazwisko opiekuna
% Instytut
% ew. Wydział ew. Uczelnia (jeżeli nie MIM UW))
\opiekun{dr hab Michał Pilipczuk\\
  Institute of Informatics\\
  }

% miesiąc i~rok:
\date{June 2022}

%Podać dziedzinę wg klasyfikacji Socrates-Erasmus:
\dziedzina{ 
%11.0 Matematyka, Informatyka:\\ 
%11.1 Matematyka\\ 
%11.2 Statystyka\\ 
11.3 Informatyka\\ 
%11.4 Sztuczna inteligencja\\ 
%11.5 Nauki aktuarialne\\
%11.9 Inne nauki matematyczne i informatyczne
}

%Klasyfikacja tematyczna wedlug AMS (matematyka) lub ACM (informatyka)
\klasyfikacja{
Theory of computation
$\rightarrow$ Design and analysis of algorithms
$\rightarrow$ Parameterized complexity and exact algorithms\\
Theory of computation
$\rightarrow$ Design and analysis of algorithms
$\rightarrow$ Approximation algorithms analysis
$\rightarrow$ Packing and covering problems \\
}

% Słowa kluczowe:
\keywords{set cover, geometric set cover,
weighted set cover, FPT, W[1]-hard,
APX-hard, ETH, grid tiling, MAX-(3,3)-SAT, $\delta$-extension}

% Tu jest dobre miejsce na Twoje własne makra i~środowiska:

\newcommand{\points}{\mathcal{C}}
\newcommand{\sets}{\mathcal{P}}
\newcommand{\sol}{\mathcal{R}}
\newcommand{\then}{\Rightarrow}

\usepackage{amsfonts}
\usepackage{amsmath}
\usepackage{graphicx}
\usepackage{xcolor}
\usepackage[nospace, noadjust]{cite}
\usepackage{lineno}
\usepackage{enumitem}
\usepackage{amsthm}
\usepackage{thmtools}
\usepackage{mathtools}  
\usepackage{makecell}
\usepackage{tikz}
\usetikzlibrary{calc,math}
\usepackage{hyperref}
\hypersetup{
    colorlinks,
    citecolor=black,
    filecolor=black,
    linkcolor=black,
    urlcolor=black
}
\linenumbers

\mathtoolsset{showonlyrefs}  

\theoremstyle{plain}
\newtheorem{claim}{Claim}[chapter]
%\newtheorem{defi}{Definition}[section]
\newtheorem{tw}{Theorem}[chapter]
\newtheorem{lemma}{Lemma}[chapter]
\newtheorem{corollary}{Corollary}[chapter]
\newtheorem{remark}{Remark}[chapter]

\theoremstyle{definition}
\newtheorem{defi}{Definition}[chapter]

\setcounter{secnumdepth}{3}
\setcounter{tocdepth}{3}


% koniec definicji

\begin{document}
\maketitle

\newcommand{\SetCover}{\textsc{Set Cover}}
\newcommand{\GeometricSetCover}{\textsc{Geometric Set Cover}}
\newcommand{\SegmentSetCover}{\textsc{Segment Set Cover}}
\newcommand{\WeightedSegmentSetCover}{\textsc{Weighted Segment Set Cover}}
\newcommand{\WeightedGeometricSetCover}{\textsc{Weighted Geometric Set Cover}}

%tu idzie streszczenie na strone poczatkowa
\begin{abstract}
	In this thesis we analyze approximation
	and parametrization of $\GeometricSetCover$ with segments.
	To achieve interesting results we define a concept
	of $\GeometricSetCover$ with $\delta$-extension, where we
	need to cover the points by geometric objects,
	which are extended by a tiny fraction,
	but we compare the solution size to the optimum solution without extension.
	We prove that $\SegmentSetCover$ is APX-hard
	even if we restrict segments to be axis-pararell
	and allow $\frac{1}{2}$-extension (Chapter \ref{chapter:segment_apx}).
	We provide an FPT algorithm for unweighted $\SegmentSetCover$
	parametrized by the size of the solution $k$
	(Section \ref{section:fpt_unweighted})
	and for $\WeightedSegmentSetCover$ with $\delta$-extension
	(Section \ref{section:fpt_weighted}).
	Finally, we prove that $\WeightedSegmentSetCover$
	is W[1]-hard and there does not exist an algorithm running
	in time $f(k)\cdot n^{o(\sqrt{k})}$ solving this problem
	even if we restrict the segments to 3 directions
	(Chapter \ref{chapter:w1_hard}).
\end{abstract}

\tableofcontents
%\listoffigures
%\listoftables
\chapter{Introduction}

Some of the problems are very well known to be NP-complete, like
Set Cover problem. Even so they still can have a variety
of different properties. Set Cover problem has been proven
in literature to be APX-hard and W[2]-hard \cite{platypus_book}.
We can also restrict a problem to look at it in various setting,
what can yield more insteresting complexity and apporximation results.
In this paper we take a closer look at Geometric Set Cover problem
on a plane, where points to cover are points on a plane
and sets to cover them with are geometric objects.

\paragraph{Approximation}
Over the years there have been a lot of work related to approximation
of Geometric Set Cover. Notabely 
Geometric Set Cover with unweighted unit disks admits PTAS, see
Corollary 1.1 in \cite{unit_disks}. When we consider the same problem
with weighted unit disks (or unit squares), the problem admits QPTAS
\cite{settling_apx_hardness}, later improved in
Theorem 2 in \cite{voronoi_true}.
Although \cite{rectangles_apx_hard} 
proves that cover with unweighted axis-parallel rectangles
is APX-hard as well as set cover with many other
standard geometric objects.

\paragraph{Parametrization}
We consider Geometric Set Cover 
parameterized by the size of solution in paremetrized setting.
Geomtric Set Cover with unit squares is proven to be W[1]-hard 
Theorem 5 in \cite{marx05}.

We also consider parametrization in weighted setting.
There does not seem to be a consensus of what parametrization
in weighted setting is exactly, there
was an attempt to introduce quite complicated general
framework of weighted parametrized setting in \cite{weighted_framework}.
Kernels for several well known weighted problems
such as Subset Sum or Knapsack are presented in \cite{kernel_weighted}.
Another work \cite{weighted_flow} presents weighted
parametrization of Weighted Directed Feedback Set and Weighted $st$-Cut.

\paragraph{$\delta$-extensions}
In this paper, we focus on Geometric Set Cover wih segments with $\delta$-extensions.
$\delta$-extensions is a problem relaxation method based on
$\delta$-shrinking model which was introduced in \cite{shrinking_original}
and used is later works \cite{shrinking1} and \cite{shrinking2}.

Similar model is used to prove that Geometric Set Cover with fat polygons
relaxed with $\delta$-extensions admits EPTAS \cite{harpeled12}.

\section*{Our contribution}
\begin{enumerate}
\item Approximation of uweighted geometric set cover with axis-parallel segments
(even if we relax it with  $\frac{1}{2}$-extensions) is APX-hard
(Theorem \ref{segment_cover_apx_hard})
	\begin{enumerate}
	\item Stronger result than cover with unweighted axis-parallel
	rectangles being APX-hard in \cite{rectangles_apx_hard}
	\item This proves that assumption about polgons being fat for
	the EPTAS in \cite{harpeled12} is a necessary condition
	\end{enumerate}
\item unweighted segments admit FPT algorithm (Theorem \ref{segment_cover_fpt})
\item weighted segments relaxed with $\delta$-extensions admit FPT (Theorem \ref{fpt_weighted_segment}),
    see Figure \ref{tab:weighted_fpt}
\item weighted axis-parallel segments is W[1]-hard (Theorem \ref{w1_hard}),
    see Figure \ref{tab:weighted_fpt}

\end{enumerate}


\begin{figure}[h]
\begin{center}
\begin{tabular}{ | c | c | c | }
\hline
                & exact     & $\delta$-extensions \\ 
\hline                
 axis-parallel   & W[1]-hard & FPT* \\  
\hline                
 any             & W[1]-hard* & FPT \\
\hline                
\end{tabular}
\caption{\textbf{Our results for Geometric Set Cover problem with weighted segments 
parametrized by the size of solution.}}

Results marked with * directly follow from more or less restricted settings.
\label{tab:weighted_fpt}
\end{center}
\end{figure}


\iffalse

The Set Cover problem is one of the most common NP-complete problems.
[tutaj referencja]
We are given a family of sets and have to choose the smallest
subfamily of these sets that cover all their elements.
This problem naturally extends to settings
were we put different weights on the sets
and look for the subfamily of the minimal weight.
This problem is NP-complete even 
without weights and if we put
restrictions on what the sets can be.
One of such variants is Vertex Cover problem,
where sets have size 2 (they are edges in a graph).

In this work we focus on another such variant where the sets correspond
to some geometric shapes and
only some points of the plane have to be covered.
When these shapes are rectangles with edges parallel
to the axis, the problem can be proven to
be W[1]-complete (solution of size $k$ cannot be found
in $n^o(k)$ time),
APX-complete (for suffciently small $\epsilon > 0$, the problem
does not admit $1+\epsilon$-approximation scheme)
[refrencje].

Some of these settings are very easy.
Set cover with lines parallel to one of the axis
can be solved in polynomial time.

There is a notion of $\delta$-expansions,
which loosen the restrictions on geometric set cover.
We allow the objects to cover the points
after $\delta$-expansion and compare
the result to the original setting.
This way we can produce both FPT and EPTAS
for the rectangle set cover with $\delta$-extensions
[referencje].



\paragraph{Our contribution.}
In this work, we prove that unweighted geometric set cover
with segments is fixed parameter tractable (FPT).

Moreover, we show that geometric set cover with segments
is APX-complete for unweighted axis-parallel segments,
even with 1/2-extensions.
So the problem for very thin rectangles
also cannot admit PTAS.
Therefore, in the efficient polynomial-time approximation scheme (EPTAS)
for \textit{fat polygons} by \cite{harpeled12},
the assumption about polygons being fat is necessary. 

Finally, we show that geometric set cover with weighted segments in
3 directions is W[1]-complete.
However, geometric set cover with weighted segments is FPT if we allow
$\delta$-extension.

This result is especially interesting,
since it's counter-intuitive that
the unweighed setting is FPT and the weighted
setting is W[1]-complete.
Most of such problems (like vertex cover or [wiecej przykladow])
are equally hard in both weighted and unweighted settings.

\fi

\chapter{Preliminaries}

In this chapter we present some basic definitions that
will be used later.

\section{Geometric set cover}
\label{section:def:geometric__set_cover}
Whenever speaking about geometric set cover,
we consider it in the 2-dimensional plane.

In the geometric set cover problem we are are given
$\sets$ --- a set of objects, which are connected
subsets of the plane and $\points$ --- a set of points in the plane.
The task is to choose $\sol \subseteq \sets$ such that
every point in $\points$ is inside some object from $\sol$
and $|\sol|$ is minimized. We will mostly consider the case where
$\sets$ consists of segments in the plane.

In the weighted setting, there is some given weight function
$f : \sets \rightarrow \mathbb{R^+}$
and we would like to find a solution $\sol$
that minimizes $\sum_{R \in \sol} f(R)$.

\begin{defi}
Segment is \textbf{axis-parallel} if it lies on line that is
either horizontal $x = c$ or vertical $y = c$.
\end{defi}

\begin{defi}
	A line is \textbf{right-diagonal} if it is
	described by linear function $x + y = d$ for some $d \in \mathbb{R}$.
	Segment is \textbf{right-diagonal} if its
	direction is a right-diagonal line.
\end{defi}

\section{Parameterization}

In the parameterized setting of the Geometric Set Cover
for a given $k$,
our task is to either find a solution $\sol$ such that $|\sol| \le k$
or decide that there is no such solution.

\begin{defi}
A \textbf{Fixed-parameter Tractable (FPT)} algorithm 
for a problem with parameter $k$ and instance size $n$
is an algorithm running in time $f(k) \cdot n^c$
for some constant $c$ and some computable function $f$.
\end{defi}

\begin{defi}
\label{definition:cnf}
Boolean formula is in \textbf{conjunctive normal form (CNF)} if
it is a conjunction of one or more formulas,
which are disjunction of literals.
\textbf{$k$-CNF} formula is a CNF formula, where
every disjunction consists of at most $k$ literals.
\end{defi}

\begin{defi}
\textbf{$k$-SAT} problem is 
a boolean satisfiability problem of $k$-CNF formulas.
Given $k$-CNF formula, one must answer if there
exists any variables assignment that satisfies the formula.
\end{defi}

\begin{defi}
For $k \ge 3$ set us define $S_k$ as a set of constants $\sigma$
such that there exists an algorithm solving $k$-SAT running in time
$\mathcal{O}^{*}(2^{\sigma n})$.
Set us define $s_k$ as the infimum  of the set $S_k$.

\textbf{Exponential Time Hypothesis (ETH)} is a conjecture
that $s_3 > 0$. This conjecture implies that
there does not exist an algorithm solving 3-SAT
running in time $2^{o(n)}$.
\end{defi}

We provide the main theorem that we use in this thesis for W[1]-hard
problems. To see the definition of a W[1]-hard problem,
see Chapter 13.3 of \cite{platypus_book}.

\begin{tw}
Problem parameterized by $k$ is \textbf{W[1]-hard} if assuming ETH there
does no algorithm solving this problem running in time
$f(k)\cdot n^{o(k)}$.
\end{tw}

\section{Approximation}

Let us recall some definitions related to optimization problems.

\begin{defi}
A \textbf{polynomial-time approximation scheme (PTAS)}
for a minimization problem $\Pi$
is a family of algorithms $\cal{A}_\epsilon$ for
every $\epsilon > 0$
such that $\cal{A}_\epsilon$ takes an instance $I$ of~$\Pi$
and in polynomial time
finds a solution that is within a factor
of ($1+\epsilon$) of being optimal.
This means that the reported solution has weight at most
$(1+\epsilon)opt(I)$, where $opt(I)$ is the weight
of an optimal solution to $I$.
\end{defi}

\begin{defi}
A problem $\Pi$ is \textbf{APX-hard} if assuming P $\neq$ NP,
there exists $\epsilon > 0$
such that there is no polynomial-time $(1+\epsilon)$-approximation algorithm
for $\Pi$.
\end{defi}

\section{$\delta$-extension}
\label{section:def:delta_extension}

Another idea presented here, which can be utilized only when considering
the problems with geometric objects,
is $\delta$-extension.
We define it specifically for the geometric set cover problem
with convex centre-symmetric objects.

Intuitively, we consider a problem with slightly larger objects,
which makes the instance more permissive.
However, we aim to find a solution that
is not larger than the
optimum solution to the original problem,
so this is substantially easier than just
solving the problem for the larger objects.
It may even be the case
that we are able to find a solution
of size smaller than the optimum solution
to the original problem.

Formal definition of $\delta$-extended objects.
is present in Definition
\ref{definition:delta_extension}.

The geometric set cover problem with $\delta$-extension
is a version of geometric set cover with
the following modifications.
\begin{itemize}
\item We need to cover all the points in $\points$
by selecting objects from $\{P^{+\delta} : P \in \sets\}$ (which always 
include no fewer points than the objects
before $\delta$-extension).
\item We look for a solution that is not larger than the optimum
solution to the original problem.
Note that it does not need to be an optimal solution in
the modified problem.
\end{itemize}
Formally, we have the following.

\begin{defi}
The \textbf{geometric set cover problem
with $\delta$-extension} is the problem where for an input instance
$I=(\sets, \points)$ of geometric set cover,
the task is to output a solution $\mathcal{R} \subseteq \sets$
such that the~$\delta$-extended set
$\{ R^{+\delta} :  R \in \mathcal{R} \}$ covers $\points$
and is not larger than the optimal solution to the~problem without
extension, i.e.~$|\mathcal{R}| \le |opt(I)|$.
\end{defi}

At last, we formulate a definition of the
polynomial-time approximation scheme (PTAS)
for a problem with $\delta$-extension.

\begin{defi}
A \textbf{PTAS for geometric set cover 
with $\delta$-extension} is a family of algorithms
$\{\mathcal{A}_{\delta, \epsilon}\}_{\delta, \epsilon > 0}$ that
each takes as an input instance $I=(\sets, \points)$
of geometric set cover where objects are centre-symmetric and strongly convex,
and in polynomial-time outputs a solution $\mathcal{R} \subseteq \sets$
such that the $\delta$-extended set
$\{ R^{+\delta} :  R \in \mathcal{R} \}$ covers $\points$
and is within a $(1+\epsilon)$ factor of the optimal
solution to this problem without
extension, i.e.~$(1+\epsilon)|\mathcal{R}| \le |opt(I)|$.
\end{defi}

\section{Weighted Geometric Set Cover}

In this thesis we also consider a weighted Geometric Set Cover problem,
which is a combination
of the weighted and parameterized setting described in 
\ref{section:def:geometric__set_cover}.
We already argued in the introduction
that there is no consensus of how it is defined, but when we discuss the
weighted parameterized setting we will consider the following
definition. There is a given weight function
$f : \sets \rightarrow \mathbb{R^+}$
and we would like to find a solution $\sol$,
such that $|\sol| \le k$
that minimizes $\sum_{R \in \sol} f(R)$ among such sets $\sol$.

\begin{defi}
The \textbf{weighted geometric set cover problem
with $\delta$-extension} is the problem where for an input instance
$I=(\sets, \points, f)$ of weighted geometric set cover,
the task is to output a solution $\mathcal{R} \subseteq \sets$
such that the~$\delta$-extended set
$\{ R^{+\delta} :  R \in \mathcal{R} \}$ covers $\points$
and it has weight not larger than the optimal solution to the~problem without
extension, i.e.~$\sum_{R \in \mathcal{R}} f(R) \le |opt(I)|$.
\end{defi}

We also consider weighted parameterized setting with $\delta$-extension,
which we formally define below.

\begin{defi}
The \textbf{weighted geometric set cover problem
with $\delta$-extension parameterized by the size of a solution}
is a problem where for an input instance
${I=(\sets, \points, f, k)}$ of weighted geometric set cover
parameterized by the size of a solution $k$,
the task is to output a solution $\mathcal{R} \subseteq \sets$
such that the~$\delta$-extended set
$\{ R^{+\delta} :  R \in \mathcal{R} \}$ covers $\points$,
uses no more than $k$ sets, i.e. $|\sol| \le k$
and it has weight not larger than the optimal solution to the~problem without
extension, i.e.~$\sum_{R \in \mathcal{R}} f(R) \le |opt(I)|$.
\end{defi}

\section{APX-completeness for segments parallel to axis}
\label{section:segment_apx}

\subsection{Definition of  MAX-(3,3)-SAT problem}
Here we define MAXSAT problem.

\begin{tw}{
	\label{hastadtheorem}
	\textbf{\cite{hastad}}
	Assume NP $\not\subseteq$ $DTIME(2^{O(\log n \log \log n)})$.
	Then, there exists such constant $c > 0$, such for
	$$\epsilon' = \frac{c \log \log \log n}{\log \log n}$$ 
	satifiable 3-SAT formulas cannot be distinguished from
	3-SAT formulas where only $7/8+\epsilon'$ of~the clauses
	can be satisfied in polynomial time.
}\end{tw}

\begin{tw}{
	\label{apxconstruction}
	Given an instance of  MAX-(3,3)-SAT 
	with $n$ variables and optimal result $k$,
	we can construct an instance of axis-parallel segments in 2D,
	which optimal result (even with 1/2-extension) is exactly $17n - k$.
}\end{tw}

\begin{tw}{
	\textbf{(axis-parallel segment set cover with 1/2-extension is APX-hard)}.	
	For every $\epsilon > 0$,
	there doesn't exist an $(1-\epsilon)$-approximation scheme
	for unweighted geometric set cover
	with axis-parallel segments in 2D (even with 1/2-extension)
	(problem is APX-hard).
}\end{tw}

\paragraph{Proof.}
Take any $\epsilon > 0$.
Take such $n$, that $\epsilon'$ from \ref{hastadtheorem}
is not greater than $max(\epsilon, 1/2)$.

Let's assume that there exists an $(1-\epsilon)$-approximation scheme
for unweighted geometric set cover with axis-pararell segments in 2D.
We will construct an algorithm distinguishing instances of MAX-(3,3)-SAT
in \ref{hastadtheorem}.
Take two instances to be distinguished and using \ref{apxconstruction}
let's construct two instances of geometric set cover,
name the one constructed from satisfiable 3-SAT $I_1$
and the unsatisfiable 3-SAT as $I_0$.

Use $(1-\epsilon)$-approximation scheme for instances of geometric
set cover, let's name the result of this approximation
for an instance of problem $I$ as $approx(I)$.

$$approx(I_1) \ge (1-\epsilon)17n > 16\frac{1}{8}n - \epsilon' \ge approx(I_2)$$ 

So we can distinguish these instances, since the satifiable instance
will always yield greater result in approximation scheme.

Therefore such approximation scheme cannot exist.

\subsection{Reduction construction}

Let's take some instance of  MAX-(3,3)-SAT with
variables $x_1, x_2 \ldots x_n$
and clauses $C_1, C_2 \dots C_n$.

We will create gadgets for choosing the value
of variables (\textit{true} or \textit{false}) and checking
if the clauses are met (any of the variables were chosen).

\begin{figure}[h]
\includegraphics[width=0.7\textwidth]{segment_apx_sketch.jpg}
\caption{General scheme of reduction.}
\label{fig:segment_apx}
\end{figure}

\subsubsection{Choose $x_i$ gadget}
\begin{figure}[h]
\includegraphics[width=0.6\textwidth]{choose_x_gadget.jpg}
\caption{Scheme of choose $x_i$ gadget.}
\label{fig:choose_x_gadget}
\end{figure}
In Figure~\ref{fig:choose_x_gadget},
we show a gadget that simulates a single variable $x_i$.
It consists of six points A, B, C, D, E, F, and several segments.
Selecting the segment marked with $x_i$
to the solution will correspond to setting $x_i$ to \textit{true},
while selecting the segment marked with $\neg x_i$
to setting $x_i$ to \textit{false}.
In the following lemmas,
we show that this construction indeed models a binary variable.

First, note that in the gadget
there are exactly two sets of three segments
that cover all points $A, B, C, D, E, F$.
These two sets of segments are marked in
Figure~\ref{fig:choose_x_gadget} in blue and green, respectively.

\begin{lemma}
Points $A, B, C, D, E, F$ cannot be covered using less than
3 segments (even with $1/2$-extensions).
\end{lemma}
\paragraph{Proof.}
We need to take at least one segment on line $ABC$,
because it's the only way to cover $C$.
All other points ($D, E, F$) are not colinear,
so we need at least 2 other segments to cover them.

\begin{lemma}
If we choose both segments $x_i$ and $\neg x_i$, we need to use at
least 4 segments to cover all points $A, B, C, D, E, F$
(even with $1/2$-extensions).
\end{lemma}

\paragraph{Proof.}
Choosing both segments $x_i$ and $\neg x_i$
we only cover points $C$
(becuase $B$ is too far away to be covered with $1/2$-extension)
and $F$.

The remaining points ($A, B, D, E$) are not colinear,
so we need at least two more segments to cover them.

\paragraph{Robustness to $1/2$-extension.}
Take a look at Figure~\ref{fig:segment_apx}.
The points will be included in choose gadgets (horizontal boxes)
and clause gadgets (vertical boxes).

Since segment $AC$ is very long
and colinear with $x_i$, after $1/2$-extension
it will cover a significant part of segment $x_i$,
even though $x_i$ will not be chosen.

If we put all the clause gadgets in the area
marked with \textbf{clauses} at gadget scheme in Figure~\ref{fig:choose_x_gadget},
it is enough to prove that $AC$ will not cover any points
in the \textbf{clauses} area even with $1/2$-extensions.

\begin{lemma}
No points in \textbf{clauses} area can be covered
by $AC$ with $1/2$-extension.
\end{lemma}

\paragraph{Proof.}
Bear in mind that length of $AC$ is equal to length of $x_i$.
Area \textbf{clauses} takes a second half
of the segment $x_i$ and $AC$ after extension will cover the first
half of segment $x_i$.

\subsubsection{Clause gadget}
\includegraphics[width=0.6\textwidth]{clause_gadget.jpg}

\begin{lemma}
In order to cover $D$ ($E, F$) point at least one
of the segments $AD$ ($BE, CF$) or $x'$ ($y', z'$).
\end{lemma}

\begin{lemma}
Points $A$ and $D$ can be covered
with one additional segment $x'$
only if $x$ is already chosen.
Otherwise they can be covered with one segment
only by using $AD$.
\end{lemma}

\begin{lemma}
Points $A, B, C, D, E, F, G$ can be covered with 
3 or 4 segments, depending if at least one of the segments
$x, y, z$ was previously chosen.
\end{lemma}

\subsubsection{Or gadget}
\begin{lemma}
Points $A, B, C, D, E, F, G, H, I, J$ can be covered using
at least 4 segments even with $1/2$-extension.
\end{lemma}

\begin{lemma}
Points $A, B, C, D, E, F, G, H, I, J$ can be covered using
4 segments and segment $x \lor y$ can be chosen
even with $1/2$-extension
only if at least one of the segments $x$ or $y$ is chosen.
\end{lemma}

\subsection{Proof that construction is sound}
\begin{lemma}
If there exists setting of values of variables that exactly $k$
clauses are satisfied, we can cover all the points
with $3n + 11m + (m-k)$ segments.
\end{lemma}

\begin{lemma}
If there exists cover with $k$ segments,
then also there exists solution for MAX-(3,3)-SAT.

TODO: Formulate this lemma better.
\end{lemma}

\chapter{Fixed-parameter tractable algorithm for geometric set cover problem}
In this chapter we show fixed-parameter tractable algorithms
for the geometric set cover problem in~two different settings.
Section \ref{section:fpt_unweighted} shows 
a fixed-parameter tractable algorithm for geometric set cover with unweighted segments.
The remainder of the chapter presents
a fixed-parameter tractable algorithm for geometric set cover with weighted segments
with $\delta$-extension.
We show an algorithm for the setting with $\delta$-extension,
because the original problem with weights is W[1]-hard,
as we show in Chapter $\ref{chapter:w1_hard}$.

We start with a shared definition for this problem.
We define \textit{extreme points} for a set of~collinear points.

\begin{defi}
	For a set of collinear points $C$ in the plane,
	\textbf{extreme points} of $C$ are the endpoints
	of the smallest segment that covers all points from set $C$.
	
	If $C$ consists of one point or is empty, then
	there are 1 or 0 extreme points respectively.
\end{defi}

\section{Fixed-parameter tractable algorithm for $\SegmentSetCover$}
\label{section:fpt_unweighted}
In this section we consider fixed-parameter tractable
algorithms for unweighted geometric set cover with segments.
The setting where segments are required to be axis-parallel
(or limited to a constant number of directions) has a trivial FPT algorithm.
We present an FPT algorithm for geometric set cover
with unweighted segments, where segments are in arbitrary directions.

\subsection{Axis-parallel segments}
\begin{tw}
	\textbf{(FPT for segment cover with axis-parallel segments)}.
	There exists an algorithm that given a family $\sets$ of
	axis-parallel segments,
	a set of points $\points$
	and a parameter $k$,
	runs in time $\mathcal{O}(2^k)$,
	and outputs a solution $\sol \subseteq \sets$
	such that $|\sol| \le k$ and $\sol$ covers all points in~$\points$,
	or determines that such a set $\sol$ does not exist.
\end{tw}

\begin{proof}
We show an $\mathcal{O}(2^k)$-time branching algorithm.
In each step, the algorithm selects a point $a$ which is not yet covered,
branches to choose one of the two directions, and greedily chooses
a segment $a$ in that direction to cover.
This proceeds until either all points are covered or $k$~segments are chosen.

Let us take
the point $a=(x_a,y_a)$ which is the smallest 
among points that are not yet covered
in the lexicographic ordering
of points in $\mathbb{R}^2$.
We need to cover $a$ with some of~the~remaining segments.

Branch over the choice of one of the coordinates ($x$ or $y$);
without loss of generality, let us assume we chose $x$.
Among the segments lying on line $x = x_a$,
we greedily add to~the~solution the~one that covers the most points.
As $a$ was the smallest in the lexicographical order,
all points on the line $x = x_a$ have the $y$-coordinate larger than $y_a$.
Therefore, if we denote the~greedily chosen segment as $s$,
then any other segment on the line $x = x_a$ that covers $a$ can only
cover a subset of points covered by $s$.
Thus, greedily choosing $s$ is optimal.

In each step of the algorithm we add one segment to the solution,
thus the recursion can be stopped at depth $k$.
If no branch finds a solution, then this means
that a solution of size at most $k$ does not exist.
\end{proof}

Note that the same algorithm can be used for segments in $d$ directions,
where we branch over $d$ choices of directions, and it runs in complexity $\mathcal{O}(d^k)$.

\subsection{Segments in arbitrary directions}
\label{segments_in_arbitrary_direction}
In this section we consider the setting where segments are not constrained
to a constant number of directions. 
We present a fixed-parameter tractable algorithm,
parameterized by the size of the solution.

\segmentCoverFpt*

We will need the following lemmas proving properties of any
instance of the problem.

\begin{lemma}
   \label{fpt_reduction}
   Given an instance $(\sets, \points)$ of the segment cover problem,
   without loss of generality we can assume that
   no segment covers a superset of what another segment covers.
   That~is, for any distinct $A, B \in \sets$, we have
   $A \cap \points \not \subseteq B \cap \points$ and $A \cap \points \not \supseteq B \cap \points$.
\end{lemma}   
   
\begin{proof}
Assume towards a contradiction that there is an instance  $(\sets, \points)$,
and two distinct subsets of $\sets$,
$A, B$, such that $A \cap \points \subseteq B \cap \points$.

We construct a set $\sets' := \sets - \{A\}$.
We prove that for any solution $\sol$ of $(\sets, \points)$,
we can construct a~solution $\sol' \subseteq \sets'$,
such that $|\sol'| \le |\sol|$.
Let us take any solution $\sol$ of $(\sets, \points)$.
If $A \in \sol$, then $\sol' := \sol \cup \{B\} - \{A\}$,
otherwise $\sol' := \sol$.
Let us consider the case when $A \in \sol$,
because the other case is trivial.
Since $A \cap \points \subseteq B \cap \points$,
then $\sol \cup \{B\} - \{A\}$
covers any point from $\points$ that was covered by $\sol$.
Also, $|\sol \cup \{B\} - \{A\}| \le |\sol|$.
\end{proof}

\begin{lemma}
	\label{fpt_long_lines}
	Given an instance $(\sets, \points)$
	of the segment cover problem 
	transformed by Lemma~\ref{fpt_reduction},
	if there exists a line $L$ with at least
	$k+1$ points on it, then there exists a subset $A \subseteq \sets$,
	of size at most $k$,
	such that every solution $\sol$ with $|\sol| \le k$
	satisfies $|A \cap \sol| \ge 1$.
	Moreover, such a subset can be found in~polynomial time.
\end{lemma}
\begin{proof}
Let us enumerate the points from $\points$ that lie on $L$ as $x_1, x_2, \ldots, x_t$
in the order in which they appear on $L$.
Our proposed set is defined as:
$$A := \left\{ \text{segment collinear with } L \text{ that covers } x_i
\text{ and does not cover } x_{i-1} : i \in \{1, \ldots, k\}\right\},$$
where for $i = 1$ we just take a segment that covers $x_1$.
If such a segment does not exist for any point $x$
as above, then $x$ does not give rise to any segment in $A$.

We prove the lemma by contradiction. Let us assume that there
exists a~solution $\sol$ of size at most $k$ such that $\sol \cap A = \emptyset$.


Let $\sol_L$ be the set of segments from $\sol$ that are collinear with $L$.

Every segment that is not collinear with $L$ can cover at~most one of
the points that lie on~this line.
Hence, if $\sol_L$ was empty, then
$\sol$ would cover at most $k$ points on line $L$,
but $L$ had at least $k+1$ different points from $\points$ on it.

Therefore, we know that $\sol_L$ is not empty
and $|\sol - \sol_L| \le k-1$.
Segments from $\sol - \sol_L$ can cover at most $k-1$
points among $\{x_1, x_2, \ldots, x_k\}$, therefore at least
one of these points must be covered by segments from $\sol_L$.
We take the leftmost point from $\{x_1, x_2, \ldots, x_k\}$ that is
covered in $\sol_L$ and name it $a$.
After the transformation from Lemma \ref{fpt_reduction},
in $\sol$ there is only one segment
that starts in $a$ and is collinear with $L$,
therefore this segment must be in both $\sol$ and $A$.
This contradiction concludes the proof that $|A \cap \sol| \ge 1$
for any solution $\sol$ of size at most $k$.
\end{proof}

We are now ready to prove Theorem \ref{segment_cover_fpt}.

\begin{proof}[Proof of Theorem \ref{segment_cover_fpt}.]
We will prove this theorem by presenting a branching algorithm that
works in desired complexity. It first branches over the
choice of segments to cover the lines with \textit{many} points
and then solves a small instance (where every line has at most $k$ points)
by checking all possible solutions.

\subparagraph{Algorithm.}
We present a recursive algorithm. Given an instance of the problem:

\begin{enumerate}[label={(\arabic*)}]
\item Use Lemma \ref{fpt_reduction} to remove some redundant segments from our instance.
\item If there exists a line with at least $k+1$ points from $\points$,
we branch over the choice of adding to~the~solution
one of~the~at~most $k$ possible segments
provided by Lemma \ref{fpt_long_lines}; name this segment $s$
and name the set of points from $\points$ that lie on $s$ as $S$.
By recursion, we find a~solution $\sol$
for the instance $(\points - S, \sets - \{s\})$,
and parameter $k-1$. We return $\sol \cup \{s\}$.
Note that if Lemma \ref{fpt_long_lines} returned $\emptyset$,
then we respond \texttt{NO}.
\item If every line has at most $k$ points on it and $|\points| > k^2$,
then answer \texttt{NO}.
\item If $|\points| \le k^2$, solve the problem by brute force:
check all subsets of $\sets$ of size at most $k$.
\end{enumerate}

\subparagraph{Correctness.}

Lemma \ref{fpt_long_lines} proves that at least one segment that we
branch over in (1) must be present in every solution $\sol$ with $|\sol| \le k$.
Therefore, the recursive call can find a~solution, provided there exists one.

In (2) the answer is no, because every line covers no more than $k$ points
from $\points$, which implies the same about every segment from $\sets$.
Under this assumption
we can cover only $k^2$ points with a solution of size $k$, which is less
than $|\points|$.

Checking all possible solutions in (3) is trivially correct.


\subparagraph{Complexity.}

In the leaves of the recursion we have $|\points| \le k^2$, so $|\sets| \le k^4$,
because every segment can be uniquely identified by the two extreme points it covers
(by Lemma \ref{fpt_reduction}). Therefore, there are $\binom{k^4}{k}$
possible solutions to check, each can be checked in time $\mathcal{O}(k|\points|)$.
Thus, (3) takes time $k^{\mathcal{O}(k)}$.


In this branching algorithm our parameter $k$ is decreased with every
recursive call, so we have at most $k$ levels of recursion with
branching over $k$ possibilities. Candidates to branch over
can be found on each level in time $\mathcal{O}((|\points|\cdot|\sets|)^{\mathcal{O}(1)})$.

Reduction from Lemma \ref{fpt_reduction} can be implemented
in time $\mathcal{O}((|\points|\cdot|\sets|)^{\mathcal{O}(1)})$.

It follows that the overall complexity
is $\mathcal{O}(((|\points|\cdot|\sets|)^{\mathcal{O}(1)}) \cdot k^{\mathcal{O}(k)})$
\end{proof}


\section{Fixed-parameter tractable algorithm for weighted segments with $\delta$-extensions}
\label{section:fpt_weighted}

In this section we consider the geometric set cover problem
for weighted segments relaxed with $\delta$-extensions.
We show that this problem
admits an FPT algorithm when parameterized by the size
of the solution and $\delta$.
In the next chapter we show that the assumption
about the problem being relaxed with $\delta$-extensions is necessary:
we prove that geometric set cover problem
for weighted segments (without extensions) is W[1]-hard, which means
there does not exist any FPT algorithm parameterized by solution size for it,
assuming FPT $\neq$ W[1].

\begin{tw}[FPT for weighted segment cover with $\delta$-extensions]{
	\label{fpt_weighted_segment}
	There exists an algorithm that given a family $\sets$ of
	$n$ weighted segments (in any direction),
	a set of $m$ points $\points$, and parameters $k$ and $\delta > 0$,
	such that it
	runs in time $f(k, \delta) \cdot (nm)^c$ for some computable function $f$ and a constant $c$ and
	outputs a set $\sol$ such that:
	\begin{itemize}
	\item $\sol \subseteq \sets$,
	\item $|\sol| \le k$,
	\item $\sol^{+\delta}$ covers all points in $\points$,
	\item the weight of $\sol$ is not greater than the weight
	of an optimum solution of size at most $k$
	for this problem without $\delta$-extensions
	\end{itemize}
	or determines that there is no set $\sol$ with $|\sol| \le k$
	such that $\sol$ covers all points in $\points$.
}\end{tw}


To solve this problem we will introduce a lemma about choosing
a \textit{dense} subset of points. A dense subset of points
for a set of collinear points $C$ and parameters $k$ and $\delta$
is a subset of $C$ such that
if we cover it with at most $k$ segments,
these segments after $\delta$-extensions will cover all of the points from $C$.
We will prove that such set 
of size bounded by some function $f(k, \delta)$
always exists (Lemma \ref{dense_set_exists}).
Later, Lemma \ref{dense_set_exists} will allow us to find a kernel
for our original problem.

\begin{defi}
	For a set of collinear points $C$,
	a subset $A \subseteq C$ is \textbf{$(k,\delta)$-dense} 
	if for any set of segments $R$ that covers $A$ and
	such that $|R| \le k$, it holds that $R^{+\delta}$ covers $C$.
\end{defi}

\begin{lemma}
	\label{dense_set_exists}
	For any set of collinear points $C$, $\delta > 0$ and $k \ge 1$,
	there exists a $(k,\delta)$-dense set $A \subseteq C$ of size
	at most $(2+\frac{2}{\delta})^k$.
	Moreover, there exists an algorithm that computes the $(k,\delta)$-dense set
	in time $O(|C| \cdot (2+\frac{2}{\delta})^k)$.
\end{lemma}

\begin{proof}
We prove this for a fixed $\delta$ by induction on $k$.

\subparagraph{Inductive hypothesis.}
For any set of collinear points $C$, there exists a set $A$ such that:
\begin{itemize}
\item $A$ is subset of $C$,
\item $A$ is $(\ell, \delta)$-dense for every $1 \le \ell \le k$,
\item $|A| \le (2+\frac{2}{\delta})^k$,
\item the extreme points of $C$ are in $A$.
\end{itemize}

\subparagraph{Base case for $k = 1$.}
It is sufficient that $A$ consists of the extreme points of $C$.

If they are covered with one segment, it must be a segment 
that includes the extreme points from $C$, so it covers the whole set $C$.

There are at most 2 extreme points in $C$ and $2 < 2+\frac{2}{\delta}$.

\subparagraph{Inductive step.}
Assuming inductive hypothesis for any set of collinear points $C$
and for parameter $k$, we will prove it for $k+1$.

Let $s$ be the minimal segment that includes all points from $C$.
That is, the extreme points of $C$ are endpoints of $s$.

We define $M = \lceil1+\frac{2}{\delta}\rceil$ subsegments of $s$
by splitting $s$ into $M$ closed segments of equal length.
We name these segments $v_i$, note that
$|v_i| = \frac{|s|}{M}$ for each $1 \le i \le M$.

Let $C_i$ be the subset of $C$ consisting of points lying on $v_i$.

Let $t_i$ be the segment with endpoints being the extreme points of $C_i$.
It might be a degenerate segment if $C_i$ consists of one point,
or $t_i$ might be empty if $C_i$ is empty.

Figure $\ref{fig:fpt_v_f_def}$ presents an example
of such segments $v_i$ and $t_i$.

\begin{figure}[h]
\begin{center}
\def\svgwidth{\columnwidth}
\input{fpt_v_t_def.pdf_tex}
\end{center}
\caption{\textbf{Example of segments $v_i$ and $t_i$.}}
Example for $M = 7$ and some set of points (marked with black circles).
The top panel shows segments $v_i$ and the bottom panel shows segments $t_i$
on the same set of points.
$a$ and $b$ are the extreme points and therefore segment $s$
ends at $a$ and $b$.
Red segments depict the split into $M$ segments of equal length $v_i$.
Blue segments depict the segments $t_i$. $t_5$ is an empty segment,
because there are no points that lie on segment $v_5$.
Segments $t_3$ and $t_7$ are degenerated to one point --
$c$ and $d$ respectively.
Segments $t_1$ and $t_2$ share one point $b$.
\label{fig:fpt_v_f_def}
\end{figure}

We use the inductive hypothesis to choose $(k, \delta)$-dense sets $A_i$
for sets $C_i$. Note that if $|C_i| \le 1$, then $A_i = C_i$
and it is still a $(k, \delta)$-dense set for $C_i$.

Then we define $A = \bigcup_{i=1}^{M} A_i$.
Thus $A$ includes the extreme points of $C$,
because they are included in the sets $A_1$ and $A_M$.

The size of each $A_i$ is at most $(2+\frac{2}{\delta})^{k}$
from the inductive hypothesis, therefore size of $A$ is at most:
$$M\left(2+\frac{2}{\delta}\right)^{k} =
\left\lceil1+\frac{2}{\delta}\right\rceil\cdot\left(2+\frac{2}{\delta}\right)^{k}
\le \left(2+\frac{2}{\delta}\right)^{k+1}.$$


\subparagraph{Proof that $A$ is $(k, \delta)$-dense for $C$.}
Let us take any cover of $A$ with $k+1$ segments and call it $\sol$.

For every segment $t_i$, if there exists a segment $x$ in $\sol$ 
that is disjoint with $t_i$,
then we have a cover of $A_i$ with at most $k$
segments using $\sol - \{x\}$.
Since $A_i$ is $(k, \delta)$-dense for $t_i$ and $C_i$,
$(\sol - \{x\})^{+\delta}$ covers $C_i$.
So $\sol^{+\delta}$ covers $C_i$ as well.

If there exists a segment $t_i$ for which a segment $x$ as defined above
does not exist, then all $k+1$ segments that cover
$A_i$ intersect $t_i$.
An example of such segments is depicted in Figure~\ref{fig:fpt_tricky_case}.
Let us consider any such $t_i$.
By inductive hypothesis, the endpoints of $s$ are
in $A_1$ and $A_M$ respectively, so $\sol$ must cover them.
For each endpoint of $s$, there exists
a segment that contains this endpoint and intersects $t_i$.
Let us call these two segments $y$ and $z$. It follows that:
$|y| + |z| + |t_i| \ge |s|$.
Since $|t_i| \le |v_i| = \frac{|s|}{M} \le \frac{|s|}{1+\frac{2}{\delta}} = \frac{|s|\delta}{\delta+2}$,
we have $\max(|y|, |z|) \ge |s|(1-\frac{\delta}{\delta+2})/2 = \frac{|s|}{\delta+2}$.

\begin{figure}[h]
\begin{center}
\def\svgwidth{\columnwidth}
\input{fpt_tricky_case.pdf_tex}
\end{center}
\caption{\textbf{Example of all $k+1$ segments intersecting one segment $t_i$.}}
Both panels show the same set $\points$ (black circles),
the same as in Figure $\ref{fig:fpt_v_f_def}$.
The top panel shows blue segments $t_i$ for $M=7$.
The bottom panel shows green segments -- solution $\sol$ of size 4.
All segments from $\sol$ intersect $t_4$.
Segments $z$ and $y$ are named in the figure.
\label{fig:fpt_tricky_case}
\end{figure}

After $\delta$-extension, the longer of these segments will
expand at both ends by at least:
$$\max(|y|, |z|)\delta \ge \frac{|s|\delta}{\delta+2} =
\frac{|s|}{1+\frac{2}{\delta}} \ge \frac{|s|}{M} = |v_i| \ge |t_i|.$$

Therefore, the longer of segments $y$ and $z$ will cover the whole segment $t_i$
after $\delta$-extension. We conclude that $\sol^{+\delta}$ covers $C_i$.

Since $C = \bigcup_{i=1}^M C_i$,
it follows that $\sol^{+\delta}$ covers $C$.


\subparagraph{Algorithm.}

We can simulate the inductive proof presented above by a recursive algorithm with
the following complexity:
$$O\left(|C|+\frac{1}{\delta}\right) + O\left(|C|\cdot\left(2+\frac{2}{\delta}\right)^k\right).$$

\end{proof}

Let us now formulate some claims about the
properties for the problem parameterized by the solution size.
These properties provide bounds for different
objects in the problem instance,
which help us to find a small kernel for the problem
or conclude that the optimum
solution to this instance must be in terms of size above some treshold.

\begin{defi}
A line in the plane is \textbf{long}
if there are at least $k+1$ points from $\points$ on it.
\end{defi}

\begin{claim}
\label{few_long_lines}
If there are more than $k$ different long lines, then 
$\points$ can not be covered with $k$ segments.
\end{claim}

\begin{proof}
We prove the claim by contradiction.
Let us assume that we have at least $k+1$ different
long lines in our instance of the problem
and there is a solution $\sol$ of size at most $k$
covering points $\points$.

Choose any long line $L$.
Every segment from $\sol$ which is not collinear with $L$,
covers at most one point that lies on $L$.
$L$ is long, so there are at least $k+1$ points from $\points$ that lie on $L$.
That implies that there must be a segment in $\sol$ that is
collinear with $L$.

Since we have at least $k+1$ different long lines,
there are at least $k+1$
segments in $\sol$ collinear with different lines.
This contradicts with the assumption that $|\sol| \le k$.
\end{proof}

\begin{claim}
\label{few_points}
If there are more than $k^2$ points from $\points$
that do not lie on any long line,
then $\points$ can not be covered with $k$ segments.
\end{claim}

\begin{proof}
We prove the claim by contradiction.
Let us assume that we have at least $k^2+1$ points
from $\points$ that do not lie on any long line, call this set $A$,
and a solution $\sol$ of size at most $k$
covering all points in $\points$.

Every segment $s$ from $\sol$ covers at most $k$
points from $A$.
This is because if $s$ covered at least $k+1$ points from $A$,
then the line in the direction of $s$ would be a long line
and that contradicts the definiton of $A$.

If every segment from $\sol$ covers at most $k$ points from $A$
and $|\sol| \le k$, then at most $k^2$ points from $A$ are covered by $\sol$
and that contradicts the fact that $\sol$ is a solution to the given
geometric set cover instance.
\end{proof}

We are now ready to give a proof of Theorem \ref{fpt_weighted_segment}.

\begin{proof}[Proof of Theorem \ref{fpt_weighted_segment}]
Our goal is to either answer \texttt{NO} or to find a kernel
$(\sets', \points')$ of bounded size, such that:
\begin{itemize}
\item \textit{(Property 1)} for every solution
$\sol$ to $(\sets, \points)$ of size at most $k$,
there exists a set $\sol_1 \subseteq \sets'$ such that
$\sol_1 \le k$, weight of $\sol_1$ is not greater than weight of $\sol$
and $\sol_1$ covers $\points'$;
\item \textit{(Property 2)}
for every set $\sol_2 \subseteq \sets'$ such that $|\sol_2| \le k$
and $\sol_2$ covers points in $\points'$, $\sol_2^{+\delta}$
covers points in original instance $\points$.
\end{itemize}

If we found such sets $(\sets', \points')$,
using \textit{Property 1} we know that optimum solution 
of size at most $k$ to $(\points', \sets')$
has no greater weight than optimum solution
of size at most $k$ to $(\points, \sets)$.
Using \textit{Property 2} we know that
any solution to $(\points', \sets')$
after $\delta$-extensions covers $\points$.

Therefore finding such sets in desired complexity
is sufficient to prove Theorem \ref{fpt_weighted_segment}.

\paragraph{Definition of $\points'$ and $\sets'$.}
Let us name the number of different long lines as $l$.
Applying Claims \ref{few_long_lines} and \ref{few_points},
if we have more than $k$ different long lines
or more than $k^2$ points from $\points$
that do not lie on any long line, then we answer \texttt{NO},
becase these lemmas prove that there is no solution of size at most $k$
to this instance.

Otherwise, we can split $\points$ into at most $k+1$ sets:
\begin{itemize}
\item $D$: points that do not lie on any long line, $|D| \le k^2$;
\item $C_i$ for $1 \le i \le l$: points that lie on the $i$-th long line, $|C_i| > k$.
\end{itemize}
Note that sets $C_i$ do not need to be disjoint.

Then for every set $C_i$ we can use Lemma \ref{dense_set_exists}
to obtain a $(k,\delta)$-dense set $A_i$
for $C_i$ with $|A_i| \le (2+\frac{2}{\delta})^k$.

We define $\points':= D \cup (\bigcup A_i)$. $\points'$ has size at most
$k^2 + k(2+\frac{2}{\delta})^k$.
We define $\sets'$ as
for every pair of points $\points'$, we can choose one segment from
$\sets$ that has the lowest weight
among segments that cover these points 
or decide that there is no segment that covers them.
There are at most $|\points'|^2$ different segments in $\sets'$,
Therefore both $\sets'$ and $\points'$ have size bounded
by some function $f(k)$.

\paragraph{Proof of Property 1.}
First, we prove that
for every set $\sol_2 \subseteq \sets'$ such that $|\sol_2| \le k$
and $\sol_2$ covers points in $\points'$, $\sol_2^{+\delta}$
covers points in original instance $\points$.

Let us take such a set $\sol_2$.

$\points$ is separated into several parts -- sets $D$ and $C_i$.
Points from $D$ are covered by $\sol_2$, because $D$ is part of $\points'$.
Each point from any $A_i$ is covered, because $A_i$ is a part of $\points'$;
$A_i$ is a $(k,\delta)$-dense set for $C_i$, therefore $\sol_2^{+\delta}$
covers all points in $C_i$. Therefore $\sol_2^{+\delta}$ covers
all points in $\points$.

\paragraph{Proof of Property 2.}
Secondly, we prove that for every solution
$\sol$ to $(\sets, \points)$ of size at most $k$,
there exists a set $\sol_1 \subseteq \sets'$ such that
$\sol_1 \le k$ and
$\sol_1^{+\delta}$ covers all points in $\points$ and
weight of $\sol_1$ is not greater than weight of $\sol$.

For every segment in $\sol$, say $s$,
let us look at the points from $\points'$ that lie on $s$
and call this set of points $F$.
$F$ is a set of collinear points for course.
We can cover $F$ with any segment that covers extreme points of $F$,
because all other points lie on the segment between these points.
Therefore we can replace $s$ with a segment $s'$
that has lowest weight among the points that cover extreme points of $F$.
Such a segment belongs to $\sets'$, because this is how it was defined.
Of course segment $s'$ also have weight no greater than weight of $s$,
because $s$ also covers $F$.

Therefore we produced the set $\sol_1$ that has the same size,
weight not greather than $\sol$ and it covers $\points'$.

\paragraph{Complexity}
We find solutin of $(\points', \sets')$ by iterating
over all possible subsets of $\sets$.
Finding sets $\sets'$ and $\points'$ and then solving 
problem for kernel has overall complexity
$(|\sets| + |\points|)^{O(1)}O((2 + \frac{2}{\delta})^k) + O((k^2 + k(2 + \frac{2}{\delta})^k)^k)$.
\end{proof}


\chapter{W[1]-hardness of $\WeightedSegmentSetCover$}
\label{chapter:w1_hard}

In this chapter we consider the $\WeightedSegmentSetCover$ problem with 
axis-parallel or right-diagonal segments.
In Theorem~\ref{w1_hard} below, we prove that this problem is 
W[1]-hard when parameterized by the size of the solution.
We believe that the construction can be improved to only
utilize the axis-parallel segments.

\wOneHard*

\newcommand{\GridTiling}{\textsc{Grid Tiling}}

\section{$\GridTiling$}

In~order to prove Theorem \ref{w1_hard}
we will show a reduction from a W[1]-hard problem:
\textsc{Grid} \textsc{Tiling}.
This problem was introduced in \cite{marx_grid_tiling}
(the author called it matrix tiling instead).
It was originally described as an approximation problem,
but W[1]-hardness follows directly from the theorems stated there.
For a more contemporary description of this problem
and a proof of W[1]-hardness, see Chapter 14 of \cite{platypus_book}.

\newcommand{\pow}{\mathsf{Pow}}

\begin{defi}
We define the \textbf{powerset} of a set $A$, denoted as $\pow(A)$,
as the set of all subsets of $A$, i.e. $\pow(A) = \{B : B \subseteq A\}$.
\end{defi}

\begin{defi}
In the \textbf{$\GridTiling$} problem we are given integers $n$ and $k$,
and a function
$f : \{1, \ldots, k\} \times \{1, \ldots, k\} \rightarrow \pow(\{1, \ldots, n\} \times \{1, \ldots, n\})$
specifying the set of allowed tiles for each cell of a $k \times k$ grid.
The task is to decide whether there exist functions
$x,y : \{1, \ldots, k\} \rightarrow \{1, \ldots, n\}$
that assign colors from $\{1, \ldots, n\}$
to respectively columns and rows of the grid,
so that $(x(i), y(j)) \in f(i, j)$ for all $i,j \in \{1, \ldots, k\}$.
\end{defi}

In short, in the $\GridTiling$ problem one needs to assign numbers
to rows and columns in such a way
that for every pair of a row and a column,
the pair of colors assigned
to the row and column 
belongs to the allowed set of tiles for this pair.
The next theorem describes the complexity of this problem,
which is W[1]-hard when parameterized by the size of the grid.

\definecolor{alternative_sol}{RGB}{48, 48, 255}
\definecolor{bad_sol}{RGB}{255, 48, 48}

\begin{figure}[h]
\begin{center}
\begin{tabular}{ c|c|c|c|c| } 
          & $x(1)=3$ & $x(2)=1$ & $x(3)=3$ & $x(4) = 7$\\ 
 \hline
 $y(4)=1$
	& \makecell{$\textcolor{alternative_sol}{(2,1)};(2,2);$\\$\textbf{(3,1)};(3,9)$}
	& $\textbf{(1,1)}; (3,1)$
	& $\textbf{(3,1)}; (7,2)$
	& $\textcolor{bad_sol}{(2,1)}; \textbf{(7,1)}$\\ 
 \hline
 $y(3)=1$
	& \makecell{$\textcolor{alternative_sol}{(2,1)};\textbf{(3,1)};$\\$(4,2);(8,2)$}
	& $\textbf{(1,1)}; (1,3)$
	& $\textbf{(3,1)}; (4,3)$
	& $\textcolor{bad_sol}{\textbf{(2,2)}}; \textbf{(7,1)}$\\ 
 \hline
 $y(2)=6$
	& $\textcolor{alternative_sol}{(2,6)};\textbf{(3,6)}$
	& \makecell{$(1,2); \textbf{(1,6)};$\\$(2,6)$}
	& $(2,6); \textbf{(3,6)}$
	& $\textcolor{bad_sol}{(2,6)};\textbf{(7,6)}$\\ 
 \hline
 $y(1)=4$
	& \makecell{$\textcolor{alternative_sol}{(2,4)};(2,6);$\\$\textbf{(3,4)};\textcolor{bad_sol}{(3,9)}$}
	& $\textbf{(1,4)}; \textcolor{bad_sol}{(1,9)}$
	& $\textbf{(3,4)}; \textcolor{bad_sol}{(3,9)}$
	& $\textcolor{bad_sol}{(2,9)}; \textbf{(7,4)}$\\ 
 \hline
\end{tabular}
\end{center}
\caption{\textbf{Example of a $\GridTiling$ instance and its solution.}}
In the first row and column of the table you can see the solution:
functions $x$ and $y$.
The~tiles used in this solution are marked in \textbf{bold}.
If we instead chose the tiles marked in \textcolor{alternative_sol}{blue}
(whenever there is one, taking the tile marked in \textbf{bold} otherwise),
then that corresponds to setting $x(1)=2$, and would also form a correct solution.
On the other hand, if we instead chose the tiles marked in \textcolor{bad_sol}{red}
(as before), then this corresponds to setting ${y(1)=9}$ and $x(4)=2$
and that would $\textbf{not}$ form a correct solution.
Even though the first row is correct,
the cell with coordinates (3,4) requires tile (2,1), not (2,2)
(marked in \textbf{\textcolor{bad_sol}{bold red}}).
\label{fig:grid_tiling_exmample}
\end{figure}


\begin{tw}
\label{grid_tiling_w1_hard}
\textbf{(\cite{marx_grid_tiling}).}
$\GridTiling$ is W[1]-hard when parameterized by $k$ and
assuming ETH, there is no $f(k)\cdot n^{o(k)}$-time
algorithm solving the $\GridTiling$ problem
for any computable function $f$.
\end{tw}

The remainder of this section is devoted to proving Theorem \ref{w1_hard}
by a reduction from a~$\GridTiling$ problem instance
with parameter $k$ (number of rows in the grid)
to a \textsc{Weighted} \textsc{Segment} \textsc{Set} \textsc{Cover}
instance with parameter $k^2$ (size of solution).
This reduction is described in Lemma~\ref{w1_construction}.
This proves the W[1]-hardness of the $\WeightedSegmentSetCover$ problem,
because if we could solve it with an FPT algorithm,
then we could also solve the $\GridTiling$ problem
(which we reduced to $\WeightedSegmentSetCover$).
Therefore, $\WeightedSegmentSetCover$ with setting
described in Theorem \ref{w1_hard}
is at least as hard as the $\GridTiling$ problem.

\newcommand{\hvWeight}{W_{\mathsf{hv}}}
\newcommand{\solWeight}{\hvWeight+k^2\delta }
\newcommand{\instanceSetCover}{(\points, \sets, w, 3k^2+2k)}
\newcommand{\instanceGridTiling}{(n,k,f)}
\newcommand{\yes}{\texttt{YES}}
\newcommand{\no}{\texttt{NO}}

\section{Statement of reduction}

Let us denote an instance of $\GridTiling$ problem as $\instanceGridTiling$ consisting of:
\begin{itemize}
\item the number of colors $n$,
\item the size of the grid $k$,
\item the function specifying the allowed tiles
$f : \{1, \ldots, k\} \times \{1, \ldots, k\} \rightarrow \pow(\{1, \ldots, n\} \times \{1, \ldots, n\})$.
\end{itemize}

Let us also define constants: 
\begin{eqnarray*}
\epsilon & := & \frac{1}{2k^2} \\
\delta & := & \frac{1}{4k^4} \\
\hvWeight & := & 2k^2(n^2+1) -4k^2\epsilon -4k(1-\epsilon)
\end{eqnarray*}
which are going to be used when defining the weight of the constructed
instance of \textsc{Weighted} \textsc{Segment} \textsc{Set} \textsc{Cover}.


\begin{lemma}
\label{w1_construction}
Given an instance $\instanceGridTiling$ of the $\GridTiling$ problem,
we can construct an instance $\instanceSetCover$ of $\WeightedSegmentSetCover$
such that:
\begin{enumerate}[label={(\arabic*)}]
\item \label{part1} if the answer to $\instanceGridTiling$ is $\yes$, then there exists a solution
to $\instanceSetCover$ of weight at most $\solWeight$;
\item \label{part2} if there exists a solution to $\instanceSetCover$ of weight at most $\solWeight$,
then the answer to $\instanceGridTiling$ is $\yes$.
\end{enumerate}
\end{lemma}


First, let us prove Theorem \ref{w1_hard} using Lemma \ref{w1_construction}.

\begin{proof}[Proof of Theorem \ref{w1_hard}]
Let us take any instance $(n,l,f)$ of the $\GridTiling$ problem.
We prove the theorem by contradiction, therefore we assume
that \textsc{Weighted} \textsc{Segment} \textsc{Set} \textsc{Cover}
parameterized by solution size $k = 3l^2+2l$ admits a
$g(k)\cdot n^{o(\sqrt{k})}$-time algorithm for some computable function $g$.

Using Lemma \ref{w1_construction} let us construct an instance $I$
for $(n,l,f)$.
Let us assume that the optimum solution of size at most $k$
to the instance $I$ has weight $u$.
Using \ref{part2} we know that if $u \le \solWeight$,
then the answer to $(n,l,f)$ is $\yes$.
If $u > \solWeight$, then using \ref{part1}
we know that the answer to $(n,l,f)$ must be $\no$.

Therefore if we could find the solution in time $g(k) \cdot n^{o(\sqrt{k})}$,
then we could solve the $\GridTiling$ problem
in time $g(l)\cdot n^{o(l)}$ by constructing an instance of
\textsc{Weighted} \textsc{Segment} \textsc{Set} \textsc{Cover}, solving it 
for parameter $k$ in time $n^{o(\sqrt{3l^2+2l})}$
and then answering based on the weight
of the optimum solution.
As $\mathcal{O}(n^{o(l)}) \subseteq \mathcal{O}(n^{o(\sqrt{3l^2+2l})})$,
the existence of this algorithm contradicts Theorem
\ref{grid_tiling_w1_hard}.
Hence such an algorithm can not exist.
\end{proof}

We prove Lemma \ref{w1_construction} in subsequent sections.
First, we define a constructed instance $I$, later property
\ref{part1} is proved by Lemma \ref{set_cover_solution_exists}
and property \ref{part2} is proved by Lemma \ref{grid_tiling_exists}.

In the proof of Lemma \ref{w1_construction}
(see proof of Lemma \ref{grid_tiling_exists})
we do not use the assumption that
the solution is bounded by the size,
which the problem is parameterized by, $3k^2+2k$.
If we had a permissive FPT algorithm
that finds a solution of any size that still
has weight no more than $\solWeight$,
then we still would have a contradiction with $\GridTiling$ being W[1]-hard
in proof of Theorem \ref{w1_hard}.
Thus, this reduction
proves that the problem is not only W[1]-hard, but assuming ETH 
there also does not exist permissive FPT algorithm for this problem.
Formally we state this in Theorem $\ref{permissive_w1_hard}$ below.


\begin{restatable}{tw}{permissiveWOneHard}
\label{permissive_w1_hard}
\textbf{(Permissive FPT does not exist).}
	Consider the problem of covering a set $\points$ of points
	using segments from a set $\sets$ 
	with non-negative weights $w : \sets \rightarrow \mathbb{R^+}$
	so that the weight of the cover is minimal.
	Let $\sol^k$ be the
	optimum solution to this problem of size at most $k$.
	The task is to find a solution $\sol$ of any size
	such that weight of $\sol$ is not greater than the weight of $\sol^k$.
	
	Assuming ETH, there is no algorithm for this
	problem with running time
	$f(k)\cdot(|\points| + |\sets|)^{o(\sqrt{k})}$
	for any computable function $f$.
	Moreover, this holds even if all segments in $\sets$
	are axis-parallel or right-diagonal.
\end{restatable}

\section{Construction of the $\SegmentSetCover$ instance}
\newcommand{\order}{\mathsf{order}}
\newcommand{\matchv}{\mathsf{match}_v}
\newcommand{\matchh}{\mathsf{match}_h}

We construct an instance $\instanceSetCover$ of $\SegmentSetCover$ as follows.

First, let us choose any bijection
$\order : \{1, \ldots, n^2\} \rightarrow \{1, \ldots, n\} \times \{1, \ldots, n\}$.


Define $\matchv(i, j)$ and $\matchh(i, j)$
as Boolean functions denoting whether two points share x or y coordinate:
$$\matchv(i, j) \text{ is } \true \iff
\order(i) \text{ and } \order(j) \text{ have the same x coordinate,}$$
$$\matchh(i, j) \text{ is } \true \iff
\order(i) \text{ and } \order(j) \text{ have the same y coordinate.}$$


\subsection{Points}

For $1 \le i,j \le k$ and $1 \le t \le n^2$ define points:
	$$h_{i, j, t} := (i \cdot (n^2+1) + t, j \cdot (n^2+1)),$$
	$$v_{i, j, t} := (i \cdot (n^2+1), j \cdot (n^2+1) + t).$$
	
Let us define sets $H$ and $V$ as:
$$H := \{h_{i, j, t} : 1 \le i, j \le k, 1 \le t \le n^2\},$$
$$V := \{v_{i, j, t} : 1 \le i, j \le k, 1 \le t \le n^2\}.$$
	
Let us recall that $\epsilon = \frac{1}{2k^2}$.
For a point $p = (x, y)$ we define points:
$$p^{L} := (x - \epsilon, y),$$
$$p^{R} := (x + \epsilon, y),$$
$$p^{U} := (x, y + \epsilon),$$
$$p^{D} := (x, y - \epsilon).$$

Then we define the point set as follows:
$$\points := H \cup \{p^L : p \in H\} \cup \{p^R : p \in H\}
\cup V \cup \{p^U : p \in V\} \cup \{p^D : p \in V\}.$$

\begin{defi}
	\label{guard_def}
	For every point $p \in H$, we name point $p^L$ its \textbf{left guard}
	and point $p^R$ its \textbf{right guard}.
	
	Similarly for every points $p \in V$, we name point $p^D$ its \textbf{lower guard}
	and point $p^U$ its \textbf{upper guard}.
\end{defi}

\subsection{Segments}
\newcommand{\hor}[4]{\mathsf{hor}_{#1,#2,#3,#4}}
\newcommand{\ver}[4]{\mathsf{ver}_{#1,#2,#3,#4}}
\newcommand{\horbeg}[2]{\mathsf{horBeg}_{#1,#2}}
\newcommand{\verbeg}[2]{\mathsf{verBeg}_{#1,#2}}
\newcommand{\horend}[2]{\mathsf{horEnd}_{#1,#2}}
\newcommand{\verend}[2]{\mathsf{verEnd}_{#1,#2}}

For $1 \le i,j \le k$ and $1 \le t, t_1, t_2 \le n^2$ define segments:
\begin{eqnarray*}
\hor{i}{j}{t_1}{t_2} & := & (h^R_{i,j,t_1}, h^L_{i+1, j, t_2}), \\
\ver{i}{j}{t_1}{t_2} & := & (v^U_{i,j,t_1}, v^D_{i, j+1, t_2}), \\
\horbeg{i}{t} & := & (h^L_{1, i, 1}, h^L_{1, i, t}), \\
\horend{i}{t} & := & (h^R_{k, i, t}, h^R_{k, i, n^2}), \\
\verbeg{i}{t} & := & (v^D_{i, 1, 1}, v^D_{i, 1, t}), \\
\verend{i}{t} & := & (v^U_{i, k, t}, v^U_{i, k, n^2}). \\
\end{eqnarray*}

\newcommand{\allhor}{\mathsf{HOR}}
\newcommand{\allver}{\mathsf{VER}}
\newcommand{\alldiag}{\mathsf{DIAG}}

Next, we define sets of vertical and horizontal segments:
\begin{eqnarray*}
\allhor &:= &\{\hor{i}{j}{t_1}{t_2} : 1 \le i < k, 1 \le j \le k,
1 \le t_1, t_2 \le n^2, \matchh(t_1, t_2) \text{ holds}\} \\
&\cup &\{\horbeg{i}{t} : 1 \le i \le k, 1 \le t \le n^2\}
\\
&\cup &\{\horend{i}{t} : 1 \le i \le k, 1 \le t \le n^2\},
\end{eqnarray*}
\begin{eqnarray*}
\allver &:= &\{\ver{i}{j}{t_1}{t_2} : 1 \le i \le k, 1 \le j < k,
1 \le t_1, t_2 \le n^2, \matchv(t_1, t_2) \text{ holds}\} \\
&\cup &\{\verbeg{i}{t} : 1 \le i \le k, 1 \le t \le n^2\}
\\
&\cup &\{\verend{i}{t} : 1 \le i \le k, 1 \le t \le n^2\}.
\end{eqnarray*}
An example is depicted in Figure \ref{fig:segments_def}.

Finally, we also define a set of right-diagonal segments:
$$\alldiag := \{ (h_{i, j, t}, v_{i, j, t}) :
	1 \le i, j \le k, 1 \le t \le n^2, \order(t) \in f(i, j)\}.$$
An example of such segments is depicted in Figure \ref{fig:diag_def}.

	
\definecolor{guards_color}{RGB}{80, 120, 255}

{\tikzset{guard_h/.style={
    circle, draw=guards_color, fill, fill=guards_color, minimum size=2pt,inner sep=0pt, outer sep=0pt,
    prefix after command= {\pgfextra{\tikzset{every
    label/.style={label distance=0.1cm,rotate=90,text=guards_color,font=\footnotesize}}}}
    }
}
{\tikzset{node_h/.style={
    circle, draw=black, fill, fill=black, minimum size=4pt,inner sep=0pt, outer sep=0pt,
    prefix after command= {\pgfextra{\tikzset{every
    label/.style={label distance=0.1cm,rotate=90,text=black}}}}
    }
}
{\tikzset{guard_v/.style={
    circle, draw=guards_color, fill, fill=guards_color, minimum size=2pt,inner sep=0pt, outer sep=0pt,
    prefix after command= {\pgfextra{\tikzset{every
    label/.style={label distance=0.1cm,text=guards_color,font=\footnotesize}}}}
    }
}
{\tikzset{node_v/.style={
    circle, draw=black, fill, fill=black, minimum size=4pt,inner sep=0pt, outer sep=0pt,
    prefix after command= {\pgfextra{\tikzset{every
    label/.style={label distance=0.1cm,text=black}}}}
    }
}

\newcommand{\addNodeV}[2]{
	\node[guard_v, label={left:$v_{i,j,#2}^D$}] at (0, \l#1) {};
	\node[node_v, label={left:$v_{i,j,#2}$}] at (0,\x#1) {};
	\node[guard_v, label={left:$v_{i,j,#2}^U$}] at (0, \r#1) {};
}

\newcommand{\addNodeH}[2]{
	\node[guard_h, label={left:$h_{i,j,#2}^L$}] at (\l#1,0) {};
	\node[node_h, label={left:$h_{i,j,#2}$}] at (\x#1,0) {};
	\node[guard_h, label={left:$h_{i,j,#2}^R$}] at (\r#1,0) {};
}

\begin{figure}
\begin{center}
\begin{tikzpicture}[main/.style = {draw, circle}]
\tikzmath{
	\step=2;
	\eps=0.5;
	\x1=\step;
	\x2=\x1+\step;
	\x3=\x2+\step;
	\x4=\x3+\step;
	\x5=\x4+\step;
	\x6=\x5+\step;
	\l1=\x1-\eps;
	\r1=\x1+\eps;
	\l2=\x2-\eps;
	\r2=\x2+\eps;
	\l3=\x3-\eps;
	\r3=\x3+\eps;
	\l4=\x4-\eps;
	\r4=\x4+\eps;
	\l5=\x5-\eps;
	\r5=\x5+\eps;
	\l6=\x6-\eps;
	\r6=\x6+\eps;
}

\draw (0,\x1) -- (\x1,0) node[midway, above] {$\delta$};
\draw (0,\x2) -- (\x2,0) node[midway, above] {$\delta$};
\draw (0,\x3) -- (\x3,0) node[midway, above] {$\delta$};
\draw (0,\x5) -- (\x5,0) node[midway, above] {$\delta$};
\draw (0,\x6) -- (\x6,0) node[midway, above] {$\delta$};

\addNodeV{1}{1}
\addNodeV{2}{2}
\addNodeV{3}{3}
\filldraw[black] (0,\x4) circle (0pt) node[anchor=east] {$\ldots$};
\addNodeV{5}{n^2-1}
\addNodeV{6}{n^2}

\addNodeH{1}{1}
\addNodeH{2}{2}
\addNodeH{3}{3}
\filldraw[black] (\x4,0) circle (0pt) node[anchor=north] {$\ldots$};
\addNodeH{5}{n^2-1}
\addNodeH{6}{n^2}
\end{tikzpicture} 
\end{center}
\caption{\textbf{Vertices and segments in $\alldiag$.}}
This is an example of constructed points any $1 \le i,j \le k$.
Points from $H$ and $V$ are marked in black,
their guards are marked in \textcolor{guards_color}{blue}.
You can also see segments from $\alldiag$ with their weights (equal to $\delta$).
\label{fig:diag_def}
\end{figure}


Every segment in $\alldiag$
connects points $(i(n^2+1) + t, j(n^2+1))$
and $(i(n^2+1), j(n^2+1) + t)$
for some $1 \le i,j \le k, 1 \le t \le n^2$.
The line on which it lies can be described
by linear equation ${x+y=t+(i+j)(n^2+1)}$,
thus these segments are in fact right-diagonal.

The constructed segment set is defined as:

$$\sets := \allhor \cup \allver \cup \alldiag.$$

The weight of each segment in $\allhor \cup \allver$
is equal to its length,
while every segment in $\alldiag$ has weight
$\delta$.


\definecolor{beg_color}{RGB}{255, 40, 40}
\definecolor{seg_color1}{RGB}{40, 40, 255}
\definecolor{seg_color2}{RGB}{40, 150, 40}

\newcommand{\addNode}[4]{
	\node[guard_h, label={left:$h_{#1,j,t_{#2,#3}}^L$}] at (\l#4,0) {};
	\node[node_h, label={left:$h_{#1,j,t_{#2,#3}}$}] at (\x#4,0) {};
	\node[guard_h, label={left:$h_{#1,j,t_{#2,#3}}^R$}] at (\r#4,0) {};
}

\begin{figure}
\hspace*{-1.5cm}
\begin{tikzpicture}[main/.style = {draw, circle}]
\tikzmath{
\step=2;
\eps=0.5;
%genereted by gen_math.py
\x1=\step;
\x2=\x1+\step;
\x3=\x2+\step;
\x4=\x3+\step;
\x5=\x4+\step;
\x6=\x5+\step;
\x7=\x6+\step;
\x8=\x7+\step;
\x9=\x8+\step;
\l1=\x1-\eps;
\r1=\x1+\eps;
\l2=\x2-\eps;
\r2=\x2+\eps;
\l3=\x3-\eps;
\r3=\x3+\eps;
\l4=\x4-\eps;
\r4=\x4+\eps;
\l5=\x5-\eps;
\r5=\x5+\eps;
\l6=\x6-\eps;
\r6=\x6+\eps;
\l7=\x7-\eps;
\r7=\x7+\eps;
\l8=\x8-\eps;
\r8=\x8+\eps;
\l9=\x9-\eps;
\r9=\x9+\eps;
}

\addNode{1}{1}{1}{1}
\addNode{1}{1}{2}{2}
\addNode{1}{2}{1}{3}
\addNode{1}{2}{2}{4}
\addNode{2}{1}{1}{6}
\addNode{2}{1}{2}{7}
\addNode{2}{2}{1}{8}
\addNode{2}{2}{2}{9}

\draw [beg_color] (\l1,0) to[out=150,in=30, looseness=200] (1.49,0);
\draw [beg_color] (\l1,0) to[out=40,in=140] (\l2,0);
\draw [beg_color] (\l1,0) to[out=40,in=140] (\l3,0);
\draw [beg_color] (\l1,0) to[out=40,in=140] (\l4,0);


\draw [seg_color1] (\r1,0) to[out=20,in=160] (\l6,0);
\draw [seg_color1] (\r1,0) to[out=20,in=160] (\l8,0);
\draw [seg_color2] (\r2,0) to[out=20,in=160] (\l7,0);
\draw [seg_color2] (\r2,0) to[out=20,in=160] (\l9,0);
\draw [seg_color1] (\r3,0) to[out=20,in=160] (\l6,0);
\draw [seg_color1] (\r3,0) to[out=20,in=160] (\l8,0);
\draw [seg_color2] (\r4,0) to[out=20,in=160] (\l7,0);
\draw [seg_color2] (\r4,0) to[out=20,in=160] (\l9,0);

\end{tikzpicture} 
\caption{\textbf{Vertices and segments in $HOR$.}}
This is an example for $n=2$ and any $1 \le j \le k$.
Points from $H$ are marked in black, their guards are marked in \textcolor{guards_color}{blue}.
$t_{i,j}$ is a notation that we use for $\order^{-1}(i,j)$.
Segments are represented as arcs between endpoints.
You can see $\horbeg{j}{t}$ segments in \textcolor{beg_color}{red}.
$\horbeg{j}{1}$ is degenerated to a single point at $h_{1,1,t_{1,1}}^L$.
Segments $\hor{i}{j}{t_{x_1,y}}{t_{x_2,y}}$
are marked in \textcolor{seg_color1}{blue} and \textcolor{seg_color2}{green}.
\textcolor{seg_color1}{Blue} segments connect $t_{x_1,y}$ and $t_{x_2,y}$
such that they share y-coordinate equal to 1,
for~\textcolor{seg_color2}{green} segments it is equal to 2.

\label{fig:segments_def}
\end{figure}


\begin{equation}
w(s) =
	\begin{cases*}
	  length(s) 			& if $s \in \allhor \cup \allver$ \\
	  \delta        & if $s \in \alldiag$
	\end{cases*}
\end{equation}

\section{Proof that the reduction is correct}

Now, we prove that the constructed instance of $\WeightedSegmentSetCover$
indeed gives a correct and sound reduction
of the $\GridTiling$ problem. Lemma \ref{set_cover_solution_exists}
proves that if a solution to the instance of the $\GridTiling$ instance exists,
then there exists a solution with suitably bounded size and weight
of the constructed instance of
\textsc{Weighted} \textsc{Segment} \textsc{Set} \textsc{Cover}.
Then Lemma \ref{grid_tiling_exists} proves that if
there is a solution to the
\textsc{Weighted} \textsc{Segment} \textsc{Set} \textsc{Cover}
instance with bounded weight,
then there exists a solution to the original $\GridTiling$ instance.

\begin{lemma}
\label{set_cover_solution_exists}
	If there exists a~solution to the $\GridTiling$ instance $(n,k,f)$,
	then there exists a~solution to the instance $\instanceSetCover$
	of $\WeightedSegmentSetCover$ with weight $\solWeight$.
\end{lemma}

\begin{proof}
Suppose there exists a solution $x,y$ of the instance $(n,k,f)$
of the $\GridTiling$ problem.
	
We define the proposed solution $\sol \subseteq \sets$ of the instance
of \textsc{Weighted} \textsc{Segment} \textsc{Set} \textsc{Cover}
in three parts: $D \subseteq \alldiag$, $A \subseteq \allhor$ and $B \subseteq \allver$:
\begin{eqnarray*}
	D & := & \{(v_{i, j, t}, h_{i, j, t}) : 1 \le i, j \le k, t = \order^{-1}(x(i), y(j))\}, \\
	A & := & \{\horbeg{i}{\order^{-1}(x(1), y(i))} : 1 \le i \le k\} \\
	& \cup & \{\horend{i}{\order^{-1}(x(k), y(i))} : 1 \le i \le k\} \\
	& \cup & \{\hor{i}{j}{\order^{-1}(x(i), y(j))}{\order^{-1}(x(i+1), y(j))} : 1 \le i < k, 1 \le j \le k\}, \\
	B & := & \{\verbeg{i}{\order^{-1}(x(i), y(1))} : 1 \le i \le k\} \\
	& \cup & \{\verend{i}{\order^{-1}(x(i), y(k))} : 1 \le i \le k\} \\
	& \cup & \{\ver{i}{j}{\order^{-1}(x(i), y(j))}{\order^{-1}(x(i), y(j+1))} : 1 \le i \le k, 1 \le j < k\},
\end{eqnarray*}
	$$\sol := D \cup A \cup B.$$

Since $\points = H \cup V$, we show that $\sol$ covers the whole set $H$;
the proof for $V$ is analogous.

Fix any $1 \le j \le k$ and define $t_{i} := \order^{-1}(x(i), y(j))$.
The two leftmost segments in~$A$ for this $j$ are
$\horbeg{j}{t_1} = (h_{1,j,1}^L, h_{1, j, t_1}^L)$ and
$\hor{1}{j}{t_1}{t_2} = (h_{1,j, t_1}^R, h_{2,j,t_2}^L)$.
Therefore, points $h_{1,j,x}, h_{1,j,x}^L$ and $h_{1,j,x}^R$
for all $1 \le x \le n^2$ ale covered by $\horbeg{j}{t_1}$ and $\hor{1}{j}{t_1}{t_2}$,
excluding point $h_{1,j,t_1}$.

Analogously for $2 \le i \le k-1$,
the two consecutive segments $\hor{i-1}{j}{t_{i-1}}{t_i}$
and $\hor{i}{j}{t_i}{t_{i+1}}$ cover points $h_{i,j,x}, h_{i,j,x}^L$ and $h_{i,j,x}^R$
for all $1 \le x \le n^2$,
excluding point $h_{i,j,t_i}$.

Finally $\hor{k-1}{j}{t_{k-1}}{t_k}$ and $\horend{j}{t_k}$
cover all points $h_{k,j,x}, h_{k,j,x}^L$ and $h_{k,j,x}^R$
for ${1 \le x \le n^2}$, excluding point $h_{k,j,t_k}$.

$D$ covers all points $h_{i,j,t_i}$ and $v_{i,j,t_i}$.
As $j$ was chosen arbitrarily, all points in $H$ are covered.

The size of this proposed solution is:
$$|\sol| = |D| + |A| + |B| = k^2 + (k+1)k + (k+1)k = 3k^2+2k.$$

Then, we need to compute the total weight of the solution $\sol$.
First, we compute the sum of weights of segments in $A$.
Fix $1 \le j \le k$ and consider segments collinear
with the $j$-th horizontal line.
All points $h_{i,j,t}$, $h_{i,j,t}^L$ and $h_{i,j,t}^R$
for every $1 \le i \le k$ and $1 \le t \le n^2$ are covered by $A$
excluding points $h_{i,j,\order^{-1}(x(i),y(j))}$.
Every such point leaves a gap of length $2\epsilon$ between
$h_{i,j,\order^{-1}(x(i),y(j))}^L$ and $h_{i,j,\order^{-1}(x(i),y(j))}^R$.
Therefore, the total weight of segments in $A$
that lie on the line in question equals the length of the segment
$(h_{i,1,1}^L, h_{i,k,n^2}^R)$
minus $2\epsilon k$, which is $k(n^2+1) -2(1-\epsilon)-2k\epsilon$.
We need to multiply that by $k$, as we consider all possible values of $j$.

Computation for vertical segments is analogous and yields the same result.
Every segment in~$D$ has weight $\delta$, therefore the sum of all weights
is equal to:

$$2k(k(n^2+1) -2(1-\epsilon)-2k\epsilon) + k^2\delta= \solWeight.\qedhere$$
\end{proof}

Now we present a few additional properties of the constructed instance
of the \textsc{Weighted} \textsc{Segment} \textsc{Set} \textsc{Cover}
that help us to prove Lemma \ref{grid_tiling_exists}.

\begin{claim}
\label{guards}
In any solution to the instance $\instanceSetCover$:
\begin{itemize}
\item the left and right guards of points in $H$
(points in $\{p^L : p \in H\} \cup \{p^R : p \in H\}$)
have to~be~covered with segments from $\allhor$,
\item the lower and upper guards of points in $V$
(points in $\{p^D : p \in V\} \cup \{p^U : p \in V\}$)
have to~be~covered with segments from $\allver$.
\end{itemize}
\end{claim}

\begin{proof}
We prove the claim for the points from $H$
as the proof for points from $V$ is analogous.

Every segment in $\allver$ is vertical and 
has x-coordinate equal to $i(n^2+1)$ for some $1\le i \le k$,
so they all have different x-coordinate
than any left or right guard of points in $H$.

For every point $x$ which is a left or right guard of a point in $H$,
there are $kn^2$ segments from $\alldiag$ that intersect with the horizontal
line that goes through $x$. All of these segments intersect with
this line in points from set $H$, therefore none of them
covers any of the guards.

Therefore none of the segments from $\allver$ or $\alldiag$ covers any
of the guards of the points in $H$.
\end{proof}

\begin{claim}
\label{one_diag_in_square}
For any $1 \le i, j \le n$
and any solution to the instance $\instanceSetCover$,
all but at most one point $h_{i, j, t}$
and at most one point $v_{i, j, t}$
for $1 \le t \le n^2$
must be
covered with segments from $\allhor$ or $\allver$.
\end{claim}

\begin{proof}
We prove the claim for horizontal segments,
as the proof for vertical segments is analogous.

We prove this by contradiction. Assume that we
have two points $h_{i,j,t_1}, h_{i,j,t_2},1 \le t_1 <  t_2 \le n^2$,
such that they are not covered with segments from $\allhor$.

Point $h^R_{i, j, t_1}$ has to be covered with a segment from $\allhor$
by Claim $\ref{guards}$.
Every segment in $\allhor$ covering $h^R_{i, j, t_1}$
but not $h_{i,j,t_1}$ must start at $h^R_{i, j, t_1}$
and all such segments cover also $h_{i, j, t_2}$.
This contradicts the assumption, which concludes the proof.
\end{proof}

\begin{lemma}
\label{vertical_horizontal_sum}
For every solution $\sol$ to the instance $\instanceSetCover$,
the sum of weights of segments chosen
from sets $\allhor$ and $\allver$ is at least
$\hvWeight$.
\end{lemma}

\begin{proof}
Let us fix $1 \le i \le k$.

We provide a lower bound for the sum of lengths
of vertical segments from $\sol \cap \allver$.
This bound is the same for each $i$ and is the same
for horizontal lines, thus we need to multiply such a bound by $2k$.

\begin{enumerate}[label={(\arabic*)}]
\item The total length between $v^D_{i, 1, 1}$ and $v^U_{i, k, n^2}$ is:
$$(k(n^2+1) + n^2 +\epsilon) - ((n^2+1)+1 -\epsilon) = k(n^2+1) - 2(1 - \epsilon).$$

\item For every $1 \le j \le k$ there exists at most one $1 \le t \le n^2$
such that $v_{i,j,t}$ is not covered by segments from $\allver$
(Claim \ref{one_diag_in_square}).
Its guards (see Definition \ref{guard_def}) $v^U_{i,j,t}$ and $v^D_{i,j,t}$
have to be covered in $\allver$ (Claim \ref{guards}).
Therefore, at most $k$ spaces of length $2\epsilon$ can be left
not covered by segments from $\allver$ between $v_{i,1,1}^D$ and $v_{i,k,n^2}^U$.

\end{enumerate}
The sum of these lower bounds for vertical and horizontal lines is:
$$2k(k(n^2+1) -2k\epsilon -2(1-\epsilon)) = 2k^2(n^2+1) -4k^2\epsilon -4k(1-\epsilon) = \hvWeight.\qedhere$$
\end{proof}

\begin{lemma}
\label{diag_correct}
Let $\sol$ be a solution to a constructed instance $\instanceSetCover$
with weight at most  $\solWeight$.
Then for every $1 \le i,j \le k$
there exists $1 \le t \le n^2$ such that:
\begin{enumerate}[label={(\arabic*)}]
\item $v_{i,j,t}, h_{i,j,t}$ are not covered by segments from $\allver$ or $\allhor$;
\item segment $(v_{i,j,t}, h_{i,j,t})$ is in solution $\sol$;
\item $\order(t) \in f(i,j)$, that is, $\order(t)$ is an allowed tile for $(i,j)$;
\item for every $1 \le s\le n^2$, $s \neq t$, $v_{i,j,s}$ is covered in $\allver$;
\item for every $1 \le s\le n^2$, $s \neq t$, $h_{i,j,s}$ is covered in $\allhor$.
\end{enumerate}
\end{lemma}

\begin{proof}
At most one of the points $\{h_{i,j,t_x} : 1 \le t_x \le n^2\}$
and one of the points $\{v_{i,j,t_y} : 1 \le t_y \le n^2\}$
is covered with $\alldiag$
(Claim \ref{one_diag_in_square}).
	
Moreover, exactly one such point $h_{i,j,t_x}$ and one such point $v_{i,j,t_y}$
is covered with $\alldiag$,
because if none of them were covered, then the solution would have to
have weight at least $\hvWeight + 2\epsilon$ (see the proof of Lemma \ref{vertical_horizontal_sum}),
which is more than $\solWeight$.

We observe that points $h_{i,j,t_x}$ and $v_{i,j,t_y}$
have to be covered with the same segment from $\alldiag$.
Indeed we need to use at least $k^2$ of them to use
exactly one DIAG segment for every pair of $1 \le i,j \le k$,
if we used 2 segments from $\alldiag$
for one pair $(i,j)$,
then we would have used total weight at least
$\hvWeight + k^2\delta + \delta$ (Lemma \ref{vertical_horizontal_sum}),
which is more than $\solWeight$.
Since points $h_{i,j,t_x}$ and $v_{i,j,t_y}$ are covered by
a single segment from $\alldiag$, we have $t_x = t_y$.

Therefore $t_x = t_y$
and $\order(t_x)$ is an allowed tile for $(i,j)$
because the corresponding segment is in $\alldiag$.
\end{proof}

\newcommand{\diagonal}{\mathsf{diagonal}}
We refer to the function mapping from a pair $(i,j)$, where $1\le i,j \le k$
to a number $t_x$ from Lemma \ref{diag_correct}
as $\diagonal : \{1, \ldots, k\} \times \{1, \ldots, k\} \rightarrow \{1, \ldots, n^2\}$.

\begin{lemma}
\label{vertical_horizontal_synchronized}
Let $\sol$ be any solution
of a constructed instance $\instanceSetCover$
with weight at most $\solWeight$. Then:
\begin{enumerate}
\item 
for any $1 \le i < k, 1 \le j \le k$,
$\matchh(\diagonal(i, j),\diagonal(i+1, j))$ is $\true$;
\item 
for any $1 \le i \le k, 1 \le j < k$,
$\matchv(\diagonal(i, j),\diagonal(i, j+1))$ is $\true$.
\end{enumerate}
\end{lemma}

\begin{proof}
We prove (1) by contradiction, the proof of (2) is analogous.

Let us take any $1 \le i < k, 1 \le j \le k$
and name $t_1 = \diagonal(i, j)$ and $t_2 = \diagonal(i+1, j)$.
We also assume that $\matchh(t_1,t_2)$ is \false,
which is equivalent to the fact that
segment $(h_{i,j,t_1}^R, h_{i+1,j,t_2}^L)$
is not in set $\allhor$.

Therefore $h_{i,j,t_1}$ and $h_{i+1,j,t_2}$
are not covered by segments from $\allhor$ (Lemma \ref{diag_correct}),
while $h^R_{i,j,t_1}$ and $h^L_{i+1,j,t_2}$
have to be covered by segments from $\allhor$ (Claim \ref{guards}).


Every segment from $\allhor$ either:
\begin{itemize}
\item starts at point $h^R_{x,y,z_1}$
and ends at point $h^L_{x+1,y,z_2}$ for some
$1 \le x < k$,$1 \le y \le k$ and $1 \le z_1, z_2 \le n^2$; or
\item is $\horbeg{y}{z}$ 
and starts at $h^L_{1,y,1}$ and ends at $h^L_{1,y,z}$ for some $1 \le y \le k$ and $1 \le z \le n^2$; or
\item is $\horend{y}{z}$
and starts at $h^R_{k,y,z}$ and ends at $h^R_{k,y,n^2}$ for some $1 \le y \le k$ and $1 \le z \le n^2$.
\end{itemize}
All of the points between $h^R_{i,j,t_1}$ and $h^L_{i+1,j,t_2}$
are covered by segments in $\allhor$ 
and there is no segment $(h^R_{i,j,t_1}, h^L_{i+1,j,t_2})$ in $\allhor$.
Hence, there are at least two different segments covering them.
If both of these segments are neither $\horbeg{y}{z}$ nor $\horend{y}{z}$,
then one of them must begin
at $h^R_{i,j,t_1}$ and end at $h^L_{i+1,j,z_2}$
and there must be other one that begins at $h^R_{i,j,z_1}$
and ends at $h^L_{i+1,j,t_2}$
for some $1 \le z_1, z_2 \le n^2$.

Thus, the space between $h^R_{i,j,z_1}$ and $h^L_{i,j+1,z_2}$
would be covered twice and is longer than $\epsilon$.
The case when one of them is $\horbeg{y}{z}$ or $\horend{y}{z}$ is analogous.
Note that they cannot be both $\horbeg{y}{z}$ or $\horend{y}{z}$.

By the proof of Lemma \ref{vertical_horizontal_sum},
the lower bound for weight of such a solution is $\hvWeight + \epsilon$
which is more than $\solWeight$.

Therefore ${h^R_{i,j,t_1}}$ and ${h^L_{i+1,j,t_2}}$ must be covered
by one segment from $\allhor$, namely \linebreak ${(h^R_{i,j,t_1}, h^L_{i+1,j,t_2})}$.
Hence $(h^R_{i,j,t_1}, h^L_{i+1,j,t_2})$ is a segment in $\allhor$
and $\matchh(t_1,t_2)$ is $\true$.
\end{proof}


\begin{lemma}
\label{grid_tiling_exists}
	If there exists a solution to instance $\instanceSetCover$
	with weight at most $\solWeight$,
	then there exists a solution to the $\GridTiling$ instance $(n,k,f)$.
\end{lemma}

\begin{proof}
Take $\diagonal$ function from Lemma \ref{diag_correct}.

To define the $x$ function 
for every $1 \le i \le k$ set $x(i) := x_i$
where $(x_i, a) = \order(v_{i,1})$.
Similarly, to define the $y$ function,
for every $1 \le i \le k$ set $y(i) := y_i$
where $(b, y_i) = \order(h_{1,i})$

To prove that this is a correct solution to $\GridTiling$,
we need to prove that 
for every $1 \le i,j \le k$, $(x(i), y(j))$ is in
the allowed tiles set $f(i,j)$.

Let us take any $1 \le i,j \le k$.
By Lemma \ref{vertical_horizontal_synchronized}
and simple induction,
we know that $\matchh(\diagonal(1, j),\diagonal(i, j))$ and
$\matchv(\diagonal(i, 1),\diagonal(i, j))$ are $\true$.
Therefore $\order(\diagonal(i,j)) = (x(i), y(j))$.
By Lemma \ref{diag_correct} we know that 
$\order(\diagonal(i,j))$ is in $f(i,j)$.
Therefore 
$(x(i), y(j))$
is in $f(i,j)$.
\end{proof}



\bibliographystyle{apalike}
\bibliography{bibl}

\end{document}


%%% Local Variables:
%%% mode: latex
%%% TeX-master: t
%%% coding: latin-2
%%% End:
