%
% Niniejszy plik stanowi przykład formatowania pracy magisterskiej na
% Wydziale MIM UW.  Szkielet użytych poleceń można wykorzystywać do
% woli, np. formatujac wlasna prace.
%
% Zawartosc merytoryczna stanowi oryginalnosiagniecie
% naukowosciowe Marcina Wolinskiego.  Wszelkie prawa zastrzeżone.
%
% Copyright (c) 2001 by Marcin Woliński <M.Wolinski@gust.org.pl>
% Poprawki spowodowane zmianami przepisów - Marcin Szczuka, 1.10.2004
% Poprawki spowodowane zmianami przepisow i ujednolicenie 
% - Seweryn Karłowicz, 05.05.2006
% Dodanie wielu autorów i tłumaczenia na angielski - Kuba Pochrybniak, 29.11.2016

% dodaj opcję [licencjacka] dla pracy licencjackiej
% dodaj opcję [en] dla wersji angielskiej (mogą być obie: [licencjacka,en])
\documentclass[en]{pracamgr}

% Dane magistranta:
\autor{Katarzyna Kowalska}{371053}

% Dane magistrantów:
%\autor{Autor Zerowy}{342007}
%\autori{Autor Pierwszy}{342013}
%\autorii{Drugi Autor-Z-Rzędu}{231023}
%\autoriii{Trzeci z Autorów}{777321}
%\autoriv{Autor nr Cztery}{432145}
%\autorv{Autor nr Pięć}{342011}

\title{Approximation and Parametrized Algorithms for Segment Set Cover}
\titlepl{Algorytmy parametryzowania i
trudność aproksymacji problemu pokrywania zbiorów
odcinkami na płaszczyźnie}

%\tytulang{An implementation of a difference blabalizer based on the theory of $\sigma$ -- $\rho$ phetors}

%kierunek: 
% - matematyka, informacyka, ...
% - Mathematics, Computer Science, ...
\kierunek{Computer Science}

% informatyka - nie okreslamy zakresu (opcja zakomentowana)
% matematyka - zakres moze pozostac nieokreslony,
% a jesli ma byc okreslony dla pracy mgr,
% to przyjmuje jedna z wartosci:
% {metod matematycznych w finansach}
% {metod matematycznych w ubezpieczeniach}
% {matematyki stosowanej}
% {nauczania matematyki}
% Dla pracy licencjackiej mamy natomiast
% mozliwosc wpisania takiej wartosci zakresu:
% {Jednoczesnych Studiow Ekonomiczno--Matematycznych}

% \zakres{Tu wpisac, jesli trzeba, jedna z opcji podanych wyzej}

% Praca wykonana pod kierunkiem:
% (podać tytuł/stopień imię i nazwisko opiekuna
% Instytut
% ew. Wydział ew. Uczelnia (jeżeli nie MIM UW))
\opiekun{dr Michał Pilipczuk\\
  Instytut Informatyki\\
  }

% miesiąc i~rok:
\date{June 2020}

%Podać dziedzinę wg klasyfikacji Socrates-Erasmus:
\dziedzina{ 
%11.0 Matematyka, Informatyka:\\ 
%11.1 Matematyka\\ 
%11.2 Statystyka\\ 
11.3 Informatyka\\ 
%11.4 Sztuczna inteligencja\\ 
%11.5 Nauki aktuarialne\\
%11.9 Inne nauki matematyczne i informatyczne
}

%Klasyfikacja tematyczna wedlug AMS (matematyka) lub ACM (informatyka)
\klasyfikacja{D. Software\\
  D.127. Blabalgorithms\\
  D.127.6. Numerical blabalysis}

% Słowa kluczowe:
\keywords{set cover, geometric set cover, FPT, W[1]-completeness,
APX-completeness, PCP theorem, NP-completeness}

% Tu jest dobre miejsce na Twoje własne makra i~środowiska:

\newcommand{\points}{\mathcal{C}}
\newcommand{\sets}{\mathcal{P}}
\newcommand{\sol}{\mathcal{R}}
\newcommand{\then}{\Rightarrow}

\usepackage{amsfonts}
\usepackage{amsmath}
\usepackage{graphicx}
\usepackage{xcolor}
\usepackage[nospace, noadjust]{cite}
\usepackage{lineno}
\usepackage{enumitem}
\usepackage{amsthm}
\usepackage{mathtools}  
\usepackage{makecell}
\usepackage{tikz}
\usetikzlibrary{calc,math}
\linenumbers

\mathtoolsset{showonlyrefs}  

\theoremstyle{plain}
\newtheorem{claim}{Claim}[chapter]
%\newtheorem{defi}{Definition}[section]
\newtheorem{tw}{Theorem}[chapter]
\newtheorem{lemma}{Lemma}[chapter]
\newtheorem{corollary}{Corollary}[chapter]
\newtheorem{remark}{Remark}[chapter]

\theoremstyle{definition}
\newtheorem{defi}{Definition}[chapter]

\setcounter{secnumdepth}{3}
\setcounter{tocdepth}{3}


% koniec definicji

\begin{document}
\maketitle

%tu idzie streszczenie na strone poczatkowa
\begin{abstract}
  The work presents a study
  of different geometric set cover problems.
  It mostly focuses on segment set cover
  and its connection to the polygon set cover.
\end{abstract}

\tableofcontents
%\listoffigures
%\listoftables

\chapter{Introduction}

The Set Cover problem is one of the most common NP-complete problems.
[tutaj referencja]
We are given a family of sets and have to choose the smallest
subfamily of these sets that cover all their elements.
This problem naturally extends to settings
were we put different weights on the sets
and look for the subfamily of the minimal weight.
This problem is NP-complete even 
without weights and if we put
restrictions on what the sets can be.
One of such variants is Vertex Cover problem,
where sets have size 2 (they are edges in a graph).

In this work we focus on another such variant where the sets correspond
to some geometric shapes and
only some points of the plane have to be covered.
When these shapes are rectangles with edges parallel
to the axis, the problem can be proven to
be W[1]-complete (solution of size $k$ cannot be found
in $n^o(k)$ time),
APX-complete (for suffciently small $\epsilon > 0$, the problem
does not admit $1+\epsilon$-approximation scheme)
[refrencje].

Some of these settings are very easy.
Set cover with lines parallel to one of the axis
can be solved in polynomial time.

There is a notion of $\delta$-expansions,
which loosen the restrictions on geometric set cover.
We allow the objects to cover the points
after $\delta$-expansion and compare
the result to the original setting.
This way we can produce both FPT and EPTAS
for the rectangle set cover with $\delta$-extensions
[referencje].



\paragraph{Our contribution.}
In this work, we prove that unweighted geometric set cover
with segments is fixed parameter tractable (FPT).

Moreover, we show that geometric set cover with segments
is APX-complete for unweighted axis-parallel segments,
even with 1/2-extensions.
So the problem for very thin rectangles
also cannot admit PTAS.
Therefore, in the efficient polynomial-time approximation scheme (EPTAS)
for \textit{fat polygons} by \cite{harpeled12},
the assumption about polygons being fat is necessary. 

Finally, we show that geometric set cover with weighted segments in
3 directions is W[1]-complete.
However, geometric set cover with weighted segments is FPT if we allow
$\delta$-extension.

This result is especially interesting,
since it's counter-intuitive that
the unweighed setting is FPT and the weighted
setting is W[1]-complete.
Most of such problems (like vertex cover or [wiecej przykladow])
are equally hard in both weighted and unweighted settings.

\chapter{Preliminaries}

In this chapter we present some basic definitions that
will be used later.

\section{Geometric set cover}
\label{section:def:geometric__set_cover}
Whenever speaking about geometric set cover,
we consider it in the 2-dimensional plane.

In the geometric set cover problem we are are given
$\sets$ --- a set of objects, which are connected
subsets of the plane and $\points$ --- a set of points in the plane.
The task is to choose $\sol \subseteq \sets$ such that
every point in $\points$ is inside some object from $\sol$
and $|\sol|$ is minimized. We will mostly consider the case where
$\sets$ consists of segments in the plane.

In the weighted setting, there is some given weight function
$f : \sets \rightarrow \mathbb{R^+}$
and we would like to find a solution $\sol$
that minimizes $\sum_{R \in \sol} f(R)$.

\begin{defi}
Segment is \textbf{axis-parallel} if it lies on line that is
either horizontal $x = c$ or vertical $y = c$.
\end{defi}

\begin{defi}
	A line is \textbf{right-diagonal} if it is
	described by linear function $x + y = d$ for some $d \in \mathbb{R}$.
	Segment is \textbf{right-diagonal} if its
	direction is a right-diagonal line.
\end{defi}

\section{Parameterization}

In the parameterized setting of the Geometric Set Cover
for a given $k$,
our task is to either find a solution $\sol$ such that $|\sol| \le k$
or decide that there is no such solution.

\begin{defi}
A \textbf{Fixed-parameter Tractable (FPT)} algorithm 
for a problem with parameter $k$ and instance size $n$
is an algorithm running in time $f(k) \cdot n^c$
for some constant $c$ and some computable function $f$.
\end{defi}

\begin{defi}
\label{definition:cnf}
Boolean formula is in \textbf{conjunctive normal form (CNF)} if
it is a conjunction of one or more formulas,
which are disjunction of literals.
\textbf{$k$-CNF} formula is a CNF formula, where
every disjunction consists of at most $k$ literals.
\end{defi}

\begin{defi}
\textbf{$k$-SAT} problem is 
a boolean satisfiability problem of $k$-CNF formulas.
Given $k$-CNF formula, one must answer if there
exists any variables assignment that satisfies the formula.
\end{defi}

\begin{defi}
For $k \ge 3$ set us define $S_k$ as a set of constants $\sigma$
such that there exists an algorithm solving $k$-SAT running in time
$\mathcal{O}^{*}(2^{\sigma n})$.
Set us define $s_k$ as the infimum  of the set $S_k$.

\textbf{Exponential Time Hypothesis (ETH)} is a conjecture
that $s_3 > 0$. This conjecture implies that
there does not exist an algorithm solving 3-SAT
running in time $2^{o(n)}$.
\end{defi}

We provide the main theorem that we use in this thesis for W[1]-hard
problems. To see the definition of a W[1]-hard problem,
see Chapter 13.3 of \cite{platypus_book}.

\begin{tw}
Problem parameterized by $k$ is \textbf{W[1]-hard} if assuming ETH there
does no algorithm solving this problem running in time
$f(k)\cdot n^{o(k)}$.
\end{tw}

\section{Approximation}

Let us recall some definitions related to optimization problems.

\begin{defi}
A \textbf{polynomial-time approximation scheme (PTAS)}
for a minimization problem $\Pi$
is a family of algorithms $\cal{A}_\epsilon$ for
every $\epsilon > 0$
such that $\cal{A}_\epsilon$ takes an instance $I$ of~$\Pi$
and in polynomial time
finds a solution that is within a factor
of ($1+\epsilon$) of being optimal.
This means that the reported solution has weight at most
$(1+\epsilon)opt(I)$, where $opt(I)$ is the weight
of an optimal solution to $I$.
\end{defi}

\begin{defi}
A problem $\Pi$ is \textbf{APX-hard} if assuming P $\neq$ NP,
there exists $\epsilon > 0$
such that there is no polynomial-time $(1+\epsilon)$-approximation algorithm
for $\Pi$.
\end{defi}

\section{$\delta$-extension}
\label{section:def:delta_extension}

Another idea presented here, which can be utilized only when considering
the problems with geometric objects,
is $\delta$-extension.
We define it specifically for the geometric set cover problem
with convex centre-symmetric objects.

Intuitively, we consider a problem with slightly larger objects,
which makes the instance more permissive.
However, we aim to find a solution that
is not larger than the
optimum solution to the original problem,
so this is substantially easier than just
solving the problem for the larger objects.
It may even be the case
that we are able to find a solution
of size smaller than the optimum solution
to the original problem.

Formal definition of $\delta$-extended objects.
is present in Definition
\ref{definition:delta_extension}.

The geometric set cover problem with $\delta$-extension
is a version of geometric set cover with
the following modifications.
\begin{itemize}
\item We need to cover all the points in $\points$
by selecting objects from $\{P^{+\delta} : P \in \sets\}$ (which always 
include no fewer points than the objects
before $\delta$-extension).
\item We look for a solution that is not larger than the optimum
solution to the original problem.
Note that it does not need to be an optimal solution in
the modified problem.
\end{itemize}
Formally, we have the following.

\begin{defi}
The \textbf{geometric set cover problem
with $\delta$-extension} is the problem where for an input instance
$I=(\sets, \points)$ of geometric set cover,
the task is to output a solution $\mathcal{R} \subseteq \sets$
such that the~$\delta$-extended set
$\{ R^{+\delta} :  R \in \mathcal{R} \}$ covers $\points$
and is not larger than the optimal solution to the~problem without
extension, i.e.~$|\mathcal{R}| \le |opt(I)|$.
\end{defi}

At last, we formulate a definition of the
polynomial-time approximation scheme (PTAS)
for a problem with $\delta$-extension.

\begin{defi}
A \textbf{PTAS for geometric set cover 
with $\delta$-extension} is a family of algorithms
$\{\mathcal{A}_{\delta, \epsilon}\}_{\delta, \epsilon > 0}$ that
each takes as an input instance $I=(\sets, \points)$
of geometric set cover where objects are centre-symmetric and strongly convex,
and in polynomial-time outputs a solution $\mathcal{R} \subseteq \sets$
such that the $\delta$-extended set
$\{ R^{+\delta} :  R \in \mathcal{R} \}$ covers $\points$
and is within a $(1+\epsilon)$ factor of the optimal
solution to this problem without
extension, i.e.~$(1+\epsilon)|\mathcal{R}| \le |opt(I)|$.
\end{defi}

\section{Weighted Geometric Set Cover}

In this thesis we also consider a weighted Geometric Set Cover problem,
which is a combination
of the weighted and parameterized setting described in 
\ref{section:def:geometric__set_cover}.
We already argued in the introduction
that there is no consensus of how it is defined, but when we discuss the
weighted parameterized setting we will consider the following
definition. There is a given weight function
$f : \sets \rightarrow \mathbb{R^+}$
and we would like to find a solution $\sol$,
such that $|\sol| \le k$
that minimizes $\sum_{R \in \sol} f(R)$ among such sets $\sol$.

\begin{defi}
The \textbf{weighted geometric set cover problem
with $\delta$-extension} is the problem where for an input instance
$I=(\sets, \points, f)$ of weighted geometric set cover,
the task is to output a solution $\mathcal{R} \subseteq \sets$
such that the~$\delta$-extended set
$\{ R^{+\delta} :  R \in \mathcal{R} \}$ covers $\points$
and it has weight not larger than the optimal solution to the~problem without
extension, i.e.~$\sum_{R \in \mathcal{R}} f(R) \le |opt(I)|$.
\end{defi}

We also consider weighted parameterized setting with $\delta$-extension,
which we formally define below.

\begin{defi}
The \textbf{weighted geometric set cover problem
with $\delta$-extension parameterized by the size of a solution}
is a problem where for an input instance
${I=(\sets, \points, f, k)}$ of weighted geometric set cover
parameterized by the size of a solution $k$,
the task is to output a solution $\mathcal{R} \subseteq \sets$
such that the~$\delta$-extended set
$\{ R^{+\delta} :  R \in \mathcal{R} \}$ covers $\points$,
uses no more than $k$ sets, i.e. $|\sol| \le k$
and it has weight not larger than the optimal solution to the~problem without
extension, i.e.~$\sum_{R \in \mathcal{R}} f(R) \le |opt(I)|$.
\end{defi}

\chapter{APX-hardness geometric set cover problem}
\section{APX-completeness for segments parallel to axis}
\label{section:segment_apx}

\subsection{Definition of  MAX-(3,3)-SAT problem}
Here we define MAXSAT problem.

\begin{tw}{
	\label{hastadtheorem}
	\textbf{\cite{hastad}}
	Assume NP $\not\subseteq$ $DTIME(2^{O(\log n \log \log n)})$.
	Then, there exists such constant $c > 0$, such for
	$$\epsilon' = \frac{c \log \log \log n}{\log \log n}$$ 
	satifiable 3-SAT formulas cannot be distinguished from
	3-SAT formulas where only $7/8+\epsilon'$ of~the clauses
	can be satisfied in polynomial time.
}\end{tw}

\begin{tw}{
	\label{apxconstruction}
	Given an instance of  MAX-(3,3)-SAT 
	with $n$ variables and optimal result $k$,
	we can construct an instance of axis-parallel segments in 2D,
	which optimal result (even with 1/2-extension) is exactly $17n - k$.
}\end{tw}

\begin{tw}{
	\textbf{(axis-parallel segment set cover with 1/2-extension is APX-hard)}.	
	For every $\epsilon > 0$,
	there doesn't exist an $(1-\epsilon)$-approximation scheme
	for unweighted geometric set cover
	with axis-parallel segments in 2D (even with 1/2-extension)
	(problem is APX-hard).
}\end{tw}

\paragraph{Proof.}
Take any $\epsilon > 0$.
Take such $n$, that $\epsilon'$ from \ref{hastadtheorem}
is not greater than $max(\epsilon, 1/2)$.

Let's assume that there exists an $(1-\epsilon)$-approximation scheme
for unweighted geometric set cover with axis-pararell segments in 2D.
We will construct an algorithm distinguishing instances of MAX-(3,3)-SAT
in \ref{hastadtheorem}.
Take two instances to be distinguished and using \ref{apxconstruction}
let's construct two instances of geometric set cover,
name the one constructed from satisfiable 3-SAT $I_1$
and the unsatisfiable 3-SAT as $I_0$.

Use $(1-\epsilon)$-approximation scheme for instances of geometric
set cover, let's name the result of this approximation
for an instance of problem $I$ as $approx(I)$.

$$approx(I_1) \ge (1-\epsilon)17n > 16\frac{1}{8}n - \epsilon' \ge approx(I_2)$$ 

So we can distinguish these instances, since the satifiable instance
will always yield greater result in approximation scheme.

Therefore such approximation scheme cannot exist.

\subsection{Reduction construction}

Let's take some instance of  MAX-(3,3)-SAT with
variables $x_1, x_2 \ldots x_n$
and clauses $C_1, C_2 \dots C_n$.

We will create gadgets for choosing the value
of variables (\textit{true} or \textit{false}) and checking
if the clauses are met (any of the variables were chosen).

\begin{figure}[h]
\includegraphics[width=0.7\textwidth]{segment_apx_sketch.jpg}
\caption{General scheme of reduction.}
\label{fig:segment_apx}
\end{figure}

\subsubsection{Choose $x_i$ gadget}
\begin{figure}[h]
\includegraphics[width=0.6\textwidth]{choose_x_gadget.jpg}
\caption{Scheme of choose $x_i$ gadget.}
\label{fig:choose_x_gadget}
\end{figure}
In Figure~\ref{fig:choose_x_gadget},
we show a gadget that simulates a single variable $x_i$.
It consists of six points A, B, C, D, E, F, and several segments.
Selecting the segment marked with $x_i$
to the solution will correspond to setting $x_i$ to \textit{true},
while selecting the segment marked with $\neg x_i$
to setting $x_i$ to \textit{false}.
In the following lemmas,
we show that this construction indeed models a binary variable.

First, note that in the gadget
there are exactly two sets of three segments
that cover all points $A, B, C, D, E, F$.
These two sets of segments are marked in
Figure~\ref{fig:choose_x_gadget} in blue and green, respectively.

\begin{lemma}
Points $A, B, C, D, E, F$ cannot be covered using less than
3 segments (even with $1/2$-extensions).
\end{lemma}
\paragraph{Proof.}
We need to take at least one segment on line $ABC$,
because it's the only way to cover $C$.
All other points ($D, E, F$) are not colinear,
so we need at least 2 other segments to cover them.

\begin{lemma}
If we choose both segments $x_i$ and $\neg x_i$, we need to use at
least 4 segments to cover all points $A, B, C, D, E, F$
(even with $1/2$-extensions).
\end{lemma}

\paragraph{Proof.}
Choosing both segments $x_i$ and $\neg x_i$
we only cover points $C$
(becuase $B$ is too far away to be covered with $1/2$-extension)
and $F$.

The remaining points ($A, B, D, E$) are not colinear,
so we need at least two more segments to cover them.

\paragraph{Robustness to $1/2$-extension.}
Take a look at Figure~\ref{fig:segment_apx}.
The points will be included in choose gadgets (horizontal boxes)
and clause gadgets (vertical boxes).

Since segment $AC$ is very long
and colinear with $x_i$, after $1/2$-extension
it will cover a significant part of segment $x_i$,
even though $x_i$ will not be chosen.

If we put all the clause gadgets in the area
marked with \textbf{clauses} at gadget scheme in Figure~\ref{fig:choose_x_gadget},
it is enough to prove that $AC$ will not cover any points
in the \textbf{clauses} area even with $1/2$-extensions.

\begin{lemma}
No points in \textbf{clauses} area can be covered
by $AC$ with $1/2$-extension.
\end{lemma}

\paragraph{Proof.}
Bear in mind that length of $AC$ is equal to length of $x_i$.
Area \textbf{clauses} takes a second half
of the segment $x_i$ and $AC$ after extension will cover the first
half of segment $x_i$.

\subsubsection{Clause gadget}
\includegraphics[width=0.6\textwidth]{clause_gadget.jpg}

\begin{lemma}
In order to cover $D$ ($E, F$) point at least one
of the segments $AD$ ($BE, CF$) or $x'$ ($y', z'$).
\end{lemma}

\begin{lemma}
Points $A$ and $D$ can be covered
with one additional segment $x'$
only if $x$ is already chosen.
Otherwise they can be covered with one segment
only by using $AD$.
\end{lemma}

\begin{lemma}
Points $A, B, C, D, E, F, G$ can be covered with 
3 or 4 segments, depending if at least one of the segments
$x, y, z$ was previously chosen.
\end{lemma}

\subsubsection{Or gadget}
\begin{lemma}
Points $A, B, C, D, E, F, G, H, I, J$ can be covered using
at least 4 segments even with $1/2$-extension.
\end{lemma}

\begin{lemma}
Points $A, B, C, D, E, F, G, H, I, J$ can be covered using
4 segments and segment $x \lor y$ can be chosen
even with $1/2$-extension
only if at least one of the segments $x$ or $y$ is chosen.
\end{lemma}

\subsection{Proof that construction is sound}
\begin{lemma}
If there exists setting of values of variables that exactly $k$
clauses are satisfied, we can cover all the points
with $3n + 11m + (m-k)$ segments.
\end{lemma}

\begin{lemma}
If there exists cover with $k$ segments,
then also there exists solution for MAX-(3,3)-SAT.

TODO: Formulate this lemma better.
\end{lemma}

\chapter{Fixed-parameter tractable algorithm for geometric set cover problem}
In this chapter we show fixed-parameter tractable algorithms
for the geometric set cover problem in~two different settings.
Section \ref{section:fpt_unweighted} shows 
a fixed-parameter tractable algorithm for geometric set cover with unweighted segments.
The remainder of the chapter presents
a fixed-parameter tractable algorithm for geometric set cover with weighted segments
with $\delta$-extension.
We show an algorithm for the setting with $\delta$-extension,
because the original problem with weights is W[1]-hard,
as we show in Chapter $\ref{chapter:w1_hard}$.

We start with a shared definition for this problem.
We define \textit{extreme points} for a set of~collinear points.

\begin{defi}
	For a set of collinear points $C$ in the plane,
	\textbf{extreme points} of $C$ are the endpoints
	of the smallest segment that covers all points from set $C$.
	
	If $C$ consists of one point or is empty, then
	there are 1 or 0 extreme points respectively.
\end{defi}

\section{Fixed-parameter tractable algorithm for $\SegmentSetCover$}
\label{section:fpt_unweighted}
In this section we consider fixed-parameter tractable
algorithms for unweighted geometric set cover with segments.
The setting where segments are required to be axis-parallel
(or limited to a constant number of directions) has a trivial FPT algorithm.
We present an FPT algorithm for geometric set cover
with unweighted segments, where segments are in arbitrary directions.

\subsection{Axis-parallel segments}
\begin{tw}
	\textbf{(FPT for segment cover with axis-parallel segments)}.
	There exists an algorithm that given a family $\sets$ of
	axis-parallel segments,
	a set of points $\points$
	and a parameter $k$,
	runs in time $\mathcal{O}(2^k)$,
	and outputs a solution $\sol \subseteq \sets$
	such that $|\sol| \le k$ and $\sol$ covers all points in~$\points$,
	or determines that such a set $\sol$ does not exist.
\end{tw}

\begin{proof}
We show an $\mathcal{O}(2^k)$-time branching algorithm.
In each step, the algorithm selects a point $a$ which is not yet covered,
branches to choose one of the two directions, and greedily chooses
a segment $a$ in that direction to cover.
This proceeds until either all points are covered or $k$~segments are chosen.

Let us take
the point $a=(x_a,y_a)$ which is the smallest 
among points that are not yet covered
in the lexicographic ordering
of points in $\mathbb{R}^2$.
We need to cover $a$ with some of~the~remaining segments.

Branch over the choice of one of the coordinates ($x$ or $y$);
without loss of generality, let us assume we chose $x$.
Among the segments lying on line $x = x_a$,
we greedily add to~the~solution the~one that covers the most points.
As $a$ was the smallest in the lexicographical order,
all points on the line $x = x_a$ have the $y$-coordinate larger than $y_a$.
Therefore, if we denote the~greedily chosen segment as $s$,
then any other segment on the line $x = x_a$ that covers $a$ can only
cover a subset of points covered by $s$.
Thus, greedily choosing $s$ is optimal.

In each step of the algorithm we add one segment to the solution,
thus the recursion can be stopped at depth $k$.
If no branch finds a solution, then this means
that a solution of size at most $k$ does not exist.
\end{proof}

Note that the same algorithm can be used for segments in $d$ directions,
where we branch over $d$ choices of directions, and it runs in complexity $\mathcal{O}(d^k)$.

\subsection{Segments in arbitrary directions}
\label{segments_in_arbitrary_direction}
In this section we consider the setting where segments are not constrained
to a constant number of directions. 
We present a fixed-parameter tractable algorithm,
parameterized by the size of the solution.

\segmentCoverFpt*

We will need the following lemmas proving properties of any
instance of the problem.

\begin{lemma}
   \label{fpt_reduction}
   Given an instance $(\sets, \points)$ of the segment cover problem,
   without loss of generality we can assume that
   no segment covers a superset of what another segment covers.
   That~is, for any distinct $A, B \in \sets$, we have
   $A \cap \points \not \subseteq B \cap \points$ and $A \cap \points \not \supseteq B \cap \points$.
\end{lemma}   
   
\begin{proof}
Assume towards a contradiction that there is an instance  $(\sets, \points)$,
and two distinct subsets of $\sets$,
$A, B$, such that $A \cap \points \subseteq B \cap \points$.

We construct a set $\sets' := \sets - \{A\}$.
We prove that for any solution $\sol$ of $(\sets, \points)$,
we can construct a~solution $\sol' \subseteq \sets'$,
such that $|\sol'| \le |\sol|$.
Let us take any solution $\sol$ of $(\sets, \points)$.
If $A \in \sol$, then $\sol' := \sol \cup \{B\} - \{A\}$,
otherwise $\sol' := \sol$.
Let us consider the case when $A \in \sol$,
because the other case is trivial.
Since $A \cap \points \subseteq B \cap \points$,
then $\sol \cup \{B\} - \{A\}$
covers any point from $\points$ that was covered by $\sol$.
Also, $|\sol \cup \{B\} - \{A\}| \le |\sol|$.
\end{proof}

\begin{lemma}
	\label{fpt_long_lines}
	Given an instance $(\sets, \points)$
	of the segment cover problem 
	transformed by Lemma~\ref{fpt_reduction},
	if there exists a line $L$ with at least
	$k+1$ points on it, then there exists a subset $A \subseteq \sets$,
	of size at most $k$,
	such that every solution $\sol$ with $|\sol| \le k$
	satisfies $|A \cap \sol| \ge 1$.
	Moreover, such a subset can be found in~polynomial time.
\end{lemma}
\begin{proof}
Let us enumerate the points from $\points$ that lie on $L$ as $x_1, x_2, \ldots, x_t$
in the order in which they appear on $L$.
Our proposed set is defined as:
$$A := \left\{ \text{segment collinear with } L \text{ that covers } x_i
\text{ and does not cover } x_{i-1} : i \in \{1, \ldots, k\}\right\},$$
where for $i = 1$ we just take a segment that covers $x_1$.
If such a segment does not exist for any point $x$
as above, then $x$ does not give rise to any segment in $A$.

We prove the lemma by contradiction. Let us assume that there
exists a~solution $\sol$ of size at most $k$ such that $\sol \cap A = \emptyset$.


Let $\sol_L$ be the set of segments from $\sol$ that are collinear with $L$.

Every segment that is not collinear with $L$ can cover at~most one of
the points that lie on~this line.
Hence, if $\sol_L$ was empty, then
$\sol$ would cover at most $k$ points on line $L$,
but $L$ had at least $k+1$ different points from $\points$ on it.

Therefore, we know that $\sol_L$ is not empty
and $|\sol - \sol_L| \le k-1$.
Segments from $\sol - \sol_L$ can cover at most $k-1$
points among $\{x_1, x_2, \ldots, x_k\}$, therefore at least
one of these points must be covered by segments from $\sol_L$.
We take the leftmost point from $\{x_1, x_2, \ldots, x_k\}$ that is
covered in $\sol_L$ and name it $a$.
After the transformation from Lemma \ref{fpt_reduction},
in $\sol$ there is only one segment
that starts in $a$ and is collinear with $L$,
therefore this segment must be in both $\sol$ and $A$.
This contradiction concludes the proof that $|A \cap \sol| \ge 1$
for any solution $\sol$ of size at most $k$.
\end{proof}

We are now ready to prove Theorem \ref{segment_cover_fpt}.

\begin{proof}[Proof of Theorem \ref{segment_cover_fpt}.]
We will prove this theorem by presenting a branching algorithm that
works in desired complexity. It first branches over the
choice of segments to cover the lines with \textit{many} points
and then solves a small instance (where every line has at most $k$ points)
by checking all possible solutions.

\subparagraph{Algorithm.}
We present a recursive algorithm. Given an instance of the problem:

\begin{enumerate}[label={(\arabic*)}]
\item Use Lemma \ref{fpt_reduction} to remove some redundant segments from our instance.
\item If there exists a line with at least $k+1$ points from $\points$,
we branch over the choice of adding to~the~solution
one of~the~at~most $k$ possible segments
provided by Lemma \ref{fpt_long_lines}; name this segment $s$
and name the set of points from $\points$ that lie on $s$ as $S$.
By recursion, we find a~solution $\sol$
for the instance $(\points - S, \sets - \{s\})$,
and parameter $k-1$. We return $\sol \cup \{s\}$.
Note that if Lemma \ref{fpt_long_lines} returned $\emptyset$,
then we respond \texttt{NO}.
\item If every line has at most $k$ points on it and $|\points| > k^2$,
then answer \texttt{NO}.
\item If $|\points| \le k^2$, solve the problem by brute force:
check all subsets of $\sets$ of size at most $k$.
\end{enumerate}

\subparagraph{Correctness.}

Lemma \ref{fpt_long_lines} proves that at least one segment that we
branch over in (1) must be present in every solution $\sol$ with $|\sol| \le k$.
Therefore, the recursive call can find a~solution, provided there exists one.

In (2) the answer is no, because every line covers no more than $k$ points
from $\points$, which implies the same about every segment from $\sets$.
Under this assumption
we can cover only $k^2$ points with a solution of size $k$, which is less
than $|\points|$.

Checking all possible solutions in (3) is trivially correct.


\subparagraph{Complexity.}

In the leaves of the recursion we have $|\points| \le k^2$, so $|\sets| \le k^4$,
because every segment can be uniquely identified by the two extreme points it covers
(by Lemma \ref{fpt_reduction}). Therefore, there are $\binom{k^4}{k}$
possible solutions to check, each can be checked in time $\mathcal{O}(k|\points|)$.
Thus, (3) takes time $k^{\mathcal{O}(k)}$.


In this branching algorithm our parameter $k$ is decreased with every
recursive call, so we have at most $k$ levels of recursion with
branching over $k$ possibilities. Candidates to branch over
can be found on each level in time $\mathcal{O}((|\points|\cdot|\sets|)^{\mathcal{O}(1)})$.

Reduction from Lemma \ref{fpt_reduction} can be implemented
in time $\mathcal{O}((|\points|\cdot|\sets|)^{\mathcal{O}(1)})$.

It follows that the overall complexity
is $\mathcal{O}(((|\points|\cdot|\sets|)^{\mathcal{O}(1)}) \cdot k^{\mathcal{O}(k)})$
\end{proof}


\section{Fixed-parameter tractable algorithm for weighted segments with $\delta$-extensions}
\label{section:fpt_weighted}

In this section we consider the geometric set cover problem
for weighted segments relaxed with $\delta$-extensions.
We show that this problem
admits an FPT algorithm when parameterized by the size
of the solution and $\delta$.
In the next chapter we show that the assumption
about the problem being relaxed with $\delta$-extensions is necessary:
we prove that geometric set cover problem
for weighted segments (without extensions) is W[1]-hard, which means
there does not exist any FPT algorithm parameterized by solution size for it,
assuming FPT $\neq$ W[1].

\begin{tw}[FPT for weighted segment cover with $\delta$-extensions]{
	\label{fpt_weighted_segment}
	There exists an algorithm that given a family $\sets$ of
	$n$ weighted segments (in any direction),
	a set of $m$ points $\points$, and parameters $k$ and $\delta > 0$,
	such that it
	runs in time $f(k, \delta) \cdot (nm)^c$ for some computable function $f$ and a constant $c$ and
	outputs a set $\sol$ such that:
	\begin{itemize}
	\item $\sol \subseteq \sets$,
	\item $|\sol| \le k$,
	\item $\sol^{+\delta}$ covers all points in $\points$,
	\item the weight of $\sol$ is not greater than the weight
	of an optimum solution of size at most $k$
	for this problem without $\delta$-extensions
	\end{itemize}
	or determines that there is no set $\sol$ with $|\sol| \le k$
	such that $\sol$ covers all points in $\points$.
}\end{tw}


To solve this problem we will introduce a lemma about choosing
a \textit{dense} subset of points. A dense subset of points
for a set of collinear points $C$ and parameters $k$ and $\delta$
is a subset of $C$ such that
if we cover it with at most $k$ segments,
these segments after $\delta$-extensions will cover all of the points from $C$.
We will prove that such set 
of size bounded by some function $f(k, \delta)$
always exists (Lemma \ref{dense_set_exists}).
Later, Lemma \ref{dense_set_exists} will allow us to find a kernel
for our original problem.

\begin{defi}
	For a set of collinear points $C$,
	a subset $A \subseteq C$ is \textbf{$(k,\delta)$-dense} 
	if for any set of segments $R$ that covers $A$ and
	such that $|R| \le k$, it holds that $R^{+\delta}$ covers $C$.
\end{defi}

\begin{lemma}
	\label{dense_set_exists}
	For any set of collinear points $C$, $\delta > 0$ and $k \ge 1$,
	there exists a $(k,\delta)$-dense set $A \subseteq C$ of size
	at most $(2+\frac{2}{\delta})^k$.
	Moreover, there exists an algorithm that computes the $(k,\delta)$-dense set
	in time $O(|C| \cdot (2+\frac{2}{\delta})^k)$.
\end{lemma}

\begin{proof}
We prove this for a fixed $\delta$ by induction on $k$.

\subparagraph{Inductive hypothesis.}
For any set of collinear points $C$, there exists a set $A$ such that:
\begin{itemize}
\item $A$ is subset of $C$,
\item $A$ is $(\ell, \delta)$-dense for every $1 \le \ell \le k$,
\item $|A| \le (2+\frac{2}{\delta})^k$,
\item the extreme points of $C$ are in $A$.
\end{itemize}

\subparagraph{Base case for $k = 1$.}
It is sufficient that $A$ consists of the extreme points of $C$.

If they are covered with one segment, it must be a segment 
that includes the extreme points from $C$, so it covers the whole set $C$.

There are at most 2 extreme points in $C$ and $2 < 2+\frac{2}{\delta}$.

\subparagraph{Inductive step.}
Assuming inductive hypothesis for any set of collinear points $C$
and for parameter $k$, we will prove it for $k+1$.

Let $s$ be the minimal segment that includes all points from $C$.
That is, the extreme points of $C$ are endpoints of $s$.

We define $M = \lceil1+\frac{2}{\delta}\rceil$ subsegments of $s$
by splitting $s$ into $M$ closed segments of equal length.
We name these segments $v_i$, note that
$|v_i| = \frac{|s|}{M}$ for each $1 \le i \le M$.

Let $C_i$ be the subset of $C$ consisting of points lying on $v_i$.

Let $t_i$ be the segment with endpoints being the extreme points of $C_i$.
It might be a degenerate segment if $C_i$ consists of one point,
or $t_i$ might be empty if $C_i$ is empty.

Figure $\ref{fig:fpt_v_f_def}$ presents an example
of such segments $v_i$ and $t_i$.

\begin{figure}[h]
\begin{center}
\def\svgwidth{\columnwidth}
\input{fpt_v_t_def.pdf_tex}
\end{center}
\caption{\textbf{Example of segments $v_i$ and $t_i$.}}
Example for $M = 7$ and some set of points (marked with black circles).
The top panel shows segments $v_i$ and the bottom panel shows segments $t_i$
on the same set of points.
$a$ and $b$ are the extreme points and therefore segment $s$
ends at $a$ and $b$.
Red segments depict the split into $M$ segments of equal length $v_i$.
Blue segments depict the segments $t_i$. $t_5$ is an empty segment,
because there are no points that lie on segment $v_5$.
Segments $t_3$ and $t_7$ are degenerated to one point --
$c$ and $d$ respectively.
Segments $t_1$ and $t_2$ share one point $b$.
\label{fig:fpt_v_f_def}
\end{figure}

We use the inductive hypothesis to choose $(k, \delta)$-dense sets $A_i$
for sets $C_i$. Note that if $|C_i| \le 1$, then $A_i = C_i$
and it is still a $(k, \delta)$-dense set for $C_i$.

Then we define $A = \bigcup_{i=1}^{M} A_i$.
Thus $A$ includes the extreme points of $C$,
because they are included in the sets $A_1$ and $A_M$.

The size of each $A_i$ is at most $(2+\frac{2}{\delta})^{k}$
from the inductive hypothesis, therefore size of $A$ is at most:
$$M\left(2+\frac{2}{\delta}\right)^{k} =
\left\lceil1+\frac{2}{\delta}\right\rceil\cdot\left(2+\frac{2}{\delta}\right)^{k}
\le \left(2+\frac{2}{\delta}\right)^{k+1}.$$


\subparagraph{Proof that $A$ is $(k, \delta)$-dense for $C$.}
Let us take any cover of $A$ with $k+1$ segments and call it $\sol$.

For every segment $t_i$, if there exists a segment $x$ in $\sol$ 
that is disjoint with $t_i$,
then we have a cover of $A_i$ with at most $k$
segments using $\sol - \{x\}$.
Since $A_i$ is $(k, \delta)$-dense for $t_i$ and $C_i$,
$(\sol - \{x\})^{+\delta}$ covers $C_i$.
So $\sol^{+\delta}$ covers $C_i$ as well.

If there exists a segment $t_i$ for which a segment $x$ as defined above
does not exist, then all $k+1$ segments that cover
$A_i$ intersect $t_i$.
An example of such segments is depicted in Figure~\ref{fig:fpt_tricky_case}.
Let us consider any such $t_i$.
By inductive hypothesis, the endpoints of $s$ are
in $A_1$ and $A_M$ respectively, so $\sol$ must cover them.
For each endpoint of $s$, there exists
a segment that contains this endpoint and intersects $t_i$.
Let us call these two segments $y$ and $z$. It follows that:
$|y| + |z| + |t_i| \ge |s|$.
Since $|t_i| \le |v_i| = \frac{|s|}{M} \le \frac{|s|}{1+\frac{2}{\delta}} = \frac{|s|\delta}{\delta+2}$,
we have $\max(|y|, |z|) \ge |s|(1-\frac{\delta}{\delta+2})/2 = \frac{|s|}{\delta+2}$.

\begin{figure}[h]
\begin{center}
\def\svgwidth{\columnwidth}
\input{fpt_tricky_case.pdf_tex}
\end{center}
\caption{\textbf{Example of all $k+1$ segments intersecting one segment $t_i$.}}
Both panels show the same set $\points$ (black circles),
the same as in Figure $\ref{fig:fpt_v_f_def}$.
The top panel shows blue segments $t_i$ for $M=7$.
The bottom panel shows green segments -- solution $\sol$ of size 4.
All segments from $\sol$ intersect $t_4$.
Segments $z$ and $y$ are named in the figure.
\label{fig:fpt_tricky_case}
\end{figure}

After $\delta$-extension, the longer of these segments will
expand at both ends by at least:
$$\max(|y|, |z|)\delta \ge \frac{|s|\delta}{\delta+2} =
\frac{|s|}{1+\frac{2}{\delta}} \ge \frac{|s|}{M} = |v_i| \ge |t_i|.$$

Therefore, the longer of segments $y$ and $z$ will cover the whole segment $t_i$
after $\delta$-extension. We conclude that $\sol^{+\delta}$ covers $C_i$.

Since $C = \bigcup_{i=1}^M C_i$,
it follows that $\sol^{+\delta}$ covers $C$.


\subparagraph{Algorithm.}

We can simulate the inductive proof presented above by a recursive algorithm with
the following complexity:
$$O\left(|C|+\frac{1}{\delta}\right) + O\left(|C|\cdot\left(2+\frac{2}{\delta}\right)^k\right).$$

\end{proof}

Let us now formulate some claims about the
properties for the problem parameterized by the solution size.
These properties provide bounds for different
objects in the problem instance,
which help us to find a small kernel for the problem
or conclude that the optimum
solution to this instance must be in terms of size above some treshold.

\begin{defi}
A line in the plane is \textbf{long}
if there are at least $k+1$ points from $\points$ on it.
\end{defi}

\begin{claim}
\label{few_long_lines}
If there are more than $k$ different long lines, then 
$\points$ can not be covered with $k$ segments.
\end{claim}

\begin{proof}
We prove the claim by contradiction.
Let us assume that we have at least $k+1$ different
long lines in our instance of the problem
and there is a solution $\sol$ of size at most $k$
covering points $\points$.

Choose any long line $L$.
Every segment from $\sol$ which is not collinear with $L$,
covers at most one point that lies on $L$.
$L$ is long, so there are at least $k+1$ points from $\points$ that lie on $L$.
That implies that there must be a segment in $\sol$ that is
collinear with $L$.

Since we have at least $k+1$ different long lines,
there are at least $k+1$
segments in $\sol$ collinear with different lines.
This contradicts with the assumption that $|\sol| \le k$.
\end{proof}

\begin{claim}
\label{few_points}
If there are more than $k^2$ points from $\points$
that do not lie on any long line,
then $\points$ can not be covered with $k$ segments.
\end{claim}

\begin{proof}
We prove the claim by contradiction.
Let us assume that we have at least $k^2+1$ points
from $\points$ that do not lie on any long line, call this set $A$,
and a solution $\sol$ of size at most $k$
covering all points in $\points$.

Every segment $s$ from $\sol$ covers at most $k$
points from $A$.
This is because if $s$ covered at least $k+1$ points from $A$,
then the line in the direction of $s$ would be a long line
and that contradicts the definiton of $A$.

If every segment from $\sol$ covers at most $k$ points from $A$
and $|\sol| \le k$, then at most $k^2$ points from $A$ are covered by $\sol$
and that contradicts the fact that $\sol$ is a solution to the given
geometric set cover instance.
\end{proof}

We are now ready to give a proof of Theorem \ref{fpt_weighted_segment}.

\begin{proof}[Proof of Theorem \ref{fpt_weighted_segment}]
Our goal is to either answer \texttt{NO} or to find a kernel
$(\sets', \points')$ of bounded size, such that:
\begin{itemize}
\item \textit{(Property 1)} for every solution
$\sol$ to $(\sets, \points)$ of size at most $k$,
there exists a set $\sol_1 \subseteq \sets'$ such that
$\sol_1 \le k$, weight of $\sol_1$ is not greater than weight of $\sol$
and $\sol_1$ covers $\points'$;
\item \textit{(Property 2)}
for every set $\sol_2 \subseteq \sets'$ such that $|\sol_2| \le k$
and $\sol_2$ covers points in $\points'$, $\sol_2^{+\delta}$
covers points in original instance $\points$.
\end{itemize}

If we found such sets $(\sets', \points')$,
using \textit{Property 1} we know that optimum solution 
of size at most $k$ to $(\points', \sets')$
has no greater weight than optimum solution
of size at most $k$ to $(\points, \sets)$.
Using \textit{Property 2} we know that
any solution to $(\points', \sets')$
after $\delta$-extensions covers $\points$.

Therefore finding such sets in desired complexity
is sufficient to prove Theorem \ref{fpt_weighted_segment}.

\paragraph{Definition of $\points'$ and $\sets'$.}
Let us name the number of different long lines as $l$.
Applying Claims \ref{few_long_lines} and \ref{few_points},
if we have more than $k$ different long lines
or more than $k^2$ points from $\points$
that do not lie on any long line, then we answer \texttt{NO},
becase these lemmas prove that there is no solution of size at most $k$
to this instance.

Otherwise, we can split $\points$ into at most $k+1$ sets:
\begin{itemize}
\item $D$: points that do not lie on any long line, $|D| \le k^2$;
\item $C_i$ for $1 \le i \le l$: points that lie on the $i$-th long line, $|C_i| > k$.
\end{itemize}
Note that sets $C_i$ do not need to be disjoint.

Then for every set $C_i$ we can use Lemma \ref{dense_set_exists}
to obtain a $(k,\delta)$-dense set $A_i$
for $C_i$ with $|A_i| \le (2+\frac{2}{\delta})^k$.

We define $\points':= D \cup (\bigcup A_i)$. $\points'$ has size at most
$k^2 + k(2+\frac{2}{\delta})^k$.
We define $\sets'$ as
for every pair of points $\points'$, we can choose one segment from
$\sets$ that has the lowest weight
among segments that cover these points 
or decide that there is no segment that covers them.
There are at most $|\points'|^2$ different segments in $\sets'$,
Therefore both $\sets'$ and $\points'$ have size bounded
by some function $f(k)$.

\paragraph{Proof of Property 1.}
First, we prove that
for every set $\sol_2 \subseteq \sets'$ such that $|\sol_2| \le k$
and $\sol_2$ covers points in $\points'$, $\sol_2^{+\delta}$
covers points in original instance $\points$.

Let us take such a set $\sol_2$.

$\points$ is separated into several parts -- sets $D$ and $C_i$.
Points from $D$ are covered by $\sol_2$, because $D$ is part of $\points'$.
Each point from any $A_i$ is covered, because $A_i$ is a part of $\points'$;
$A_i$ is a $(k,\delta)$-dense set for $C_i$, therefore $\sol_2^{+\delta}$
covers all points in $C_i$. Therefore $\sol_2^{+\delta}$ covers
all points in $\points$.

\paragraph{Proof of Property 2.}
Secondly, we prove that for every solution
$\sol$ to $(\sets, \points)$ of size at most $k$,
there exists a set $\sol_1 \subseteq \sets'$ such that
$\sol_1 \le k$ and
$\sol_1^{+\delta}$ covers all points in $\points$ and
weight of $\sol_1$ is not greater than weight of $\sol$.

For every segment in $\sol$, say $s$,
let us look at the points from $\points'$ that lie on $s$
and call this set of points $F$.
$F$ is a set of collinear points for course.
We can cover $F$ with any segment that covers extreme points of $F$,
because all other points lie on the segment between these points.
Therefore we can replace $s$ with a segment $s'$
that has lowest weight among the points that cover extreme points of $F$.
Such a segment belongs to $\sets'$, because this is how it was defined.
Of course segment $s'$ also have weight no greater than weight of $s$,
because $s$ also covers $F$.

Therefore we produced the set $\sol_1$ that has the same size,
weight not greather than $\sol$ and it covers $\points'$.

\paragraph{Complexity}
We find solutin of $(\points', \sets')$ by iterating
over all possible subsets of $\sets$.
Finding sets $\sets'$ and $\points'$ and then solving 
problem for kernel has overall complexity
$(|\sets| + |\points|)^{O(1)}O((2 + \frac{2}{\delta})^k) + O((k^2 + k(2 + \frac{2}{\delta})^k)^k)$.
\end{proof}


In this chapter we consider geometric set cover problem with weighted segments.
Theorem \ref{w1_hard} proves that this problem is 
W[1]-hard when parametrized by the size of the solution.
We additionally restrict the problem to only use segments
in three directions to achieve a stronger result.
W[1]-hardness is proved by reduction to a grid tiling problem,
which was introduced in \cite{marx_grid_tiling}.

\begin{defi}
	Line is \textbf{right-diagonal} if it is
	described by linear function $y = x + d$ for any $d \in \mathbb{R}$.
	Segment is \textbf{right-diagonal} if its
	direction is a right-diagonal line.
\end{defi}

\begin{tw}
\label{w1_hard}
	Consider the problem of covering a set $\points$ of points
	by selecting at most $k$ segments
	from a set of segments $\sets$ 
	with non-negative weights $w : \sets \rightarrow \mathbb{R^+}$
	so that the weight of the cover is minimal.
	Then this problem is W[1]-hard parametrized by $k$ and
	assuming ETH, there is no algorithm for this
	problem with running time
	$f(k)\cdot(|\points| + |\sets|)^{o(\sqrt{k})}$
	for any computable function $f$.
	Moreover, this holds even if all segments in $\sets$
	are axis-parallel or right-diagonal.
\end{tw}

Theorem \ref{w1_hard} is also true for less
restricted problem where segments have any direction.
We prove more tight setting in this section.

In order to prove Theorem \ref{w1_hard}
we will show reduction from a W[1]-hard problem.
We introduce the grid tiling problem, which is proven
to be W[1]-complete in literature.

\newcommand{\pow}{\mathsf{Pow}}

\begin{defi}
We define \textbf{powerset} of some set $A$, denoted as $\pow(A)$,
as a set of all subsets of $A$, ie. $\pow(A) = \{B : B \subseteq A\}$.
\end{defi}

\begin{defi}
In the \textbf{grid tiling} problem we are given integers $n$ and $k$,
and a function
$f : \{1 \ldots k\} \times \{1 \ldots k\} \rightarrow \pow(\{1 \ldots n\} \times \{1 \ldots n\})$
specifying the set of allowed tiles for each cell of a $k \times k$ grid.
The task is to find functions
$x,y : \{1 \ldots k\} \rightarrow \{1 \ldots n\}$
that assign numbers from $\{1 \ldots n\}$
to respectively columns and rows of the grid,
so that $(x(i), y(j)) \in f(i, j)$ for all valid $i$ and $j$,
or conclude that such an assignment does not exist.
\end{defi}

In short, in grid tiling problem you need to assign numbers
to rows and columns in such a way,
that for every pair of a row and a column,
the pair of numbers assigned
to the row and column 
belongs to the allowed set corresponding to the intersection
of the row and column in question.
The next theorem describes the complexity of this problem,
which is W[1]-hard when parametrized by the size of the grid.

\begin{tw}
\label{grid_tiling_w1_hard}
\textbf{\cite{marx_grid_tiling}}
Grid tiling is W[1]-hard parametrized by $k$ and
assuming ETH, there is no $f(k)\cdot n^{o(\sqrt{k})}$-time
algorithm solving the grid tiling problem
for any computable function $f$.
\end{tw}

The reminder of this section is proving Theorem \ref{w1_hard}
by reduction of grid tiling problem to geometric set cover.
That proves the W[1]-hardness of geometric set cover,
because if we could solve it with an FPT algorithm,
then we could also solve grid tiling problem
(which we reduced to geometric set cover).
Therefore geometric set cover with setting
described in Theorem \ref{w1_hard}
is at least as hard as the grid tiling problem.

We start with an instance of the grid tiling problem $(n, k, f)$.
The instance consists of:
\begin{itemize}
\item size of the grid $k$,
\item number of colors $n$,
\item function of allowed tiles
$f : \{1, \ldots, k\} \times \{1, \ldots, k\} \rightarrow \pow(\{1, \ldots, n\} \times \{1, \ldots, n\})$.
\end{itemize}

TODO: nice picture of instance of grid tiling with solution

\paragraph{Construction.}
\newcommand{\order}{\mathsf{order}}
\newcommand{\matchv}{\mathsf{match}_v}
\newcommand{\matchh}{\mathsf{match}_h}
\newcommand{\instanceSetCover}{(\points, \sets, w, 3k^2+2k)}
We construct an instance $\instanceSetCover$ of geometric set cover as follows.

First let us choose any bijection
$\order : \{1, \ldots, n^2\} \rightarrow \{1, \ldots, n\} \times \{1, \ldots, n\}$.


Define $\matchv(i, j)$ and $\matchh(i, j)$
as boolean functions denoting whether two points share x or y coordinate:
$$\matchv(i, j) \text{ is } \texttt{true} \iff
\order(i) \text{ and } \order(j) \text{ have the same x coordinate,}$$
$$\matchh(i, j) \text{ is } \texttt{true} \iff
\order(i) \text{ and } \order(j) \text{ have the same y coordinate.}$$


\subparagraph{Points.}

For $1 \le i,j \le k$ and $1 \le t \le n^2$ define points:
	$$h_{i, j, t} := (i \cdot (n^2+1) + t, j \cdot (n^2+1)),$$
	$$v_{i, j, t} := (i \cdot (n^2+1), j \cdot (n^2+1) + t).$$
	
Let us define sets $H$ and $V$ as:
$$H := \{h_{i, j, t} : 1 \le i, j, \le k, 1 \le t \le n^2\},$$
$$V := \{v_{i, j, t} : 1 \le i, j, \le k, 1 \le t \le n^2\}.$$
	
Let $\epsilon = \frac{1}{2k^2}$.
For a point $p = (x, y)$ we define points:
$$p^{L} := (x - \epsilon, y),$$
$$p^{R} := (x + \epsilon, y),$$
$$p^{U} := (x, y + \epsilon),$$
$$p^{D} := (x, y - \epsilon).$$

Then we define the point set as follows:
$$\points := H \cup \{p^L : p \in H\} \cup \{p^R : p \in H\}
\cup V \cup \{p^U : p \in V\} \cup \{p^D : p \in V\}.$$

\begin{defi}
	\label{guard_def}
	For every point $p \in H$, we name point $p^L$ its \textbf{left guard}
	and point $p^R$ its \textbf{right guard}.
	
	Similarily for every points $p \in V$, we name point $p^D$ its \textbf{lower guard}
	and point $p^U$ its \textbf{upper guard}.
\end{defi}

\subparagraph{Segments.}
\newcommand{\hor}[4]{\mathsf{hor}_{#1,#2,#3,#4}}
\newcommand{\ver}[4]{\mathsf{ver}_{#1,#2,#3,#4}}
\newcommand{\horbeg}[2]{\mathsf{horBeg}_{#1,#2}}
\newcommand{\verbeg}[2]{\mathsf{verBeg}_{#1,#2}}
\newcommand{\horend}[2]{\mathsf{horEnd}_{#1,#2}}
\newcommand{\verend}[2]{\mathsf{verEnd}_{#1,#2}}

For $1 \le i,j \le k$ and $1 \le t_1, t_2 \le n^2$ define segments:

$$\hor{i}{j}{t_1}{t_2} := (h^R_{i,j,t_1}, h^L_{i+1, j, t_2}),$$
$$\ver{i}{j}{t_1}{t_2} := (v^U_{i,j,t_1}, v^D_{i, j+1, t_2}),$$

$$\horbeg{i}{t} := (h^L_{1, i, 1}, h^L_{1, i, t}),$$
$$\horend{i}{t} := (h^R_{k, i, t}, h^R_{k, i, n^2}),$$

$$\verbeg{i}{t} := (v^D_{i, 1, 1}, v^D_{i, 1, t}),$$
$$\verend{i}{t} := (v^U_{i, k, t}, v^U_{i, k, n^2}).$$

\newcommand{\allhor}{\mathsf{HOR}}
\newcommand{\allver}{\mathsf{VER}}
\newcommand{\alldiag}{\mathsf{DIAG}}

Next, we define sets of vertical and horizontal segments:
\begin{eqnarray*}
\allhor &:= &\{\hor{i}{j}{t_1}{t_2} : 1 \le i < k, 1 \le j \le k,
1 \le t_1, t_2 \le n^2, \matchh(t_1, t_2) \text{ holds}\} \\
&\cup &\{\horbeg{i}{t} : 1 \le i \le k, 1 \le t \le n^2\}
\\
&\cup &\{\horend{i}{t} : 1 \le i \le k, 1 \le t \le n^2\},
\end{eqnarray*}

\begin{eqnarray*}
\allver &:= &\{\ver{i}{j}{t_1}{t_2} : 1 \le i \le k, 1 \le j < k,
1 \le t_1, t_2 \le n^2, \matchv(t_1, t_2) \text{ holds}\} \\
&\cup &\{\verbeg{i}{t} : 1 \le i \le k, 1 \le t \le n^2\}
\\
&\cup &\{\verend{i}{t} : 1 \le i \le k, 1 \le t \le n^2\}.
\end{eqnarray*}

Finally, we also define a set of right-diagonal segments:
$$\alldiag := \{ (h_{i, j, t}, v_{i, j, t}) :
	1 \le i, j \le k, 1 \le t \le n^2, \order(t) \in f(i, j)\}$$

TODO: explain that these segments are in fact diagonal

The constructed segment set is:

$$\sets := \allhor \cup \allver \cup \alldiag.$$

The weight of each segment in $\allhor \cup \allver$
is equal to the segment,
while every segment in $\alldiag$ has weight
$\delta = \frac{1}{4k^4}$.

TODO: Put a picture of small instance like 3x3 with n=2

\begin{equation}
w(s) =
	\begin{cases*}
	  length(s) 			& if $s \in \allhor \cup \allver$ \\
	  \delta        & if $s \in \alldiag$
	\end{cases*}
\end{equation}

\newcommand{\solWeight}{2k^2(n^2+1) - 4k^2\epsilon -4k(1-\epsilon) +k^2\delta }

Now, we prove that the constructed instance of geometric set cover
with weighted segments is indeed a correct and sound reduction
of the grid tiling problem. Lemma \ref{set_cover_solution_exists}
proves that if the solution of the grid tiling instance exists,
then there exists a solution with bounded size and weight
of constructed geometric set cover instance exists.

Then Lemma \ref{grid_tiling_exists} proves that if the solution
of the geometric set cover instance with bounded weight exists,
then there exists a solution to the original grid tiling instance.

\begin{lemma}
\label{set_cover_solution_exists}
	If there exists a solution of the grid tiling instance $(f_{i,j})$,
	then there exists a solution for the instance $\instanceSetCover$
	of geometric set cover of size at most $3k^2+2k$
	with weight $\solWeight$.
\end{lemma}

\begin{proof}
Suppose there exists a solution $x,y$ to the grid tiling problem.
	
We define subset of $\sets$ -- a proposed solution $\sol$
in three parts $D \subset \alldiag$, $A \subset \allhor$ and $B \subset \allver$:
\begin{eqnarray*}
	D & := & \{(v_{i, j, t}, h_{i, j, t}) : 1 \le i, j \le k, t = \order^{-1}(x(i), y(j))\}, \\
	A & := & \{\horbeg{i}{\order^{-1}(x(1), y(i))} : 1 \le i \le k\} \\
	& \cup & \{\horend{i}{\order^{-1}(x(k), y(i))} : 1 \le i \le k\} \\
	& \cup & \{\hor{i}{j}{\order^{-1}(x(i), y(j))}{\order^{-1}(x(i+1), y(j))} : 1 \le i < k, 1 \le j \le k\}, \\
	B & := & \{\verbeg{i}{\order^{-1}(x(i), y(1))} : 1 \le i \le k\} \\
	& \cup & \{\verend{i}{\order^{-1}(x(i), y(k))} : 1 \le i \le k\} \\
	& \cup & \{\ver{i}{j}{\order^{-1}(x(i), y(j))}{\order^{-1}(x(i), y(j+1))} : 1 \le i \le k, 1 \le j < k\},
\end{eqnarray*}
	$$\sol := D \cup A \cup B.$$

Since $\points = H \cup V$, we show that this covers the whole set $H$,
proof for $V$ is analogical.

Take any $1 \le j \le k$ and define $t_{i,j} := \order^{-1}(x(i), y(j)$:
$\horbeg{j}{t_1} = (h_{1,j,1}^L, h_{1, j, t_1}^L)$ and next segment
$\hor{1}{j}{t_1}{t_2} = (h_{1,j, t_1}^R, h_{2,j,t_2}^L)$.
Therefore points $h_{1,j,x}, h_{1,j,x}^L$ and $h_{1,j,x}^R$
for all $1 \le x \le n^2$ ale covered by $\horbeg{j}{t_1} and \hor{1}{j}{t_1}{t_2}$,
excluding point $h_{1,j,t_i}$.

$D$ covers all points $h_{i,j, t_{i,j}}$ and $v_{i,j, t_{i,j}}$, therefore
all points are covered.

Size of this proposed solution is:
$$|\sol| = |D| + |A| + |B| = k^2 + (k+1)k + (k+1)k = 3k^+2k.$$

TODO: Calculate weight of $\sol$
Whatevs, przeciez widac
	
Then consider a solution that covers
all points
$$\{h_{i, j, t} : 1 \le i, j \le k, \order(t)=(x(i), y(j))\}
\cup \{v_{i, j, t} : 1 \le i, j \le k, \order(t)=(x(i), y(j))\}$$

with $k^2$ segments from $\alldiag$
and the rest in $\allver$ or $\allhor$.
This solution has weight $\solWeight$.
\end{proof}


\begin{claim}
\label{guards}
In any solution of the instance $\instanceSetCover$:
\begin{itemize}
\item left and right guards of points in $H$
(points in $\{p^L : p \in H\} \cup \{p^R : p \in H\}$)
have to be covered with sgements from $\allhor$,
\item lower and upper guard of points in $V$
(points in $\{p^D : p \in V\} \cup \{p^U : p \in V\}$)
have to be covered with segments from $\allver$.
\end{itemize}
\end{claim}

\begin{proof}
We prove the claim for the points from $H$
as the proof for points from $V$ is analogical.

Every segment in $\allver$ is vertical and 
their x-coordinate is equal to $i(n^2+1)$ for some $1\le i \le k$,
so they all have different x-coordinate
than any left or right guard of points in $H$.

Every point $x$, which is a left or right guard of points in $H$
have $kn^2$ segments from $\alldiag$ that intersect with the horizontal
line that goes through $x$. All of these segments intersect with
this line in points from set $H$, therefore none of them
cover any of the guards.

Therefore none of the segments from $\allver$ or $\alldiag$ cover any
of the guards of the points in $H$.
\end{proof}

Now we present a few additional properties of the constructed instance
of the geometric set cover that help us to prove
the Lemma \ref{grid_tiling_exists}.

\begin{claim}
\label{one_diag_in_square}
For any $1 \le i, j \le n$
and any solution of the instance $\instanceSetCover$
all but at most one points $h_{i, j, t_1}, h_{i, j, t_2}$
($v_{i, j, t_1}, v_{i, j, t_2}$)
for $1 \le t_1 < t_2 \le n^2$
must be
covered with segments from $\allhor$ ($\allver$).
\end{claim}

\begin{proof}
We prove the claim for horizontal segments,
as the proof for vertical segments is analoguous.

Assume point $h_{i, j, t_1}$ is not covered with
segments from $\allhor$.
Point $h^R_{i, j, t_1}$ has to be covered with $\allhor$
by Claim $\ref{guards}$.
Every segment in $\allhor$ covering $h^R_{i, j, t_1}$,
but not $h_{i,j,t_1}$ covers also $h_{i, j, t_2}$.
\end{proof}

\begin{lemma}
\label{vertical_horizontal_sum}
For every solution of the instance $\instanceSetCover$,
the sum of weights of segments chosen
from sets $\allhor$ and $\allver$ at least
$W_{hv} = 2k^2(n^2+1) -4k^2\epsilon -4k(1-\epsilon)$.
\end{lemma}

\begin{proof}
We know that for all $1 \le i,j \le k$
there exists at most one $t_x$ and at most one $t_y$ such that
$v_{i,j,t_x}$ and $h_{i,j,t_y}$
are not covered by
segments from $\allhor$ and $\allver$ (Claim \ref{one_diag_in_square}),

TODO: Rephrase the sentence below

We sum the lower bound for sum of length for horizontal/vertical
lines for a single vertical line
(the bound is the same for every horizontal line).

Let us fix $1 \le i \le k$.

\begin{enumerate}[label={(\arabic*)}]
\item The total length between $v^D_{i, 1, 1}$ and $v^U_{i, k, n^2}$ is:
$$(k(n^2+1) + n^2 +\epsilon) - ((n^2+1)+1 -\epsilon) = k(n^2+1) - 2(1 - \epsilon).$$

\item For every $1 \le j \le k$ there exists at most one $1 \le t \le n^2$
such that $v_{i,j,t}$ is not covered by segments from $\allver$
(Claim \ref{one_diag_in_square}).
Its guards (see Definition \ref{guard_def}) $v^U_{i,j,t}$ and $v^D_{i,j,t}$
have to be covered in $\allver$ (Claim \ref{guards}).
Therefore at most $k$ spaces of length $2\epsilon$ can be left
not covered by segments from $\allver$ between $v_{i,1,1}^D$ and $v_{i,k,n^2}^U$.

Proof for horizontal segments is analogous.

\end{enumerate}
The sum of these lower bounds for vertical and horizontal lines is:
$$2k(k(n^2+1) -2k\epsilon -2(1-\epsilon)) = 2k^2(n^2+1) -4k^2\epsilon -4k(1-\epsilon)$$
\end{proof}

\begin{lemma}
\label{diag_correct}
For a constructed instance $\instanceSetCover$
for any solution $\sol$ with weight equal to $\solWeight$,
for every $1 \le i,j \le k$
there exists such $1 \le t \le n^2$ that:
\begin{enumerate}[label={(\arabic*)}]
\item $v_{i,j,t}, h_{i,j,t}$ are not covered by segments from $\allver$ or $\allhor$;
\item segment $(v_{i,j,t}, h_{i,j,t})$ is in solution $\sol$;
\item $\order(t) \in f(i,j)$, ie. it is an allowed tile for $(i,j)$;
\item for every $1 \le s\le n^2$, $s \neq t$, $v_{i,j,s}$ is covered in $\allver$;
\item for every $1 \le s\le n^2$, $s \neq t$, $h_{i,j,s}$ is covered in $\allhor$.
\end{enumerate}
Name the function of this $t$ as
$diagonal : \{1 \ldots k\} \times \{1 \ldots k\} \rightarrow \{1 \ldots n^2\}$.
\end{lemma}

\begin{proof}
At most one $h_{i,j,t_x}$ and $v_{i,j,t_y}$
points are covered with $\alldiag$
(Claim \ref{one_diag_in_square}).
	
Exactly one $h_{i,j,t_x}$ and $v_{i,j,t_y}$
points are covered with $\alldiag$,
because if one of them were not, then we would use too much weight
$$W_{hv} + 2\epsilon > \solWeight$$

This points are covered with the same segment from $\alldiag$,
because we need to use at least $k^2$ of them to use
exactly one DIAG segment for every pair of $1 \le i,j \le k$,
if we used 2 segments from $\alldiag$
for one pair $(i,j)$,
then we would have used too much weight by $\delta$.
Since these points $h_{i,j,t_x}$ and $v_{i,j,t_y}$ are covered by
segment from $\alldiag$, therefore $t_x = t_y$.

Therefore $diagonal(i,j) = t_x = t_y$
and $\order(t_x)$ is an allowed tile for $(i,j)$
because the respective segment is in $\alldiag$.

\end{proof}

\begin{lemma}
\label{vertical_horizontal_synchronized}
For a constructed instance $\instanceSetCover$
for any solution of weight $\solWeight$ it holds that $diagonal$ function
from Lemma \ref{diag_correct}:
\begin{enumerate}
\item 
for any $1 \le i < k, 1 \le j \le k$,
$\matchh(diagonal(i, j),diagonal(i+1, j))$ must be true;
\item 
for any $1 \le i \le k, 1 \le j < k$,
$\matchv(diagonal(i, j),diagonal(i, j+1))$ must be true.
\end{enumerate}
\end{lemma}

\begin{proof}
Every space between points in $H$ and $V$ covered by $\allhor$/$\allver$
has to be covered by only one segment, because otherwise it would
use $W_{hv} + \epsilon > \solWeight$,
therefore such solution would be too costly.

Proof for vertical (2), proof for horizontal is analogous.

Let us take any $1 \le i < k, 1 \le j \le k$
and name $t_1 = diagonal(i, j)$ and $t_2 = diagonal(i+1, j)$.
Therefore $h_{i,j,t_1}$ and $h_{i+1,j,t_2}$
are not covered by segments from $\allhor$,
$h^R_{i,j,t_1}$ and $h^L_{i+1,j,t_2}$
have to be covered by segments from $\allhor$ (Claim \ref{guards}).
Every segment from $\allhor$ starts at $h^R_{x,y,z_1}$
segment and finishes at $h^L_{x,y+1,z_2}$ segment for some
$1 \le x \le k$,$1 \le z < k$ and $1 \le z_1, z_2 \le n^2$.
Since all of the points between $h^R_{i,j,t_1}$ and $h^L_{i+1,j,t_2}$
are covered by segments in $\allhor$,
and there are two different segments covering
these points, one of them must begin
at $h^R_{i,j,t_1}$ and end at $h^L_{i,j+1,z_2}$
and the other one begin at $h^R_{i,j,z_1}$
and end at $h^L_{i+1,j,t_2}$
for some $1 \le z_1, z_2 \le n^2$.
Therefore space between $h^R_{i,j,z_1}$ and $h^L_{i,j+1,z_2}$
is covered twice and is longer than $\epsilon$.
By Lemma \ref{vertical_horizontal_sum}
lower bound for weight of such solution is $W_{hv} + \epsilon$
which contradicts that the given solution has size $\solWeight < W_{hv} + \epsilon$
Therefore $h^R_{i,j,t_1}$ and $h^L_{i+1,j,t_2}$ must be covered
by one segment from $\allhor$ and these points are ends of this segment.

$h^R_{i,j,t_1}$ and $h^L_{i+1,j,t_2}$
are ends of a segment from $\allhor$,
therefore $\matchh(t_1,t_2)$ must be true.
\end{proof}

\begin{corollary}
\label{vertical_horizontal_synchronized_inductive}
For a constructed instance $\instanceSetCover$
for any solution of weight $\solWeight$ it holds that $diagonal$ function
from Lemma \ref{diag_correct}:
\begin{enumerate}
\item 
for any $1 \le i, j \le k$,
$\matchh(diagonal(1, j),diagonal(i, j))$ must be true;
\item 
for any $1 \le i, j \le k$,
$\matchv(diagonal(i, 1),diagonal(i, j))$ must be true.
\end{enumerate}
\end{corollary}
\begin{proof}
Simple inductive proof based on Lemma \ref{vertical_horizontal_synchronized}.
\end{proof}

\begin{lemma}
\label{grid_tiling_exists}
	If there exists solution of instance $\instanceSetCover$
	with weight at most $\solWeight$,
	then there exists a solution for grid tiling instance.
\end{lemma}

\begin{proof}
Take $diagonal$ function from Lemma \ref{diag_correct}.

To define $x$ funtion 
for every $1 \le i \le k$ as $x(i) = x_i$
where $(x_i, a) = \order(v_{i,1})$
and $y$ function 
for every $1 \le i \le k$ as $y(i) = y_i$
where $(b, y_i) = \order(h_{1,i})$

To prove that it is a correct solution for grid tiling,
we need to prove that 
for every $1 \le i,j \le k$ $(x(i), y(j))$ is in
allowed tiles set $f(i,j)$.

Let us take any $1 \le i,j \le k$,
By Corollary \ref{vertical_horizontal_synchronized_inductive}
we know that $\matchh(diagonal(1, j),diagonal(i, j))$ and
$\matchv(diagonal(i, 1),diagonal(i, j))$ are true.
Therefore $\order(diagonal(i,j)) = (x(i), y(i))$.
By Lemma \ref{diag_correct} we know that 
$\order(diagonal(i,j))$ is in $f(i,j)$.
Therefore 
$(x(i), y(i))$
is in $f(i,j)$.

\end{proof}

\begin{proof}[Proof of Theorem \ref{w1_hard}]
Based on Lemmas \ref{set_cover_solution_exists} and \ref{grid_tiling_exists}
this is true.
\end{proof}


\chapter{Geometric Set Cover with lines}
\section{Lines parallel to one of the axis}
When $\mathcal{R}$ consists only of lines parallel to
one of the axis, the problem can be solved in
polynomial time.

We create bipartial graph $G$ with node for every line on the input
split into sets: $H$ -- horizontal lines and $V$ -- vertical lines.
If any two lines cover the same point from $\mathcal{C}$, then
we add edge between them.

Of course there will be no edges between nodes inside $H$,
because all of them are pararell and if they share 
one point, they are the same lines. Similar argument for $V$.
So the graph is bipartial.

Now Geometric Set Cover can be solved with Vertex Cover on graph $G$.
Since Vertex Cover (even in weighted setting) 
on bipartial graphs can be solved in polynomial time.

Short note for myself just to remember how to this in polynomial time:

Non-weighted setting - Konig theorem + max matching

Weighted setting - Min cut in graph of $\neg A$ or $\neg B$
(edges directed from $V$ to $H$)

\section{FPT for arbitrary lines}
You can find this is Platypus book.
We will show FPT kernel of size at most $k^2$.

(Maybe we need to reduce lines with one point/points with one line).

For every line if there is more than $k$ points on it,
you have to take it. At the end, if there is more than $k^2$ points,
return NO.
Otherwise there is no more than $k^4$ lines.

In weighted settings among the same lines with different weights
you leave the cheapest one and use the same algorithm.

\section{APX-completeness for arbitrary lines}
We will show a reduction from Vertex Cover problem.
Let's take an instance of the Vertex Cover problem for graph $G$.
We will create a set of $|V(G)|$ pairwise non-pararell lines,
such that no three of them share a common point.

Then for every edge in $(v, w) \in E(G)$
we put a point on crossing of lines for vertices $v$ and $w$.
They are not pararell, so there exists exactly one such point
and any other line do not cover this point (any three of them do not
cross in the same point).

Solution of Geometric Set Cover for this instance would yield
a sound solution of Vertex Cover for graph $G$.
For every point (edge) we need to choose at least one of
lines (vertices) $v$ or $w$ to cover this point.

Vertex Cover for arbitrary graph is APX-complete,
so this problem in also APX-complete.

\section{2-approximation for arbitrary lines}
Vertex Cover has an easy 2-approximation algorithm,
but here very many lines can cross through
the same point, so we can do $d$-approximation,
where $d$ is the biggest number of lines crossing through the same point.
So for set where any 3 lines do not cross in the same point
it yields 2-approximation.

The problematic cases are where through all points
cross at least $k$ points and all lines have at least $k$ points on them.
It can be created by casting $k$-grid in $k$-D space on 2D space.

Greedy algorithm yields $\log |\mathcal{R}|$-approximation,
but I have example for this for bipartial graph and
reduction with taking all lines crossing through some point
(if there are no more than $k$) would solve this case.
So maybe it works.

Unfortunaly I have not done this :(

I can link some papers telling it's hard to do.

\section{Connection with general set cover}
Problem with finite set of lines with more dimensions
is equivalent
to problem in 2D, because we can project
lines on the plane which is not perpendicular
to any plane created by pairs of
(point from $\mathcal{C}$, line from $\mathcal{P}$).

Of course every two lines have at most one common point,
so is every family of sets that have at most one point
in common equivalent to some geometric set cover with lines?

No, because of Desargues's theorem.
Have to write down exactly what configuration is banned.


\chapter{Geometric Set Cover with polygons}
\section{State of the art}

Covering points with weighted discs admits PTAS \cite{li}
and with fat polygons with $\delta$-extensions with unit weights
admits EPTAS \cite{harpeled12}.

Although with thin objects, even if we allow $\delta$-expansion,
the Set Cover with rectangles
is APX-complete (for~$\delta = 1/2$),
it follows from APX-completeness for segments with $\delta$-expansion
in Section \ref{section:segment_apx}.

Covering points with squares is W[1]-hard \cite{marx05}.
It can be proven that assuming $SETH$,
there is no $f(k)\cdot(|\points|+|\sets|)^{k-\epsilon}$ time algorithm
for any computable function $f$ and $\epsilon >0$ that decides if there
are $k$ polygons in $\sets$ that together cover $\points$,
\textit{Theorem 1.9} in \cite{voronoi}.








\chapter{Conclusions}
We do not know FPT for axis-parallel segments without $\delta$-extensions.

\bibliographystyle{apalike}
\bibliography{bibl}

\end{document}


%%% Local Variables:
%%% mode: latex
%%% TeX-master: t
%%% coding: latin-2
%%% End:
