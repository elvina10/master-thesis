
\definecolor{beg_color}{RGB}{255, 40, 40}
\definecolor{seg_color1}{RGB}{40, 40, 255}
\definecolor{seg_color2}{RGB}{40, 150, 40}

\newcommand{\addNode}[4]{
	\node[guard_h, label={left:$h_{#1,j,t_{#2,#3}}^L$}] at (\l#4,0) {};
	\node[node_h, label={left:$h_{#1,j,t_{#2,#3}}$}] at (\x#4,0) {};
	\node[guard_h, label={left:$h_{#1,j,t_{#2,#3}}^R$}] at (\r#4,0) {};
}

\begin{figure}
\hspace*{-1.5cm}
\begin{tikzpicture}[main/.style = {draw, circle}]
\tikzmath{
\step=2;
\eps=0.5;
%genereted by gen_math.py
\x1=\step;
\x2=\x1+\step;
\x3=\x2+\step;
\x4=\x3+\step;
\x5=\x4+\step;
\x6=\x5+\step;
\x7=\x6+\step;
\x8=\x7+\step;
\x9=\x8+\step;
\l1=\x1-\eps;
\r1=\x1+\eps;
\l2=\x2-\eps;
\r2=\x2+\eps;
\l3=\x3-\eps;
\r3=\x3+\eps;
\l4=\x4-\eps;
\r4=\x4+\eps;
\l5=\x5-\eps;
\r5=\x5+\eps;
\l6=\x6-\eps;
\r6=\x6+\eps;
\l7=\x7-\eps;
\r7=\x7+\eps;
\l8=\x8-\eps;
\r8=\x8+\eps;
\l9=\x9-\eps;
\r9=\x9+\eps;
}

\draw [beg_color] (\l1,0) to[out=150,in=30, looseness=200] (1.49,0);
\draw [beg_color] (\l1,0) to[out=40,in=140] (\l2,0);
\draw [beg_color] (\l1,0) to[out=40,in=140] (\l3,0);
\draw [beg_color] (\l1,0) to[out=40,in=140] (\l4,0);


\draw [seg_color1] (\r1,0) to[out=20,in=160] (\l6,0);
\draw [seg_color1] (\r1,0) to[out=20,in=160] (\l8,0);
\draw [seg_color2] (\r2,0) to[out=20,in=160] (\l7,0);
\draw [seg_color2] (\r2,0) to[out=20,in=160] (\l9,0);
\draw [seg_color1] (\r3,0) to[out=20,in=160] (\l6,0);
\draw [seg_color1] (\r3,0) to[out=20,in=160] (\l8,0);
\draw [seg_color2] (\r4,0) to[out=20,in=160] (\l7,0);
\draw [seg_color2] (\r4,0) to[out=20,in=160] (\l9,0);

\addNode{1}{1}{1}{1}
\addNode{1}{1}{2}{2}
\addNode{1}{2}{1}{3}
\addNode{1}{2}{2}{4}
\addNode{2}{1}{1}{6}
\addNode{2}{1}{2}{7}
\addNode{2}{2}{1}{8}
\addNode{2}{2}{2}{9}

\end{tikzpicture} 
\caption{\textbf{Vertices and segments in $HOR$.}}
This is an example for $n=2$ and any $1 \le j \le k$.
Points from $H$ are marked in black, their guards are marked in \textcolor{guards_color}{light blue}.
$t_{i,j}$ is a notation that we use for $\order^{-1}(i,j)$.
Segments are represented as arcs between endpoints.
You can see $\horbeg{j}{t}$ segments in \textcolor{beg_color}{red}.
$\horbeg{j}{1}$ is degenerated to a single point at $h_{1,1,t_{1,1}}^L$.
Segments $\hor{i}{j}{t_{x_1,y}}{t_{x_2,y}}$
are marked in \textcolor{seg_color1}{blue} and \textcolor{seg_color2}{green}.
\textcolor{seg_color1}{Blue} segments connect $t_{x_1,y}$ and $t_{x_2,y}$
such that they share y-coordinate equal to 1,
for~\textcolor{seg_color2}{green} segments it is equal to 2.

\label{fig:segments_def}
\end{figure}
