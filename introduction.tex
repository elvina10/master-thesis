\chapter{Introduction}

Some problems in Computer Science are known to be NP-complete,
meaning that assuming P$\neq$NP there is no polynomial time
algorithm that can solve these problems.
Even so, they still can be amenable to different approaches,
such as approximation or parametrization.

\begin{defi}
In the \textbf{Set Cover} problem we are given a set of elements (universe)
$\points$ and a family of sets $\sets$ that are subsets of the universe $\points$
and sum up to the whole $\points$.
Our task is to find a set $\sol \subseteq \sets$
such that $\bigcup \sol = \points$ and the size of $\sol$ is minimum possible.
\end{defi}

Set Cover is a classical example of an NP-complete problem,
which has been proven
in the literature to be
inapproximable with factor $(1-o(1))\ln n$ asssuming P $\neq$ NP
(which is a stronger result than APX-hardness),
and W[2]-hard with the natural parametrization.
TODO: citation
However restricting the problem to various specialized settings
can lead to more tractable special cases.
In this thesis we take a closer look at the Geometric Set Cover problem
in the plane, where elements to cover are points in the plane
and sets to cover them with are geometric objects.

\paragraph{Approximation}
Over the years there has been a lot of work related to approximation
algorithms for Geometric Set Cover. Notably,
Geometric Set Cover with unweighted unit disks admits a PTAS (see
Corollary 1.1 in \cite{unit_disks}). When we consider the same problem
with weighted unit disks (or unit squares), the problem admits a QPTAS
\cite{settling_apx_hardness}, see also \cite{voronoi_true}.
On the other hand, \cite{rectangles_apx_hard} 
proves that Geometric Set Cover with unweighted axis-parallel rectangles
is APX-hard; they also show similar hardness
for Geometric Set Cover with many other standard geometric objects.

\paragraph{Parametrization}
We consider Geometric Set Cover 
parameterized by the size of solution.
Geometric Set Cover with unit squares was first proven to be W[1]-hard 
in \cite{marx05} (Theorem 5). A later follow-up work \cite{voronoi}
shows that there is an algorithm running in time $n^{\mathcal{O}(\sqrt{k})}$
that solves Geometric Set Cover with unit squares or disks
and that there is no algorithm running in time $f(k) \cdot n^{o(\sqrt{k})}$
for any computable $f$ under the Exponential-Time Hyphothesis,
so this is a tight bound for this problem.

We also consider parametrization of weighted problems.
There does not seem to be a consensus of what parametrization
in the weighted setting is exactly; there
was an attempt to introduce a quite complicated general
framework of weighted parameterized setting in \cite{weighted_framework}.
Kernels for several well-known weighted problems
such as Subset Sum or Knapsack are presented in \cite{kernel_weighted}.
Another work \cite{weighted_flow} considers weighted
parametrization of Weighted Directed Feedback Set and Weighted $st$-Cut.

\paragraph{$\delta$-extension}
In this paper, we focus on Geometric Set Cover with segments with $\delta$-extension.
$\delta$-extension is a problem relaxation method based on the
$\delta$-shrinking model which was introduced in \cite{shrinking_original}
and later used in \cite{shrinking2} and \cite{shrinking1}.

\begin{defi}
\label{definition:delta_extension}
For any $\delta > 0$ and a centre-symmetric convex object $L$ with
centre of symmetry $S = (x_s, y_s)$,
the \textbf{$\delta$-extension} of $L$ is the object $L^{+\delta} =
\{(1 + \delta)\cdot(x - x_s, y - y_s) + (x_s, y_s) : (x, y) \in L\}$.
That is, $L^{+\delta}$ is the image of $L$ under homothety centred
at $S$ with scale $(1+\delta)$.
\end{defi}

TODO: Do some more stuff about extensions

Similar model is used to prove that Geometric Set Cover with fat polygons
relaxed with $\delta$-extension admits EPTAS \cite{harpeled12}.

\section*{Our contribution}
In this paper we make the following contributions.

We show that approximation of uweighted Geometric Set Cover with axis-parallel segments
(even if we relax the problem with  $\frac{1}{2}$-extension) is APX-hard
(Theorem \ref{segment_cover_apx_hard}).

\begin{restatable}{tw}{segmentCoverApxHard}{
\label{segment_cover_apx_hard}
	\textbf{(Axis-parallel segment set cover with $\frac{1}{2}$-extension is APX-hard)}.	
	Unweighted geometric set cover
	with axis-parallel segments in the 2D plane
	(even with $\frac{1}{2}$-extension) is APX-hard.
	That is, assuming $P\neq NP$, there does not exist a PTAS
	for this problem.
}\end{restatable}

This expands the previous result of \cite{rectangles_apx_hard} 
that Geometric Set Cover
with unweighted axis-parallel rectangles being APX-hard.
This also proves that the assumption in \cite{harpeled12}
for EPTAS about polygons being fat is necessary, because
cover with arbitrary polygons with $\delta$-extension is APX-hard.

We also provide two FPT algorithms for parameterized Geometric Set Cover 	
with unweighted segments (Theorem~\ref{segment_cover_fpt})
and weighted segments relaxed with $\delta$-extension
(Theorem~\ref{fpt_weighted_segment}).

\begin{restatable}{tw}{segmentCoverFpt}{
	\label{segment_cover_fpt}
	\textbf{(FPT for segment cover).}
	There exists an algorithm that given a family $\sets$ of
	segments (in any direction),
	a set of points $\points$
	and a parameter $k$,
	runs in time ${k^{\mathcal{O}(k)} (|\points|\cdot|\sets|)^2}$,
	and outputs a solution $\sol \subseteq \sets$
	such that $|\sol| \le k$ and $\sol$ covers all points in~$\points$,
	or determines that such a set $\sol$ does not exist.
}\end{restatable}

\begin{restatable}{tw}{fptWeightedSegment}{
	\label{fpt_weighted_segment}
	\textbf{(FPT for weighted segment cover with $\delta$-extension).}
	There exists an algorithm that given a family $\sets$ of
	$n$ weighted segments (in any direction),
	a set of $m$ points $\points$, and parameters $k$ and $\delta > 0$,
	such that it
	runs in time $f(k, \delta) \cdot (nm)^c$ for some computable function $f$ and a constant $c$ and
	outputs a set $\sol$ such that:
	\begin{itemize}
	\item $\sol \subseteq \sets$,
	\item $|\sol| \le k$,
	\item $\sol^{+\delta}$ covers all points in $\points$,
	\item the weight of $\sol$ is not greater than the weight
	of an optimum solution of size at most $k$
	for this problem without $\delta$-extension
	\end{itemize}
	or determines that there is no set $\sol$ with $|\sol| \le k$
	such that $\sol$ covers all points in $\points$.
}\end{restatable}

On the other hand, we prove that Geometric Set Cover with weighted
axis-parallel segments is W[1]-hard (Theorem~\ref{w1_hard})
and assuming ETH there does not exist algorithm for this problem
that runs in time ${f(k)(|\points| + |\sets|)^{o(\sqrt{k})}}$.
See Figure \ref{tab:weighted_fpt} for a summary of parameterized
results for the weighted setting.

\begin{restatable}{tw}{wOneHard}
\label{w1_hard}
	Consider the problem of covering a set $\points$ of points
	by selecting at most $k$ segments
	from a set of segments $\sets$ 
	with non-negative weights $w : \sets \rightarrow \mathbb{R^+}$
	so that the weight of the cover is minimal.
	Then this problem is W[1]-hard when parameterized by $k$ and
	assuming ETH, there is no algorithm for this
	problem with running time
	$f(k)\cdot(|\points| + |\sets|)^{o(\sqrt{k})}$
	for any computable function $f$.
	Moreover, this holds even if all segments in $\sets$
	are axis-parallel or right-diagonal.
\end{restatable}

Please see Section \ref{section:def:geometric__set_cover}
for exact definitions of axis-parallel and right-diagonal segments.

TODO: Write something about persmissive as a side thingy

Not that the result of theorem \ref{w1_hard} is not tight:
there exists a simple algorithm 
running in time ${\mathcal{O}(f(k)(|\points| + |\sets|)^k)}$.
So the question whether there exists an algorithm
for this problem running in time ${f(k)\cdot(|\points| + |\sets|)^{o(k)}}$
is still open.

\begin{figure}[h]
\begin{center}
\begin{tabular}{ | c | c | c | }
\hline
                & exact     & $\delta$-extension \\ 
\hline                
 axis-parallel   & ? & FPT* \\  
\hline                
 3 directions    & W[1]-hard & FPT* \\  
\hline                
 any direction   & W[1]-hard* & FPT \\
\hline                
\end{tabular}
\caption{Our results for Geometric Set Cover problem with weighted segments 
parameterized by the size of solution.
Results marked with * directly follow from more or less restricted settings.
TODO: Fix unclear wording here}
\label{tab:weighted_fpt}
\end{center}
\end{figure}
