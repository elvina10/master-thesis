{\tikzset{clause/.style={
    prefix after command= {\pgfextra{\tikzset{every
    label/.style={label distance=-0.35cm,rotate=90}}}}
    }
}

\newcommand{\variableSegments}[3]{
	\draw (\xbeg,\y#1) -- (\x6,\y#1) node[pos=0.15, above] {$x_#3 = \false$};
	\draw (\xbeg,\y#2) -- (\x6,\y#2) node[pos=0.15, above] {$x_#3 = \true$};
}

\begin{figure}
\centering
\begin{tikzpicture}
\tikzmath{
\width=5;
\height=1;
\step=1;
\y0=0;
\y1=\height;
\y2=\y1+\height;
\y3=\y2+\height;
\y4=\y3+\height;
\y5=\y4+\height;
\y6=\y5+\height;
\x0=0;
\xbeg=\width;
\x1=\width+3;
\x2=\x1+\step;
\x3=\x2+\step;
\x4=\x3+\step;
\x5=\x4+\step;
\x6=\x5+\step;
}

\filldraw [fill=lime, draw=black] (0,\y0) rectangle (\width, \y1) node[pos=.5] {VARIABLE-gadget$_1$};
\filldraw [fill=lime, draw=black] (0,\y2) rectangle (\width, \y3) node[pos=.5] {VARIABLE-gadget$_2$};
\filldraw [fill=lime, draw=black] (0,\y4) rectangle (\width, \y5) node[pos=.5] {VARIABLE-gadget$_3$};

\variableSegments{0}{1}{1}
\variableSegments{2}{3}{2}
\variableSegments{4}{5}{3}

\filldraw [fill=cyan!60, draw=black] (\x1,\y0) rectangle (\x2, \y6) node[clause, pos=.5, label= {CLAUSE-gadget$_1$}] {};
\filldraw [fill=cyan!60, draw=black] (\x3,\y0) rectangle (\x4, \y6) node[clause, pos=.5, label= {CLAUSE-gadget$_2$}] {};
\filldraw [fill=cyan!60, draw=black] (\x5,\y0) rectangle (\x6, \y6) node[clause, pos=.5, label= {CLAUSE-gadget$_3$}] {};
\end{tikzpicture}
\caption{\textbf{Schema of the whole construction.}}
General layout of VARIABLE-gadgets and CLAUSE-gadgets and how they
interact with each other.
\label{fig:segment_apx_whole}
\end{figure}
