In this chapter we consider geometric set cover problem with weighted segments.
Theorem \ref{w1_hard} proves that this problem is 
W[1]-hard when parametrized by the size of the solution.
We additionally restrict the problem to only use segments
in three directions to achieve a stronger result.
W[1]-hardness is proved by reduction to a grid tiling problem,
which was introduced in \cite{marx_grid_tiling}.

\begin{defi}
	Line is \textbf{right-diagonal} if it is
	described by linear function $y = -x + d$ for any $d \in \mathbb{R}$.
	Segment is \textbf{right-diagonal} if its
	direction is a right-diagonal line.
\end{defi}

\begin{tw}
\label{w1_hard}
	Consider the problem of covering a set $\points$ of points
	by selecting at most $k$ segments
	from a set of segments $\sets$ 
	with non-negative weights $w : \sets \rightarrow \mathbb{R^+}$
	so that the weight of the cover is minimal.
	Then this problem is W[1]-hard parametrized by $k$ and
	assuming ETH, there is no algorithm for this
	problem with running time
	$f(k)\cdot(|\points| + |\sets|)^{o(\sqrt{k})}$
	for any computable function $f$.
	Moreover, this holds even if all segments in $\sets$
	are axis-parallel or right-diagonal.
\end{tw}

Theorem \ref{w1_hard} is also true for less
restricted problem where segments have any direction.
We prove more tight setting in this section.

In order to prove Theorem \ref{w1_hard}
we will show reduction from a W[1]-hard problem.
We introduce the grid tiling problem, which is proven
to be W[1]-complete in literature.

\newcommand{\pow}{\mathsf{Pow}}

\begin{defi}
We define \textbf{powerset} of some set $A$, denoted as $\pow(A)$,
as a set of all subsets of $A$, ie. $\pow(A) = \{B : B \subseteq A\}$.
\end{defi}

\begin{defi}
In the \textbf{grid tiling} problem we are given integers $n$ and $k$,
and a function
$f : \{1 \ldots k\} \times \{1 \ldots k\} \rightarrow \pow(\{1 \ldots n\} \times \{1 \ldots n\})$
specifying the set of allowed tiles for each cell of a $k \times k$ grid.
The task is to find functions
$x,y : \{1 \ldots k\} \rightarrow \{1 \ldots n\}$
that assign numbers from $\{1 \ldots n\}$
to respectively columns and rows of the grid,
so that $(x(i), y(j)) \in f(i, j)$ for all valid $i$ and $j$,
or conclude that such an assignment does not exist.
\end{defi}

In short, in grid tiling problem you need to assign numbers
to rows and columns in such a way,
that for every pair of a row and a column,
the pair of numbers assigned
to the row and column 
belongs to the allowed set corresponding to the intersection
of the row and column in question.
The next theorem describes the complexity of this problem,
which is W[1]-hard when parametrized by the size of the grid.

\definecolor{alternative_sol}{RGB}{48, 48, 255}
\definecolor{bad_sol}{RGB}{255, 48, 48}

\begin{figure}[h]
\begin{center}
\begin{tabular}{ c|c|c|c|c| } 
          & $x(1)=3$ & $x(2)=1$ & $x(3)=3$ & $x(4) = 7$\\ 
 \hline
 $y(4)=1$
	& \makecell{$\textcolor{alternative_sol}{(2,1)};(2,2);$\\$\textbf{(3,1)};(3,9)$}
	& $\textbf{(1,1)}; (3,1)$
	& $\textbf{(3,1)}; (7,2)$
	& $\textcolor{bad_sol}{(2,1)}; \textbf{(7,1)}$\\ 
 \hline
 $y(3)=1$
	& \makecell{$\textcolor{alternative_sol}{(2,1)};\textbf{(3,1)};$\\$(4,2);(8,2)$}
	& $\textbf{(1,1)}; (1,3)$
	& $\textbf{(3,1)}; (4,3)$
	& $\textcolor{bad_sol}{\textbf{(2,2)}}; \textbf{(7,1)}$\\ 
 \hline
 $y(2)=6$
	& $\textcolor{alternative_sol}{(2,6)};\textbf{(3,6)}$
	& \makecell{$(1,2); \textbf{(1,6)};$\\$(2,6)$}
	& $(2,6); \textbf{(3,6)}$
	& $\textcolor{bad_sol}{(2,6)};\textbf{(7,6)}$\\ 
 \hline
 $y(1)=4$
	& \makecell{$\textcolor{alternative_sol}{(2,4)};(2,6);$\\$\textbf{(3,4)};\textcolor{bad_sol}{(3,9)}$}
	& $\textbf{(1,4)}; \textcolor{bad_sol}{(1,9)}$
	& $\textbf{(3,4)}; \textcolor{bad_sol}{(3,9)}$
	& $\textcolor{bad_sol}{(2,9)}; \textbf{(7,4)}$\\ 
 \hline
\end{tabular}
\end{center}
\caption{\textbf{Example of a grid tiling instance with solution.}}
In the first row and column of the table you can see the solution $x$ and $y$.
The tiles used in this solution are marked in \textbf{bold}.
Change marked in \textcolor{alternative_sol}{blue} $x(1)=2$ would be still a corect solution
of this instance without any other changes.
Change marked in \textcolor{bad_sol}{red} $y(1)=9$ and $x(4)=2$ is not a correct solution.
Even though the first row is correct,
tile with coordinates (3,4) should have value (2,1), not (2,2).
\label{fig:grid_tiling_exmample}
\end{figure}

\begin{tw}
\label{grid_tiling_w1_hard}
\textbf{\cite{marx_grid_tiling}}
Grid tiling is W[1]-hard parametrized by $k$ and
assuming ETH, there is no $f(k)\cdot n^{o(\sqrt{k})}$-time
algorithm solving the grid tiling problem
for any computable function $f$.
\end{tw}

The reminder of this section is proving Theorem \ref{w1_hard}
by reduction of grid tiling problem to geometric set cover.
That proves the W[1]-hardness of geometric set cover,
because if we could solve it with an FPT algorithm,
then we could also solve grid tiling problem
(which we reduced to geometric set cover).
Therefore geometric set cover with setting
described in Theorem \ref{w1_hard}
is at least as hard as the grid tiling problem.

We start with an instance of the grid tiling problem $(n, k, f)$.
The instance consists of:
\begin{itemize}
\item size of the grid $k$,
\item number of colors $n$,
\item function of allowed tiles
$f : \{1, \ldots, k\} \times \{1, \ldots, k\} \rightarrow \pow(\{1, \ldots, n\} \times \{1, \ldots, n\})$.
\end{itemize}

\paragraph{Construction.}
\newcommand{\order}{\mathsf{order}}
\newcommand{\matchv}{\mathsf{match}_v}
\newcommand{\matchh}{\mathsf{match}_h}
\newcommand{\instanceSetCover}{(\points, \sets, w, 3k^2+2k)}
We construct an instance $\instanceSetCover$ of geometric set cover as follows.

First let us choose any bijection
$\order : \{1, \ldots, n^2\} \rightarrow \{1, \ldots, n\} \times \{1, \ldots, n\}$.


Define $\matchv(i, j)$ and $\matchh(i, j)$
as boolean functions denoting whether two points share x or y coordinate:
$$\matchv(i, j) \text{ is } \texttt{true} \iff
\order(i) \text{ and } \order(j) \text{ have the same x coordinate,}$$
$$\matchh(i, j) \text{ is } \texttt{true} \iff
\order(i) \text{ and } \order(j) \text{ have the same y coordinate.}$$


\subparagraph{Points.}

For $1 \le i,j \le k$ and $1 \le t \le n^2$ define points:
	$$h_{i, j, t} := (i \cdot (n^2+1) + t, j \cdot (n^2+1)),$$
	$$v_{i, j, t} := (i \cdot (n^2+1), j \cdot (n^2+1) + t).$$
	
Let us define sets $H$ and $V$ as:
$$H := \{h_{i, j, t} : 1 \le i, j, \le k, 1 \le t \le n^2\},$$
$$V := \{v_{i, j, t} : 1 \le i, j, \le k, 1 \le t \le n^2\}.$$
	
Let $\epsilon = \frac{1}{2k^2}$.
For a point $p = (x, y)$ we define points:
$$p^{L} := (x - \epsilon, y),$$
$$p^{R} := (x + \epsilon, y),$$
$$p^{U} := (x, y + \epsilon),$$
$$p^{D} := (x, y - \epsilon).$$

Then we define the point set as follows:
$$\points := H \cup \{p^L : p \in H\} \cup \{p^R : p \in H\}
\cup V \cup \{p^U : p \in V\} \cup \{p^D : p \in V\}.$$

\begin{defi}
	\label{guard_def}
	For every point $p \in H$, we name point $p^L$ its \textbf{left guard}
	and point $p^R$ its \textbf{right guard}.
	
	Similarily for every points $p \in V$, we name point $p^D$ its \textbf{lower guard}
	and point $p^U$ its \textbf{upper guard}.
\end{defi}

\subparagraph{Segments.}
\newcommand{\hor}[4]{\mathsf{hor}_{#1,#2,#3,#4}}
\newcommand{\ver}[4]{\mathsf{ver}_{#1,#2,#3,#4}}
\newcommand{\horbeg}[2]{\mathsf{horBeg}_{#1,#2}}
\newcommand{\verbeg}[2]{\mathsf{verBeg}_{#1,#2}}
\newcommand{\horend}[2]{\mathsf{horEnd}_{#1,#2}}
\newcommand{\verend}[2]{\mathsf{verEnd}_{#1,#2}}

For $1 \le i,j \le k$ and $1 \le t_1, t_2 \le n^2$ define segments:
\begin{eqnarray*}
\hor{i}{j}{t_1}{t_2} & := & (h^R_{i,j,t_1}, h^L_{i+1, j, t_2}), \\
\ver{i}{j}{t_1}{t_2} & := & (v^U_{i,j,t_1}, v^D_{i, j+1, t_2}), \\
\horbeg{i}{t} & := & (h^L_{1, i, 1}, h^L_{1, i, t}), \\
\horend{i}{t} & := & (h^R_{k, i, t}, h^R_{k, i, n^2}), \\
\verbeg{i}{t} & := & (v^D_{i, 1, 1}, v^D_{i, 1, t}), \\
\verend{i}{t} & := & (v^U_{i, k, t}, v^U_{i, k, n^2}). \\
\end{eqnarray*}

\newcommand{\allhor}{\mathsf{HOR}}
\newcommand{\allver}{\mathsf{VER}}
\newcommand{\alldiag}{\mathsf{DIAG}}

Next, we define sets of vertical and horizontal segments:
\begin{eqnarray*}
\allhor &:= &\{\hor{i}{j}{t_1}{t_2} : 1 \le i < k, 1 \le j \le k,
1 \le t_1, t_2 \le n^2, \matchh(t_1, t_2) \text{ holds}\} \\
&\cup &\{\horbeg{i}{t} : 1 \le i \le k, 1 \le t \le n^2\}
\\
&\cup &\{\horend{i}{t} : 1 \le i \le k, 1 \le t \le n^2\},
\end{eqnarray*}
\begin{eqnarray*}
\allver &:= &\{\ver{i}{j}{t_1}{t_2} : 1 \le i \le k, 1 \le j < k,
1 \le t_1, t_2 \le n^2, \matchv(t_1, t_2) \text{ holds}\} \\
&\cup &\{\verbeg{i}{t} : 1 \le i \le k, 1 \le t \le n^2\}
\\
&\cup &\{\verend{i}{t} : 1 \le i \le k, 1 \le t \le n^2\}.
\end{eqnarray*}
You can see an example of these segments in Figure \ref{fig:segments_def}.

Finally, we also define a set of right-diagonal segments:
$$\alldiag := \{ (h_{i, j, t}, v_{i, j, t}) :
	1 \le i, j \le k, 1 \le t \le n^2, \order(t) \in f(i, j)\}.$$
You can see an example of such segments in Figure \ref{fig:diag_def}.
	
\definecolor{guards_color}{RGB}{80, 120, 255}

{\tikzset{guard_h/.style={
    circle, draw=guards_color, fill, fill=guards_color, minimum size=2pt,inner sep=0pt, outer sep=0pt,
    prefix after command= {\pgfextra{\tikzset{every
    label/.style={label distance=0.1cm,rotate=90,text=guards_color,font=\footnotesize}}}}
    }
}
{\tikzset{node_h/.style={
    circle, draw=black, fill, fill=black, minimum size=4pt,inner sep=0pt, outer sep=0pt,
    prefix after command= {\pgfextra{\tikzset{every
    label/.style={label distance=0.1cm,rotate=90,text=black}}}}
    }
}
{\tikzset{guard_v/.style={
    circle, draw=guards_color, fill, fill=guards_color, minimum size=2pt,inner sep=0pt, outer sep=0pt,
    prefix after command= {\pgfextra{\tikzset{every
    label/.style={label distance=0.1cm,text=guards_color,font=\footnotesize}}}}
    }
}
{\tikzset{node_v/.style={
    circle, draw=black, fill, fill=black, minimum size=4pt,inner sep=0pt, outer sep=0pt,
    prefix after command= {\pgfextra{\tikzset{every
    label/.style={label distance=0.1cm,text=black}}}}
    }
}

\newcommand{\addNodeV}[2]{
	\node[guard_v, label={left:$v_{i,j,#2}^D$}] at (0, \l#1) {};
	\node[node_v, label={left:$v_{i,j,#2}$}] at (0,\x#1) {};
	\node[guard_v, label={left:$v_{i,j,#2}^U$}] at (0, \r#1) {};
}

\newcommand{\addNodeH}[2]{
	\node[guard_h, label={left:$h_{i,j,#2}^L$}] at (\l#1,0) {};
	\node[node_h, label={left:$h_{i,j,#2}$}] at (\x#1,0) {};
	\node[guard_h, label={left:$h_{i,j,#2}^R$}] at (\r#1,0) {};
}

\begin{figure}
\begin{center}
\begin{tikzpicture}[main/.style = {draw, circle}]
\tikzmath{
	\step=2;
	\eps=0.5;
	\x1=\step;
	\x2=\x1+\step;
	\x3=\x2+\step;
	\x4=\x3+\step;
	\x5=\x4+\step;
	\x6=\x5+\step;
	\l1=\x1-\eps;
	\r1=\x1+\eps;
	\l2=\x2-\eps;
	\r2=\x2+\eps;
	\l3=\x3-\eps;
	\r3=\x3+\eps;
	\l4=\x4-\eps;
	\r4=\x4+\eps;
	\l5=\x5-\eps;
	\r5=\x5+\eps;
	\l6=\x6-\eps;
	\r6=\x6+\eps;
}

\addNodeV{1}{1}
\addNodeV{2}{2}
\addNodeV{3}{3}
\filldraw[black] (0,\x4) circle (0pt) node[anchor=east] {$\ldots$};
\addNodeV{5}{n^2-1}
\addNodeV{6}{n^2}

\addNodeH{1}{1}
\addNodeH{2}{2}
\addNodeH{3}{3}
\filldraw[black] (\x4,0) circle (0pt) node[anchor=north] {$\ldots$};
\addNodeH{5}{n^2-1}
\addNodeH{6}{n^2}
\draw (0,\x1) -- (\x1,0) node[midway, above] {$\delta$};
\draw (0,\x2) -- (\x2,0) node[midway, above] {$\delta$};
\draw (0,\x3) -- (\x3,0) node[midway, above] {$\delta$};
\draw (0,\x5) -- (\x5,0) node[midway, above] {$\delta$};
\draw (0,\x6) -- (\x6,0) node[midway, above] {$\delta$};
\end{tikzpicture} 
\end{center}
\caption{\textbf{Vertices and segments in $\alldiag$.}}
Nodes from $H$ and $V$ are marked in black, their guards are marked in blue.
You can see segments in $\alldiag$ with their weights (equal to $\delta$).
This is an example for any $1 \le i,j \le k$.
\label{fig:diag_def}
\end{figure}


Every segment in $\alldiag$
connects points $(i(n^2+1) + t, j \cdot (n^2+1))$
and $(i \cdot (n^2+1), j(n^2+1) + t)$
for some $1 \le i,j \le k, 1 \le t \le n^2$.
The line on which it lies can be described by $y=-x+(t+(i+j)(n^2+1))$
and these segments are in fact right-diagonal.

The constructed segment set is:

$$\sets := \allhor \cup \allver \cup \alldiag.$$

The weight of each segment in $\allhor \cup \allver$
is equal to the segment,
while every segment in $\alldiag$ has weight
$\delta := \frac{1}{4k^4}$.


\definecolor{beg_color}{RGB}{255, 40, 40}
\definecolor{seg_color1}{RGB}{40, 40, 255}
\definecolor{seg_color2}{RGB}{40, 150, 40}

\newcommand{\addNode}[4]{
	\node[guard_h, label={left:$h_{#1,j,t_{#2,#3}}^L$}] at (\l#4,0) {};
	\node[node_h, label={left:$h_{#1,j,t_{#2,#3}}$}] at (\x#4,0) {};
	\node[guard_h, label={left:$h_{#1,j,t_{#2,#3}}^R$}] at (\r#4,0) {};
}

\begin{figure}
\begin{tikzpicture}[main/.style = {draw, circle}]
\tikzmath{
\step=2;
\eps=0.5;
%genereted by gen_math.py
\x1=\step;
\x2=\x1+\step;
\x3=\x2+\step;
\x4=\x3+\step;
\x5=\x4+\step;
\x6=\x5+\step;
\x7=\x6+\step;
\x8=\x7+\step;
\x9=\x8+\step;
\l1=\x1-\eps;
\r1=\x1+\eps;
\l2=\x2-\eps;
\r2=\x2+\eps;
\l3=\x3-\eps;
\r3=\x3+\eps;
\l4=\x4-\eps;
\r4=\x4+\eps;
\l5=\x5-\eps;
\r5=\x5+\eps;
\l6=\x6-\eps;
\r6=\x6+\eps;
\l7=\x7-\eps;
\r7=\x7+\eps;
\l8=\x8-\eps;
\r8=\x8+\eps;
\l9=\x9-\eps;
\r9=\x9+\eps;
}

\addNode{1}{1}{1}{1}
\addNode{1}{1}{2}{2}
\addNode{1}{2}{1}{3}
\addNode{1}{2}{2}{4}
\addNode{2}{1}{1}{6}
\addNode{2}{1}{2}{7}
\addNode{2}{2}{1}{8}
\addNode{2}{2}{2}{9}

\draw [beg_color] (\l1,0) to[out=40,in=140] (\l2,0);
\draw [beg_color] (\l1,0) to[out=40,in=140] (\l3,0);
\draw [beg_color] (\l1,0) to[out=40,in=140] (\l4,0);


\draw [seg_color1] (\r1,0) to[out=20,in=160] (\l6,0);
\draw [seg_color1] (\r1,0) to[out=20,in=160] (\l8,0);
\draw [seg_color2] (\r2,0) to[out=20,in=160] (\l7,0);
\draw [seg_color2] (\r2,0) to[out=20,in=160] (\l9,0);
\draw [seg_color1] (\r3,0) to[out=20,in=160] (\l6,0);
\draw [seg_color1] (\r3,0) to[out=20,in=160] (\l8,0);
\draw [seg_color2] (\r4,0) to[out=20,in=160] (\l7,0);
\draw [seg_color2] (\r4,0) to[out=20,in=160] (\l9,0);

\end{tikzpicture} 
\caption{\textbf{Vertices and segments in $HOR$.}}
Nodes from $H$ are marked in black, their guards are marked in blue.
You can see $\horbeg_{i}$ segments in \textcolor{beg_color}{red}
and segments $\hor_{i}{j}{t_{x,y_2}}{t_{x,y_2}}$
in \textcolor{seg_color1}{blue} and \textcolor{seg_color2}{green}.
\textcolor{seg_color1}{Blue} segments connect $t_x$
such that they share y-coordinate equal to 1,
for \textcolor{seg_color2}{green} it is equal to 2.
This is an example for any $1 \le j \le k$.

\label{fig:segments_def}
\end{figure}

\begin{equation}
w(s) =
	\begin{cases*}
	  length(s) 			& if $s \in \allhor \cup \allver$ \\
	  \delta        & if $s \in \alldiag$
	\end{cases*}
\end{equation}

\newcommand{\solWeight}{2k^2(n^2+1) - 4k^2\epsilon -4k(1-\epsilon) +k^2\delta }

Now, we prove that the constructed instance of geometric set cover
with weighted segments is indeed a correct and sound reduction
of the grid tiling problem. Lemma \ref{set_cover_solution_exists}
proves that if the solution of the grid tiling instance exists,
then there exists a solution with bounded size and weight
of constructed geometric set cover instance.

Then Lemma \ref{grid_tiling_exists} proves that if the solution
of the geometric set cover instance with bounded weight exists,
then there exists a solution to the original grid tiling instance.

\begin{lemma}
\label{set_cover_solution_exists}
	If there exists a solution of the grid tiling instance $(f_{i,j})$,
	then there exists a solution for the instance $\instanceSetCover$
	of geometric set cover with weight $\solWeight$.
\end{lemma}

\begin{proof}
Suppose there exists a solution $x,y$ to the grid tiling problem.
	
We define the proposed solution $\sol \subset \sets$
in three parts $D \subset \alldiag$, $A \subset \allhor$ and $B \subset \allver$:
\begin{eqnarray*}
	D & := & \{(v_{i, j, t}, h_{i, j, t}) : 1 \le i, j \le k, t = \order^{-1}(x(i), y(j))\}, \\
	A & := & \{\horbeg{i}{\order^{-1}(x(1), y(i))} : 1 \le i \le k\} \\
	& \cup & \{\horend{i}{\order^{-1}(x(k), y(i))} : 1 \le i \le k\} \\
	& \cup & \{\hor{i}{j}{\order^{-1}(x(i), y(j))}{\order^{-1}(x(i+1), y(j))} : 1 \le i < k, 1 \le j \le k\}, \\
	B & := & \{\verbeg{i}{\order^{-1}(x(i), y(1))} : 1 \le i \le k\} \\
	& \cup & \{\verend{i}{\order^{-1}(x(i), y(k))} : 1 \le i \le k\} \\
	& \cup & \{\ver{i}{j}{\order^{-1}(x(i), y(j))}{\order^{-1}(x(i), y(j+1))} : 1 \le i \le k, 1 \le j < k\},
\end{eqnarray*}
	$$\sol := D \cup A \cup B.$$

Since $\points = H \cup V$, we show that this covers the whole set $H$,
proof for $V$ is analogous.

Take any $1 \le j \le k$ and define $t_{i} := \order^{-1}(x(i), y(j)$.
There first two segments in $A$ are
$\horbeg{j}{t_1} = (h_{1,j,1}^L, h_{1, j, t_1}^L)$ and
$\hor{1}{j}{t_1}{t_2} = (h_{1,j, t_1}^R, h_{2,j,t_2}^L)$.
Therefore points $h_{1,j,x}, h_{1,j,x}^L$ and $h_{1,j,x}^R$
for all $1 \le x \le n^2$ ale covered by $\horbeg{j}{t_1}$ and $\hor{1}{j}{t_1}{t_2}$,
excluding point $h_{1,j,t_1}$.

Analogically for $2 \le i \le k-1$
for two consecutive segments $\hor{i-1}{j}{t_{i-1}}{t_i}$
and $\hor{i}{j}{t_i}{t_{i+1}}$ we prove that 
all points $h_{i,j,x}, h_{i,j,x}^L$ and $h_{i,j,x}^R$
for all $1 \le x \le n^2$ ale covered by these segments
excluding point $h_{i,j,t_i}$.

Finally $\hor{k-1}{j}{t_{k-1}}{t_k}$ and $\horend{j}{t_k}$
cover all points $h_{k,j,x}, h_{k,j,x}^L$ and $h_{k,j,x}^R$
for all $1 \le x \le n^2$ excluding point $h_{k,j,t_k}$.

$D$ covers all points $h_{i,j, t_i}$ and $v_{i,j, t_i}$, therefore
all points are covered.

Size of this proposed solution is:
$$|\sol| = |D| + |A| + |B| = k^2 + (k+1)k + (k+1)k = 3k^2+2k.$$

Then, we need to compute the total weight of solution $\sol$.
First, we compute the sum of weights of segments in $A$.
Fix $1 \le j \le k$, all points $h_{i,j,t}$, $h_{i,j,t}^L$ and $h_{i,j,t}^R$
for every $1 \le i \le k$ and $1 \le t \le n^2$ are covered by $A$
excluding points $h_{i,j,\order^{-1}(x(i),y(j))}$ for each $1 \le i \le k$.
Every such point leaves a gap of length $2\epsilon$ between
$h_{i,j,\order^{-1}(x(i),y(j))}^L$ and $h_{i,j,\order^{-1}(x(i),y(j))}^R$.
Therefore, the total weight of segments in $A$
corresponding to the chosen value of $j$ equals the length of the segment
$(h_{i,1,1}^L, h_{i,k,n^2}^R)$
minus $2\epsilon k$, which is $k(n^2+1) -2k\epsilon -2(1-\epsilon)$.
We need to multiply that by $k$, as we consider all possible values of $j$.

Calculation for vertical segments is analogous and has the same result.
Every segment in $D$ has weight $\delta$, therefore the sum of all weights
is equal to:

$$2k(k(n^2+1) -2k\epsilon -2(1-\epsilon)) + k^2\delta= \solWeight$$
\end{proof}


\begin{claim}
\label{guards}
In any solution of the instance $\instanceSetCover$:
\begin{itemize}
\item left and right guards of points in $H$
(points in $\{p^L : p \in H\} \cup \{p^R : p \in H\}$)
have to be covered with segments from $\allhor$,
\item lower and upper guard of points in $V$
(points in $\{p^D : p \in V\} \cup \{p^U : p \in V\}$)
have to be covered with segments from $\allver$.
\end{itemize}
\end{claim}

\begin{proof}
We prove the claim for the points from $H$
as the proof for points from $V$ is analogous.

Every segment in $\allver$ is vertical and 
has x-coordinate equal to $i(n^2+1)$ for some $1\le i \le k$,
so they all have different x-coordinate
than any left or right guard of points in $H$.

Every point $x$, which is a left or right guard of points in $H$
have $kn^2$ segments from $\alldiag$ that intersect with the horizontal
line that goes through $x$. All of these segments intersect with
this line in points from set $H$, therefore none of them
cover any of the guards.

Therefore none of the segments from $\allver$ or $\alldiag$ cover any
of the guards of the points in $H$.
\end{proof}

Now we present a few additional properties of the constructed instance
of the geometric set cover that help us to prove
the Lemma \ref{grid_tiling_exists}.

\begin{claim}
\label{one_diag_in_square}
For any $1 \le i, j \le n$
and any solution of the instance $\instanceSetCover$
all but at most one points $h_{i, j, t_1}, h_{i, j, t_2}$
($v_{i, j, t_1}, v_{i, j, t_2}$)
for $1 \le t_1 < t_2 \le n^2$
must be
covered with segments from $\allhor$ ($\allver$).
\end{claim}

\begin{proof}
We prove the claim for horizontal segments,
as the proof for vertical segments is analoguous.

Assume point $h_{i, j, t_1}$ is not covered with
segments from $\allhor$.
Point $h^R_{i, j, t_1}$ has to be covered with $\allhor$
by Claim $\ref{guards}$.
Every segment in $\allhor$ covering $h^R_{i, j, t_1}$,
but not $h_{i,j,t_1}$ covers also $h_{i, j, t_2}$.
\end{proof}

\begin{lemma}
\label{vertical_horizontal_sum}
For every solution of the instance $\instanceSetCover$,
the sum of weights of segments chosen
from sets $\allhor$ and $\allver$ at least
$2k^2(n^2+1) -4k^2\epsilon -4k(1-\epsilon)$.
\end{lemma}

\begin{proof}
We prove the lemma for vertical lines,
as the proof for horizontal segments is analogous.

We know that for all $1 \le i,j \le k$
there exists at most one $t_x$ such that
$v_{i,j,t_x}$ is not covered by
segments from $\allver$ (Claim \ref{one_diag_in_square}),

Let us fix $1 \le i \le k$.

We provide the lower bound for the sum of length
of vertical segments between points $v_{i,j_1,t_1}^L$ and $v_{i,j_2,t_2}^R$
for any $1\le j_1, j_2 \le k$ and $1 \le t_1, t_2 \le n^2$.
This bound is the same for each $i$ and is the same
for horizontal lines, thus we need to multiply such bound by $2k$.

\begin{enumerate}[label={(\arabic*)}]
\item The total length between $v^D_{i, 1, 1}$ and $v^U_{i, k, n^2}$ is:
$$(k(n^2+1) + n^2 +\epsilon) - ((n^2+1)+1 -\epsilon) = k(n^2+1) - 2(1 - \epsilon).$$

\item For every $1 \le j \le k$ there exists at most one $1 \le t \le n^2$
such that $v_{i,j,t}$ is not covered by segments from $\allver$
(Claim \ref{one_diag_in_square}).
Its guards (see Definition \ref{guard_def}) $v^U_{i,j,t}$ and $v^D_{i,j,t}$
have to be covered in $\allver$ (Claim \ref{guards}).
Therefore at most $k$ spaces of length $2\epsilon$ can be left
not covered by segments from $\allver$ between $v_{i,1,1}^D$ and $v_{i,k,n^2}^U$.

\end{enumerate}
The sum of these lower bounds for vertical and horizontal lines is:
$$2k(k(n^2+1) -2k\epsilon -2(1-\epsilon)) = 2k^2(n^2+1) -4k^2\epsilon -4k(1-\epsilon)$$
\end{proof}

Let us name the bound from the previous lemma
as $W_{hv} := 2k^2(n^2+1) -4k^2\epsilon -4k(1-\epsilon)$
for the further reference.

\begin{lemma}
\label{diag_correct}
Let $\sol$ be a solution 
of a constructed instance $\instanceSetCover$
with weight at most  $\solWeight$.
Then for every $1 \le i,j \le k$
there exists such $1 \le t \le n^2$ that:
\begin{enumerate}[label={(\arabic*)}]
\item $v_{i,j,t}, h_{i,j,t}$ are not covered by segments from $\allver$ or $\allhor$;
\item segment $(v_{i,j,t}, h_{i,j,t})$ is in solution $\sol$;
\item $\order(t) \in f(i,j)$ that is, $\order(t)$ is an allowed tile for $(i,j)$;
\item for every $1 \le s\le n^2$, $s \neq t$, $v_{i,j,s}$ is covered in $\allver$;
\item for every $1 \le s\le n^2$, $s \neq t$, $h_{i,j,s}$ is covered in $\allhor$.
\end{enumerate}
\end{lemma}

\begin{proof}
At most one of points $\{h_{i,j,t_x} : 1 \le t_x \le n^2\}$
and one of points $\{v_{i,j,t_y} : 1 \le t_y \le n^2\}$
is covered with $\alldiag$
(Claim \ref{one_diag_in_square}).
	
Moreover exactly one such point $h_{i,j,t_x}$ and one such point $v_{i,j,t_y}$
is covered with $\alldiag$,
because if none of them were covered, then the solution would have to
have weight at least $W_{hv} + 2\epsilon$, which is more than $\solWeight$.

We observe that points $h_{i,j,t_x}$ and $v_{i,j,t_y}$
have to be covered with the same segment from $\alldiag$.
Indeed we need to use at least $k^2$ of them to use
exactly one DIAG segment for every pair of $1 \le i,j \le k$,
if we used 2 segments from $\alldiag$
for one pair $(i,j)$,
then we would have used $W_{hv} + k^2\delta + \delta$,
which if more than $\solWeight$.
Since points $h_{i,j,t_x}$ and $v_{i,j,t_y}$ are covered by
a single segment from $\alldiag$, we have $t_x = t_y$.

Therefore $t_x = t_y$
and $\order(t_x)$ is an allowed tile for $(i,j)$
because the corresponding segment is in $\alldiag$.
\end{proof}

\newcommand{\diagonal}{\mathsf{diagonal}}

We refer to the function of the sequence $t_x$ from Lemma \ref{diag_correct}
as $\diagonal : \{1 \ldots k\} \times \{1 \ldots k\} \rightarrow \{1 \ldots n^2\}$.

\begin{lemma}
\label{vertical_horizontal_synchronized}
For any solution $\sol$
of a constructed instance $\instanceSetCover$
with weight at most $\solWeight$:
\begin{enumerate}
\item 
for any $1 \le i < k, 1 \le j \le k$,
$\matchh(\diagonal(i, j),\diagonal(i+1, j))$ is \true;
\item 
for any $1 \le i \le k, 1 \le j < k$,
$\matchv(\diagonal(i, j),\diagonal(i, j+1))$ is \true.
\end{enumerate}
\end{lemma}

\begin{proof}
We prove (1) by contradiction, the proof of (2) is analogous.

Let us take any $1 \le i < k, 1 \le j \le k$
and name $t_1 = \diagonal(i, j)$ and $t_2 = \diagonal(i+1, j)$.
We also assume that $\matchh(t_1,t_2)$ is \false,
which is equivalent to the fact that
segment $(h_{i,j,t_1}^R, h{i+1,j,t_2}^L)$
is not in set $\allhor$.

Therefore $h_{i,j,t_1}$ and $h_{i+1,j,t_2}$
are not covered by segments from $\allhor$ (Lemma \ref{diag_correct}),
while $h^R_{i,j,t_1}$ and $h^L_{i+1,j,t_2}$
have to be covered by segments from $\allhor$ (Claim \ref{guards}).

Every segment from $\allhor$ starts at $h^R_{x,y,z_1}$
segment and ends at $h^L_{x+1,y,z_2}$ segment for some
$1 \le x \le k$,$1 \le y < k$ and $1 \le z_1, z_2 \le n^2$.
All of the points between $h^R_{i,j,t_1}$ and $h^L_{i+1,j,t_2}$
are covered by segments in $\allhor$ 
and there is no segment $(h^R_{i,j,t_1}, h^L_{i+1,j,t_2})$ in $\allhor$.
Hence, there are at least two different segments covering them.
One of them must begin
at $h^R_{i,j,t_1}$ and end at $h^L_{i+1,j,z_2}$
and there must be other one that begins at $h^R_{i,j,z_1}$
and ends at $h^L_{i+1,j,t_2}$
for some $1 \le z_1, z_2 \le n^2$.

Thus, the space between $h^R_{i,j,z_1}$ and $h^L_{i,j+1,z_2}$
would be covered twice and is longer than $\epsilon$.
By Lemma \ref{vertical_horizontal_sum},
the lower bound for weight of such a solution is $W_{hv} + \epsilon$
which is more than $\solWeight$.

Therefore $h^R_{i,j,t_1}$ and $h^L_{i+1,j,t_2}$ must be covered
by one segment from $\allhor$,
$(h^R_{i,j,t_1}, h^L_{i+1,j,t_2})$ is a segment in $\allhor$
and $\matchh(t_1,t_2)$ is $\true$.
\end{proof}


\begin{lemma}
\label{grid_tiling_exists}
	If there exists solution of instance $\instanceSetCover$
	with weight at most $\solWeight$,
	then there exists a solution for the grid tiling instance $(f_{i,j})$.
\end{lemma}

\begin{proof}
Take $\diagonal$ function from Lemma \ref{diag_correct}.

To define the $x$ funtion 
for every $1 \le i \le k$ set $x(i) := x_i$
where $(x_i, a) = \order(v_{i,1})$.
Similarily, to define the $y$ function,
for every $1 \le i \le k$ set $y(i) := y_i$
where $(b, y_i) = \order(h_{1,i})$

To prove that it is a correct solution for grid tiling,
we need to prove that 
for every $1 \le i,j \le k$ $(x(i), y(j))$ is in
allowed tiles set $f(i,j)$.

Let us take any $1 \le i,j \le k$.
By Lemma \ref{vertical_horizontal_synchronized}
and simple induction,
we know that $\matchh(\diagonal(1, j),\diagonal(i, j))$ and
$\matchv(\diagonal(i, 1),\diagonal(i, j))$ are $\true$.
Therefore $\order(\diagonal(i,j)) = (x(i), y(i))$.
By Lemma \ref{diag_correct} we know that 
$\order(\diagonal(i,j))$ is in $f(i,j)$.
Therefore 
$(x(i), y(i))$
is in $f(i,j)$.
\end{proof}

\begin{proof}[Proof of Theorem \ref{w1_hard}]
Follows from Lemmas \ref{set_cover_solution_exists} and \ref{grid_tiling_exists}.
\end{proof}
