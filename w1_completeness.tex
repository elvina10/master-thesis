\section{W[1]-completeness for weighted segments in 3 directions}

\begin{tw}
\label{w1_hard}
	\textbf{Weighted geometric set cover with segments in 3 directions is W[1]-hard}.
	Consider the problem of covering a set $\points$ of points
	by selecting $k$ axis-parallel or right-diagonal weighted segments
	from a set $\sets$ 
	with weights $w : \sets \rightarrow \mathbb{R}$
	with minimal weight.
	Assuming ETH, there is no algorithm for this
	problem with running time
	$f(k)\cdot(|\points| + |\sets|)^{o(\sqrt{k})}$
	for any computable function $f$.
\end{tw}

\begin{corollary}
	\textbf{Weighted geometric set cover is W[1]-hard}.
	Assuming ETH, there is no algorithm for weighted geometric set cover
	with running time
	$f(k)\cdot(|\points| + |\sets|)^{o(\sqrt{k})}$
	for any computable function $f$.
\end{corollary}

\begin{proof}
Trivial from Theorem \ref{w1_hard}. 
\end{proof}

In order to prove theorem \ref{w1_hard} we will show reduction from grid tiling problem.

\begin{defi}
In the \textbf{grid tiling} problem we are given integers $n$ and $k$,
and a function
$f : \{1 \ldots k\} \times \{1 \ldots k\} \rightarrow \mathcal{P}(\{1 \ldots n\} \times \{1 \ldots n\})$
specifying the set of allowed tiles for each cell of a $k \times k$ grid.
The task is to find functions
$x,y : \{1 \ldots k\} \rightarrow \{1 \ldots n\}$
that assign numbers from $\{1 \ldots n\}$
to respectively rows and columns of the grid,
so that $(x(i), y(j)) \in f(i, j)$ for all valid $i$ and $j$,
or conclude that such assignment does not exist.
\end{defi}


\begin{tw}
\label{grid_tiling_w1_hard}
Assuming ETH, there is no algorithm for grid tiling problem
$f(k)\cdot n^{o(\sqrt{k})}$
for any computable function $f$.
\end{tw}

TODO: proof from reference in literature

Let us have an instance of grid tiling problem -- size of the
grid $k$, number of colors $n$
and function of allowed tiles
$f : \{1 \ldots k\} \times \{1 \ldots k\} \rightarrow \{1 \ldots n\} \times \{1 \ldots n\}$.

\paragraph{Construction.}
We construct a set $\sets$ of segments and a set $\points$
of points.

First let us choose any ordering of $n^2$ elements
and denote it with bijective function
$order : \{1 \ldots n\} \times \{1 \ldots n\} \rightarrow \{1 \ldots n^2\}$.

Define $match_v(i, j)$ and $match_h(i, j)$
as functions denoting whether two points share x or y coordinate.

$$match_v(i, j) \iff
order^{-1}(i) = \{x_i, y_i\} \land order^{-1}(j) = \{x_j, y_j\} \land x_i = x_j$$

$$match_h(i, j) \iff
order^{-1}(i) = \{x_i, y_i\} \land order^{-1}(j) = \{x_j, y_j\} \land y_i = y_j$$


\subparagraph{Points.}

Define points:
	$$h_{i, j, t} = (i \cdot (n^2+1) + t, j \cdot (n^2+1))$$
	$$v_{i, j, t} = (i \cdot (n^2+1), j \cdot (n^2+1) + t)$$
	
Let's define sets $H$ and $V$ as:
$$H = \{h_{i, j, t} : 1 \le i, j, \le k, 1 \le t \le n^2\}$$
$$V = \{v_{i, j, t} : 1 \le i, j, \le k, 1 \le t \le n^2\}$$
	
Let us define $\epsilon = \frac{1}{2k^2}$.
For a point $p = \{x, y\}$ we define points:
$$p^{L} = \{x - \epsilon, y\},$$
$$p^{R} = \{x + \epsilon, y\},$$
$$p^{U} = \{x, y + \epsilon\},$$
$$p^{D} = \{x, y - \epsilon\}.$$

Then we define:
$$\points := H \cup \{p^L : p \in H\} \cup \{p^R : p \in H\}
\cup V \cup \{p^U : p \in V\} \cup \{p^D : p \in V\} $$


\subparagraph{Segments.}
Define horizontal segments.

\newcommand{\hor}[4]{\mathsf{hor}_{#1,#2,#3,#4}}
\newcommand{\ver}[4]{\mathsf{ver}_{#1,#2,#3,#4}}
\newcommand{\horbeg}[2]{\mathsf{horBeg}_{#1,#2}}
\newcommand{\verbeg}[2]{\mathsf{verBeg}_{#1,#2}}
\newcommand{\horend}[2]{\mathsf{horEnd}_{#1,#2}}
\newcommand{\verend}[2]{\mathsf{verEnd}_{#1,#2}}

$$\hor{i}{j}{t_1}{t_2} = (h^R_{i,j,t_1}, h^L_{i+1, j, t_2})$$
$$\ver{i}{j}{t_1}{t_2} = (v^D_{i,j,t_1}, v^U_{i, j+1, t_2})$$

$$\horbeg{i}{t} = (h^L_{1, i, 1}, h^L_{1, i, t})$$
$$\horend{i}{t} = (h^R_{k, i, t}, h^R_{k, i, n^2})$$

$$\verbeg{i}{t} = (v^U_{i, 1, 1}, v^U_{i, 1, t})$$
$$\verend{i}{t} = (v^D_{i, k, t}, v^D_{i, k, n^2})$$

\begin{eqnarray*}
HOR &= &\{\hor{i}{j}{t_1}{t_2} : 1 \le i < k, 1 \le j \le k,
1 \le t_1, t_2 \le n^2, match_h(t_1, t_2)\} \\
&\cup &\{\horbeg{i}{t} : 1 \le i \le k, 1 \le t \le n^2\}
\\
&\cup &\{\horend{i}{t} : 1 \le i \le k, 1 \le t \le n^2\}
\end{eqnarray*}

\begin{eqnarray*}
VER &= &\{\ver{i}{j}{t_1}{t_2} : 1 \le i \le k, 1 \le j < k,
1 \le t_1, t_2 \le n^2, match_v(t_1, t_2)\} \\
&\cup &\{\verbeg{i}{t} : 1 \le i \le k, 1 \le t \le n^2\}
\\
&\cup &\{\verend{i}{t} : 1 \le i \le k, 1 \le t \le n^2\}
\end{eqnarray*}

$$DIAG := \{ (h_{i, j, t}, v_{i, j, t}) :
	1 \le i, j \le k, 1 \le t \le n^2, a_t \in f(i, j)\}$$

TODO: explain that these segments are in fact diagonal

$$\sets := HOR \cup VER \cup DIAG$$

Weight function is equal to length of the segment for $HOR$ and $VER$
and equal to $\delta = \frac{1}{4k^4}$ for $DIAG$.

\begin{equation}
w(s) =
	\begin{cases*}
	  |s| 			& if $s \in HOR \cup VER$ \\
	  \delta        & if $s \in DIAG$
	\end{cases*}
\end{equation}

\newcommand{\solWeight}{2k^2(n^2+1) - 2k^2\epsilon +k^2\delta -4(k-\epsilon)}

\begin{lemma}
	If there exists solution for grid tiling,
	then there exists solution of instance $(\points, \sets)$
	of geometric set cover
	with weight $\solWeight$.
\end{lemma}

\begin{proof}
	If there exists a solution to the grid tiling problem
	$x, y$,
	then there exists a solution that covers
	all points
	$$\{h_{i, j, t} : 1 \le i, j \le k, t=order(x(i), y(j))\}
	\cup \{v_{i, j, t} : 1 \le i, j \le k, t=order(x(i), y(j))\}$$
	
	with $k^2$ segments from $DIAG$
	and the rest in $VER$ or $HOR$.
	This solution has weight $\solWeight$.
\end{proof}

\begin{lemma}
	If there exists solution  of instance $(\points, \sets)$
	of geometric set cover
	with weight at most $\solWeight$,
	then there exists a solution for grid tiling.
\end{lemma}

\begin{proof}
1. Any solution has to have sum of weights from HOR and VER at least
$2k^2(n^2+1) -2k^2\epsilon -4(k-\epsilon)$
\begin{itemize}
\item We know that for every $i,j$ only one $v_{i,j,t}$ and $h_{i,j,t}$
can be uncovered by HOR and VER (Claim \ref{one_diag_in_square}),
the guards have to bo covered in HOR and VER (Claim \ref{guards}),
therefore sum of these segments are
$2k^2(n^2+1) -2k^2\epsilon -4(k-\epsilon)$
\end{itemize}

2. Solution has to have at most one DIAG segment in every square
\begin{itemize}
\item At least one h and v points are covered with DIAG for i,j
	(Claim \ref{one_diag_in_square})
\item Exactly one $h$ and $v$ points are covered with DIAG for every $i,j$,
because if one of them were not, then we would use too much weight
$$2k^2(n^2+1) -2k^2\epsilon -4(k-\epsilon) + \epsilon > \solWeight$$
\item This points are covered with the same segment from DIAG,
because we need to use at least $k^2$ of them to use
exactly one DIAG segment for every $i,j$, if we used 2 DIAGs
for one of $(i,j)$, then we would have used too much weight by $\delta$
\item Therefore $t_x$ and $t_y$ are allowed tile for i,j,
because respective segment was in DIAG
\end{itemize}

3. Points from previous point are synchronized in one row/column
$match_h(t_1,t_2)$ and $match_v(t_1,t_2)$ must be true
\begin{itemize}
\item Every space between points in H and V covered by HOR/VER
has to be covered by only one segment, because otherwise it would
use $\epsilon > k^2\delta$ additional weight, which is larger than
weight of solution
\item Therefore for every $t_1$ and $t_2$ that are consecutive
uncovered points, there must be a segment connecting
$h^R_{i,j,t_1}$ and $h^L_{i+1,j,t_2}$, because these points
have to be covered in HOR (Claim \ref{guards}),
therefore $match_h(t_1,t_2)$ must be true
\end{itemize}
\end{proof}

\begin{claim}
\label{guards}
Points $p^{L}, p^R, p^U, p^D$ cannot be covered with
any segments from $DIAG$.
\end{claim}

\begin{claim}
\label{hor_ver_sound}
Points in $H \cup \{p^L : p \in H\} \cup \{p^R : p \in H\}$
cannot be covered with any segments from $VER$.

Points in $V \cup \{p^U : p \in V\} \cup \{p^D : p \in V\} $
cannot be covered with any segments from $HOR$.
\end{claim}


\begin{claim}
\label{guards_covered}
(Follows from two previous claims, I guess this might be enough?).
Points in $H \cup \{p^L : p \in H\} \cup \{p^R : p \in H\}$
have to be covered with sgements from $HOR$.

Points in $V \cup \{p^U : p \in V\} \cup \{p^D : p \in V\} $
have to be covered with segments from $VER$.
\end{claim}

\begin{claim}
For given $i, j$ if none of the points $h_{i, j, t}$ ($v_{i, j, t}$)
for $1 \le t \le n^2$ are covered with $DIAG$,
then some spaces between neighbouring points were covered twice.
\end{claim}


\begin{claim}
\label{one_diag_in_square}
For given $i, j$ two points $h_{i, j, t_1}, h_{i, j, t_2}$
($v_{i, j, t_1}, v_{i, j, t_2}$)
for $1 \le t_1 < t_2 \le n^2$ are covered with $DIAG$,
then one of them had to be also covered with a segment from $HOR$
($VER$).
\end{claim}
\begin{proof}
Assume point $h_{i, j, t_1}$ is not covered with
segments from $HOR$.
Point $h^R_{i, j, t_1}$ has to be covered with $HOR$
from Claim $\ref{guards_covered}$.
And every segment in $HOR$ covering $h^R_{i, j, t_1}$,
covers also $h_{i, j, t_2}$.
\end{proof}


\section{What is missing}
We don't know FPT for axis-parallel segments without $\delta$-extensions.
