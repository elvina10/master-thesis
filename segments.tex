In this chapter we show two fixed-parameter algorithms
for geometric set cover problem in two different settings.
Section \ref{section:fpt_unweighted} shows 
a fixed-parameter algorithm for geometric set cover with unweighted segments.

The reminder of the chapter presents
a fixed-parameter algorithm for geometric set cover with weighted segments
with $\delta$-extensions. We show algorithm for setting with $\delta$-extensions,
because the original problem with weights is W[1]-hard,
what we show in Chapter $\ref{chapter:w1_hard}$.

We start with shared definition for this problem.
We define \textit{extreme points} for a set of collinear points.

\begin{defi}
	For a set of collinear points $C$,
	\textbf{extreme points} are the ends
	of the smallest segment that covers all points from set $C$.
	
	If $C$ consists of one point or is empty, then
	there exists 1 or 0 extreme points respectively.
\end{defi}

\section{Fixed-parameter tractable algorithm for unweighted segments}
\label{section:fpt_unweighted}
In this section we consider the fixed-parameter tractable
algorithms for unweighted geometric set cover with segments.
Setting where segments are limited to be axis-parallel
(or limited to constant number of directions) has an FPT
algorithm already present in literature.
We present an FPT algorithm for geometric set cover
with unweighted segments, where segments are in arbitrary directions.

\subsection{Axis-parallel segments}
You can find this simple algorithm in Patemetrized Algorithms book \cite{platypus_book}.

We show an $\mathcal{O}(2^k)$-time branching algorithm.
In each step, the algorithm selects a point $a$ which is not yet covered,
branches to choose one of the two directions, and greedily chooses
a segment in that direction to cover $a$.
This proceeds until either all points are covered or $k$ segments are chosen.

Let us take
the point $a=(x_a,y_a)$ which is the smallest 
among points that are not yet covered
in the lexicographic ordering
of points in $\mathbb{R}^2$.
We need to cover $a$ with some of the remaining segments.

Branch over the choice of one of the coordinates ($x$ or $y$);
without loss of generality, let us assume we chose $x$.
Among the segments lying on line $x = x_a$,
we greedily add to the solution the one that covers the most points.
As $a$ was the smallest in the lexicographical order,
then all points on line $x = x_a$ have the $y$-coordinate larger than $y_a$.
Therefore, if we denote the greedily chosen segment as $s$,
then any other segment on $x = x_a$ that covers $a$ can only
cover a (possibly improper) subset of points covered by $s$.
Thus, greedily choosing $s$ is optimal.

In each step of the algorithm we add one segment to the solution,
thus each branch can stop at depth $k$.
If no branch finds a solution, then that means a solution of size at most $k$ does not exist.


\begin{remark}
The same algorithm can be used for segments in $d$ directions,
where we branch over $d$ directions and it runs in complexity $\mathcal{O}(d^k)$.
\end{remark}

\subsection{Segments in arbitrary directions}
\label{segments_in_arbitrary_direction}
In this section we consider setting where segments are not constrained
to only $d$ directions. 
We present a fixed-parameter tractable algorithm,
where parameter is the size of the solution.

\begin{tw}{
	\label{segment_cover_fpt}
	\textbf{(FPT for segment cover)}.
	There exists an algorithm that given a family $\sets$ of
	$n$ segments (in any direction),
	a set of $m$ points $\points$
	and a parameter $k$,
	runs in time $k^{O(k)} \cdot (nm)^2$,
	and outputs a subfamily $\sol \subseteq \sets$
	such that $|\sol| \le k$ and $\sol$ covers all points in $\points$,
	or determines that such a set $\sol$ does not exist.
}\end{tw}

We will need the following lemmas.

\begin{lemma}
   \label{fpt_reduction}
   Given an instance $(\sets, \points)$ of the segment cover problem,
   without a loss of generality we can assume that
   no segment covers a superset of what another segment covers.
   That is, for any distinct $A, B \in \sets$, we have
   $A \cap C \not \subseteq B \cap C$ and $A \cap C \not \supseteq B \cap C$.
\end{lemma}   
   
\begin{proof} Trivial. \end{proof}

\begin{lemma}
	\label{fpt_long_lines}
	Given an instance $(\sets, \points)$ of the segment cover problem,
	if there exists a line $L$ with at least
	$k+1$ points on it, then there exists a subset $\mathcal{A} \subseteq \sets$,
	$|\mathcal{A}| \le k$,
	such that every solution $\sol$ with $|\sol| \le k$
	satisfies $|\mathcal{A} \cap \sol| \ge 1$.
	Moreover, such a subset can be found in polunomial time.
\end{lemma}

\begin{proof}

First we use Lemma \ref{fpt_reduction}.

Let us enumerate the points from $\points$ that lie on $L$ as $x_1, x_2, \ldots x_t$
in the order in which they appear on $L$.
Every segment that is not collinear with $L$ can cover at most one of these
points. Therefore, in any solution of size not larger than $k$,
among any $k$ of these points at least one must
be covered with segment collinear with $L$.

Therefore, every solution needs to take one of the segments collinear
with $L$ that covers any of the points
$x_1, x_2, \ldots x_k$. After using reduction from Lemma \ref{fpt_reduction},
there are at most $k$ such segments that are distinct.
\end{proof}

We are ready to prove Theorem \ref{segment_cover_fpt}.

\begin{proof}[Proof of Theorem \ref{segment_cover_fpt}.]\leavevmode

We will prove this theorem by presenting a branching algorithm that
works in desired complexity. It branches over the
choice of segments to cover lines with \textit{a lot} of points,
then finally solving the small instance, where
every line has at most $k$ points by checking all possible solutions.

\subparagraph{Algorithm.}
First we use Lemma \ref{fpt_reduction}.

Next, we present a recursive algorithm. Given an instance of the problem:

\begin{enumerate}[label={(\arabic*)}]
\item If there exist a line with at least $k+1$ points from $\points$, we branch over
adding to the solution one of the at most $k$ possible segments
provided by Lemma \ref{fpt_long_lines}; name this segment $S$.
Then we find a solution $\sol$
for the problem for points $\points - S$, segments $\sets - \{S\}$,
and parameter $k-1$. We return $\sol \cup \{S\}$.
\item If every line has at most $k$ points on it and $|\points| > k^2$,
then answer \texttt{NO}.
\item If $|\points| \le k^2$, solve the problem by brute force:
check all subsets of $\sets$ of size at most $k$.
\end{enumerate}

\subparagraph{Correctness.}

Lemma \ref{fpt_long_lines} proves that at least one segment that we
branch over in (1) must be present in every solution $\sol$ with $|\sol| \le k$.
Therefore, the recursive call can find a solution, provided there exists one.

In (2) the answer is no, because every line covers no more than $k$ points
from $\points$, which implies the same about every segment from $\sets$.
Under this assumption
we can cover only $k^2$ points with a solution of size $k$, which is less
than $|\points|$.

Checking all possible solutions in (3) is trivially correct.


\subparagraph{Complexity.}

In the leaves of recursion we have $|\points| \le k^2$, so $|\sets| \le k^4$, because
every segments can be uniquely identified by the two extreme points it covers
(by Lemma \ref{fpt_reduction}). Therefore, there are $\binom{k^4}{k}$
possible solutions to check, each can be checked in time $O(k|\points|)$.
Therefore, (3) takes time $k^{O(k)}$.


In this branching algorithm our parameter $k$ is decreased with every
recursive call, so we have at most $k$ levels of recursion with
branching over $k$ possibilites. Candidates to branch over
can be found on each level in time $O((nm)^2)$.

Reduction from Lemma \ref{fpt_reduction} can be implemented in time $O(n^2m)$.

It follows that the overall complexity is $O((nm)^2 \cdot k^{O(k)})$
\end{proof}

