\chapter{Definitions}

In this chapter we present some definitions that are later used
across the different chapters.

\section{Geometric set cover}
Every time we refer to geometric set cover,
we consider a geometric set cover problem
on a~2-dimensional plane.

In the geometric set cover problem we are are given
$\sets$ -- a set of objects, which are connected
subsets of the plane and $\points$ -- a set of points in the plane.
The task is to choose $\sol \subseteq \sets$ such that
every point in $\points$ is inside some object from $\sol$
and $|\sol|$ is minimized.

In the parameterized setting for a given $k$,
we only look for a solution $\sol$ such that $|\sol| \le k$
or decide that there is no such set $\sol$.

In the weighted setting, there is some given weight function
$f : \sets \rightarrow \mathbb{R^+}$
and we would like to find a solution $\sol$
that minimizes $\sum_{R \in \sol} f(R)$.

\section{Approximation}

Let us recall some definitions related to optimization problems
that are used in the following sections.

\begin{defi}
A \textbf{polynomial-time approximation scheme (PTAS)}
for a minimization problem $\Pi$
is a family of algorithms $\cal{A}_\epsilon$ for
every $\epsilon > 0$
such that $\cal{A}_\epsilon$ takes an instance $I$ of~$\Pi$
and in polynomial time
finds a solution that is within a factor
of ($1+\epsilon$) of being optimal.
That means the reported solution has weight at most
$(1+\epsilon)opt(I)$, where $opt(I)$ is the weight
of an optimal solution to $I$.
\end{defi}

\begin{defi}
A problem $\Pi$ is \textbf{APX-hard} if assuming P $\neq$ NP,
there exists $\epsilon > 0$
such that there is no polynomial-time $(1+\epsilon)$-approximation algorithm
for $\Pi$.
\end{defi}

\section{Problem modification with $\delta$-extension}

Another idea presented here, much less versatile than the previous concepts,
is $\delta$-extension.
We define it specificaly for the geometric set cover problem.

It is based on the similar idea of $\delta$-shrinking
for the geometric independent set problem,
which is presented in \cite{shrinking}.

Intuitively, we consider a problem with slightly larger objects,
which makes the instance more permissive.
However, we aim to find a solution that
is not larger than the
optimum solution to the original problem,
so this is substantially easier than just
solving the problem for the larger objects.
It may even be the case
that we are able to find the solution
of size smaller than the optimum solution
to the original problem.

First, we formally define $\delta$-extended objects.

\begin{defi}
For any $\delta > 0$ and a center-symmetric object $L$ with
centre of symmetry $S = (x_s, y_s)$,
the \textbf{$\delta$-extension} of $L$ is the object $L^{+\delta} =
\{(1 + \delta)\cdot(x - x_s, y - y_s) + (x_s, y_s) : (x, y) \in L\}$,
that is, $L^{+\delta}$ is the image of $L$ under homothety centered
at $S$ with scale $(1+\delta)$.
\end{defi}


The geometric set cover problem with $\delta$-extension
is a modified version of geometric set cover with
the following modifications.
\begin{itemize}
\item We need to cover all the points in $\points$
with objects from $\{P^{+\delta} : P \in \sets\}$ (which always 
include no fewer points than the objects
before $\delta$-extension).
\item We look for a solution that is not larger than the optimum
solution to the original problem.
Note that it does not need to be an optimal solution in
the modified problem.
\end{itemize}

Formally, we have the following.

\begin{defi}
The \textbf{geometric set cover problem
with $\delta$-extension} is the problem where for an input instance
$I=(\sets, \points)$,
the task is to output a solution $\mathcal{R} \subseteq \sets$
such that the~$\delta$-extended set
$\{ R^{+\delta} :  R \in \mathcal{R} \}$ covers $\points$
and is not larger than the optimal solution to the~problem without
extension, i.e.~$|\mathcal{R}| \le |opt(I)|$.
\end{defi}

At last, we formulate a definition of the
polynomial-time approximation scheme (PTAS) of
the problem with $\delta$-extension.

\begin{defi}
We define a \textbf{PTAS for geometric set cover 
with $\delta$-extension} as a family of algorithms
$\{\mathcal{A}_{\delta, \epsilon}\}_{\delta, \epsilon > 0}$ that
each takes as an input instance $I=(\sets, \points)$,
and in polynomial-time outputs a solution $\mathcal{R} \subseteq \sets$
such that the $\delta$-extended set
$\{ R^{+\delta} :  R \in \mathcal{R} \}$ covers $\points$
and is within a $(1+\epsilon)$ factor of the optimal
solution to this problem without
extension, i.e.~$(1+\epsilon)|\mathcal{R}| \le |opt(I)|$.

\end{defi}
