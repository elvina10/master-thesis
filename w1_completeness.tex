TODO: some introduction

\begin{tw}
\label{w1_hard}
	\textbf{Weighted geometric set cover with segments in 3 directions is W[1]-hard}.
	Consider the problem of covering a set $\points$ of points
	by selecting $k$ axis-parallel or right-diagonal weighted segments
	from a set $\sets$ 
	with weights $w : \sets \rightarrow \mathbb{R}$
	with minimal weight.
	Assuming ETH, there is no algorithm for this
	problem with running time
	$f(k)\cdot(|\points| + |\sets|)^{o(\sqrt{k})}$
	for any computable function $f$.
\end{tw}

\begin{corollary}
	\textbf{Weighted geometric set cover is W[1]-hard}.
	Assuming ETH, there is no algorithm for weighted geometric set cover
	with running time
	$f(k)\cdot(|\points| + |\sets|)^{o(\sqrt{k})}$
	for any computable function $f$.
\end{corollary}

\begin{proof}
Trivial from Theorem \ref{w1_hard}. 
\end{proof}

In order to prove theorem \ref{w1_hard} we will show reduction from grid tiling problem.

\begin{defi}
In the \textbf{grid tiling} problem we are given integers $n$ and $k$,
and a function
$f : \{1 \ldots k\} \times \{1 \ldots k\} \rightarrow \wp(\{1 \ldots n\} \times \{1 \ldots n\})$
specifying the set of allowed tiles for each cell of a $k \times k$ grid.
The task is to find functions
$x,y : \{1 \ldots k\} \rightarrow \{1 \ldots n\}$
that assign numbers from $\{1 \ldots n\}$
to respectively columns and rows of the grid,
so that $(x(i), y(j)) \in f(i, j)$ for all valid $i$ and $j$,
or conclude that such assignment does not exist.
\end{defi}


\begin{tw}
\label{grid_tiling_w1_hard}
Assuming ETH, there is no $f(k)\cdot n^{o(\sqrt{k})}$-time
algorithm for grid tiling problem
for any computable function $f$.
\end{tw}

TODO: proof from reference in literature (platypus book page 490)

Let us have an instance of grid tiling problem -- size of the
grid $k$, number of colors $n$
and function of allowed tiles
$f : \{1 \ldots k\} \times \{1 \ldots k\} \rightarrow \wp\{1 \ldots n\} \times \{1 \ldots n\}$.

TODO: nice picture of instance of grid tiling with solution

\paragraph{Construction.}
We construct an instance of Geometric Set Cover with segments in 3 directions
with weights $(\points, \sets, w)$.

First let us choose any ordering of $n^2$ elements
and denote it as bijective function
$order : \{1 \ldots n^2\} \rightarrow \{1 \ldots n\} \times \{1 \ldots n\}$.

Define $match_v(i, j)$ and $match_h(i, j)$
as functions denoting whether two points share x or y coordinate.

$$match_v(i, j) \iff
order(i) = \{x_i, y_i\} \land order(j) = \{x_j, y_j\} \land x_i = x_j$$

$$match_h(i, j) \iff
order(i) = \{x_i, y_i\} \land order(j) = \{x_j, y_j\} \land y_i = y_j$$


\subparagraph{Points.}

Define points:
	$$h_{i, j, t} = (i \cdot (n^2+1) + t, j \cdot (n^2+1))$$
	$$v_{i, j, t} = (i \cdot (n^2+1), j \cdot (n^2+1) + t)$$
	
Let's define sets $H$ and $V$ as:
$$H = \{h_{i, j, t} : 1 \le i, j, \le k, 1 \le t \le n^2\}$$
$$V = \{v_{i, j, t} : 1 \le i, j, \le k, 1 \le t \le n^2\}$$
	
Let us define $\epsilon = \frac{1}{2k^2}$.
For a point $p = \{x, y\}$ we define points:
$$p^{L} = \{x - \epsilon, y\},$$
$$p^{R} = \{x + \epsilon, y\},$$
$$p^{U} = \{x, y + \epsilon\},$$
$$p^{D} = \{x, y - \epsilon\}.$$

Then we define:
$$\points := H \cup \{p^L : p \in H\} \cup \{p^R : p \in H\}
\cup V \cup \{p^U : p \in V\} \cup \{p^D : p \in V\} $$


\subparagraph{Segments.}
Define horizontal segments.

\newcommand{\hor}[4]{\mathsf{hor}_{#1,#2,#3,#4}}
\newcommand{\ver}[4]{\mathsf{ver}_{#1,#2,#3,#4}}
\newcommand{\horbeg}[2]{\mathsf{horBeg}_{#1,#2}}
\newcommand{\verbeg}[2]{\mathsf{verBeg}_{#1,#2}}
\newcommand{\horend}[2]{\mathsf{horEnd}_{#1,#2}}
\newcommand{\verend}[2]{\mathsf{verEnd}_{#1,#2}}

$$\hor{i}{j}{t_1}{t_2} = (h^R_{i,j,t_1}, h^L_{i+1, j, t_2})$$
$$\ver{i}{j}{t_1}{t_2} = (v^U_{i,j,t_1}, v^D_{i, j+1, t_2})$$

$$\horbeg{i}{t} = (h^L_{1, i, 1}, h^L_{1, i, t})$$
$$\horend{i}{t} = (h^R_{k, i, t}, h^R_{k, i, n^2})$$

$$\verbeg{i}{t} = (v^D_{i, 1, 1}, v^D_{i, 1, t})$$
$$\verend{i}{t} = (v^U_{i, k, t}, v^U_{i, k, n^2})$$

\begin{eqnarray*}
HOR &= &\{\hor{i}{j}{t_1}{t_2} : 1 \le i < k, 1 \le j \le k,
1 \le t_1, t_2 \le n^2, match_h(t_1, t_2)\} \\
&\cup &\{\horbeg{i}{t} : 1 \le i \le k, 1 \le t \le n^2\}
\\
&\cup &\{\horend{i}{t} : 1 \le i \le k, 1 \le t \le n^2\}
\end{eqnarray*}

\begin{eqnarray*}
VER &= &\{\ver{i}{j}{t_1}{t_2} : 1 \le i \le k, 1 \le j < k,
1 \le t_1, t_2 \le n^2, match_v(t_1, t_2)\} \\
&\cup &\{\verbeg{i}{t} : 1 \le i \le k, 1 \le t \le n^2\}
\\
&\cup &\{\verend{i}{t} : 1 \le i \le k, 1 \le t \le n^2\}
\end{eqnarray*}

$$DIAG := \{ (h_{i, j, t}, v_{i, j, t}) :
	1 \le i, j \le k, 1 \le t \le n^2, order(t) \in f(i, j)\}$$

TODO: explain that these segments are in fact diagonal

$$\sets := HOR \cup VER \cup DIAG$$

Weight function is equal to length of the segment for $HOR$ and $VER$
and equal to $\delta = \frac{1}{4k^4}$ for $DIAG$.

TODO: Put a picture of small instance like 3x3 with n=2

\begin{equation}
w(s) =
	\begin{cases*}
	  |s| 			& if $s \in HOR \cup VER$ \\
	  \delta        & if $s \in DIAG$
	\end{cases*}
\end{equation}

\newcommand{\solWeight}{2k^2(n^2+1) - 4k^2\epsilon -4k(1-\epsilon) +k^2\delta }

\begin{lemma}
\label{set_cover_solution_exists}
	If there exists solution for grid tiling,
	then there exists solution of instance $(\points, \sets, w)$
	of geometric set cover
	with weight $\solWeight$.
\end{lemma}

\begin{proof}
	If there exists a solution to the grid tiling problem
	$x, y$,
	then there exists a solution that covers
	all points
	$$\{h_{i, j, t} : 1 \le i, j \le k, order(t)=(x(i), y(j))\}
	\cup \{v_{i, j, t} : 1 \le i, j \le k, order(t)=(x(i), y(j))\}$$
	
	with $k^2$ segments from $DIAG$
	and the rest in $VER$ or $HOR$.
	This solution has weight $\solWeight$.
\end{proof}


\begin{claim}
\label{guards}
Points in $\{p^L : p \in H\} \cup \{p^R : p \in H\}$
have to be covered with sgements from $HOR$.

Points in $\{p^U : p \in V\} \cup \{p^D : p \in V\} $
have to be covered with segments from $VER$.
\end{claim}


\begin{claim}
\label{one_diag_in_square}
For given $1 \le i, j \le n$
and any solution of an instance $(\points, \sets, w)$
no two points $h_{i, j, t_1}, h_{i, j, t_2}$
($v_{i, j, t_1}, v_{i, j, t_2}$)
for $1 \le t_1 < t_2 \le n^2$
can be not covered with segments from $HOR$ ($VER$).
\end{claim}

\begin{proof}
Proof for horizontal segments. Proof for vertical is analoguous.

Assume point $h_{i, j, t_1}$ is not covered with
segments from $HOR$.
Point $h^R_{i, j, t_1}$ has to be covered with $HOR$
from Claim $\ref{guards}$.
And every segment in $HOR$ covering $h^R_{i, j, t_1}$,
but not $h_{i,j,t_1}$ covers also $h_{i, j, t_2}$.
\end{proof}

\begin{lemma}
\label{vertical_horizontal_sum}
For a constructed instance $(\points, \sets, w)$
any solution has to have sum of weights from sets $HOR$ and $VER$ at least
$W_{hv} = 2k^2(n^2+1) -4k^2\epsilon -4k(1-\epsilon)$.
\end{lemma}

\begin{proof}
We know that for every $1 \le i,j \le k$
only one there exists only one $t_x$ and $t_y$ such that
$v_{i,j,t_x}$ and $h_{i,j,t_y}$
can be not covered by
segments from $HOR$ and $VER$ (Claim \ref{one_diag_in_square}),


We sum the lower bound for sum of length for horizontal/vertical
lines for a single vertical line
(the bound is the same for every horizontal line).

Let us fix $1 \le i \le k$.

\begin{enumerate}[label={(\arabic*)}]
\item Length between $v^D_{i, 1, 1}$ and $v^U_{i, k, n^2}$ is:
$$(k(n^2+1) + n^2 +\epsilon) - ((n^2+1)+1 -\epsilon) = k(n^2+1) - 2(1 - \epsilon).$$

\item For every $1 \le j \le k$ there exists at most one $1 \le t \le n^2$
such that $v_{i,j,t}$ is not covered by segments from $VER$
(Claim \ref{one_diag_in_square}).
Its guards $v^U_{i,j,t}$ and $v^D_{i,j,t}$
have to be covered in $VER$ (Claim \ref{guards}).
Therefore at most $k$ spaces of length $2\epsilon$ can be left
not covered by segments from $VER$.

\end{enumerate}
Therefore sum of these lower bounds for vertical and horizontal lines are:
$$2k(k(n^2+1) -2k\epsilon -2(1-\epsilon)) = 2k^2(n^2+1) -4k^2\epsilon -4k(1-\epsilon)$$
\end{proof}

\begin{lemma}
\label{diag_correct}
For a constructed instance $(\points, \sets, w)$
for any solution $\sol$ with weight equal to $\solWeight$,
for every $1 \le i,j \le k$
there exists such $1 \le t \le n^2$ that:
\begin{enumerate}[label={(\arabic*)}]
\item $v_{i,j,t}, h_{i,j,t}$ are not covered by segments from $VER$ or $HOR$;
\item segment $(v_{i,j,t}, h_{i,j,t})$ is in solution $\sol$;
\item $order(t) \in f(i,j)$, ie. it is an allowed tile for $(i,j)$;
\item for every $1 \le s\le n^2$, $s \neq t$, $v_{i,j,s}$ is covered in $VER$;
\item for every $1 \le s\le n^2$, $s \neq t$, $h_{i,j,s}$ is covered in $HOR$.
\end{enumerate}
Name the function of this $t$ as
$diagonal : \{1 \ldots k\} \times \{1 \ldots k\} \rightarrow \{1 \ldots n^2\}$.
\end{lemma}

\begin{proof}
At most one $h_{i,j,t_x}$ and $v_{i,j,t_y}$
points are covered with $DIAG$
(Claim \ref{one_diag_in_square}).
	
Exactly one $h_{i,j,t_x}$ and $v_{i,j,t_y}$
points are covered with $DIAG$,
because if one of them were not, then we would use too much weight
$$W_{hv} + 2\epsilon > \solWeight$$

This points are covered with the same segment from $DIAG$,
because we need to use at least $k^2$ of them to use
exactly one DIAG segment for every pair of $1 \le i,j \le k$,
if we used 2 segments from $DIAG$
for one pair $(i,j)$,
then we would have used too much weight by $\delta$.
Since these points $h_{i,j,t_x}$ and $v_{i,j,t_y}$ are covered by
segment from $DIAG$, therefore $t_x = t_y$.

Therefore $diagonal(i,j) = t_x = t_y$
and $order(t_x)$ is an allowed tile for $(i,j)$
because the respective segment is in $DIAG$.

\end{proof}

\begin{lemma}
\label{vertical_horizontal_synchronized}
For a constructed instance $(\points, \sets, w)$
for any solution of weight $\solWeight$ it holds that $diagonal$ function
from Lemma \ref{diag_correct}:
\begin{enumerate}
\item 
for any $1 \le i < k, 1 \le j \le k$,
$match_h(diagonal(i, j),diagonal(i+1, j))$ must be true;
\item 
for any $1 \le i \le k, 1 \le j < k$,
$match_v(diagonal(i, j),diagonal(i, j+1))$ must be true.
\end{enumerate}
\end{lemma}

\begin{proof}
Every space between points in $H$ and $V$ covered by $HOR$/$VER$
has to be covered by only one segment, because otherwise it would
use $W_{hv} + \epsilon > \solWeight$,
therefore such solution would be too costly.

Proof for vertical (2), proof for horizontal is analogous.

Let us take any $1 \le i < k, 1 \le j \le k$
and name $t_1 = diagonal(i, j)$ and $t_2 = diagonal(i+1, j)$.
Therefore $h_{i,j,t_1}$ and $h_{i+1,j,t_2}$
are not covered by segments from $HOR$,
$h^R_{i,j,t_1}$ and $h^L_{i+1,j,t_2}$
have to be covered by segments from $HOR$ (Claim \ref{guards}).
Every segment from $HOR$ starts at $h^R_{x,y,z_1}$
segment and finishes at $h^L_{x,y+1,z_2}$ segment for some
$1 \le x \le k$,$1 \le z < k$ and $1 \le z_1, z_2 \le n^2$.
Since all of the points between $h^R_{i,j,t_1}$ and $h^L_{i+1,j,t_2}$
are covered by segments in $HOR$,
and there are two different segments covering
these points, one of them must begin
at $h^R_{i,j,t_1}$ and end at $h^L_{i,j+1,z_2}$
and the other one begin at $h^R_{i,j,z_1}$
and end at $h^L_{i+1,j,t_2}$
for some $1 \le z_1, z_2 \le n^2$.
Therefore space between $h^R_{i,j,z_1}$ and $h^L_{i,j+1,z_2}$
is covered twice and is longer than $\epsilon$.
By Lemma \ref{vertical_horizontal_sum}
lower bound for weight of such solution is $W_{hv} + \epsilon$
which contradicts that the given solution has size $\solWeight < W_{hv} + \epsilon$
Therefore $h^R_{i,j,t_1}$ and $h^L_{i+1,j,t_2}$ must be covered
by one segment from $HOR$ and these points are ends of this segment.

$h^R_{i,j,t_1}$ and $h^L_{i+1,j,t_2}$
are ends of a segment from $HOR$,
therefore $match_h(t_1,t_2)$ must be true.
\end{proof}

\begin{corollary}
\label{vertical_horizontal_synchronized_inductive}
For a constructed instance $(\points, \sets, w)$
for any solution of weight $\solWeight$ it holds that $diagonal$ function
from Lemma \ref{diag_correct}:
\begin{enumerate}
\item 
for any $1 \le i, j \le k$,
$match_h(diagonal(1, j),diagonal(i, j))$ must be true;
\item 
for any $1 \le i, j \le k$,
$match_v(diagonal(i, 1),diagonal(i, j))$ must be true.
\end{enumerate}
\end{corollary}
\begin{proof}
Simple inductive proof based on Lemma \ref{vertical_horizontal_synchronized}.
\end{proof}

\begin{lemma}
\label{grid_tiling_exists}
	If there exists solution of instance $(\points, \sets, w)$
	with weight at most $\solWeight$,
	then there exists a solution for grid tiling instance.
\end{lemma}

\begin{proof}
Take $diagonal$ function from Lemma \ref{diag_correct}.

To define $x$ funtion 
for every $1 \le i \le k$ as $x(i) = x_i$
where $(x_i, a) = order(v_{i,1})$
and $y$ function 
for every $1 \le i \le k$ as $y(i) = y_i$
where $(b, y_i) = order(h_{1,i})$

To prove that it is a correct solution for grid tiling,
we need to prove that 
for every $1 \le i,j \le k$ $(x(i), y(j))$ is in
allowed tiles set $f(i,j)$.

Let us take any $1 \le i,j \le k$,
By Corollary \ref{vertical_horizontal_synchronized_inductive}
we know that $match_h(diagonal(1, j),diagonal(i, j))$ and
$match_v(diagonal(i, 1),diagonal(i, j))$ are true.
Therefore $order(diagonal(i,j)) = (x(i), y(i))$.
By Lemma \ref{diag_correct} we know that 
$order(diagonal(i,j))$ is in $f(i,j)$.
Therefore 
$(x(i), y(i))$
is in $f(i,j)$.

\end{proof}

\begin{proof}[Proof of Theorem \ref{w1_hard}]
Based on Lemmas \ref{set_cover_solution_exists} and \ref{grid_tiling_exists}
this is true.
\end{proof}
