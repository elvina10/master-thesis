%
% Niniejszy plik stanowi przykład formatowania pracy magisterskiej na
% Wydziale MIM UW.  Szkielet użytych poleceń można wykorzystywać do
% woli, np. formatujac wlasna prace.
%
% Zawartosc merytoryczna stanowi oryginalnosiagniecie
% naukowosciowe Marcina Wolinskiego.  Wszelkie prawa zastrzeżone.
%
% Copyright (c) 2001 by Marcin Woliński <M.Wolinski@gust.org.pl>
% Poprawki spowodowane zmianami przepisów - Marcin Szczuka, 1.10.2004
% Poprawki spowodowane zmianami przepisow i ujednolicenie 
% - Seweryn Karłowicz, 05.05.2006
% Dodanie wielu autorów i tłumaczenia na angielski - Kuba Pochrybniak, 29.11.2016

% dodaj opcję [licencjacka] dla pracy licencjackiej
% dodaj opcję [en] dla wersji angielskiej (mogą być obie: [licencjacka,en])
\documentclass[en]{pracamgr}

% Dane magistranta:
\autor{Katarzyna Kowalska}{371053}

% Dane magistrantów:
%\autor{Autor Zerowy}{342007}
%\autori{Autor Pierwszy}{342013}
%\autorii{Drugi Autor-Z-Rzędu}{231023}
%\autoriii{Trzeci z Autorów}{777321}
%\autoriv{Autor nr Cztery}{432145}
%\autorv{Autor nr Pięć}{342011}

\title{Approximation and Parameterized Algorithms for Segment Set Cover}
\titlepl{Algorytmy aproksymacyjne i parametryzowane dla problemu
pokrywania punktów odcinkami na płaszczyźnie}

%\tytulang{An implementation of a difference blabalizer based on the theory of $\sigma$ -- $\rho$ phetors}

%kierunek: 
% - matematyka, informacyka, ...
% - Mathematics, Computer Science, ...
\kierunek{Computer Science}

% informatyka - nie okreslamy zakresu (opcja zakomentowana)
% matematyka - zakres moze pozostac nieokreslony,
% a jesli ma byc okreslony dla pracy mgr,
% to przyjmuje jedna z wartosci:
% {metod matematycznych w finansach}
% {metod matematycznych w ubezpieczeniach}
% {matematyki stosowanej}
% {nauczania matematyki}
% Dla pracy licencjackiej mamy natomiast
% mozliwosc wpisania takiej wartosci zakresu:
% {Jednoczesnych Studiow Ekonomiczno--Matematycznych}

% \zakres{Tu wpisac, jesli trzeba, jedna z opcji podanych wyzej}

% Praca wykonana pod kierunkiem:
% (podać tytuł/stopień imię i nazwisko opiekuna
% Instytut
% ew. Wydział ew. Uczelnia (jeżeli nie MIM UW))
\opiekun{dr Michał Pilipczuk\\
  Institute of Informatics\\
  }

% miesiąc i~rok:
\date{June 2022}

%Podać dziedzinę wg klasyfikacji Socrates-Erasmus:
\dziedzina{ 
%11.0 Matematyka, Informatyka:\\ 
%11.1 Matematyka\\ 
%11.2 Statystyka\\ 
11.3 Informatyka\\ 
%11.4 Sztuczna inteligencja\\ 
%11.5 Nauki aktuarialne\\
%11.9 Inne nauki matematyczne i informatyczne
}

%Klasyfikacja tematyczna wedlug AMS (matematyka) lub ACM (informatyka)
\klasyfikacja{D. Software\\
  D.127. Blabalgorithms\\
  D.127.6. Numerical blabalysis}

% Słowa kluczowe:
\keywords{set cover, geometric set cover, FPT, W[1]-completeness,
APX-completeness, PCP theorem, NP-completeness}

% Tu jest dobre miejsce na Twoje własne makra i~środowiska:

\newcommand{\points}{\mathcal{C}}
\newcommand{\sets}{\mathcal{P}}
\newcommand{\sol}{\mathcal{R}}
\newcommand{\then}{\Rightarrow}

\usepackage{amsfonts}
\usepackage{amsmath}
\usepackage{graphicx}
\usepackage{xcolor}
\usepackage[nospace, noadjust]{cite}
\usepackage{lineno}
\usepackage{enumitem}
\usepackage{amsthm}
\usepackage{mathtools}  
\usepackage{makecell}
\usepackage{tikz}
\usetikzlibrary{calc,math}
\usepackage{hyperref}
\hypersetup{
    colorlinks,
    citecolor=black,
    filecolor=black,
    linkcolor=black,
    urlcolor=black
}
\linenumbers

\mathtoolsset{showonlyrefs}  

\theoremstyle{plain}
\newtheorem{claim}{Claim}[chapter]
%\newtheorem{defi}{Definition}[section]
\newtheorem{tw}{Theorem}[chapter]
\newtheorem{lemma}{Lemma}[chapter]
\newtheorem{corollary}{Corollary}[chapter]
\newtheorem{remark}{Remark}[chapter]

\theoremstyle{definition}
\newtheorem{defi}{Definition}[chapter]

\setcounter{secnumdepth}{3}
\setcounter{tocdepth}{3}


% koniec definicji

\begin{document}
\maketitle

%tu idzie streszczenie na strone poczatkowa
\begin{abstract}
  The work presents a study
  of different geometric set cover problems.
  It mostly focuses on segment set cover
  and its connection to the polygon set cover.
\end{abstract}

\tableofcontents
%\listoffigures
%\listoftables
\chapter{Introduction}
\section{Background}
Some problems in Computer Science are known to be NP-complete,
meaning that assuming P$\neq$NP there is no polynomial-time
algorithm that can solve these problems.
Even so, they still can be amenable to different approaches,
such as approximation or parameterization.

\begin{defi}
In the \textbf{$\SetCover$} problem we are given a set of elements (universe)
$\points$ and~a~family of sets $\sets$ that are subsets of the universe $\points$
and sum up to the whole $\points$.
Our~task is to find a set $\sol \subseteq \sets$
such that $\bigcup \sol = \points$ and the size of $\sol$ is minimum possible.
\end{defi}

$\SetCover$ is a classical example of an NP-complete problem,
which has been proven
in \cite{set_cover_inapproximation} to be
inapproximable with factor $(1-o(1))\ln n$ assuming P~$\neq$~NP
(which is a stronger result than APX-hardness),
and W[2]-complete with the natural parameterization,
see Theorem 13.21 in \cite{platypus_book}.
However, restricting the problem to various specialized settings
can lead to more tractable special cases.
In this thesis we take a closer look at the $\GeometricSetCover$ problem
in the plane, where elements to cover are~points in the plane
and sets to cover them with are geometric objects.

\begin{defi}
\textbf{$\SegmentSetCover$} is $\GeometricSetCover$ where
objects that we cover the points with are segments in the plane.
\end{defi}

\paragraph{Approximation}
Over the years there has been a lot of work related to approximation
algorithms for $\GeometricSetCover$. Notably,
$\GeometricSetCover$ with unweighted unit disks admits a PTAS (see
Corollary 1.1 in \cite{unit_disks}). When we consider the same problem
with weighted unit disks (or unit squares), the problem admits a QPTAS
\cite{settling_apx_hardness}, see also \cite{voronoi_true}.
On the other hand, \cite{rectangles_apx_hard} 
proved that $\GeometricSetCover$ with unweighted axis-parallel fat rectangles
is APX-hard; they also show similar hardness
for $\GeometricSetCover$ with many other standard geometric objects.

\paragraph{Parameterization}
We consider $\GeometricSetCover$ 
parameterized by the size of solution.
$\GeometricSetCover$ with unit squares was first proven to be W[1]-hard 
in \cite{marx05} (Theorem 5). A later follow-up work \cite{voronoi}
shows that there is an~algorithm running in time $n^{\mathcal{O}(\sqrt{k})}$
that solves $\GeometricSetCover$ with unit squares or disks
and that there is no algorithm running in time $f(k) \cdot n^{o(\sqrt{k})}$
for any computable $f$ under the~Exponential-Time Hypothesis,
so this is a tight bound for this problem.

We also consider parameterization of weighted problems.
There does not seem to be a~consensus of what parameterization
in the weighted setting is exactly; there
was an attempt to introduce a quite complicated general
framework of weighted parameterized setting in \cite{weighted_framework}.
Kernels for several well-known weighted problems
such as \textsc{Weighted Subset Sum} or~\textsc{Weighted Knapsack} are presented in \cite{kernel_weighted}.
Another work \cite{weighted_flow} considers weighted
parameterization of \textsc{Weighted Directed Feedback Set} and \textsc{Weighted $st$-Cut}.

\paragraph{$\delta$-extension}
In this paper, we focus on $\SegmentSetCover$ with $\delta$-extension.
$\delta$-extension is a problem relaxation method based on the
$\delta$-shrinking model which was introduced in \cite{shrinking_original}
to provide interesting results for
the \textsc{Maximum Weight Independent Set of Rectangles} problem.
In this problem one is given a family of weighted rectangles
and needs to find a set of non-overlapping rectangles
with the largest possible total weight.
In the $\delta$-shrinking relaxed problem
the returned set of rectangles must be non-overlapping
after all the rectangles are shrunk by a tiny fraction $\delta$
towards the centre of symmetry.
This problem is easier, because we compare the
weight of the obtained solution
to the optimum result before the shrinking. It might
even lead to finding a set with result better than the optimum
for the original problem.
The authors in~\cite{shrinking_original} present a PTAS
for \textsc{Maximum Weight Independent Set of Rectangles} with $\delta$-shrinking,
which was later improved to an EPTAS in \cite{shrinking1}, alongside
with presenting a new FPT algorithm
for this problem with the natural parameterization.
Later, a similar $\delta$-shrinking model was used in \cite{shrinking2}
to present a PTAS for
\textsc{Maximum Weight Independent Set of Polygons} with $\delta$-shrinking.

\newcommand{\Int}{\mathsf{Int}}
\begin{defi}
\label{definition:delta_extension}
For any $\delta > 0$ and a centre-symmetric convex object $L$ with
centre of symmetry $S = (x_s, y_s)$,
the \textbf{$\delta$-extension} of $L$ is the open set of points:
$$L^{+\delta} = \Int\{(1 + \delta)\cdot(x - x_s, y - y_s) + (x_s, y_s) : (x, y) \in L\},$$
where $\Int$ denotes the interior of a set of points.
That is, $L^{+\delta}$ is interior of the image of $L$ under homothety centred
at $S$ with scale $(1+\delta)$.
\end{defi}

Analogous to $\delta$-shrinking,
$\delta$-extension provides a framework for relaxing
\textsc{Geometric} \textsc{Set} \textsc{Cover} problems, where we allow the returned set of
objects $\sol$ to \textit{almost} cover the points in the universe
by requiring that they are covered by $\sol$ after $\delta$-extension,
i.e. by the set $\sol^{+\delta}$.
The same concept could be used for \textsc{Geometric Hitting Set} problems.
 
For a longer discussion of this concept see Section
\ref{section:def:delta_extension}.

Similar model is used to prove that $\GeometricSetCover$ with fat polygons
relaxed with $\delta$-extension admits an EPTAS \cite{harpeled12}.
The $\delta$-extension model presented there is well-defined only
for fat polygons. An object is extended by all the points that
are at distance to the closest point in the object $P$
no larger than $\delta\cdot \mathsf{rad}(P)$, where $\mathsf{rad}(P)$
is the largest radius of a circle inscribed into $P$.
Since segments do not have any circle inscribed into them,
the definition presented there cannot be utilized
for the setting of segments considered here.
Polygons extended by $\delta$-extension
defined in Definition \ref{definition:delta_extension}
covers a superset of points that the polygon extended
by $\delta$-extension defined in \cite{harpeled12}.
Since our definition is more permissive for any polygon,
the EPTAS from \cite{harpeled12}
also works for polygons extended
according to our definition of $\delta$-extension.

\section{Our contribution}
In this thesis we make the following contributions.

We show that approximation of $\SegmentSetCover$,
even if segments are axis-parallel and we relax the problem with  $\frac{1}{2}$-extension,
is APX-hard (Theorem \ref{segment_cover_apx_hard}).

\begin{restatable}{tw}{segmentCoverApxHard}{
\label{segment_cover_apx_hard}
	\textbf{($\SegmentSetCover$ is APX-hard)}.	
	$\SegmentSetCover$
	is APX-hard even when 
	relaxed with $\frac{1}{2}$-extension
	and segments are axis-parallel.
	That is, assuming $P\neq NP$, there does not exist a PTAS
	for this problem.
}\end{restatable}

Theorem \ref{segment_cover_apx_hard} implies the following.
Note that segments are just degenerated rectangles.

\begin{corollary}{
\label{rectangle_cover_apx_hard}
	\textbf{($\GeometricSetCover$ with rectangles is APX-hard)}.	
	\textsc{Geometric} \textsc{Set} \textsc{Cover}
	with axis-parallel rectangles is APX-hard
	even when relaxed with $\frac{1}{2}$-extension.
}\end{corollary}

This expands the previous result of \cite{rectangles_apx_hard} 
that $\GeometricSetCover$
with axis-parallel fat rectangles is APX-hard,
we improved the result that rectangles no longer
have to be fat (Corollary \ref{rectangle_cover_apx_hard})
and it holds when problem is relaxed with $\frac{1}{2}$-extension.
It also proves that the assumption in \cite{harpeled12}
about polygons being fat is necessary, because
covering with arbitrary polygons with $\frac{1}{2}$-extension is APX-hard.

We also provide two FPT algorithms for parameterized $\SegmentSetCover$ 	
(Theorem~\ref{segment_cover_fpt})
and with $\WeightedSegmentSetCover$ relaxed with $\delta$-extension
(Theorem~\ref{fpt_weighted_segment}).

\begin{restatable}{tw}{segmentCoverFpt}{
	\label{segment_cover_fpt}
	\textbf{(FPT for $\SegmentSetCover$).}
	There exists an algorithm that given a family $\sets$ of
	segments (in any direction),
	a set of points $\points$
	and a parameter $k$,
	runs in time ${k^{\mathcal{O}(k)} (|\points|\cdot|\sets|)^2}$,
	and outputs a solution $\sol \subseteq \sets$
	such that $|\sol| \le k$ and $\sol$ covers all points in~$\points$,
	or determines that such a set $\sol$ does not exist.
}\end{restatable}

\begin{restatable}{tw}{fptWeightedSegment}{
	\label{fpt_weighted_segment}
	\textbf{(FPT for $\WeightedSegmentSetCover$ with $\delta$-extension).}
	There exists an algorithm that given a family $\sets$ of
	$n$ weighted segments (in any direction),
	a set of $m$ points $\points$, and parameters $k$ and $\delta > 0$,
	runs in time $f(k, \delta) \cdot (nm)^c$ for some computable function $f$ and a constant $c$ and
	outputs a set $\sol$ such that:
	\begin{itemize}
	\item $\sol \subseteq \sets$,
	\item $|\sol| \le k$,
	\item $\sol^{+\delta}$ covers all points in $\points$,
	\item the weight of $\sol$ is not greater than the weight
	of an optimum solution of size at most $k$
	for this problem without $\delta$-extension,
	\end{itemize}
	or determines that there is no set $\sol$ with $|\sol| \le k$
	such that $\sol$ covers all points in $\points$.
}\end{restatable}

On the other hand, we prove that $\WeightedSegmentSetCover$ 
is W[1]-hard even when segments are limited to 3 directions (Theorem~\ref{w1_hard})
and assuming ETH there does not exist algorithm for this problem
that runs in time ${f(k)(|\points| + |\sets|)^{o(\sqrt{k})}}$.
See Figure \ref{tab:weighted_fpt} for a summary of parameterized
results for $\WeightedSegmentSetCover$.
Similar table for unweighted problem is present in Figure \ref{tab:unweighted_fpt}.

\begin{restatable}{tw}{wOneHard}
\label{w1_hard}
	\textbf{($\WeightedSegmentSetCover$ is W[1]-hard).}
	Consider the problem of covering a set $\points$ of points
	by selecting at most $k$ segments
	from a set of segments $\sets$ 
	with non-negative weights $w : \sets \rightarrow \mathbb{R^+}$
	so that the weight of the cover is minimal.
	Then this problem is W[1]-hard when parameterized by $k$ and
	assuming ETH, there is no algorithm for this
	problem with running time
	$f(k)\cdot(|\points| + |\sets|)^{o(\sqrt{k})}$
	for any computable function $f$.
	Moreover, this holds even if all segments in $\sets$
	are axis-parallel or right-diagonal.
\end{restatable}

See Section \ref{section:def:geometric__set_cover}
for exact definitions of axis-parallel and right-diagonal segments.

This result is particularly interesting,
because the problem without weights is FPT,
while the weighted variant is W[1]-hard.
Moreover, $\delta$-extension allowed us to provide an FPT algorithm
for the problem which is W[1]-hard otherwise.

Note that the result of Theorem \ref{w1_hard} is not tight:
there exists a simple algorithm 
running in time ${f(k)(|\points| + |\sets|)^k}$.
So the question whether there exists an algorithm
for this problem running in time ${f(k)\cdot(|\points| + |\sets|)^{o(k)}}$
is still open.

Permissive FPT is a relaxed FPT problem, where 
we need to find solution of \textit{any} size in FPT-time,
but we compare it to the optimum solution of size at most $k$.
Idea for permissive FPT in local search was presented
in \cite{permissive_problem1}, \cite{permissive_problem2}.
Theorem \ref{w1_hard} can be improved to show that a permissive FPT
algorithm does not exist.
This is formulated precisely in Theorem \ref{permissive_w1_hard}.

\begin{figure}[h]
\begin{center}
\begin{tabular}{ | c | c | c | }
\hline
                & exact     & $\delta$-extension \\ 
\hline                
 axis-parallel   & ? & FPT* \\  
\hline                
 3 directions    & W[1]-hard & FPT* \\  
\hline                
 any direction   & W[1]-hard* & FPT \\
\hline                
\end{tabular}
\caption{Our results for $\WeightedSegmentSetCover$
parameterized by the size of solution.
Results marked with * are not explicitly given in this thesis,
but they trivially follow from stronger results shown in the other cells of the table.}
\label{tab:weighted_fpt}
\end{center}
\end{figure}

\begin{figure}[h]
\begin{center}
\begin{tabular}{ | c | c | c | }
\hline
                & exact     & $\delta$-extension \\ 
\hline                
 axis-parallel   & FPT* & FPT* \\  
\hline                
 3 directions    & FPT* & FPT* \\  
\hline                
 any direction   & FPT  & FPT* \\
\hline                
\end{tabular}
\caption{Our results for unweighted $\SegmentSetCover$
parameterized by the size of solution.
Results marked with * are not explicitly given in this thesis,
but they trivially follow from stronger results shown in the other cells of the table.}
\label{tab:unweighted_fpt}
\end{center}
\end{figure}

\paragraph{Future work.} There are two aforementioned problems
that relate to Theorem \ref{w1_hard} and were not solved in this thesis.
We have not given a W[1]-hardness proof
for \textsc{Weighted} \textsc{Segment} \textsc{Set} \textsc{Cover} where segments are limited to 3 directions,
but the segments in the construction may be also right-diagonal.
However, it may be possible to improve this construction to use segments
in 2 directions instead of 3 directions. 
The other question is what is the tight bound for this problem.
The simple algorithm solving
this problem is running in time ${f(k)(|\points| + |\sets|)^k}$,
while our lower bound refutes running time ${f(k)(|\points| + |\sets|)^{\sqrt{k}}}$.

Another problem to consider is whether
\textsc{Geometric Hitting Set} relaxed with $\delta$-extension
can yield some better results.

\chapter{Preliminaries}

In this chapter we present some basic definitions that
will be used later.

\section{Geometric set cover}
\label{section:def:geometric__set_cover}
Whenever speaking about geometric set cover,
we consider it in the 2-dimensional plane.

In the geometric set cover problem we are are given
$\sets$ --- a set of objects, which are connected
subsets of the plane and $\points$ --- a set of points in the plane.
The task is to choose $\sol \subseteq \sets$ such that
every point in $\points$ is inside some object from $\sol$
and $|\sol|$ is minimized. We will mostly consider the case where
$\sets$ consists of segments in the plane.

In the weighted setting, there is some given weight function
$f : \sets \rightarrow \mathbb{R^+}$
and we would like to find a solution $\sol$
that minimizes $\sum_{R \in \sol} f(R)$.

\begin{defi}
Segment is \textbf{axis-parallel} if it lies on line that is
either horizontal $x = c$ or vertical $y = c$.
\end{defi}

\begin{defi}
	A line is \textbf{right-diagonal} if it is
	described by linear function $x + y = d$ for some $d \in \mathbb{R}$.
	Segment is \textbf{right-diagonal} if its
	direction is a right-diagonal line.
\end{defi}

\section{Parameterization}

In the parameterized setting of the Geometric Set Cover
for a given $k$,
our task is to either find a solution $\sol$ such that $|\sol| \le k$
or decide that there is no such solution.

\begin{defi}
A \textbf{Fixed-parameter Tractable (FPT)} algorithm 
for a problem with parameter $k$ and instance size $n$
is an algorithm running in time $f(k) \cdot n^c$
for some constant $c$ and some computable function $f$.
\end{defi}

\begin{defi}
\label{definition:cnf}
Boolean formula is in \textbf{conjunctive normal form (CNF)} if
it is a conjunction of one or more formulas,
which are disjunction of literals.
\textbf{$k$-CNF} formula is a CNF formula, where
every disjunction consists of at most $k$ literals.
\end{defi}

\begin{defi}
\textbf{$k$-SAT} problem is 
a boolean satisfiability problem of $k$-CNF formulas.
Given $k$-CNF formula, one must answer if there
exists any variables assignment that satisfies the formula.
\end{defi}

\begin{defi}
For $k \ge 3$ set us define $S_k$ as a set of constants $\sigma$
such that there exists an algorithm solving $k$-SAT running in time
$\mathcal{O}^{*}(2^{\sigma n})$.
Set us define $s_k$ as the infimum  of the set $S_k$.

\textbf{Exponential Time Hypothesis (ETH)} is a conjecture
that $s_3 > 0$. This conjecture implies that
there does not exist an algorithm solving 3-SAT
running in time $2^{o(n)}$.
\end{defi}

We provide the main theorem that we use in this thesis for W[1]-hard
problems. To see the definition of a W[1]-hard problem,
see Chapter 13.3 of \cite{platypus_book}.

\begin{tw}
Problem parameterized by $k$ is \textbf{W[1]-hard} if assuming ETH there
does no algorithm solving this problem running in time
$f(k)\cdot n^{o(k)}$.
\end{tw}

\section{Approximation}

Let us recall some definitions related to optimization problems.

\begin{defi}
A \textbf{polynomial-time approximation scheme (PTAS)}
for a minimization problem $\Pi$
is a family of algorithms $\cal{A}_\epsilon$ for
every $\epsilon > 0$
such that $\cal{A}_\epsilon$ takes an instance $I$ of~$\Pi$
and in polynomial time
finds a solution that is within a factor
of ($1+\epsilon$) of being optimal.
This means that the reported solution has weight at most
$(1+\epsilon)opt(I)$, where $opt(I)$ is the weight
of an optimal solution to $I$.
\end{defi}

\begin{defi}
A problem $\Pi$ is \textbf{APX-hard} if assuming P $\neq$ NP,
there exists $\epsilon > 0$
such that there is no polynomial-time $(1+\epsilon)$-approximation algorithm
for $\Pi$.
\end{defi}

\section{$\delta$-extension}
\label{section:def:delta_extension}

Another idea presented here, which can be utilized only when considering
the problems with geometric objects,
is $\delta$-extension.
We define it specifically for the geometric set cover problem
with convex centre-symmetric objects.

Intuitively, we consider a problem with slightly larger objects,
which makes the instance more permissive.
However, we aim to find a solution that
is not larger than the
optimum solution to the original problem,
so this is substantially easier than just
solving the problem for the larger objects.
It may even be the case
that we are able to find a solution
of size smaller than the optimum solution
to the original problem.

Formal definition of $\delta$-extended objects.
is present in Definition
\ref{definition:delta_extension}.

The geometric set cover problem with $\delta$-extension
is a version of geometric set cover with
the following modifications.
\begin{itemize}
\item We need to cover all the points in $\points$
by selecting objects from $\{P^{+\delta} : P \in \sets\}$ (which always 
include no fewer points than the objects
before $\delta$-extension).
\item We look for a solution that is not larger than the optimum
solution to the original problem.
Note that it does not need to be an optimal solution in
the modified problem.
\end{itemize}
Formally, we have the following.

\begin{defi}
The \textbf{geometric set cover problem
with $\delta$-extension} is the problem where for an input instance
$I=(\sets, \points)$ of geometric set cover,
the task is to output a solution $\mathcal{R} \subseteq \sets$
such that the~$\delta$-extended set
$\{ R^{+\delta} :  R \in \mathcal{R} \}$ covers $\points$
and is not larger than the optimal solution to the~problem without
extension, i.e.~$|\mathcal{R}| \le |opt(I)|$.
\end{defi}

At last, we formulate a definition of the
polynomial-time approximation scheme (PTAS)
for a problem with $\delta$-extension.

\begin{defi}
A \textbf{PTAS for geometric set cover 
with $\delta$-extension} is a family of algorithms
$\{\mathcal{A}_{\delta, \epsilon}\}_{\delta, \epsilon > 0}$ that
each takes as an input instance $I=(\sets, \points)$
of geometric set cover where objects are centre-symmetric and strongly convex,
and in polynomial-time outputs a solution $\mathcal{R} \subseteq \sets$
such that the $\delta$-extended set
$\{ R^{+\delta} :  R \in \mathcal{R} \}$ covers $\points$
and is within a $(1+\epsilon)$ factor of the optimal
solution to this problem without
extension, i.e.~$(1+\epsilon)|\mathcal{R}| \le |opt(I)|$.
\end{defi}

\section{Weighted Geometric Set Cover}

In this thesis we also consider a weighted Geometric Set Cover problem,
which is a combination
of the weighted and parameterized setting described in 
\ref{section:def:geometric__set_cover}.
We already argued in the introduction
that there is no consensus of how it is defined, but when we discuss the
weighted parameterized setting we will consider the following
definition. There is a given weight function
$f : \sets \rightarrow \mathbb{R^+}$
and we would like to find a solution $\sol$,
such that $|\sol| \le k$
that minimizes $\sum_{R \in \sol} f(R)$ among such sets $\sol$.

\begin{defi}
The \textbf{weighted geometric set cover problem
with $\delta$-extension} is the problem where for an input instance
$I=(\sets, \points, f)$ of weighted geometric set cover,
the task is to output a solution $\mathcal{R} \subseteq \sets$
such that the~$\delta$-extended set
$\{ R^{+\delta} :  R \in \mathcal{R} \}$ covers $\points$
and it has weight not larger than the optimal solution to the~problem without
extension, i.e.~$\sum_{R \in \mathcal{R}} f(R) \le |opt(I)|$.
\end{defi}

We also consider weighted parameterized setting with $\delta$-extension,
which we formally define below.

\begin{defi}
The \textbf{weighted geometric set cover problem
with $\delta$-extension parameterized by the size of a solution}
is a problem where for an input instance
${I=(\sets, \points, f, k)}$ of weighted geometric set cover
parameterized by the size of a solution $k$,
the task is to output a solution $\mathcal{R} \subseteq \sets$
such that the~$\delta$-extended set
$\{ R^{+\delta} :  R \in \mathcal{R} \}$ covers $\points$,
uses no more than $k$ sets, i.e. $|\sol| \le k$
and it has weight not larger than the optimal solution to the~problem without
extension, i.e.~$\sum_{R \in \mathcal{R}} f(R) \le |opt(I)|$.
\end{defi}

\chapter{APX-hardness of geometric set cover problem}
\newcommand{\setCoverInstance}{(\points, \sets)}
\newcommand{\true}{\texttt{true}}
\newcommand{\false}{\texttt{false}}

\label{chapter:segment_apx}

In this section we analyze whether there exists 
a PTAS for geometric set cover for rectangles.
We show that we can restrict this problem
to a very simple setting:
segments parallel to axes and allow (1/2)-extension,
and the problem is still APX-hard.
Note that segments are just degenerated rectangles
with one side being very narrow.


Our results can be summarized in the following
theorem and this section aims to prove it.

\begin{tw}{
\label{segment_cover_apx_hard}
	\textbf{(axis-parallel segment set cover with 1/2-extension is APX-hard)}.	
	Unweighted geometric set cover
	with axis-parallel segments in 2D (even with 1/2-extension)
	is APX-hard.
	That is, assuming $P\neq NP$, there does not exist a PTAS
	for this problem.
}\end{tw}
 
Theorem \ref{segment_cover_apx_hard} implies the following.

\begin{corollary}{
\label{rectangle_cover_apx_hard}
	\textbf{(rectangle set cover is APX-hard)}.	
	Unweighted geometric set cover
	with axis-parallel rectangles (even with 1/2-extension) is APX-hard.
}\end{corollary}


We prove Theorem \ref{segment_cover_apx_hard}
by taking a problem that is APX-hard
and showing a reduction.
For this problem we choose
MAX-(3,3)-SAT which we define below.


\section{MAX-(3,3)-SAT and statement of reduction}
\begin{defi}
\textbf{MAX-3SAT} is the following maximization problem. We are given a 3-CNF
formula, and need to find an assignment of variables
that satisfies the most clauses.
\end{defi}

\begin{defi}
\textbf{MAX-(3,3)-SAT} is a variant of MAX-3SAT with an additional
restriction that every variable appears in exactly 3 clauses
and every clause contains exactly 3 literals of 3 different variables.
Note that thus, the number of clauses is equal to the number of variables.
\end{defi}

In our proof of Theorem \ref{segment_cover_apx_hard} we use
hardness of approximation of MAX-(3,3)-SAT proved
in \cite{hastad} and described in
Theorem \ref{hastadtheorem} below.

\begin{defi}[$\alpha$-satisfiable MAX-3SAT formula]
MAX-3SAT formula with $m$ clauses is at most $\alpha$-satisfiable, if
every assignment of variables satisfies no more than $\alpha m$
clauses. 
\end{defi}

\begin{tw}{
	\label{hastadtheorem}
	\textbf{\cite{hastad}}
	For any $\epsilon > 0$, it is NP-hard to distinguish satisfiable
	\linebreak
	(3,3)-SAT formulas from
	at most
	\mbox{$(7/8 + \epsilon)$-satisfiable}
	(3,3)-SAT formulas.
}\end{tw}


Given an instance $I$ of MAX-(3,3)-SAT,
we construct an instance $J$ of 
axis-parallel segment set cover problem
such that for a sufficiently small $\epsilon > 0$,
a polynomial time $(1+\epsilon)$-approximation algorithm for $J$
would be able to distinguish  whether an instance $I$ of MAX-(3,3)-SAT
is fully satisfiable
or is at most $(7/8 + \epsilon)$-satisfiable.
However, according to Theorem \ref{hastadtheorem} the latter problem
is NP-hard.
This would imply P = NP, contradicting the assumption.

The following lemma encapsulates the properties
of the reduction described in this section,
and it allows us to prove Theorem \ref{segment_cover_apx_hard}.

\begin{lemma}{
	\label{apxconstruction}
	Given an instance $S$ of  MAX-(3,3)-SAT 
	with $n$ variables and optimum value $opt(S)$,
	we can construct an instance $I$ of geometric set cover with
	axis-parallel segments in 2D such that:
	\begin{enumerate}[label={(\arabic*)}]
	\item For every solution $X$ of instance $I$,
	there exists a solution to $S$ that satisfies at least  $15n - |X|$
	clauses.
	
	\item For every solution to instance $S$ that satisfies $w$ clauses,
	there exists a solution to $I$ of size $15n - w$.
	
	\item \label{lemma:apxconstruction:enumerate:extensions}
	Every solution with $1/2$-extensions of $I$
	is also a solution to the original instance $I$.
\end{enumerate}
Therefore, the optimum size of a solution to $I$
is $opt(I) = 15n - opt(S)$. 
	
}\end{lemma}

We prove Lemma \ref{apxconstruction} in
subsequent sections, but meanwhile let us prove
Theorem \ref{segment_cover_apx_hard} using Lemma \ref{apxconstruction}
and Theorem \ref{hastadtheorem}.

\begin{proof}[Proof of Theorem \ref{segment_cover_apx_hard}]
Consider any $0 < \epsilon < 1/(15 \cdot 8)$.

Let us assume that there exists a polynomial-time
$(1+\epsilon)$-approximation algorithm
for unweighted geometric set cover with axis-parallel segments in 2D
with (1/2)-extensions.
We construct an algorithm that solves the problem stated in 
Theorem \ref{hastadtheorem}, thereby proving that P~=~NP.

Take an instance~$S$ of MAX-(3,3)-SAT to be distinguished
and construct an instance of geometric set cover $I$
using Lemma \ref{apxconstruction}.
We now use the $(1+\epsilon)$-approximation algorithm
for geometric set cover on $I$.
Denote the size of the solution returned by this algorithm as $approx(I)$.
We prove that 
if in $S$
one can satisfy at most $(\frac{7}{8}+\epsilon)n$ clauses,
then $approx(I) \ge 15n - (\frac{7}{8} + \epsilon)n$
and if $S$ is
satisfiable, then $approx(I) < 15n - (\frac{7}{8} + \epsilon)n$.


\textbf{Assume $S$ satisfiable.}
From the definition of $S$ being satisfiable, we have:
$$opt(S) = n.$$

From Lemma \ref{apxconstruction} we have:

$$opt(I) = 14n.$$

Therefore,
$$approx(I) \le (1+\epsilon)opt(I) = 14n(1+\epsilon)
	= 14n + 14\epsilon\cdot n =$$ 
	$$= 14n + (15\epsilon - \epsilon)n < 
  14n + \left(\frac{1}{8} - \epsilon\right)n 
= 15n - \left(\frac{7}{8} + \epsilon\right)n.$$
\textbf{Assume $S$ is at most 
$\left(\frac{7}{8} + \epsilon\right)$ satisfiable.}
From the defintion of $S$ being at most 
$\left(\frac{7}{8} + \epsilon\right)n$ satisfiable, we have:
$$opt(S) \le \left(\frac{7}{8} + \epsilon\right)n$$

From Lemma \ref{apxconstruction} we have:
$$opt(I) \ge 15n - \left(\frac{7}{8} + \epsilon\right)n$$

Since a solution to $I$ with $\frac{1}{2}$-extension is
also a solution without any extention, by 
Lemma \ref{apxconstruction} \ref{lemma:apxconstruction:enumerate:extensions}, we have:

$$approx(I) \ge opt(I) = 15n - \left(\frac{7}{8} + \epsilon\right)n$$


Therefore, by using the assumed $(1+\epsilon)$-approximation
algorithm,
it is possible to distinguish the case when
$S$ is satisfiable: from the case when it is
at most $(\frac{7}{8} + \epsilon)n$ satisfiable,
it suffices to compare $approx(I)$ with $15n - (\frac{7}{8}+\epsilon)n$.
Hence, the assumed approximation algorithm cannot exist, unless P = NP.
\end{proof}

\section{Reduction}
\label{construction_description}
We proceed to the proof of Lemma \ref{apxconstruction}.
That is, we show a reduction from the MAX-(3,3)-SAT problem
to geometric set cover with segments
parallel to axis. Moreover, the obtained instance
of geometric set cover will be robust
to 1/2-extensions (have the same optimal solution
after 1/2-extension).

The construction will be composed of 2 types of gadgets:
\textbf{VARIABLE-gadgets} and \textbf{CLAUSE-gadgets}.
CLAUSE-gadgets will be constructed using two \textbf{OR-gadgets}
connected together.

\subsection{VARIABLE-gadget}

VARIABLE-gadget is responsible for choosing the value of a variable
in a CNF formula. It allows two minimum solutions of size 3 each.
These two choices correspond to the two Boolean values of the variable
corresponding to this gadget.

\paragraph{Points.}

Define points $a,b,c,d,e,f,g,h$ as follows, where $L = 22n$:


\newcommand{\pointsVarNoArg}{\mathsf{pointsVariable} }
\newcommand{\pointsVar}[1]{\mathsf{pointsVariable}_{#1} }
\newcommand{\chooseVar}[2]{\mathsf{chooseVariable}^{#1}_{#2} }
\newcommand{\segmentsVar}[1]{\mathsf{segmentsVariable}_{#1} }

\definecolor{x_true_colour}{RGB}{40, 40, 255}
\definecolor{x_false_colour}{RGB}{255, 40, 40}

{\tikzset{point/.style={
    circle, draw=black, fill, fill=black, minimum size=4pt,inner sep=0pt, outer sep=0pt,
    prefix after command= {\pgfextra{\tikzset{every
    label/.style={label distance=0.05cm,text=black}}}}
    }
}

{\tikzset{point_not_cover/.style={
    circle, draw=black, fill, fill=white, minimum size=4pt,inner sep=0pt, outer sep=0pt,
    prefix after command= {\pgfextra{\tikzset{every
    label/.style={label distance=0.05cm,text=black}}}}
    }
}

\begin{figure}[h]
\centering
\begin{tikzpicture}
\tikzmath{
\stepx=1.5;
\stepy=1;
\y1=0;
\y2=\y1+\stepy;
\y3=\y2+\stepy;
\x1=0;
\x2=\x1+\stepx;
\x3=\x2+\stepx;
\xend=\x3+2*\stepx;
}


\draw[x_false_colour,very thick] (\x1,\y1) -- (\x3,\y1);
\draw[x_false_colour,very thick] (\x1,\y2) -- (\x2,\y2);
\draw[x_false_colour,very thick] (\x2,\y3) -- (\xend,\y3);
\draw[x_true_colour,very thick] (\x1,\y1) -- (\x1,\y2);
\draw[x_true_colour,very thick] (\x2,\y1) -- (\x2,\y3);
\draw[x_true_colour,very thick] (\x3,\y1) -- (\xend,\y1);

\node[point,label={below:$a_i$}] at (\x1,\y1) {};
\node[point,label={below:$b_i$}] at (\x2,\y1) {};
\node[point,label={below:$c_i$}] at (\x3,\y1) {};
\node[point,label={left:$d_i$}] at (\x1,\y2) {};
\node[point,label={above left:$e_i$}] at (\x2,\y2) {};
\node[point,label={above left:$f_i$}] at (\x2,\y3) {};
\node[point_not_cover,label={right:$g_i$}] at (\xend,\y1) {};
\node[point_not_cover,label={right:$h_i$}] at (\xend,\y3) {};


\end{tikzpicture}
\caption{\textbf{VARIABLE-gadget.}
We denote the set of points marked with black circles as $\pointsVar{i}$,
and they need to be covered (are part of the set $\points$).
Note that some of the points are not marked as black dots
and exists only to name segments for further reference.
We denote the set of \textcolor{x_false_colour}{red} segments as $\chooseVar{\false}{i}$
and the set of \textcolor{x_true_colour}{blue} segments as $\chooseVar{\true}{i}$.}
\label{fig:apx_choose_variable}
\end{figure}


\begin{center}
\begin{tabular}{ l l l l}
	$a = (-3L, 0)$ &
	$b = (-2L, 0)$ & 
	$c = (-L, 0)$ & 
	$d = (-3L, 1)$ \\  
	$e = (-2L, 1)$ & 
	$f = (-2L, 2)$ &
	$g = (L, 0)$ &
	$h = (L, 2)$
\end{tabular}
\end{center}


Let us define:
$$\pointsVarNoArg =  \{a, b, c, d, e, f\}$$
and, for any $1 \le i \le n$,
$$\pointsVar{i} = \pointsVarNoArg + (0, 4i).$$

We denote $a_i := a + (0,4i)$ etc.

\paragraph{Segments.}

\newcommand{\xTrueSegmentDef}[1]{(c_{#1}, g_{#1})}
\newcommand{\xFalseSegmentDef}[1]{(f_{#1}, h_{#1})}
\newcommand{\xTrueSegment}[1]{\mathsf{xTrueSegment}_{#1}}
\newcommand{\xFalseSegment}[1]{\mathsf{xFalseSegment}_{#1}}
\newcommand{\orTrueSegment}[2]{(t_{#1, #2}, v_{#1, #2})}

Let us define:

$$\chooseVar{true}{i} :=\{ (a_i, d_i), (b_i, f_i), (c_i, g_i)\},$$
$$\chooseVar{false}{i} := \{(a_i, c_i), (d_i, e_i), (f_i, h_i)\},$$
$$\segmentsVar{i} := \chooseVar{true}{i} \cup \chooseVar{false}{i}.$$

We also name two of these segment for future reference:
$\xTrueSegment{i} := \xTrueSegmentDef{i},$
$\xFalseSegment{i} := \xFalseSegmentDef{i}.$

\begin{lemma}
\label{choose_variables_solution}
For any $1 \le i \le n$, points in $\pointsVar{i}$
can be covered using 3 segments from $\segmentsVar{i}$.
\end{lemma}

\begin{proof}
We can use either set $\chooseVar{true}{i}$ or $\chooseVar{false}{i}$.
\end{proof}

\begin{lemma}
\label{choose_variables_no_less}
For any $1 \le i \le n$, points in $\pointsVar{i}$
can not be covered with fewer than 3 segments from $\segmentsVar{i}$.
\end{lemma}

\begin{proof}
No segment of $\segmentsVar{i}$ covers more than one point from
$\{d_i, f_i, c_i\}$, therefore $\pointsVar{i}$ can
not be covered with fewer than 3 segments.
\end{proof}

\begin{lemma}
\label{choose_variables_both}
For every set $A \subseteq \segmentsVar{i}$ such that $A$ covers $\pointsVar{i}$
and $\xTrueSegment{i}, \xFalseSegment{i} \in A$,
it holds that $|A| \ge 4$.
\end{lemma}
\begin{proof}
No segment from $\segmentsVar{i}$ covers more than one point from
$\{a_i, e_i\}$,
therefore 
$\pointsVar{i} - \{c_i, f_i, g_i, h_i\}$
can not be covered with fewer than 2 segments.
\end{proof}


\subsection{OR-gadget}

OR-gadget connects input and output segments (see Figure \ref{fig:apx_or_gadget})
in a way that is supposed to simulate a binary $or$ function.

Input segments are the only segments that cover points outside of the gadget,
as their left ends lie outside of it.
Point $v_{i,j}$ is the only one that can be covered
by segments that do not belong to the gadget.

The OR-gadget has the property that every set of segments
that covers all the points in the gadget uses at least 3 segments from it..
Moreover, the output segment belongs to the solution to size 3
only if at least one of the input segments belong to the solution.
Therefore, optimum solutions restricted to the OR-gadget behave
like a binary $or$ function for the input segments.


\paragraph{Points.}

\newcommand{\chooseOr}[3]{\mathsf{chooseOr}^{#1}_{#2,#3}}
\newcommand{\orMoveVariable}[2]{\mathsf{orMoveVariable}_{#1,#2}}
\newcommand{\pointsOr}[2]{\mathsf{pointsOr}_{#1,#2}}
\newcommand{\segmentsOr}[2]{\mathsf{segmentsOr}_{#1,#2}}

%https://davidmathlogic.com/colorblind/#%23646464-%2389DA6D-%23FFCF3C-%2338D1F1-%23FF0909-%23238CD2
\definecolor{environment}{RGB}{100,100,100}
\definecolor{move_variable1}{RGB}{137, 218, 109}
\definecolor{move_variable2}{RGB}{255, 207, 60}
\definecolor{choose_true1}{RGB}{56, 209, 241}
\definecolor{choose_true2}{RGB}{35, 140, 210}
\definecolor{choose_false}{RGB}{255, 9, 9}

{\tikzset{point/.style={
    circle, draw=black, fill, fill=black, minimum size=4pt,inner sep=0pt, outer sep=0pt,
    prefix after command= {\pgfextra{\tikzset{every
    label/.style={label distance=0.05cm,text=black}}}}
    }
}

\begin{figure}[h]
\centering
\def\svgwidth{0.5\columnwidth}
\begin{tikzpicture}
\tikzmath{
\stepx=1.5;
\stepy=1.5;
\xbeg=0;
\x1=\xbeg+3*\stepx;
\x2=\x1+\stepx;
\x3=\x2+\stepx;
\xend=\x3+\stepx;
\y1=0;
\y2=\y1+\stepy;
\y3=\y2+\stepy;
\y4=\y3+\stepy;
\y5=\y4+\stepy;
}

\draw[environment,ultra thick] (\xbeg,\y1) -- (\x1,\y1) node[black,pos=0.15, above] {$input_x$};
\draw[environment,ultra thick] (\xbeg,\y5) -- (\x1,\y5) node[black,pos=0.15, above] {$input_y$};
\draw[move_variable1, ultra thick] (\x1,\y1) -- (\x1,\y3);
\draw[move_variable2, ultra thick] (\x1,\y5) -- (\x1,\y3);
\draw[choose_true1, ultra thick] (\x1,\y2) -- (\x3,\y2);
\draw[choose_true1, ultra thick] (\x1,\y4) -- (\x3,\y4);
\draw[choose_true2, ultra thick] (\x3,\y3) -- (\xend,\y3) node [black,pos=0.5, below] {$output$};
\draw[choose_false, ultra thick] (\x2,\y2) -- (\x2,\y4);
\draw[choose_false, ultra thick] (\x3,\y2) -- (\x3,\y4);

\node[point,label={above left:$l_{i,j}$}] at (\x1,\y1) {};
\node[point,label={left:$m_{i,j}$}] at (\x1,\y2) {};
\node[point,label={left:$n_{i,j}$}] at (\x1,\y3) {};
\node[point,label={left:$o_{i,j}$}] at (\x1,\y4) {};
\node[point,label={above left:$p_{i,j}$}] at (\x1,\y5) {};

\node[point,label={below:$q_{i,j}$}] at (\x2,\y2) {};
\node[point,label={above:$r_{i,j}$}] at (\x2,\y4) {};
\node[point,label={below:$s_{i,j}$}] at (\x3,\y2) {};
\node[point,label={left:$t_{i,j}$}] at (\x3,\y3) {};
\node[point,label={above:$u_{i,j}$}] at (\x3,\y4) {};
\node[point,label={above:$v_{i,j}$}] at (\xend,\y3) {};

\end{tikzpicture}
\caption{
	\textbf{OR-gadget.} Segments from $\chooseOr{\false}{i}{j}$ are \textcolor{choose_false}{red},
	segments from $\chooseOr{\true}{i}{j}$ are blue
	(both \textcolor{choose_true1}{light blue} and \textcolor{choose_true2}{dark blue}),
	segments from $\orMoveVariable{i}{j}$ are \textcolor{move_variable1}{green} and \textcolor{move_variable2}{yellow}.
	\textcolor{choose_true2}{Dark blue} segment is the $output$ segment.
	\textcolor{environment}{Grey segments} $input_x$ and $input_y$ are input segments that
	are not part of $\segmentsOr{i}{j}$.
}
\label{fig:apx_or_gadget}
\end{figure}


\newcommand{\clauseXFactor}{20}

\begin{center}
	\begin{tabular}{ l l l l}
		$l_0 := (0, 0)$ &
		$m_0 := (0, 1)$ &
		$n_0 := (0, 2)$ &
		$o_0 := (0, 3)$ \\
		$p_0 := (0, 4)$ &
		$q_0 := (1, 1)$ &
		$r_0 := (1, 3)$ &
		$s_0 := (2, 1)$ \\
		$t_0 := (2, 2)$ &
		$u_0 := (2, 3)$ &
		$v_0 := (3, 2)$ &
	\end{tabular}
\end{center}


$$vec_{i, j} := (\clauseXFactor i + 3 + 3j, 4(n+1) + 2j)$$

For integers $i,j$,
define 
$\{ l_{i, j}, m_{i, j} \ldots v_{i, j} \}$
as $\{l_0, m_0 \ldots v_0\}$ shifted by $vec_{i, j}$,
i.e. $l_{i,j} = l_0 + vec_{i,j}$ etc.

Note that $v_{i, 0} = l_{i, 1}$ (see Figure~\ref{fig:apx_clause})
 
$$\pointsOr{i}{j} := 
 \{l_{i, j}, m_{i, j}, n_{i, j}, o_{i, j},
 p_{i, j}, q_{i, j}, r_{i, j}, s_{i, j}, t_{i, j}, u_{i, j} \}
 $$
 
Note that $\pointsOr{i}{j}$ does not include the point $v_{i,j}$
 
\paragraph{Segments.}

We define set of segments in several parts:
 
$$\chooseOr{false}{i}{j} :=
\{ (q_{i, j}, r_{i, j}), (s_{i, j}, u_{i, j})\},$$
$$\chooseOr{true}{i}{j} :=
\{ (m_{i, j}, s_{i, j}), (o_{i, j}, u_{i, j}),
(t_{i, j}, v_{i, j}) \},$$

$$\orMoveVariable{i}{j} :=
\{ (l_{i, j}, n_{i, j}), (n_{i, j}, p_{i, j})\}.$$

Finally all segments in OR-gadget are defined as:

$$\segmentsOr{i}{j} := 
  \chooseOr{false}{i}{j} \cup \chooseOr{true}{i}{j} \cup \orMoveVariable{i}{j}
$$


\begin{lemma}
\label{cover_or_true}
For any $1 \le i \le n, j \in \{0, 1\}$ and 
 $x \in \{l_{i, j}, p_{i, j}\}$, points in
$\pointsOr_{i, j} - \{ x\} \cup \{v_{i, j}\}$
can be covered
with 4 segments from $\segmentsOr{i}{j}$.
\end{lemma}

\begin{proof}
We can do that using one segment from
$\orMoveVariable{i}{j}$, the one that does not cover $x$,
and all segments from $\chooseOr{true}{i}{j}$.
\end{proof}

\begin{lemma}
\label{cover_or_false}
For any $1 \le i \le n, j \in \{0, 1\}$, points in
$\pointsOr{i}{j}$ can be covered
with 4 segments from $\segmentsOr{i}{j}$.
\end{lemma}

\begin{proof}
We can do that using segments from $\orMoveVariable{i}{j} \cup \chooseOr{false}{i}{j}$.
\end{proof}


\subsection{CLAUSE-gadget}
A CLAUSE-gadget is responsible for determining whether
variable values assigned in variable gadgets
satisfy the corresponding clause in the input formula $\phi$.
It has a minimum solution to weight $w$
if and only if the clause is satisfied, i.e. at least one
of the respective variables is assigned the correct value.
Otherwise, its minimum solution has weight $w+1$.
In this way, by analyzing the cost of the minimum solution
for the entire constructed instance, we will be able to tell
how many clauses it was possible to satisfy
in the optimum solution to $\phi$.


\paragraph{Points.}

\newcommand{\pointsClause}[1]{\mathsf{pointsClause}_{#1}}

\definecolor{environment}{RGB}{150,150,150}

{\tikzset{point/.style={
    circle, draw=black, fill, fill=black, minimum size=4pt,inner sep=0pt, outer sep=0pt,
    prefix after command= {\pgfextra{\tikzset{every
    label/.style={label distance=0.05cm,text=black,font=\footnotesize}}}}
    }
}

\begin{figure}[h]
\centering
\begin{tikzpicture}
\tikzmath{
\stepx=1.7;
\stepy=1.5;
\boxsize=1.2*\stepx;
\xbeg=0;
\x1=\xbeg+2*\stepx;
\x2=\x1+\stepx;
\x3=\x2+\stepx;
\x4=\x3+\stepx;
\x5=\x4+\boxsize;
\x6=\x5+\stepx;
\x7=\x6+\boxsize;
\x8=\x7+\stepx;
\y1=0;
\y2=\y1+\stepy;
\y3=\y2+\stepy;
\y4=\y3+2*\stepy;
\y5=\y4+0.5*\stepy;
\y6=\y5+0.5*\stepy;
\y7=\y6+0.5*\stepy;
}

\draw[thick] (\x1,\y3) -- (\x1,\y4);
\draw[thick] (\x2,\y1) -- (\x2,\y6);
\draw[thick] (\x3,\y2) -- (\x3,\y7);
\draw[thick] (\x1,\y4) -- (\x4,\y4);
\draw[thick] (\x2,\y6) -- (\x4,\y6);
\draw[thick] (\x3,\y7) -- (\x6,\y7);
\draw[thick] (\x5,\y5) -- (\x6,\y5);
\draw[thick] (\x7,\y6) -- (\x8,\y6);
\draw [fill=green!50, draw=green!50] (\x4,\y6) rectangle (\x5,\y4) node[pos=0.5] {\scriptsize OR-gadget};
\draw [fill=green!50, draw=green!50] (\x6,\y5) rectangle (\x7,\y7) node[pos=0.5] {\scriptsize OR-gadget};

\draw[environment,thick] (\xbeg, \y1) -- (\x8, \y1) node[black,pos=0.7, below] {$variable_b^{true}$};
\draw[environment,thick] (\xbeg, \y2) -- (\x8, \y2) node[black,pos=0.7, below] {$variable_c^{false}$};
\draw[environment,thick] (\xbeg, \y3) -- (\x8, \y3) node[black,pos=0.7, above] {$variable_a^{true}$};

\node[point,label={above left:$x_{i,0}$}] at (\x1,\y3) {};
\node[point,label={above left:$y_{i,0}$}] at (\x2,\y1) {};
\node[point,label={above left:$z_{i,0}$}] at (\x3,\y2) {};
\node[point,label={above left:$x_{i,1}$}] at (\x1,\y4) {};
\node[point,label={above left:$y_{i,1}$}] at (\x2,\y6) {};
\node[point,label={above left:$z_{i,1}$}] at (\x3,\y7) {};
\node[point,label={above left:$l_{i,0}$}] at (\x4,\y4) {};
\node[point,label={above left:$p_{i,0}$}] at (\x4,\y6) {};
\node[point,label={above right:$t_{i,0}$}] at (\x5,\y5) {};
\node[point,label={below:$v_{i,0}=l_{i,1}$}] at (\x6,\y5) {};
\node[point,label={above:$p_{i,1}$}] at (\x6,\y7) {};
\node[point,label={above right:$t_{i,1}$}] at (\x7,\y6) {};
\node[point,label={above right:$v_{i,1}$}] at (\x8,\y6) {};
\end{tikzpicture}

\caption{\textbf{CLAUSE-gadget for a clause $a \lor b \lor \neg c$.}
Every green rectangle is an OR-gadget.
$y$-coordinates of $x_{i, 0}$, $y_{i, 0}$ and $z_{i,0}$
depend on the variables in the $i$-th clause.
Grey segments corresponds to the values of variables
satistying the $i$-th clause.
}
\label{fig:apx_clause}
\end{figure}

First, we define auxiliary functions for literals. 
For a literal $w$, let $idx(w)$ be the index of the variable in $w$,
and $neg(w)$ be the Boolean value whether the variable is negated in $w$
or not.

Let us assume that clause $C_i = a \lor b \lor c$
for any literals $a,b,c$. Then, we define points in the gadget as:

\begin{center}
\begin{tabular}{ l l }
	$x_{i, 0} := (\clauseXFactor i, 4\cdot idx(a) + 2\cdot neg(c)),$ &
	$x_{i, 1} := (\clauseXFactor i, 4(n+1)),$ \\
	$y_{i, 0} := (\clauseXFactor i+1, 4\cdot idx(b) + 2\cdot neg(b)),$ &
	$y_{i, 1} := (\clauseXFactor i+1, 4(n+1) + 4),$ \\
	$z_{i, 0} := (\clauseXFactor i+2, 4\cdot idx(c) + 2\cdot neg(c)),$ &
	$z_{i, 1} := (\clauseXFactor i+2, 4(n+1) + 6).$
\end{tabular}
\end{center}

\newcommand{\segmentsClause}{\mathsf{segmentsClause}}	
 
We are now ready to define set of points:
 
 $$\mathsf{moveVariable}_i := 
 \{x_{i, j} : j \in \{0, 1\}\} \cup
 \{y_{i, j} : j \in \{0, 1\}\} \cup
 \{z_{i, j} : j \in \{0, 1\}\},
 $$
 
 $$\pointsClause{i} := 
 \mathsf{moveVariable}_i \cup \pointsOr{i}{0}
 \cup \pointsOr{i}{1} \cup \{v_{i, 1} \}.
 $$
 
Note that these two points are equal: $v_{i,0} = l_{i,1}$.
This translates to the fact, that output of the one OR-gadget
is an input to the other OR-gadget to create $or$ of 3 segments.

\paragraph{Segments.}
We also define segments for the clause gadget as below:

\begin{eqnarray*}
\segmentsClause_i & := & \{ (x_{i, 0}, x_{i, 1}),
(y_{i, 0}, y_{i, 1}),
(z_{i, 0}, z_{i, 1}),
(x_{i, 1}, l_{i, 0}),
(y_{i, 1}, p_{i, 0}),
(z_{i, 1}, p_{i, 1}),
\} \cup \\
& & \cup \ \segmentsOr{i}{0} \cup \segmentsOr{i}{1}.
\end{eqnarray*}

\newcommand{\segmentsClauseSolTrue}[1]{\mathsf{solClause}^{true,#1}}
\newcommand{\segmentsClauseSolFalse}{\mathsf{solClause}^{false}}

The CLAUSE-gadgets consist of two OR-gadgets.
Ideally, we would place the $i$-th CLAUSE-gadget close to the
$\xTrueSegment{j_1}$ or $\xFalseSegment{j_1}$ segments
corresponding to the literals that occur in the $i$-th clause.
It would be inconvenient to position them there,
because between these segments there may be additional
$\xTrueSegment{j_2}$ or $\xFalseSegment{j_2}$
segments corresponding to the other literals.

Instead, we use simple auxiliary gadgets to
\textit{transfer} whether the segment
is in a solution, i.e. segments
$(x_{i, 0}, x_{i, 1}), (y_{i, 0}, y_{i, 1}), (z_{i, 0}, z_{i, 1})$ in this gadget.
Each gadget consists of two segments $(x_{i, 0}, x_{i, 1}), (x_{i, 1}, a)$.
These are the only segments that can cover $x_{i,1}$.
We place $x_{i,0}$ on a segment that we want to transfer (i.e.
segment responsible for choosing the variable value satisfying the
corresponding literal).
If in some solution $x_{i,0}$ is already covered by this segment, then
we can cover $x_{i,1}$ by $(x_{i,1}, a)$, thus also covering $a$.
If $x_{i,0}$ is not covered by this segment,
then the only way to cover $x_{i,0}$ is to use segment $(x_{i, 0}, x_{i, 1})$.
Intuitively,
in any optimal solution the two segments \textit{transfer} the state of whether $x_{i,0}$
is covered onto whether $a$ is covered.
Therefore, the number of segments in the optimal solution is increased by one,
and we get a point $a$ that was effectively placed
on some segment $s$, but it can be placed anywhere on the plane instead,
consequently simplifying the construction.

\begin{lemma}
\label{cover_clauses_solution_true}
For any $1 \le i \le n$ and $a \in \{ x_{i, 0}, y_{i, 0}, z_{i, 0}\}$,
there is a set $\segmentsClauseSolTrue{a}_i \subseteq \segmentsClause_i$
with $|\segmentsClauseSolTrue{a}_i| = 11$
that covers all points in $\pointsClause{i} - \{a\}$.
\end{lemma}

\begin{proof}
For $a = x_{i, 0}$ (analogous proof for $y_{i, 0}$):
First we use Lemma~\ref{cover_or_true} twice with excluded $x = l_{i, 0}$ and
$x = l_{i, 1} = v_{i, 0}$,
resulting with 8 segments in $\chooseOr{true}{i}{0} \cup \chooseOr{true}{i}{1}$
which cover all required points apart from
$x_{i, 1}, y_{i, 0}, y_{i, 1}, z_{i, 0}, z_{i, 1}, l_{i, 0}$.
We cover those using additional 3 segments:
$\{ (x_{i, 1}, l_{i, 0}), (y_{i, 0}, y_{i, 1}),
(z_{i, 0}, z_{i, 1}) \}$

For $a = z_{0, i}$:
Using Lemma~\ref{cover_or_false} and Lemma~\ref{cover_or_true} with
$x = p_{i, 1}$,
we obtain 8 segments in $\chooseOr{false}{i}{0} \cup \chooseOr{true}{i}{1}$
which cover all required points apart from
$x_{i, 0}, x_{i, 1}, y_{i, 0}, y_{i, 1}, z_{i, 1}, p_{i, 1}$.
We cover those using additional 3 segments:
$\{ (x_{i, 0}, x_{i, 1}), (y_{i, 0}, y_{i, 1}),
(z_{i, 1}, p_{i, 1}) \}$.
\end{proof}

\begin{lemma}
\label{cover_clauses_solution_false}
For any $1 \le i \le n$ there is
a set $\segmentsClauseSolFalse_i \subseteq \segmentsClause_i$
with $|\segmentsClauseSolFalse_i| = 12$
that covers all points in $\pointsClause{i}$.
\end{lemma}

\begin{proof}
Using Lemma \ref{cover_or_false} twice we can
cover $\pointsOr{i}{0}$ and  $\pointsOr{i}{1}$
with 8 segments.
To cover the remaining points we additionally use:
$\{ (x_{i, 0}, x_{i, 1}), (y_{i, 0}, y_{i, 1}),
(z_{i, 0}, z_{i, 1}), (t_{i, 1}, v_{i, 1}) \}$
\end{proof}

\begin{lemma}
\label{cover_clauses_segments_no_less}
For any $1 \le i \le n$:
\begin{enumerate}[label={(\arabic*)}]
	\item points in $\pointsClause{i}$ can not be covered 
	using any subset of segments
	from $\segmentsClause_i$ of size smaller than 12;
	\item points in $\pointsClause{i} - \{ x_{i, 0}, y_{i, 0}, z_{i, 0}\}$
	can not be covered using any subset of segments
	from $\segmentsClause_i$ of size smaller than 11.
\end{enumerate}
\end{lemma}


\begin{proof}[Proof of (1).]
No segment in $\segmentsClause_i$ covers more than 1 point from
$$\{ x_{i, 0}, y_{i, 0}, z_{i, 0}, l_{i, 0}, p_{i, 0}, q_{i, 0},
u_{i, 0}, v_{i, 0} = l_{i, 1}, p_{i, 1}, q_{i, 1}, u_{i, 1}, v_{i, 1} \}.$$

Therefore we need to use at least 12 segments.
\end{proof}

\begin{proof}[Proof of (2).]

We can define disjoint sets $X, Y, Z$ such that
$X \cup Y \cup Z \subseteq \pointsClause{i} - \{x_{i, 0}, y_{i, 0}, z_{i, 0}\}$
such that there are no segments in $\segmentsClause_i$ covering points from different sets.
And we prove a lower bound for each of these sets.
First, let:

$$X := \{x_{i, 1}, y_{i, 1}, z_{i, 1}\}.$$

No two points in $X$ can be covered with one segment
of $\segmentsClause_i$, so it must be covered with 3 different segments.
Next we define other sets:

$$Y := \pointsOr{i}{0} - \{l_{i, 0}, p_{i, 0}\},$$
$$Z := \pointsOr{i}{1} - \{l_{i, 1}, p_{i, 1}\}.$$


For both $Y$ and $Z$ we can check all of the subsets of 3 segments
of $\segmentsClause_i$
to conclude that none of them cover the considered,
so both $Y$ and $Z$ have to be covered with 
disjoint sets of 4 segments each.

Therefore, $\pointsClause{i} - \{x_{i, 0}, y_{i, 0}, z_{i, 0}\}$
must be covered with at least 3 + 4 + 4 = 11 segments from $\segmentsClause_i$.
\end{proof}

\subsection{Summary}

Add some smart lemmas that sets will be exclusive to each other.

\begin{lemma}
\textbf{Robustness to 1/2-extensions}. For every segment $s \in \sets$,
$s$ and $s^{+\frac{1}{2}}$ cover the same points from $\points$.
\end{lemma}

\begin{proof}
We can just check every segment. Most of the segments $s$
are collinear only with points that lay on $s$,
so trivially $s^{+\frac{1}{2}}$ cannot cover more points than $s$ does.

Within VARIABLE-gadget for any $1 \le i \le n$ after $\frac{1}{2}$-extension:
$(c_i,g_i)$ does not cover $b_i$.

Within OR-gadget some of the segments are collinear and share one point;
specifically, for any $1 \le i \le n$ and $j \in \{0,1\}$, after $\frac{1}{2}$-extension:
\begin{itemize}
\item $(l_{i,j}, n_{i,j})$ does not cover $o_{i,j}$,
\item $(n_{i,j}, p_{i,j})$ does not cover $m_{i,j}$,
\item $(t_{i,j}, v_{i,j})$ does not cover $n_{i,j}$.
\end{itemize}
Within CLAUSE-gadget, for any $1 \le i \le n$ after $\frac{1}{2}$-extension:
\begin{itemize}
\item $(o_{i,0}, u_{i,0})$ does not cover $m_{i,1}$,
\item $(m_{i,1}, s_{i,1})$ does not cover $u_{i,0}$,
\item $(y_{i,1}, p_{i,0})$ does not cover $n_{i,1}$.
\end{itemize}
For two consequitive VARIABLE-gadgets, for any $1 \le i < n$ after $\frac{1}{2}$-extension:
$(b_i, f_i)$ does not cover $b_{i+1}$ (nor $f_{i-1}$ for $i>1$).
Similiarily $(a_i,d_i)$ does not cover $a_{i+1}$ (nor $d_{i-1}$ for $i>1$),
because this segment is shorter than the previous one and $a_i$ and $b_i$
share y-coordinate.

For two consequtive CLAUSE-gadgets,
segments from one do not cover anything from the other,
as the gadgets have width 9 and
every lefmost x-coordnate is divisible by $\clauseXFactor$.
Hence two different gadgets do not interact with each other
after $\frac{1}{2}$-extension.

Next we need to check whether VARIABLE-gadget's segments
do not cover any points $x_{i,0}, y_{i,0}$ or $z_{i,0}$ from CLAUSE-gadget.
For any $1 \le i \le n$ and $1 \le j \le n$, all points $x_{j,0}, y_{j,0}$ and $z_{j,0}$
have x-coordinate strictly positive. Segment $(a_i, c_i)$ have length $2L$
and $c_i$ has x-coordinate equal to $-L$, so after $\frac{1}{2}$-extension
this segment does not cover any points with a positive x-coordinate.


\end{proof}


\subsection{Summary of construction}
{\tikzset{clause/.style={
    prefix after command= {\pgfextra{\tikzset{every
    label/.style={label distance=-0.35cm,rotate=90}}}}
    }
}

\newcommand{\variableSegments}[3]{
	\draw (\xbeg,\y#1) -- (\x6,\y#1) node[pos=0.15, above] {$x_#3 = \true$};
	\draw (\xbeg,\y#2) -- (\x6,\y#2) node[pos=0.15, above] {$x_#3 = \false$};
}

\begin{figure}
\centering
\begin{tikzpicture}
\tikzmath{
\width=5;
\height=1;
\step=1;
\eps = 0.1;
\widtheps = 5+\eps;
\meps = -\eps;
\y0=0;
\yeps0 = \y0 - \eps;
\y1=\height;
\yx5=\y0+(\height/2);
\yeps1 = \y1 + \eps;
\y2=\y1+\height;
\y3=\y2+\height;
\y4=\y3+\height;
\y5=\y4+\height;
\y6=\y5+\height;
\x0=0;
\xx5=\x0+(\width/2);
\xbeg=\width;
\x1=\width+3;
\x2=\x1+\step;
\x3=\x2+\step;
\x4=\x3+\step;
\x5=\x4+\step;
\x6=\x5+\step;
}

\filldraw [fill=lime, draw=black] (0,0) rectangle (\width, \y1) node[pos=.5] {VARIABLE-gadget$_2$};
\filldraw [fill=lime, draw=black] (0,\y2) rectangle (\width, \y3) node[pos=.5] {VARIABLE-gadget$_2$};
\filldraw [fill=lime, draw=black] (0,\y4) rectangle (\width, \y5) node[pos=.5] {VARIABLE-gadget$_3$};

\variableSegments{0}{1}{1}
\variableSegments{2}{3}{2}
\variableSegments{4}{5}{3}

\filldraw [fill=cyan!60, draw=black] (\x1,\y0) rectangle (\x2, \y6) node[clause, pos=.5, label= {CLAUSE-gadget$_1$}] {};
\filldraw [fill=cyan!60, draw=black] (\x3,\y0) rectangle (\x4, \y6) node[clause, pos=.5, label= {CLAUSE-gadget$_2$}] {};
\filldraw [fill=cyan!60, draw=black] (\x5,\y0) rectangle (\x6, \y6) node[clause, pos=.5, label= {CLAUSE-gadget$_3$}] {};
\end{tikzpicture}
\caption{\textbf{Schema of the whole construction.}}
General layout of VARIABLE-gadgets and CLAUSE-gadgets and how they
interact with each other.
\label{fig:segment_apx_whole}
\end{figure}


{

{\tikzset{node/.style={
    prefix after command= {\pgfextra{\tikzset{every
    label/.style={font=\footnotesize}}}}
    }
}
\begin{figure}
\centering
\begin{tikzpicture}
\tikzmath{
\x0=0;
\y0=0;
\mepsx0=-0.2;
\yalabel=1.5;
\xalabel=12.5;
\xblabel=2.5;
\xclabel=5.0;
\yblabel=0.75;
\mepsy0=-0.2;
\epsxclabel=5.2;
\epsyalabel=1.7;
\yclabel=1.5;
\ydlabel=2.5;
\yelabel=1.5;
\yflabel=2.5;
\yglabel=4.0;
\xdlabel=12.5;
\xelabel=2.5;
\xflabel=5.0;
\yhlabel=3.25;
\mepsydlabel=2.3;
\epsxflabel=5.2;
\epsyglabel=4.2;
\yilabel=1.5;
\yajlabel=2.5;
\yaalabel=4.0;
\yablabel=5.0;
\yaclabel=1.5;
\yadlabel=2.5;
\yaelabel=4.0;
\yaflabel=5.0;
\yaglabel=6.5;
\xglabel=12.5;
\xhlabel=2.5;
\xilabel=5.0;
\yahlabel=5.75;
\mepsyablabel=4.8;
\epsxilabel=5.2;
\epsyaglabel=6.7;
\xajlabel=7.5;
\xaalabel=7.5;
\xablabel=9.5;
\yailabel=1.5;
\ybjlabel=2.5;
\ybalabel=4.0;
\ybblabel=5.0;
\ybclabel=6.5;
\ybdlabel=8.5;
\ybelabel=1.5;
\ybflabel=2.5;
\ybglabel=4.0;
\ybhlabel=5.0;
\ybilabel=6.5;
\xaclabel=7.7;
\ycjlabel=7.166666666666667;
\xadlabel=7.9;
\ycalabel=7.833333333333333;
\xaelabel=8.1;
\ycblabel=8.166666666666666;
\xaflabel=8.3;
\xaglabel=8.9;
\mepsxajlabel=7.3;
\epsxablabel=9.7;
\epsybdlabel=8.7;
\ycclabel=1.5;
\ycdlabel=2.5;
\ycelabel=1.5;
\ycflabel=2.5;
\ycglabel=4.0;
\ychlabel=5.0;
\ycilabel=1.5;
\ydjlabel=7.833333333333334;
\ydalabel=7.333333333333334;
\ydblabel=7.5;
\ydclabel=7.666666666666667;
\xahlabel=8.5;
\xailabel=8.7;
\xbjlabel=8.899999999999999;
\yddlabel=7.499999999999999;
\ydelabel=8.166666666666666;
\ydflabel=7.666666666666666;
\ydglabel=7.833333333333332;
\ydhlabel=7.999999999999999;
\xbalabel=9.1;
\xbblabel=9.299999999999999;
\xbclabel=9.499999999999998;
\xbdlabel=7.5;
\xbelabel=9.5;
\xbflabel=10.5;
\xbglabel=7.5;
\xbhlabel=9.5;
\xbilabel=10.5;
\xcjlabel=12.5;
\ydilabel=1.5;
\yejlabel=2.5;
\yealabel=4.0;
\yeblabel=5.0;
\yeclabel=6.5;
\yedlabel=8.5;
\yeelabel=1.5;
\yeflabel=2.5;
\yeglabel=4.0;
\yehlabel=5.0;
\yeilabel=6.5;
\xcalabel=10.7;
\yfjlabel=7.166666666666667;
\xcblabel=10.9;
\yfalabel=7.833333333333333;
\xcclabel=11.1;
\yfblabel=8.166666666666666;
\xcdlabel=11.3;
\xcelabel=11.9;
\mepsxbflabel=10.3;
\epsxcjlabel=12.7;
\epsyedlabel=8.7;
\yfclabel=1.5;
\yfdlabel=2.5;
\yfelabel=4.0;
\yfflabel=5.0;
\yfglabel=6.5;
\yfhlabel=1.5;
\yfilabel=2.5;
\ygjlabel=7.833333333333334;
\ygalabel=7.333333333333334;
\ygblabel=7.5;
\ygclabel=7.666666666666667;
\xcflabel=11.5;
\xcglabel=11.7;
\xchlabel=11.899999999999999;
\ygdlabel=7.499999999999999;
\ygelabel=8.166666666666666;
\ygflabel=7.666666666666666;
\ygglabel=7.833333333333332;
\yghlabel=7.999999999999999;
\xcilabel=12.1;
\xdjlabel=12.299999999999999;
\xdalabel=12.499999999999998;
}

\filldraw [fill=cyan!30, draw=black] (\mepsxajlabel,\mepsy0) rectangle (\epsxablabel, \epsybdlabel);
\draw (\xaclabel,\ycdlabel) -- (\xaclabel,\ycjlabel);
\draw (\xaclabel,\ycjlabel) -- (\xaflabel,\ycjlabel);
\draw (\xadlabel,\ychlabel) -- (\xadlabel,\ycalabel);
\draw (\xadlabel,\ycalabel) -- (\xaflabel,\ycalabel);
\draw (\xaelabel,\ycilabel) -- (\xaelabel,\ycblabel);
\draw (\xaelabel,\ycblabel) -- (\xaglabel,\ycblabel);
\node[above right] at (\mepsxajlabel,\epsybdlabel) {CLAUSE-gadget$_1$};
\draw (\xaflabel,\ycjlabel) -- (\xaflabel,\ydjlabel);
\draw (\xaflabel,\ydalabel) -- (\xailabel,\ydalabel);
\draw (\xaflabel,\ydclabel) -- (\xailabel,\ydclabel);
\draw (\xahlabel,\ydalabel) -- (\xahlabel,\ydclabel);
\draw (\xailabel,\ydalabel) -- (\xailabel,\ydclabel);
\draw (\xailabel,\ydblabel) -- (\xbjlabel,\ydblabel);
\draw (\xaglabel,\yddlabel) -- (\xaglabel,\ydelabel);
\draw (\xaglabel,\ydflabel) -- (\xbblabel,\ydflabel);
\draw (\xaglabel,\ydhlabel) -- (\xbblabel,\ydhlabel);
\draw (\xbalabel,\ydflabel) -- (\xbalabel,\ydhlabel);
\draw (\xbblabel,\ydflabel) -- (\xbblabel,\ydhlabel);
\draw (\xbblabel,\ydglabel) -- (\xbclabel,\ydglabel);

\filldraw [fill=cyan!30, draw=black] (\mepsxbflabel,\mepsy0) rectangle (\epsxcjlabel, \epsyedlabel);
\draw (\xcalabel,\yfglabel) -- (\xcalabel,\yfjlabel);
\draw (\xcalabel,\yfjlabel) -- (\xcdlabel,\yfjlabel);
\draw (\xcblabel,\y0) -- (\xcblabel,\yfalabel);
\draw (\xcblabel,\yfalabel) -- (\xcdlabel,\yfalabel);
\draw (\xcclabel,\yfilabel) -- (\xcclabel,\yfblabel);
\draw (\xcclabel,\yfblabel) -- (\xcelabel,\yfblabel);
\node[above right] at (\mepsxbflabel,\epsyedlabel) {CLAUSE-gadget$_2$};
\draw (\xcdlabel,\yfjlabel) -- (\xcdlabel,\ygjlabel);
\draw (\xcdlabel,\ygalabel) -- (\xcglabel,\ygalabel);
\draw (\xcdlabel,\ygclabel) -- (\xcglabel,\ygclabel);
\draw (\xcflabel,\ygalabel) -- (\xcflabel,\ygclabel);
\draw (\xcglabel,\ygalabel) -- (\xcglabel,\ygclabel);
\draw (\xcglabel,\ygblabel) -- (\xchlabel,\ygblabel);
\draw (\xcelabel,\ygdlabel) -- (\xcelabel,\ygelabel);
\draw (\xcelabel,\ygflabel) -- (\xdjlabel,\ygflabel);
\draw (\xcelabel,\yghlabel) -- (\xdjlabel,\yghlabel);
\draw (\xcilabel,\ygflabel) -- (\xcilabel,\yghlabel);
\draw (\xdjlabel,\ygflabel) -- (\xdjlabel,\yghlabel);
\draw (\xdjlabel,\ygglabel) -- (\xdalabel,\ygglabel);
\draw (\xclabel,\y0) -- (\xalabel,\y0) node[pos=0.17, above] {$x_1 = \true$};
\draw (\xclabel,\yalabel) -- (\xalabel,\yalabel) node[pos=0.17, above] {$x_1 = \false$};
\filldraw [fill=lime!30, draw=black] (\mepsx0,\mepsy0) rectangle (\epsxclabel, \epsyalabel);
\draw (\x0,\y0) -- (\xalabel,\y0);
\draw (\x0,\y0) -- (\x0,\yblabel);
\draw (\xblabel,\y0) -- (\xblabel,\yalabel);
\draw (\xblabel,\yalabel) -- (\xalabel,\yalabel);
\draw (\x0,\yblabel) -- (\xblabel,\yblabel);
\node[above right] at (\mepsx0,\epsyalabel) {VARIABLE-gadget$_1$};
\draw (\xflabel,\ydlabel) -- (\xdlabel,\ydlabel) node[pos=0.17, above] {$x_2 = \true$};
\draw (\xflabel,\yglabel) -- (\xdlabel,\yglabel) node[pos=0.17, above] {$x_2 = \false$};
\filldraw [fill=lime!30, draw=black] (\mepsx0,\mepsydlabel) rectangle (\epsxflabel, \epsyglabel);
\draw (\x0,\ydlabel) -- (\xdlabel,\ydlabel);
\draw (\x0,\ydlabel) -- (\x0,\yhlabel);
\draw (\xelabel,\ydlabel) -- (\xelabel,\yglabel);
\draw (\xelabel,\yglabel) -- (\xdlabel,\yglabel);
\draw (\x0,\yhlabel) -- (\xelabel,\yhlabel);
\node[above right] at (\mepsx0,\epsyglabel) {VARIABLE-gadget$_2$};
\draw (\xilabel,\yablabel) -- (\xglabel,\yablabel) node[pos=0.17, above] {$x_3 = \true$};
\draw (\xilabel,\yaglabel) -- (\xglabel,\yaglabel) node[pos=0.17, above] {$x_3 = \false$};
\filldraw [fill=lime!30, draw=black] (\mepsx0,\mepsyablabel) rectangle (\epsxilabel, \epsyaglabel);
\draw (\x0,\yablabel) -- (\xglabel,\yablabel);
\draw (\x0,\yablabel) -- (\x0,\yahlabel);
\draw (\xhlabel,\yablabel) -- (\xhlabel,\yaglabel);
\draw (\xhlabel,\yaglabel) -- (\xglabel,\yaglabel);
\draw (\x0,\yahlabel) -- (\xhlabel,\yahlabel);
\node[above right] at (\mepsx0,\epsyaglabel) {VARIABLE-gadget$_3$};
\end{tikzpicture}
\caption{\textbf{Scheme of the whole construction.}}
General layout of VARIABLE-gadgets and CLAUSE-gadgets and how they
interact with each other.
\label{fig:segment_apx_whole}
\end{figure}



Finally we define set of points and segments for the constructed instance:
$$\points := \bigcup_{1 \le i \le n} \pointsVar{i} \cup \pointsClause{i},$$
$$\sets := \bigcup_{1 \le i \le n} \segmentsVar{i} \cup \segmentsClause_i.$$

\section{Construction lemmas and proof of Lemma \ref{apxconstruction}}

In order to prove Lemma \ref{apxconstruction} we introduce several
auxiliary lemmas proving properties of the construction
described in the previous section.

Consider an instance $S$ of MAX-(3,3)-SAT of size $n$
with optimum solution satisfying $k$ clauses.
Let us construct an instance $\setCoverInstance$ of geometric set cover
as described in Section~\ref{construction_description}
for the instance $S$ of MAX-(3,3)-SAT.

\begin{lemma}
	\label{construction_correctness}
	Instance $\setCoverInstance$ of geometric set cover
	admits a solution to size $15n - k$.
\end{lemma}

\begin{proof}
Let the clauses in $S$ be $c_1$,~$c_2$~$\ldots$~$c_n$
and the variables be $x_1$,~$x_2$~$\ldots$~$x_n$.
Let the variable assignment in
the optimum solution to $S$ be
$\phi : \{ x_1, x_2 \ldots x_n\} \rightarrow \{\true, \false\}$.


We cover every VARIABLE-gadget with solution described in
Lemma~\ref{choose_variables_solution}, where
in the $i$-th gadget we choose the set of segments corresponding to the
value of $\phi(x_i)$.

For every clause that is satisfied, say $c_i$, 
let us name the variable that is $\true$ in it as $x_i$
and point corresponding to $x_i$ in $\pointsClause{i}$ as $a$.
Points in $\pointsClause{i}$ 
are covered with set $\segmentsClauseSolTrue{a}_i$ described in
Lemma~\ref{cover_clauses_solution_true}.
For every clause that is not satisfied, say $c_j$,
points in $\pointsClause{j}$ are covered
with set $\segmentsClauseSolFalse_i$ described in
Lemma~\ref{cover_clauses_solution_false}.

Formally we define 
sets responsible for choosing variable assignment and satisfing clauses,
$R_i$ and $C_i$ respectively, as following:

\begin{align}
	\begin{split}
	& R_i := \begin{cases}
		\chooseVar{true}{i} & \text{if}\ \phi(x_i) = \true \\
		\chooseVar{false}{i} & \text{if}\ \phi(x_i) = \false \\
		\end{cases} \\
	& C_i := \begin{cases}
		\segmentsClauseSolTrue{a}_i & \text{if}\ c_i \text{ satisfied by literal corresponding to point } a \\
		\segmentsClauseSolFalse_i & \text{if}\ c_i \text{ not satisfied}
		\end{cases} \\
	& \sol := \bigcup\limits_{i=1}^{n} \{R_i \cup C_i : 1 \le i \le n\}.
    \end{split}
\end{align}


This set covers all the points from $\points$, because
the sets $R_i$, $C_i$ individually cover their corresponding gadgets,
as proved in the respective lemmas.

All of these sets are disjoint, so the size of the obtained solution is:

$$|\sol| = \sum_{i=1}^{n} R_i + \sum_{i=1}^n C_i = 3n + 11k + 12(n-k) = 15n - k.\qedhere$$
\end{proof}

\begin{lemma}
	\label{at_most_one_var_segment}
	Suppose we have a solution $\sol$ of the instance $\setCoverInstance$
	of geometric set cover.
	Then there exists a solution $\sol'$, such that $|\sol'| \le |\sol|$,
	and $\sol'$ contains
	at most one of the segments $\xTrueSegment{i}$ and $\xFalseSegment{i}$
	from each VARIABLE-gadget.
\end{lemma}
\begin{proof}\leavevmode
Assume that we have $\{\xTrueSegment{i}, \xFalseSegment{i}\} \subseteq \sol$ for some $i$.
We will show how to modify $\sol$ into $\sol'$,
such that the number of such $i$ decreases,
while $\sol'$ is still a valid solution to $\setCoverInstance$,
and $|\sol'| \le |\sol|$. Then, by repeating this procedure,
we can eventually construct a solution satisfying the property from the Lemma.

To construct $\sol'$, 
we first remove from $\sol$ all segments belonging to $\segmentsVar{i}$.
Recall that the $i$-th VARIABLE-gadget corresponds to variable $x_i$ in $S$.
As every variable in $S$ is used in exactly 3 clauses,
then one literal $x_i$ or $\neg x_i$ must appear in at least
2 clauses.
If that literal is $x_i$, then we add to the constructed solution all segments from $\chooseVar{true}{i}$,
otherwise we add all segments from $\chooseVar{false}{i}$.

Now, there exists at most one CLAUSE-gadget which needs adjustment to make $\sol'$ valid;
assuming it is the $j$-th clause, then one of the points $x_{j,0}, y_{j,0}$ or $z_{j,0}$ for this
CLAUSE-gadget might be not covered, say $y_{j,0}$.
We amend the solution by adding $(y_{j,0}, y_{j,1})$ to $\sol'$.

By Lemma \ref{choose_variables_both} we know 
that $\sol$ used at least 4 segments from $\segmentsVar{i}$.
Therefore, we removed at least 4 segments and added at most 4 segments,
so $|\sol'| \le |\sol|$.
\end{proof}

\begin{lemma}
	\label{construction_completness}
	Suppose we have a solution $\sol$ of the instance $\setCoverInstance$
	of geometric set cover that is of size $w$.
	Then there exists a solution to $S$
	that satisfies at least $15n - w$ clauses.
\end{lemma}


\begin{proof}\leavevmode
Let the clauses in $S$ be $c_1$,~$c_2$~$\ldots$~$c_n$
and the variables be $x_1$,~$x_2$~$\ldots$~$x_n$.
Given a solution $\sol$
of the instance $\setCoverInstance$ of geometric set cover, we use Lemma~\ref{at_most_one_var_segment} to modify $\sol$ such that for any $i$ it contains at most one of $\xTrueSegment{i}$ and $\xFalseSegment{i}$; this may decrease the cost of $\sol$, but that does not matter in the subsequent construction. To simplify notation, in the remainder of this proof we use $\sol$ to refer to the modified solution.

Given $\sol$, we construct a solution to $S$ by defining an
assignment of variables:
$$\phi : \{ x_1, x_2 \ldots x_n\} \rightarrow \{\true, \false\}$$
that satisfies at least $15n-w$ clauses in $S$.

\subparagraph{Definition of $\phi$.}
Recall that due to Lemma~\ref{at_most_one_var_segment},
$\sol$ contains at most one of $\xTrueSegment{i}$ and $\xFalseSegment{i}$.

We define the value $\phi(x_i)$ for the variable $x_i$ as follows:
\begin{align}
	\begin{split}
	\label{eqn:variable_assignment}
	& \begin{cases}
	\phi(x_i) = \true & \text{if}\ \xTrueSegment{i} \in \sol \\
	\phi(x_i) = \false & \text{otherwise}
	\end{cases}
	\end{split}
\end{align}

Moreover, from Lemma~\ref{choose_variables_no_less} we get $|\segmentsVar{i} \cap \sol| \ge 3$ for every $i$.

\subparagraph{Clauses satisfied with the chosen variable assignment.}

For a clause $c_i$,
$\sol$ needs to use at least 11 segments to cover $\pointsClause{i} - \{x_{i,0}, y_{i,0}, z_{i,0}\}$
in the $i$-th CLAUSE-gadget (Lemma~\ref{cover_clauses_segments_no_less}).

Moreover, if none of the points $\{x_{i,0}, y_{i,0}, z_{i,0}\}$
are covered by the segments from $\sol~\cap~\segmentsVar{i}$,
then $\sol$ needs to cover $\pointsClause{i}$
with at least 12 segments
by Lemma~\ref{cover_clauses_segments_no_less}.

\iffalse
\begin{align}
	\begin{split}
	\label{eqn:clauses}
	|\sol \cap \segmentsClause_i| \ge
	\begin{cases}
	11 & \text{if any of points } \{x_{i,0}, y_{i,0}, z_{i,0}\} \text{ is covered by segments from } \sol~\cap~\pointsVar{i} \\
	12 & \text{otherwise} \\
	\end{cases} \\
	\end{split}
\end{align}
\fi

Let us denote $a$ as the amount of such clauses $c_i$ for which none of
the points $x_{i,0}, y_{i,0}, z_{i,0}$ in $\pointsClause{i}$ were covered by
segments from $\sol~\cap~\segmentsVar{j}$ for any $1 \le j \le n$.

Consider a clause $c_i$ for which at least one of the points
$x_{i,0}, y_{i,0}, z_{i,0}$ in $\pointsClause{i}$ were covered by
segments from $\sol~\cap~\segmentsVar{j}$ for some $1 \le j \le n$,
then denote this point as $t$ and say it corresponds to
literal $q$ and variable $x_j$.
Point $t$ can be only covered in $\segmentsVar{j}$
by a corresponding segment $\xTrueSegment{j}$ or $\xFalseSegment{j}$
(depending on whether the literal $q$ is negated or not).
From the definition of $\phi$ and the fact that one of this segment is
in $\sol$, we know that
$\phi(j)$ has the value that evaluates $w$ to be true.
Therefore, clause $c_i$ is satisfied.

Consequently, $\phi$ satisfies all but at most $a$ clauses in $S$.

To conclude,
given a solution to $\setCoverInstance$ of size $w$ we constructed
a variable assignment $\phi$
that satisfies at least $n-a$ clauses of $S$.
Finally, note that

$$w \ge 3n + 11(n-a) + 12a = 3n + 11n + a = 14n + a,$$
hence
$$15n - w  \le 15n - 14n - a = n - a.$$

Therefore $\phi$ satisfies at least $15n-w$ clauses of $S$.
\end{proof}

We are ready to conclude the proof of Lemma $\ref{apxconstruction}$.

\begin{proof}[Proof of Lemma \ref{apxconstruction}]
By Lemma~\ref{construction_correctness}, we know
that there exists a solution to $\setCoverInstance$ of size $15n-k$, so: 
$$opt(\setCoverInstance) \le 15n - k.$$
Since the optimum solution to $S$ satisfies $k$ clauses,
then according to Lemma~\ref{construction_completness}:
$$opt(\setCoverInstance) \ge 15n -k.$$
Therefore, the solution given by Lemma~\ref{construction_correctness} 
of size $15n - k$ is an optimum solution to the instance $\setCoverInstance$.
\end{proof}

\chapter{Fixed-parameter tractable algorithm for geometric set cover problem}
In this chapter we show two fixed-parameter tractable algorithms
for geometric set cover problem in~two different settings.
Section \ref{section:fpt_unweighted} shows 
a fixed-parameter tractable algorithm for geometric set cover with unweighted segments.
The reminder of the chapter presents
a fixed-parameter tractable algorithm for geometric set cover with weighted segments
with $\delta$-extensions.
We show an algorithm for the setting with $\delta$-extensions,
because the original problem with weights is W[1]-hard,
as we show in Chapter $\ref{chapter:w1_hard}$.

We start with a shared definition for this problem.
We define \textit{extreme points} for a set of~collinear points.

\begin{defi}
	For a set of collinear points $C$,
	\textbf{extreme points} are the ends
	of the smallest segment that covers all points from set $C$.
	
	If $C$ consists of one point or is empty, then
	there exists 1 or 0 extreme points respectively.
\end{defi}

\section{Fixed-parameter tractable algorithm for unweighted segments}
\label{section:fpt_unweighted}
In this section we consider fixed-parameter tractable
algorithms for unweighted geometric set cover with segments.
The setting where segments are limited to be axis-parallel
(or limited to a constant number of directions) has an FPT
algorithm already present in literature.
We present an FPT algorithm for geometric set cover
with unweighted segments, where segments are in arbitrary directions.

\subsection{Axis-parallel segments}
You can find this simple algorithm in Parameterized Algorithms book \cite{platypus_book}.

We show an $\mathcal{O}(2^k)$-time branching algorithm.
In each step, the algorithm selects a point $a$ which is not yet covered,
branches to choose one of the two directions, and greedily chooses
a segment $a$ in that direction to cover.
This proceeds until either all points are covered or $k$~segments are chosen.

Let us take
the point $a=(x_a,y_a)$ which is the smallest 
among points that are not yet covered
in the lexicographic ordering
of points in $\mathbb{R}^2$.
We need to cover $a$ with some of~the~remaining segments.

Branch over the choice of one of the coordinates ($x$ or $y$);
without loss of generality, let us assume we chose $x$.
Among the segments lying on line $x = x_a$,
we greedily add to~the~solution the~one that covers the most points.
As $a$ was the smallest in the lexicographical order,
all points on the line $x = x_a$ have the $y$-coordinate larger than $y_a$.
Therefore, if we denote the~greedily chosen segment as $s$,
then any other segment on the line $x = x_a$ that covers $a$ can only
cover a (possibly improper) subset of points covered by $s$.
Thus, greedily choosing $s$ is optimal.

In each step of the algorithm we add one segment to the solution,
thus each branch can stop at depth $k$.
If no branch finds a solution, then that means a solution of size at most $k$ does not exist.


\begin{remark}
The same algorithm can be used for segments in $d$ directions,
where we branch over $d$ choices of directions, and it runs in complexity $\mathcal{O}(d^k)$.
\end{remark}

\subsection{Segments in arbitrary directions}
\label{segments_in_arbitrary_direction}
In this section we consider the setting where segments are not constrained
to a constant number of directions. 
We present a fixed-parameter tractable algorithm,
parameterized by the size of the solution.

\begin{tw}{
	\label{segment_cover_fpt}
	\textbf{(FPT for segment cover)}.
	There exists an algorithm that given a family $\sets$ of
	segments (in any direction),
	a set of points $\points$
	and a parameter $k$,
	runs in time $k^{O(k)} \cdot (|\points|\cdot|\sets|)^2$,
	and outputs a solution $\sol \subseteq \sets$
	such that $|\sol| \le k$ and $\sol$ covers all points in~$\points$,
	or determines that such a set $\sol$ does not exist.
}\end{tw}

We will need the following lemmas proving properties of any
instance of the problem.

\begin{lemma}
   \label{fpt_reduction}
   Given an instance $(\sets, \points)$ of the segment cover problem,
   without a loss of generality we can assume that
   no segment covers a superset of what another segment covers.
   That~is, for any distinct $A, B \in \sets$, we have
   $A \cap C \not \subseteq B \cap C$ and $A \cap C \not \supseteq B \cap C$.
\end{lemma}   
   
\begin{proof} Trivial. \end{proof}

\begin{lemma}
	\label{fpt_long_lines}
	Given an instance $(\sets, \points)$
	of the segment cover problem 
	transformed by Lemma~\ref{fpt_reduction},
	if there exists a line $L$ with at least
	$k+1$ points on it, then there exists a subset $A \subseteq \sets$,
	of size at most $k$,
	such that every solution $\sol$ with $|\sol| \le k$
	satisfies $|A \cap \sol| \ge 1$.
	Moreover, such a subset can be found in~polynomial time.
\end{lemma}

\begin{proof}

Let us enumerate the points from $\points$ that lie on $L$ as $x_1, x_2, \ldots x_t$
in the order in which they appear on $L$.
Our proposed set is defined as:
$$A := \{ \text{segment that have the leftmost point in } x : x \in x_1, x_2, \ldots x_k\}.$$
If such segment does not exist for any of these points,
then set $A$ has smaller size.
We prove the lemma by contradition. Let us assume that there
exists a solution $\sol$ of size at most $k$, such that $\sol \cap A = \emptyset$.

Every segment that is not collinear with $L$ can cover at~most one of
the points that lie on this line.
Hence if all segments from $\sol$ were not collinear with $L$, then
$\sol$ would cover at most $k$ points on line $L$,
but $L$ had at least $k+1$ different points from $\points$ on it.

Therefore we know that one of the segments from $\sol$ must be collinear with $L$
and at most $k-1$ segments can be not collinear with $L$.
Segments from $\sol$, that are not collinear with $L$ can cover at most $k-1$
points among $\{x_1, x_2, \ldots x_k\}$, therefore at least
one of these points must be covered by segments from $\sol$.
We take leftmost point from $\{x_1, x_2, \ldots x_k\}$ that is
covered in $\sol$ by segment collinear with $L$ and name it $a$.
After transformation from Lemma \ref{fpt_reduction}
there is only one segment that starts in $a$, therefore
this segment must be in both $\sol$ and $A$.

This contradiction concludes the proof that $|A \cap \sol| \ge 1$
for any solution $\sol$ of size at most $k$.
\end{proof}

We are now ready to prove Theorem \ref{segment_cover_fpt}.

\begin{proof}[Proof of Theorem \ref{segment_cover_fpt}.]\leavevmode

We will prove this theorem by presenting a branching algorithm that
works in desired complexity. It branches over the
choice of segments to cover the lines with \textit{a lot} of points,
then solves a small instance (where every line has at most $k$ points)
by checking all possible solutions.

\subparagraph{Algorithm.}
First we use Lemma \ref{fpt_reduction}.

Next, we present a recursive algorithm. Given an instance of the problem:

\begin{enumerate}[label={(\arabic*)}]
\item If there exist a line with at least $k+1$ points from $\points$,
we branch over choice of adding to~the~solution one of~the~at~most $k$ possible segments
provided by Lemma \ref{fpt_long_lines}; name this segment $s$
and name set of points from $\points$ that lie on $s$ as $S$.
Then we find a solution $\sol$
for the instance $(\points - S, \sets - \{s\})$,
and parameter $k-1$. We return $\sol \cup \{s\}$.
\item If every line has at most $k$ points on it and $|\points| > k^2$,
then answer \texttt{NO}.
\item If $|\points| \le k^2$, solve the problem by brute force:
check all subsets of $\sets$ of size at most $k$.
\end{enumerate}

\subparagraph{Correctness.}

Lemma \ref{fpt_long_lines} proves that at least one segment that we
branch over in (1) must be present in every solution $\sol$ with $|\sol| \le k$.
Therefore, the recursive call can find a~solution, provided there exists one.

In (2) the answer is no, because every line covers no more than $k$ points
from $\points$, which implies the same about every segment from $\sets$.
Under this assumption
we can cover only $k^2$ points with a solution of size $k$, which is less
than $|\points|$.

Checking all possible solutions in (3) is trivially correct.


\subparagraph{Complexity.}

In the leaves of the recursion we have $|\points| \le k^2$, so $|\sets| \le k^4$,
because every segment can be uniquely identified by the two extreme points it covers
(by Lemma \ref{fpt_reduction}). Therefore, there are $\binom{k^4}{k}$
possible solutions to check, each can be checked in time $O(k|\points|)$.
Thus, (3) takes time $k^{O(k)}$.


In this branching algorithm our parameter $k$ is decreased with every
recursive call, so we have at most $k$ levels of recursion with
branching over $k$ possibilites. Candidates to branch over
can be found on each level in time $O((|\points|\cdot|\sets|)^2)$.

Reduction from Lemma \ref{fpt_reduction} can be implemented in time $O(|\points|^2|\sets|)$.

It follows that the overall complexity is $O((|\points|\cdot|\sets|)^2 \cdot k^{O(k)})$
\end{proof}


\section{Fixed-parameter tractable algorithm for weighted segments with $\delta$-extensions}
\label{section:fpt_weighted}

In this section we consider the geometric set cover problem
for weighted segments relaxed with $\delta$-extensions.
We show that this problem
admits an FPT algorithm when parameterized by the size
of the solution and $\delta$.
In the next chapter we show that the assumption
about the problem being relaxed with $\delta$-extensions is necessary:
we prove that geometric set cover problem
for weighted segments (without extensions) is W[1]-hard, which means
there does not exist any FPT algorithm parameterized by solution size for it,
assuming FPT $\neq$ W[1].

\begin{tw}[FPT for weighted segment cover with $\delta$-extensions]{
	\label{fpt_weighted_segment}
	There exists an algorithm that given a family $\sets$ of
	$n$ weighted segments (in any direction),
	a set of $m$ points $\points$, and parameters $k$ and $\delta > 0$,
	such that it
	runs in time $f(k, \delta) \cdot (nm)^c$ for some computable function $f$ and a constant $c$ and
	outputs a set $\sol$ such that:
	\begin{itemize}
	\item $\sol \subseteq \sets$,
	\item $|\sol| \le k$,
	\item $\sol^{+\delta}$ covers all points in $\points$,
	\item the weight of $\sol$ is not greater than the weight
	of an optimum solution of size at most $k$
	for this problem without $\delta$-extensions
	\end{itemize}
	or determines that there is no set $\sol$ with $|\sol| \le k$
	such that $\sol$ covers all points in $\points$.
}\end{tw}


To solve this problem we will introduce a lemma about choosing
a \textit{dense} subset of points. A dense subset of points
for a set of collinear points $C$ and parameters $k$ and $\delta$
is a subset of $C$ such that
if we cover it with at most $k$ segments,
these segments after $\delta$-extensions will cover all of the points from $C$.
We will prove that such set 
of size bounded by some function $f(k, \delta)$
always exists (Lemma \ref{dense_set_exists}).
Later, Lemma \ref{dense_set_exists} will allow us to find a kernel
for our original problem.

\begin{defi}
	For a set of collinear points $C$,
	a subset $A \subseteq C$ is \textbf{$(k,\delta)$-dense} 
	if for any set of segments $R$ that covers $A$ and
	such that $|R| \le k$, it holds that $R^{+\delta}$ covers $C$.
\end{defi}

\begin{lemma}
	\label{dense_set_exists}
	For any set of collinear points $C$, $\delta > 0$ and $k \ge 1$,
	there exists a $(k,\delta)$-dense set $A \subseteq C$ of size
	at most $(2+\frac{2}{\delta})^k$.
	Moreover, there exists an algorithm that computes the $(k,\delta)$-dense set
	in time $O(|C| \cdot (2+\frac{2}{\delta})^k)$.
\end{lemma}

\begin{proof}
We prove this for a fixed $\delta$ by induction on $k$.

\subparagraph{Inductive hypothesis.}
For any set of collinear points $C$, there exists a set $A$ such that:
\begin{itemize}
\item $A$ is subset of $C$,
\item $A$ is $(\ell, \delta)$-dense for every $1 \le \ell \le k$,
\item $|A| \le (2+\frac{2}{\delta})^k$,
\item the extreme points of $C$ are in $A$.
\end{itemize}

\subparagraph{Base case for $k = 1$.}
It is sufficient that $A$ consists of the extreme points of $C$.

If they are covered with one segment, it must be a segment 
that includes the extreme points from $C$, so it covers the whole set $C$.

There are at most 2 extreme points in $C$ and $2 < 2+\frac{2}{\delta}$.

\subparagraph{Inductive step.}
Assuming inductive hypothesis for any set of collinear points $C$
and for parameter $k$, we will prove it for $k+1$.

Let $s$ be the minimal segment that includes all points from $C$.
That is, the extreme points of $C$ are endpoints of $s$.

We define $M = \lceil1+\frac{2}{\delta}\rceil$ subsegments of $s$
by splitting $s$ into $M$ closed segments of equal length.
We name these segments $v_i$, note that
$|v_i| = \frac{|s|}{M}$ for each $1 \le i \le M$.

Let $C_i$ be the subset of $C$ consisting of points lying on $v_i$.

Let $t_i$ be the segment with endpoints being the extreme points of $C_i$.
It might be a degenerate segment if $C_i$ consists of one point,
or $t_i$ might be empty if $C_i$ is empty.

Figure $\ref{fig:fpt_v_f_def}$ presents an example
of such segments $v_i$ and $t_i$.

\begin{figure}[h]
\begin{center}
\def\svgwidth{\columnwidth}
\input{fpt_v_t_def.pdf_tex}
\end{center}
\caption{\textbf{Example of segments $v_i$ and $t_i$.}}
Example for $M = 7$ and some set of points (marked with black circles).
The top panel shows segments $v_i$ and the bottom panel shows segments $t_i$
on the same set of points.
$a$ and $b$ are the extreme points and therefore segment $s$
ends at $a$ and $b$.
Red segments depict the split into $M$ segments of equal length $v_i$.
Blue segments depict the segments $t_i$. $t_5$ is an empty segment,
because there are no points that lie on segment $v_5$.
Segments $t_3$ and $t_7$ are degenerated to one point --
$c$ and $d$ respectively.
Segments $t_1$ and $t_2$ share one point $b$.
\label{fig:fpt_v_f_def}
\end{figure}

We use the inductive hypothesis to choose $(k, \delta)$-dense sets $A_i$
for sets $C_i$. Note that if $|C_i| \le 1$, then $A_i = C_i$
and it is still a $(k, \delta)$-dense set for $C_i$.

Then we define $A = \bigcup_{i=1}^{M} A_i$.
Thus $A$ includes the extreme points of $C$,
because they are included in the sets $A_1$ and $A_M$.

The size of each $A_i$ is at most $(2+\frac{2}{\delta})^{k}$
from the inductive hypothesis, therefore size of $A$ is at most:
$$M\left(2+\frac{2}{\delta}\right)^{k} =
\left\lceil1+\frac{2}{\delta}\right\rceil\cdot\left(2+\frac{2}{\delta}\right)^{k}
\le \left(2+\frac{2}{\delta}\right)^{k+1}.$$


\subparagraph{Proof that $A$ is $(k, \delta)$-dense for $C$.}
Let us take any cover of $A$ with $k+1$ segments and call it $\sol$.

For every segment $t_i$, if there exists a segment $x$ in $\sol$ 
that is disjoint with $t_i$,
then we have a cover of $A_i$ with at most $k$
segments using $\sol - \{x\}$.
Since $A_i$ is $(k, \delta)$-dense for $t_i$ and $C_i$,
$(\sol - \{x\})^{+\delta}$ covers $C_i$.
So $\sol^{+\delta}$ covers $C_i$ as well.

If there exists a segment $t_i$ for which a segment $x$ as defined above
does not exist, then all $k+1$ segments that cover
$A_i$ intersect $t_i$.
An example of such segments is depicted in Figure~\ref{fig:fpt_tricky_case}.
Let us consider any such $t_i$.
By inductive hypothesis, the endpoints of $s$ are
in $A_1$ and $A_M$ respectively, so $\sol$ must cover them.
For each endpoint of $s$, there exists
a segment that contains this endpoint and intersects $t_i$.
Let us call these two segments $y$ and $z$. It follows that:
$|y| + |z| + |t_i| \ge |s|$.
Since $|t_i| \le |v_i| = \frac{|s|}{M} \le \frac{|s|}{1+\frac{2}{\delta}} = \frac{|s|\delta}{\delta+2}$,
we have $\max(|y|, |z|) \ge |s|(1-\frac{\delta}{\delta+2})/2 = \frac{|s|}{\delta+2}$.

\begin{figure}[h]
\begin{center}
\def\svgwidth{\columnwidth}
\input{fpt_tricky_case.pdf_tex}
\end{center}
\caption{\textbf{Example of all $k+1$ segments intersecting one segment $t_i$.}}
Both panels show the same set $\points$ (black circles),
the same as in Figure $\ref{fig:fpt_v_f_def}$.
The top panel shows blue segments $t_i$ for $M=7$.
The bottom panel shows green segments -- solution $\sol$ of size 4.
All segments from $\sol$ intersect $t_4$.
Segments $z$ and $y$ are named in the figure.
\label{fig:fpt_tricky_case}
\end{figure}

After $\delta$-extension, the longer of these segments will
expand at both ends by at least:
$$\max(|y|, |z|)\delta \ge \frac{|s|\delta}{\delta+2} =
\frac{|s|}{1+\frac{2}{\delta}} \ge \frac{|s|}{M} = |v_i| \ge |t_i|.$$

Therefore, the longer of segments $y$ and $z$ will cover the whole segment $t_i$
after $\delta$-extension. We conclude that $\sol^{+\delta}$ covers $C_i$.

Since $C = \bigcup_{i=1}^M C_i$,
it follows that $\sol^{+\delta}$ covers $C$.


\subparagraph{Algorithm.}

We can simulate the inductive proof presented above by a recursive algorithm with
the following complexity:
$$O\left(|C|+\frac{1}{\delta}\right) + O\left(|C|\cdot\left(2+\frac{2}{\delta}\right)^k\right).$$

\end{proof}

Let us now formulate some claims about the
properties for the problem parameterized by the solution size.
These properties provide bounds for different
objects in the problem instance,
which help us to find a small kernel for the problem
or conclude that the optimum
solution to this instance must be in terms of size above some treshold.

\begin{defi}
A line in the plane is \textbf{long}
if there are at least $k+1$ points from $\points$ on it.
\end{defi}

\begin{claim}
\label{few_long_lines}
If there are more than $k$ different long lines, then 
$\points$ can not be covered with $k$ segments.
\end{claim}

\begin{proof}
We prove the claim by contradiction.
Let us assume that we have at least $k+1$ different
long lines in our instance of the problem
and there is a solution $\sol$ of size at most $k$
covering points $\points$.

Choose any long line $L$.
Every segment from $\sol$ which is not collinear with $L$,
covers at most one point that lies on $L$.
$L$ is long, so there are at least $k+1$ points from $\points$ that lie on $L$.
That implies that there must be a segment in $\sol$ that is
collinear with $L$.

Since we have at least $k+1$ different long lines,
there are at least $k+1$
segments in $\sol$ collinear with different lines.
This contradicts with the assumption that $|\sol| \le k$.
\end{proof}

\begin{claim}
\label{few_points}
If there are more than $k^2$ points from $\points$
that do not lie on any long line,
then $\points$ can not be covered with $k$ segments.
\end{claim}

\begin{proof}
We prove the claim by contradiction.
Let us assume that we have at least $k^2+1$ points
from $\points$ that do not lie on any long line, call this set $A$,
and a solution $\sol$ of size at most $k$
covering all points in $\points$.

Every segment $s$ from $\sol$ covers at most $k$
points from $A$.
This is because if $s$ covered at least $k+1$ points from $A$,
then the line in the direction of $s$ would be a long line
and that contradicts the definiton of $A$.

If every segment from $\sol$ covers at most $k$ points from $A$
and $|\sol| \le k$, then at most $k^2$ points from $A$ are covered by $\sol$
and that contradicts the fact that $\sol$ is a solution to the given
geometric set cover instance.
\end{proof}

We are now ready to give a proof of Theorem \ref{fpt_weighted_segment}.

\begin{proof}[Proof of Theorem \ref{fpt_weighted_segment}]
Our goal is to either answer \texttt{NO} or to find a kernel
$(\sets', \points')$ of bounded size, such that:
\begin{itemize}
\item \textit{(Property 1)} for every solution
$\sol$ to $(\sets, \points)$ of size at most $k$,
there exists a set $\sol_1 \subseteq \sets'$ such that
$\sol_1 \le k$, weight of $\sol_1$ is not greater than weight of $\sol$
and $\sol_1$ covers $\points'$;
\item \textit{(Property 2)}
for every set $\sol_2 \subseteq \sets'$ such that $|\sol_2| \le k$
and $\sol_2$ covers points in $\points'$, $\sol_2^{+\delta}$
covers points in original instance $\points$.
\end{itemize}

If we found such sets $(\sets', \points')$,
using \textit{Property 1} we know that optimum solution 
of size at most $k$ to $(\points', \sets')$
has no greater weight than optimum solution
of size at most $k$ to $(\points, \sets)$.
Using \textit{Property 2} we know that
any solution to $(\points', \sets')$
after $\delta$-extensions covers $\points$.

Therefore finding such sets in desired complexity
is sufficient to prove Theorem \ref{fpt_weighted_segment}.

\paragraph{Definition of $\points'$ and $\sets'$.}
Let us name the number of different long lines as $l$.
Applying Claims \ref{few_long_lines} and \ref{few_points},
if we have more than $k$ different long lines
or more than $k^2$ points from $\points$
that do not lie on any long line, then we answer \texttt{NO},
becase these lemmas prove that there is no solution of size at most $k$
to this instance.

Otherwise, we can split $\points$ into at most $k+1$ sets:
\begin{itemize}
\item $D$: points that do not lie on any long line, $|D| \le k^2$;
\item $C_i$ for $1 \le i \le l$: points that lie on the $i$-th long line, $|C_i| > k$.
\end{itemize}
Note that sets $C_i$ do not need to be disjoint.

Then for every set $C_i$ we can use Lemma \ref{dense_set_exists}
to obtain a $(k,\delta)$-dense set $A_i$
for $C_i$ with $|A_i| \le (2+\frac{2}{\delta})^k$.

We define $\points':= D \cup (\bigcup A_i)$. $\points'$ has size at most
$k^2 + k(2+\frac{2}{\delta})^k$.
We define $\sets'$ as
for every pair of points $\points'$, we can choose one segment from
$\sets$ that has the lowest weight
among segments that cover these points 
or decide that there is no segment that covers them.
There are at most $|\points'|^2$ different segments in $\sets'$,
Therefore both $\sets'$ and $\points'$ have size bounded
by some function $f(k)$.

\paragraph{Proof of Property 1.}
First, we prove that
for every set $\sol_2 \subseteq \sets'$ such that $|\sol_2| \le k$
and $\sol_2$ covers points in $\points'$, $\sol_2^{+\delta}$
covers points in original instance $\points$.

Let us take such a set $\sol_2$.

$\points$ is separated into several parts -- sets $D$ and $C_i$.
Points from $D$ are covered by $\sol_2$, because $D$ is part of $\points'$.
Each point from any $A_i$ is covered, because $A_i$ is a part of $\points'$;
$A_i$ is a $(k,\delta)$-dense set for $C_i$, therefore $\sol_2^{+\delta}$
covers all points in $C_i$. Therefore $\sol_2^{+\delta}$ covers
all points in $\points$.

\paragraph{Proof of Property 2.}
Secondly, we prove that for every solution
$\sol$ to $(\sets, \points)$ of size at most $k$,
there exists a set $\sol_1 \subseteq \sets'$ such that
$\sol_1 \le k$ and
$\sol_1^{+\delta}$ covers all points in $\points$ and
weight of $\sol_1$ is not greater than weight of $\sol$.

For every segment in $\sol$, say $s$,
let us look at the points from $\points'$ that lie on $s$
and call this set of points $F$.
$F$ is a set of collinear points for course.
We can cover $F$ with any segment that covers extreme points of $F$,
because all other points lie on the segment between these points.
Therefore we can replace $s$ with a segment $s'$
that has lowest weight among the points that cover extreme points of $F$.
Such a segment belongs to $\sets'$, because this is how it was defined.
Of course segment $s'$ also have weight no greater than weight of $s$,
because $s$ also covers $F$.

Therefore we produced the set $\sol_1$ that has the same size,
weight not greather than $\sol$ and it covers $\points'$.

\paragraph{Complexity}
We find solutin of $(\points', \sets')$ by iterating
over all possible subsets of $\sets$.
Finding sets $\sets'$ and $\points'$ and then solving 
problem for kernel has overall complexity
$(|\sets| + |\points|)^{O(1)}O((2 + \frac{2}{\delta})^k) + O((k^2 + k(2 + \frac{2}{\delta})^k)^k)$.
\end{proof}


\chapter{W[1]-hardness of $\WeightedSegmentSetCover$}
\label{chapter:w1_hard}

In this chapter we consider the $\WeightedSegmentSetCover$ problem with 
axis-parallel or right-diagonal segments.
In Theorem~\ref{w1_hard} below, we prove that this problem is 
W[1]-hard when parameterized by the size of the solution.
We believe that the construction can be improved to only
utilize the axis-parallel segments.

\wOneHard*

\newcommand{\GridTiling}{\textsc{Grid Tiling}}

\section{$\GridTiling$}

In~order to prove Theorem \ref{w1_hard}
we will show a reduction from a W[1]-hard problem:
\textsc{Grid} \textsc{Tiling}.
This problem was introduced in \cite{marx_grid_tiling}
(the author called it matrix tiling instead).
It was originally described as an approximation problem,
but W[1]-hardness follows directly from the theorems stated there.
For a more contemporary description of this problem
and a proof of W[1]-hardness, see Chapter 14 of \cite{platypus_book}.

\newcommand{\pow}{\mathsf{Pow}}

\begin{defi}
We define the \textbf{powerset} of a set $A$, denoted as $\pow(A)$,
as the set of all subsets of $A$, i.e. $\pow(A) = \{B : B \subseteq A\}$.
\end{defi}

\begin{defi}
In the \textbf{$\GridTiling$} problem we are given integers $n$ and $k$,
and a function
$f : \{1, \ldots, k\} \times \{1, \ldots, k\} \rightarrow \pow(\{1, \ldots, n\} \times \{1, \ldots, n\})$
specifying the set of allowed tiles for each cell of a $k \times k$ grid.
The task is to decide whether there exist functions
$x,y : \{1, \ldots, k\} \rightarrow \{1, \ldots, n\}$
that assign colors from $\{1, \ldots, n\}$
to respectively columns and rows of the grid,
so that $(x(i), y(j)) \in f(i, j)$ for all $i,j \in \{1, \ldots, k\}$.
\end{defi}

In short, in the $\GridTiling$ problem one needs to assign numbers
to rows and columns in such a way
that for every pair of a row and a column,
the pair of colors assigned
to the row and column 
belongs to the allowed set of tiles for this pair.
The next theorem describes the complexity of this problem,
which is W[1]-hard when parameterized by the size of the grid.

\definecolor{alternative_sol}{RGB}{48, 48, 255}
\definecolor{bad_sol}{RGB}{255, 48, 48}

\begin{figure}[h]
\begin{center}
\begin{tabular}{ c|c|c|c|c| } 
          & $x(1)=3$ & $x(2)=1$ & $x(3)=3$ & $x(4) = 7$\\ 
 \hline
 $y(4)=1$
	& \makecell{$\textcolor{alternative_sol}{(2,1)};(2,2);$\\$\textbf{(3,1)};(3,9)$}
	& $\textbf{(1,1)}; (3,1)$
	& $\textbf{(3,1)}; (7,2)$
	& $\textcolor{bad_sol}{(2,1)}; \textbf{(7,1)}$\\ 
 \hline
 $y(3)=1$
	& \makecell{$\textcolor{alternative_sol}{(2,1)};\textbf{(3,1)};$\\$(4,2);(8,2)$}
	& $\textbf{(1,1)}; (1,3)$
	& $\textbf{(3,1)}; (4,3)$
	& $\textcolor{bad_sol}{\textbf{(2,2)}}; \textbf{(7,1)}$\\ 
 \hline
 $y(2)=6$
	& $\textcolor{alternative_sol}{(2,6)};\textbf{(3,6)}$
	& \makecell{$(1,2); \textbf{(1,6)};$\\$(2,6)$}
	& $(2,6); \textbf{(3,6)}$
	& $\textcolor{bad_sol}{(2,6)};\textbf{(7,6)}$\\ 
 \hline
 $y(1)=4$
	& \makecell{$\textcolor{alternative_sol}{(2,4)};(2,6);$\\$\textbf{(3,4)};\textcolor{bad_sol}{(3,9)}$}
	& $\textbf{(1,4)}; \textcolor{bad_sol}{(1,9)}$
	& $\textbf{(3,4)}; \textcolor{bad_sol}{(3,9)}$
	& $\textcolor{bad_sol}{(2,9)}; \textbf{(7,4)}$\\ 
 \hline
\end{tabular}
\end{center}
\caption{\textbf{Example of a $\GridTiling$ instance and its solution.}}
In the first row and column of the table you can see the solution:
functions $x$ and $y$.
The~tiles used in this solution are marked in \textbf{bold}.
If we instead chose the tiles marked in \textcolor{alternative_sol}{blue}
(whenever there is one, taking the tile marked in \textbf{bold} otherwise),
then that corresponds to setting $x(1)=2$, and would also form a correct solution.
On the other hand, if we instead chose the tiles marked in \textcolor{bad_sol}{red}
(as before), then this corresponds to setting ${y(1)=9}$ and $x(4)=2$
and that would $\textbf{not}$ form a correct solution.
Even though the first row is correct,
the cell with coordinates (3,4) requires tile (2,1), not (2,2)
(marked in \textbf{\textcolor{bad_sol}{bold red}}).
\label{fig:grid_tiling_exmample}
\end{figure}


\begin{tw}
\label{grid_tiling_w1_hard}
\textbf{(\cite{marx_grid_tiling}).}
$\GridTiling$ is W[1]-hard when parameterized by $k$ and
assuming ETH, there is no $f(k)\cdot n^{o(k)}$-time
algorithm solving the $\GridTiling$ problem
for any computable function $f$.
\end{tw}

The remainder of this section is devoted to proving Theorem \ref{w1_hard}
by a reduction from a~$\GridTiling$ problem instance
with parameter $k$ (number of rows in the grid)
to a \textsc{Weighted} \textsc{Segment} \textsc{Set} \textsc{Cover}
instance with parameter $k^2$ (size of solution).
This reduction is described in Lemma~\ref{w1_construction}.
This proves the W[1]-hardness of the $\WeightedSegmentSetCover$ problem,
because if we could solve it with an FPT algorithm,
then we could also solve the $\GridTiling$ problem
(which we reduced to $\WeightedSegmentSetCover$).
Therefore, $\WeightedSegmentSetCover$ with setting
described in Theorem \ref{w1_hard}
is at least as hard as the $\GridTiling$ problem.

\newcommand{\hvWeight}{W_{\mathsf{hv}}}
\newcommand{\solWeight}{\hvWeight+k^2\delta }
\newcommand{\instanceSetCover}{(\points, \sets, w, 3k^2+2k)}
\newcommand{\instanceGridTiling}{(n,k,f)}
\newcommand{\yes}{\texttt{YES}}
\newcommand{\no}{\texttt{NO}}

\section{Statement of reduction}

Let us denote an instance of $\GridTiling$ problem as $\instanceGridTiling$ consisting of:
\begin{itemize}
\item the number of colors $n$,
\item the size of the grid $k$,
\item the function specifying the allowed tiles
$f : \{1, \ldots, k\} \times \{1, \ldots, k\} \rightarrow \pow(\{1, \ldots, n\} \times \{1, \ldots, n\})$.
\end{itemize}

Let us also define constants: 
\begin{eqnarray*}
\epsilon & := & \frac{1}{2k^2} \\
\delta & := & \frac{1}{4k^4} \\
\hvWeight & := & 2k^2(n^2+1) -4k^2\epsilon -4k(1-\epsilon)
\end{eqnarray*}
which are going to be used when defining the weight of the constructed
instance of \textsc{Weighted} \textsc{Segment} \textsc{Set} \textsc{Cover}.


\begin{lemma}
\label{w1_construction}
Given an instance $\instanceGridTiling$ of the $\GridTiling$ problem,
we can construct an instance $\instanceSetCover$ of $\WeightedSegmentSetCover$
such that:
\begin{enumerate}[label={(\arabic*)}]
\item \label{part1} if the answer to $\instanceGridTiling$ is $\yes$, then there exists a solution
to $\instanceSetCover$ of weight at most $\solWeight$;
\item \label{part2} if there exists a solution to $\instanceSetCover$ of weight at most $\solWeight$,
then the answer to $\instanceGridTiling$ is $\yes$.
\end{enumerate}
\end{lemma}


First, let us prove Theorem \ref{w1_hard} using Lemma \ref{w1_construction}.

\begin{proof}[Proof of Theorem \ref{w1_hard}]
Let us take any instance $(n,l,f)$ of the $\GridTiling$ problem.
We prove the theorem by contradiction, therefore we assume
that \textsc{Weighted} \textsc{Segment} \textsc{Set} \textsc{Cover}
parameterized by solution size $k = 3l^2+2l$ admits a
$g(k)\cdot n^{o(\sqrt{k})}$-time algorithm for some computable function $g$.

Using Lemma \ref{w1_construction} let us construct an instance $I$
for $(n,l,f)$.
Let us assume that the optimum solution of size at most $k$
to the instance $I$ has weight $u$.
Using \ref{part2} we know that if $u \le \solWeight$,
then the answer to $(n,l,f)$ is $\yes$.
If $u > \solWeight$, then using \ref{part1}
we know that the answer to $(n,l,f)$ must be $\no$.

Therefore if we could find the solution in time $g(k) \cdot n^{o(\sqrt{k})}$,
then we could solve the $\GridTiling$ problem
in time $g(l)\cdot n^{o(l)}$ by constructing an instance of
\textsc{Weighted} \textsc{Segment} \textsc{Set} \textsc{Cover}, solving it 
for parameter $k$ in time $n^{o(\sqrt{3l^2+2l})}$
and then answering based on the weight
of the optimum solution.
As $\mathcal{O}(n^{o(l)}) \subseteq \mathcal{O}(n^{o(\sqrt{3l^2+2l})})$,
the existence of this algorithm contradicts Theorem
\ref{grid_tiling_w1_hard}.
Hence such an algorithm can not exist.
\end{proof}

We prove Lemma \ref{w1_construction} in subsequent sections.
First, we define a constructed instance $I$, later property
\ref{part1} is proved by Lemma \ref{set_cover_solution_exists}
and property \ref{part2} is proved by Lemma \ref{grid_tiling_exists}.

In the proof of Lemma \ref{w1_construction}
(see proof of Lemma \ref{grid_tiling_exists})
we do not use the assumption that
the solution is bounded by the size,
which the problem is parameterized by, $3k^2+2k$.
If we had a permissive FPT algorithm
that finds a solution of any size that still
has weight no more than $\solWeight$,
then we still would have a contradiction with $\GridTiling$ being W[1]-hard
in proof of Theorem \ref{w1_hard}.
Thus, this reduction
proves that the problem is not only W[1]-hard, but assuming ETH 
there also does not exist permissive FPT algorithm for this problem.
Formally we state this in Theorem $\ref{permissive_w1_hard}$ below.


\begin{restatable}{tw}{permissiveWOneHard}
\label{permissive_w1_hard}
\textbf{(Permissive FPT does not exist).}
	Consider the problem of covering a set $\points$ of points
	using segments from a set $\sets$ 
	with non-negative weights $w : \sets \rightarrow \mathbb{R^+}$
	so that the weight of the cover is minimal.
	Let $\sol^k$ be the
	optimum solution to this problem of size at most $k$.
	The task is to find a solution $\sol$ of any size
	such that weight of $\sol$ is not greater than the weight of $\sol^k$.
	
	Assuming ETH, there is no algorithm for this
	problem with running time
	$f(k)\cdot(|\points| + |\sets|)^{o(\sqrt{k})}$
	for any computable function $f$.
	Moreover, this holds even if all segments in $\sets$
	are axis-parallel or right-diagonal.
\end{restatable}

\section{Construction of the $\SegmentSetCover$ instance}
\newcommand{\order}{\mathsf{order}}
\newcommand{\matchv}{\mathsf{match}_v}
\newcommand{\matchh}{\mathsf{match}_h}

We construct an instance $\instanceSetCover$ of $\SegmentSetCover$ as follows.

First, let us choose any bijection
$\order : \{1, \ldots, n^2\} \rightarrow \{1, \ldots, n\} \times \{1, \ldots, n\}$.


Define $\matchv(i, j)$ and $\matchh(i, j)$
as Boolean functions denoting whether two points share x or y coordinate:
$$\matchv(i, j) \text{ is } \true \iff
\order(i) \text{ and } \order(j) \text{ have the same x coordinate,}$$
$$\matchh(i, j) \text{ is } \true \iff
\order(i) \text{ and } \order(j) \text{ have the same y coordinate.}$$


\subsection{Points}

For $1 \le i,j \le k$ and $1 \le t \le n^2$ define points:
	$$h_{i, j, t} := (i \cdot (n^2+1) + t, j \cdot (n^2+1)),$$
	$$v_{i, j, t} := (i \cdot (n^2+1), j \cdot (n^2+1) + t).$$
	
Let us define sets $H$ and $V$ as:
$$H := \{h_{i, j, t} : 1 \le i, j \le k, 1 \le t \le n^2\},$$
$$V := \{v_{i, j, t} : 1 \le i, j \le k, 1 \le t \le n^2\}.$$
	
Let us recall that $\epsilon = \frac{1}{2k^2}$.
For a point $p = (x, y)$ we define points:
$$p^{L} := (x - \epsilon, y),$$
$$p^{R} := (x + \epsilon, y),$$
$$p^{U} := (x, y + \epsilon),$$
$$p^{D} := (x, y - \epsilon).$$

Then we define the point set as follows:
$$\points := H \cup \{p^L : p \in H\} \cup \{p^R : p \in H\}
\cup V \cup \{p^U : p \in V\} \cup \{p^D : p \in V\}.$$

\begin{defi}
	\label{guard_def}
	For every point $p \in H$, we name point $p^L$ its \textbf{left guard}
	and point $p^R$ its \textbf{right guard}.
	
	Similarly for every points $p \in V$, we name point $p^D$ its \textbf{lower guard}
	and point $p^U$ its \textbf{upper guard}.
\end{defi}

\subsection{Segments}
\newcommand{\hor}[4]{\mathsf{hor}_{#1,#2,#3,#4}}
\newcommand{\ver}[4]{\mathsf{ver}_{#1,#2,#3,#4}}
\newcommand{\horbeg}[2]{\mathsf{horBeg}_{#1,#2}}
\newcommand{\verbeg}[2]{\mathsf{verBeg}_{#1,#2}}
\newcommand{\horend}[2]{\mathsf{horEnd}_{#1,#2}}
\newcommand{\verend}[2]{\mathsf{verEnd}_{#1,#2}}

For $1 \le i,j \le k$ and $1 \le t, t_1, t_2 \le n^2$ define segments:
\begin{eqnarray*}
\hor{i}{j}{t_1}{t_2} & := & (h^R_{i,j,t_1}, h^L_{i+1, j, t_2}), \\
\ver{i}{j}{t_1}{t_2} & := & (v^U_{i,j,t_1}, v^D_{i, j+1, t_2}), \\
\horbeg{i}{t} & := & (h^L_{1, i, 1}, h^L_{1, i, t}), \\
\horend{i}{t} & := & (h^R_{k, i, t}, h^R_{k, i, n^2}), \\
\verbeg{i}{t} & := & (v^D_{i, 1, 1}, v^D_{i, 1, t}), \\
\verend{i}{t} & := & (v^U_{i, k, t}, v^U_{i, k, n^2}). \\
\end{eqnarray*}

\newcommand{\allhor}{\mathsf{HOR}}
\newcommand{\allver}{\mathsf{VER}}
\newcommand{\alldiag}{\mathsf{DIAG}}

Next, we define sets of vertical and horizontal segments:
\begin{eqnarray*}
\allhor &:= &\{\hor{i}{j}{t_1}{t_2} : 1 \le i < k, 1 \le j \le k,
1 \le t_1, t_2 \le n^2, \matchh(t_1, t_2) \text{ holds}\} \\
&\cup &\{\horbeg{i}{t} : 1 \le i \le k, 1 \le t \le n^2\}
\\
&\cup &\{\horend{i}{t} : 1 \le i \le k, 1 \le t \le n^2\},
\end{eqnarray*}
\begin{eqnarray*}
\allver &:= &\{\ver{i}{j}{t_1}{t_2} : 1 \le i \le k, 1 \le j < k,
1 \le t_1, t_2 \le n^2, \matchv(t_1, t_2) \text{ holds}\} \\
&\cup &\{\verbeg{i}{t} : 1 \le i \le k, 1 \le t \le n^2\}
\\
&\cup &\{\verend{i}{t} : 1 \le i \le k, 1 \le t \le n^2\}.
\end{eqnarray*}
An example is depicted in Figure \ref{fig:segments_def}.

Finally, we also define a set of right-diagonal segments:
$$\alldiag := \{ (h_{i, j, t}, v_{i, j, t}) :
	1 \le i, j \le k, 1 \le t \le n^2, \order(t) \in f(i, j)\}.$$
An example of such segments is depicted in Figure \ref{fig:diag_def}.

	
\definecolor{guards_color}{RGB}{80, 120, 255}

{\tikzset{guard_h/.style={
    circle, draw=guards_color, fill, fill=guards_color, minimum size=2pt,inner sep=0pt, outer sep=0pt,
    prefix after command= {\pgfextra{\tikzset{every
    label/.style={label distance=0.1cm,rotate=90,text=guards_color,font=\footnotesize}}}}
    }
}
{\tikzset{node_h/.style={
    circle, draw=black, fill, fill=black, minimum size=4pt,inner sep=0pt, outer sep=0pt,
    prefix after command= {\pgfextra{\tikzset{every
    label/.style={label distance=0.1cm,rotate=90,text=black}}}}
    }
}
{\tikzset{guard_v/.style={
    circle, draw=guards_color, fill, fill=guards_color, minimum size=2pt,inner sep=0pt, outer sep=0pt,
    prefix after command= {\pgfextra{\tikzset{every
    label/.style={label distance=0.1cm,text=guards_color,font=\footnotesize}}}}
    }
}
{\tikzset{node_v/.style={
    circle, draw=black, fill, fill=black, minimum size=4pt,inner sep=0pt, outer sep=0pt,
    prefix after command= {\pgfextra{\tikzset{every
    label/.style={label distance=0.1cm,text=black}}}}
    }
}

\newcommand{\addNodeV}[2]{
	\node[guard_v, label={left:$v_{i,j,#2}^D$}] at (0, \l#1) {};
	\node[node_v, label={left:$v_{i,j,#2}$}] at (0,\x#1) {};
	\node[guard_v, label={left:$v_{i,j,#2}^U$}] at (0, \r#1) {};
}

\newcommand{\addNodeH}[2]{
	\node[guard_h, label={left:$h_{i,j,#2}^L$}] at (\l#1,0) {};
	\node[node_h, label={left:$h_{i,j,#2}$}] at (\x#1,0) {};
	\node[guard_h, label={left:$h_{i,j,#2}^R$}] at (\r#1,0) {};
}

\begin{figure}
\begin{center}
\begin{tikzpicture}[main/.style = {draw, circle}]
\tikzmath{
	\step=2;
	\eps=0.5;
	\x1=\step;
	\x2=\x1+\step;
	\x3=\x2+\step;
	\x4=\x3+\step;
	\x5=\x4+\step;
	\x6=\x5+\step;
	\l1=\x1-\eps;
	\r1=\x1+\eps;
	\l2=\x2-\eps;
	\r2=\x2+\eps;
	\l3=\x3-\eps;
	\r3=\x3+\eps;
	\l4=\x4-\eps;
	\r4=\x4+\eps;
	\l5=\x5-\eps;
	\r5=\x5+\eps;
	\l6=\x6-\eps;
	\r6=\x6+\eps;
}

\draw (0,\x1) -- (\x1,0) node[midway, above] {$\delta$};
\draw (0,\x2) -- (\x2,0) node[midway, above] {$\delta$};
\draw (0,\x3) -- (\x3,0) node[midway, above] {$\delta$};
\draw (0,\x5) -- (\x5,0) node[midway, above] {$\delta$};
\draw (0,\x6) -- (\x6,0) node[midway, above] {$\delta$};

\addNodeV{1}{1}
\addNodeV{2}{2}
\addNodeV{3}{3}
\filldraw[black] (0,\x4) circle (0pt) node[anchor=east] {$\ldots$};
\addNodeV{5}{n^2-1}
\addNodeV{6}{n^2}

\addNodeH{1}{1}
\addNodeH{2}{2}
\addNodeH{3}{3}
\filldraw[black] (\x4,0) circle (0pt) node[anchor=north] {$\ldots$};
\addNodeH{5}{n^2-1}
\addNodeH{6}{n^2}
\end{tikzpicture} 
\end{center}
\caption{\textbf{Vertices and segments in $\alldiag$.}}
This is an example of constructed points any $1 \le i,j \le k$.
Points from $H$ and $V$ are marked in black,
their guards are marked in \textcolor{guards_color}{blue}.
You can also see segments from $\alldiag$ with their weights (equal to $\delta$).
\label{fig:diag_def}
\end{figure}


Every segment in $\alldiag$
connects points $(i(n^2+1) + t, j(n^2+1))$
and $(i(n^2+1), j(n^2+1) + t)$
for some $1 \le i,j \le k, 1 \le t \le n^2$.
The line on which it lies can be described
by linear equation ${x+y=t+(i+j)(n^2+1)}$,
thus these segments are in fact right-diagonal.

The constructed segment set is defined as:

$$\sets := \allhor \cup \allver \cup \alldiag.$$

The weight of each segment in $\allhor \cup \allver$
is equal to its length,
while every segment in $\alldiag$ has weight
$\delta$.


\definecolor{beg_color}{RGB}{255, 40, 40}
\definecolor{seg_color1}{RGB}{40, 40, 255}
\definecolor{seg_color2}{RGB}{40, 150, 40}

\newcommand{\addNode}[4]{
	\node[guard_h, label={left:$h_{#1,j,t_{#2,#3}}^L$}] at (\l#4,0) {};
	\node[node_h, label={left:$h_{#1,j,t_{#2,#3}}$}] at (\x#4,0) {};
	\node[guard_h, label={left:$h_{#1,j,t_{#2,#3}}^R$}] at (\r#4,0) {};
}

\begin{figure}
\hspace*{-1.5cm}
\begin{tikzpicture}[main/.style = {draw, circle}]
\tikzmath{
\step=2;
\eps=0.5;
%genereted by gen_math.py
\x1=\step;
\x2=\x1+\step;
\x3=\x2+\step;
\x4=\x3+\step;
\x5=\x4+\step;
\x6=\x5+\step;
\x7=\x6+\step;
\x8=\x7+\step;
\x9=\x8+\step;
\l1=\x1-\eps;
\r1=\x1+\eps;
\l2=\x2-\eps;
\r2=\x2+\eps;
\l3=\x3-\eps;
\r3=\x3+\eps;
\l4=\x4-\eps;
\r4=\x4+\eps;
\l5=\x5-\eps;
\r5=\x5+\eps;
\l6=\x6-\eps;
\r6=\x6+\eps;
\l7=\x7-\eps;
\r7=\x7+\eps;
\l8=\x8-\eps;
\r8=\x8+\eps;
\l9=\x9-\eps;
\r9=\x9+\eps;
}

\draw [beg_color] (\l1,0) to[out=150,in=30, looseness=200] (1.49,0);
\draw [beg_color] (\l1,0) to[out=40,in=140] (\l2,0);
\draw [beg_color] (\l1,0) to[out=40,in=140] (\l3,0);
\draw [beg_color] (\l1,0) to[out=40,in=140] (\l4,0);


\draw [seg_color1] (\r1,0) to[out=20,in=160] (\l6,0);
\draw [seg_color1] (\r1,0) to[out=20,in=160] (\l8,0);
\draw [seg_color2] (\r2,0) to[out=20,in=160] (\l7,0);
\draw [seg_color2] (\r2,0) to[out=20,in=160] (\l9,0);
\draw [seg_color1] (\r3,0) to[out=20,in=160] (\l6,0);
\draw [seg_color1] (\r3,0) to[out=20,in=160] (\l8,0);
\draw [seg_color2] (\r4,0) to[out=20,in=160] (\l7,0);
\draw [seg_color2] (\r4,0) to[out=20,in=160] (\l9,0);

\addNode{1}{1}{1}{1}
\addNode{1}{1}{2}{2}
\addNode{1}{2}{1}{3}
\addNode{1}{2}{2}{4}
\addNode{2}{1}{1}{6}
\addNode{2}{1}{2}{7}
\addNode{2}{2}{1}{8}
\addNode{2}{2}{2}{9}

\end{tikzpicture} 
\caption{\textbf{Vertices and segments in $HOR$.}}
This is an example for $n=2$ and any $1 \le j \le k$.
Points from $H$ are marked in black, their guards are marked in \textcolor{guards_color}{light blue}.
$t_{i,j}$ is a notation that we use for $\order^{-1}(i,j)$.
Segments are represented as arcs between endpoints.
You can see $\horbeg{j}{t}$ segments in \textcolor{beg_color}{red}.
$\horbeg{j}{1}$ is degenerated to a single point at $h_{1,1,t_{1,1}}^L$.
Segments $\hor{i}{j}{t_{x_1,y}}{t_{x_2,y}}$
are marked in \textcolor{seg_color1}{blue} and \textcolor{seg_color2}{green}.
\textcolor{seg_color1}{Blue} segments connect $t_{x_1,y}$ and $t_{x_2,y}$
such that they share y-coordinate equal to 1,
for~\textcolor{seg_color2}{green} segments it is equal to 2.

\label{fig:segments_def}
\end{figure}


\begin{equation}
w(s) =
	\begin{cases*}
	  length(s) 			& if $s \in \allhor \cup \allver$ \\
	  \delta        & if $s \in \alldiag$
	\end{cases*}
\end{equation}

\section{Proof that the reduction is correct}

Now, we prove that the constructed instance of $\WeightedSegmentSetCover$
indeed gives a correct and sound reduction
of the $\GridTiling$ problem. Lemma \ref{set_cover_solution_exists}
proves that if a solution to the instance of the $\GridTiling$ instance exists,
then there exists a solution with suitably bounded size and weight
of the constructed instance of
\textsc{Weighted} \textsc{Segment} \textsc{Set} \textsc{Cover}.
Then Lemma \ref{grid_tiling_exists} proves that if
there is a solution to the
\textsc{Weighted} \textsc{Segment} \textsc{Set} \textsc{Cover}
instance with bounded weight,
then there exists a solution to the original $\GridTiling$ instance.

\begin{lemma}
\label{set_cover_solution_exists}
	If there exists a~solution to the $\GridTiling$ instance $(n,k,f)$,
	then there exists a~solution to the instance $\instanceSetCover$
	of $\WeightedSegmentSetCover$ with weight $\solWeight$.
\end{lemma}

\begin{proof}
Suppose there exists a solution $x,y$ of the instance $(n,k,f)$
of the $\GridTiling$ problem.
	
We define the proposed solution $\sol \subseteq \sets$ of the instance
of \textsc{Weighted} \textsc{Segment} \textsc{Set} \textsc{Cover}
in three parts: $D \subseteq \alldiag$, $A \subseteq \allhor$ and $B \subseteq \allver$:
\begin{eqnarray*}
	D & := & \{(v_{i, j, t}, h_{i, j, t}) : 1 \le i, j \le k, t = \order^{-1}(x(i), y(j))\}, \\
	A & := & \{\horbeg{i}{\order^{-1}(x(1), y(i))} : 1 \le i \le k\} \\
	& \cup & \{\horend{i}{\order^{-1}(x(k), y(i))} : 1 \le i \le k\} \\
	& \cup & \{\hor{i}{j}{\order^{-1}(x(i), y(j))}{\order^{-1}(x(i+1), y(j))} : 1 \le i < k, 1 \le j \le k\}, \\
	B & := & \{\verbeg{i}{\order^{-1}(x(i), y(1))} : 1 \le i \le k\} \\
	& \cup & \{\verend{i}{\order^{-1}(x(i), y(k))} : 1 \le i \le k\} \\
	& \cup & \{\ver{i}{j}{\order^{-1}(x(i), y(j))}{\order^{-1}(x(i), y(j+1))} : 1 \le i \le k, 1 \le j < k\},
\end{eqnarray*}
	$$\sol := D \cup A \cup B.$$

Since $\points = H \cup V$, we show that $\sol$ covers the whole set $H$;
the proof for $V$ is analogous.

Fix any $1 \le j \le k$ and define $t_{i} := \order^{-1}(x(i), y(j))$.
The two leftmost segments in~$A$ for this $j$ are
$\horbeg{j}{t_1} = (h_{1,j,1}^L, h_{1, j, t_1}^L)$ and
$\hor{1}{j}{t_1}{t_2} = (h_{1,j, t_1}^R, h_{2,j,t_2}^L)$.
Therefore, points $h_{1,j,x}, h_{1,j,x}^L$ and $h_{1,j,x}^R$
for all $1 \le x \le n^2$ ale covered by $\horbeg{j}{t_1}$ and $\hor{1}{j}{t_1}{t_2}$,
excluding point $h_{1,j,t_1}$.

Analogously for $2 \le i \le k-1$,
the two consecutive segments $\hor{i-1}{j}{t_{i-1}}{t_i}$
and $\hor{i}{j}{t_i}{t_{i+1}}$ cover points $h_{i,j,x}, h_{i,j,x}^L$ and $h_{i,j,x}^R$
for all $1 \le x \le n^2$,
excluding point $h_{i,j,t_i}$.

Finally $\hor{k-1}{j}{t_{k-1}}{t_k}$ and $\horend{j}{t_k}$
cover all points $h_{k,j,x}, h_{k,j,x}^L$ and $h_{k,j,x}^R$
for ${1 \le x \le n^2}$, excluding point $h_{k,j,t_k}$.

$D$ covers all points $h_{i,j,t_i}$ and $v_{i,j,t_i}$.
As $j$ was chosen arbitrarily, all points in $H$ are covered.

The size of this proposed solution is:
$$|\sol| = |D| + |A| + |B| = k^2 + (k+1)k + (k+1)k = 3k^2+2k.$$

Then, we need to compute the total weight of the solution $\sol$.
First, we compute the sum of weights of segments in $A$.
Fix $1 \le j \le k$ and consider segments collinear
with the $j$-th horizontal line.
All points $h_{i,j,t}$, $h_{i,j,t}^L$ and $h_{i,j,t}^R$
for every $1 \le i \le k$ and $1 \le t \le n^2$ are covered by $A$
excluding points $h_{i,j,\order^{-1}(x(i),y(j))}$.
Every such point leaves a gap of length $2\epsilon$ between
$h_{i,j,\order^{-1}(x(i),y(j))}^L$ and $h_{i,j,\order^{-1}(x(i),y(j))}^R$.
Therefore, the total weight of segments in $A$
that lie on the line in question equals the length of the segment
$(h_{i,1,1}^L, h_{i,k,n^2}^R)$
minus $2\epsilon k$, which is $k(n^2+1) -2(1-\epsilon)-2k\epsilon$.
We need to multiply that by $k$, as we consider all possible values of $j$.

Computation for vertical segments is analogous and yields the same result.
Every segment in~$D$ has weight $\delta$, therefore the sum of all weights
is equal to:

$$2k(k(n^2+1) -2(1-\epsilon)-2k\epsilon) + k^2\delta= \solWeight.\qedhere$$
\end{proof}

Now we present a few additional properties of the constructed instance
of the \textsc{Weighted} \textsc{Segment} \textsc{Set} \textsc{Cover}
that help us to prove Lemma \ref{grid_tiling_exists}.

\begin{claim}
\label{guards}
In any solution to the instance $\instanceSetCover$:
\begin{itemize}
\item the left and right guards of points in $H$
(points in $\{p^L : p \in H\} \cup \{p^R : p \in H\}$)
have to~be~covered with segments from $\allhor$,
\item the lower and upper guards of points in $V$
(points in $\{p^D : p \in V\} \cup \{p^U : p \in V\}$)
have to~be~covered with segments from $\allver$.
\end{itemize}
\end{claim}

\begin{proof}
We prove the claim for the points from $H$
as the proof for points from $V$ is analogous.

Every segment in $\allver$ is vertical and 
has x-coordinate equal to $i(n^2+1)$ for some $1\le i \le k$,
so they all have different x-coordinate
than any left or right guard of points in $H$.

For every point $x$ which is a left or right guard of a point in $H$,
there are $kn^2$ segments from $\alldiag$ that intersect with the horizontal
line that goes through $x$. All of these segments intersect with
this line in points from set $H$, therefore none of them
covers any of the guards.

Therefore none of the segments from $\allver$ or $\alldiag$ covers any
of the guards of the points in $H$.
\end{proof}

\begin{claim}
\label{one_diag_in_square}
For any $1 \le i, j \le n$
and any solution to the instance $\instanceSetCover$,
all but at most one point $h_{i, j, t}$
and at most one point $v_{i, j, t}$
for $1 \le t \le n^2$
must be
covered with segments from $\allhor$ or $\allver$.
\end{claim}

\begin{proof}
We prove the claim for horizontal segments,
as the proof for vertical segments is analogous.

We prove this by contradiction. Assume that we
have two points $h_{i,j,t_1}, h_{i,j,t_2},1 \le t_1 <  t_2 \le n^2$,
such that they are not covered with segments from $\allhor$.

Point $h^R_{i, j, t_1}$ has to be covered with a segment from $\allhor$
by Claim $\ref{guards}$.
Every segment in $\allhor$ covering $h^R_{i, j, t_1}$
but not $h_{i,j,t_1}$ must start at $h^R_{i, j, t_1}$
and all such segments cover also $h_{i, j, t_2}$.
This contradicts the assumption, which concludes the proof.
\end{proof}

\begin{lemma}
\label{vertical_horizontal_sum}
For every solution $\sol$ to the instance $\instanceSetCover$,
the sum of weights of segments chosen
from sets $\allhor$ and $\allver$ is at least
$\hvWeight$.
\end{lemma}

\begin{proof}
Let us fix $1 \le i \le k$.

We provide a lower bound for the sum of lengths
of vertical segments from $\sol \cap \allver$.
This bound is the same for each $i$ and is the same
for horizontal lines, thus we need to multiply such a bound by $2k$.

\begin{enumerate}[label={(\arabic*)}]
\item The total length between $v^D_{i, 1, 1}$ and $v^U_{i, k, n^2}$ is:
$$(k(n^2+1) + n^2 +\epsilon) - ((n^2+1)+1 -\epsilon) = k(n^2+1) - 2(1 - \epsilon).$$

\item For every $1 \le j \le k$ there exists at most one $1 \le t \le n^2$
such that $v_{i,j,t}$ is not covered by segments from $\allver$
(Claim \ref{one_diag_in_square}).
Its guards (see Definition \ref{guard_def}) $v^U_{i,j,t}$ and $v^D_{i,j,t}$
have to be covered in $\allver$ (Claim \ref{guards}).
Therefore, at most $k$ spaces of length $2\epsilon$ can be left
not covered by segments from $\allver$ between $v_{i,1,1}^D$ and $v_{i,k,n^2}^U$.

\end{enumerate}
The sum of these lower bounds for vertical and horizontal lines is:
$$2k(k(n^2+1) -2k\epsilon -2(1-\epsilon)) = 2k^2(n^2+1) -4k^2\epsilon -4k(1-\epsilon) = \hvWeight.\qedhere$$
\end{proof}

\begin{lemma}
\label{diag_correct}
Let $\sol$ be a solution to a constructed instance $\instanceSetCover$
with weight at most  $\solWeight$.
Then for every $1 \le i,j \le k$
there exists $1 \le t \le n^2$ such that:
\begin{enumerate}[label={(\arabic*)}]
\item $v_{i,j,t}, h_{i,j,t}$ are not covered by segments from $\allver$ or $\allhor$;
\item segment $(v_{i,j,t}, h_{i,j,t})$ is in solution $\sol$;
\item $\order(t) \in f(i,j)$, that is, $\order(t)$ is an allowed tile for $(i,j)$;
\item for every $1 \le s\le n^2$, $s \neq t$, $v_{i,j,s}$ is covered in $\allver$;
\item for every $1 \le s\le n^2$, $s \neq t$, $h_{i,j,s}$ is covered in $\allhor$.
\end{enumerate}
\end{lemma}

\begin{proof}
At most one of the points $\{h_{i,j,t_x} : 1 \le t_x \le n^2\}$
and one of the points $\{v_{i,j,t_y} : 1 \le t_y \le n^2\}$
is covered with $\alldiag$
(Claim \ref{one_diag_in_square}).
	
Moreover, exactly one such point $h_{i,j,t_x}$ and one such point $v_{i,j,t_y}$
is covered with $\alldiag$,
because if none of them were covered, then the solution would have to
have weight at least $\hvWeight + 2\epsilon$ (see the proof of Lemma \ref{vertical_horizontal_sum}),
which is more than $\solWeight$.

We observe that points $h_{i,j,t_x}$ and $v_{i,j,t_y}$
have to be covered with the same segment from $\alldiag$.
Indeed we need to use at least $k^2$ of them to use
exactly one DIAG segment for every pair of $1 \le i,j \le k$,
if we used 2 segments from $\alldiag$
for one pair $(i,j)$,
then we would have used total weight at least
$\hvWeight + k^2\delta + \delta$ (Lemma \ref{vertical_horizontal_sum}),
which is more than $\solWeight$.
Since points $h_{i,j,t_x}$ and $v_{i,j,t_y}$ are covered by
a single segment from $\alldiag$, we have $t_x = t_y$.

Therefore $t_x = t_y$
and $\order(t_x)$ is an allowed tile for $(i,j)$
because the corresponding segment is in $\alldiag$.
\end{proof}

\newcommand{\diagonal}{\mathsf{diagonal}}
We refer to the function mapping from a pair $(i,j)$, where $1\le i,j \le k$,
to a number $t_x$ from Lemma \ref{diag_correct}
as $\diagonal : \{1, \ldots, k\} \times \{1, \ldots, k\} \rightarrow \{1, \ldots, n^2\}$.

\begin{lemma}
\label{vertical_horizontal_synchronized}
Let $\sol$ be any solution
of a constructed instance $\instanceSetCover$
with weight at most $\solWeight$. Then:
\begin{enumerate}
\item 
for any $1 \le i < k, 1 \le j \le k$,
$\matchh(\diagonal(i, j),\diagonal(i+1, j))$ is $\true$;
\item 
for any $1 \le i \le k, 1 \le j < k$,
$\matchv(\diagonal(i, j),\diagonal(i, j+1))$ is $\true$.
\end{enumerate}
\end{lemma}

\begin{proof}
We prove (1) by contradiction, the proof of (2) is analogous.

Let us take any $1 \le i < k, 1 \le j \le k$
and name $t_1 = \diagonal(i, j)$ and $t_2 = \diagonal(i+1, j)$.
We also assume that $\matchh(t_1,t_2)$ is \false,
which is equivalent to the fact that
segment $(h_{i,j,t_1}^R, h_{i+1,j,t_2}^L)$
is not in set $\allhor$.

Therefore $h_{i,j,t_1}$ and $h_{i+1,j,t_2}$
are not covered by segments from $\allhor$ (Lemma \ref{diag_correct}),
while $h^R_{i,j,t_1}$ and $h^L_{i+1,j,t_2}$
have to be covered by segments from $\allhor$ (Claim \ref{guards}).


Every segment from $\allhor$ either:
\begin{itemize}
\item starts at point $h^R_{x,y,z_1}$
and ends at point $h^L_{x+1,y,z_2}$ for some
$1 \le x < k$,$1 \le y \le k$ and $1 \le z_1, z_2 \le n^2$; or
\item is $\horbeg{y}{z}$ 
and starts at $h^L_{1,y,1}$ and ends at $h^L_{1,y,z}$ for some $1 \le y \le k$ and $1 \le z \le n^2$; or
\item is $\horend{y}{z}$
and starts at $h^R_{k,y,z}$ and ends at $h^R_{k,y,n^2}$ for some $1 \le y \le k$ and $1 \le z \le n^2$.
\end{itemize}
All of the points between $h^R_{i,j,t_1}$ and $h^L_{i+1,j,t_2}$
are covered by segments in $\allhor$ 
and there is no segment $(h^R_{i,j,t_1}, h^L_{i+1,j,t_2})$ in $\allhor$.
Hence, there are at least two different segments covering them.
If both of these segments are neither $\horbeg{y}{z}$ nor $\horend{y}{z}$,
then one of them must begin
at $h^R_{i,j,t_1}$ and end at $h^L_{i+1,j,z_2}$
and there must be other one that begins at $h^R_{i,j,z_1}$
and ends at $h^L_{i+1,j,t_2}$
for some $1 \le z_1, z_2 \le n^2$.

Thus, the space between $h^R_{i,j,z_1}$ and $h^L_{i,j+1,z_2}$
would be covered twice and is longer than $\epsilon$.
The case when one of them is $\horbeg{y}{z}$ or $\horend{y}{z}$ is analogous.
Note that they cannot be both $\horbeg{y}{z}$ or $\horend{y}{z}$.

By the proof of Lemma \ref{vertical_horizontal_sum},
the lower bound for weight of such a solution is $\hvWeight + \epsilon$
which is more than $\solWeight$.

Therefore ${h^R_{i,j,t_1}}$ and ${h^L_{i+1,j,t_2}}$ must be covered
by one segment from $\allhor$, namely \linebreak ${(h^R_{i,j,t_1}, h^L_{i+1,j,t_2})}$.
Hence $(h^R_{i,j,t_1}, h^L_{i+1,j,t_2})$ is a segment in $\allhor$
and $\matchh(t_1,t_2)$ is $\true$.
\end{proof}


\begin{lemma}
\label{grid_tiling_exists}
	If there exists a solution to instance $\instanceSetCover$
	with weight at most $\solWeight$,
	then there exists a solution to the $\GridTiling$ instance $(n,k,f)$.
\end{lemma}

\begin{proof}
Take $\diagonal$ function from Lemma \ref{diag_correct}.

To define the $x$ function 
for every $1 \le i \le k$ set $x(i) := x_i$
where $(x_i, a) = \order(v_{i,1})$.
Similarly, to define the $y$ function,
for every $1 \le i \le k$ set $y(i) := y_i$
where $(b, y_i) = \order(h_{1,i})$

To prove that this is a correct solution to $\GridTiling$,
we need to prove that 
for every $1 \le i,j \le k$, $(x(i), y(j))$ is in
the allowed tiles set $f(i,j)$.

Let us take any $1 \le i,j \le k$.
By Lemma \ref{vertical_horizontal_synchronized}
and simple induction,
we know that $\matchh(\diagonal(1, j),\diagonal(i, j))$ and
$\matchv(\diagonal(i, 1),\diagonal(i, j))$ are $\true$.
Therefore $\order(\diagonal(i,j)) = (x(i), y(j))$.
By Lemma \ref{diag_correct} we know that 
$\order(\diagonal(i,j))$ is in $f(i,j)$.
Therefore 
$(x(i), y(j))$
is in $f(i,j)$.
\end{proof}



\chapter{Geometric Set Cover with lines}
\section{Lines parallel to one of the axis}
When $\mathcal{R}$ consists only of lines parallel to
one of the axis, the problem can be solved in
polynomial time.

We create bipartial graph $G$ with node for every line on the input
split into sets: $H$ -- horizontal lines and $V$ -- vertical lines.
If any two lines cover the same point from $\mathcal{C}$, then
we add edge between them.

Of course there will be no edges between nodes inside $H$,
because all of them are pararell and if they share 
one point, they are the same lines. Similar argument for $V$.
So the graph is bipartial.

Now Geometric Set Cover can be solved with Vertex Cover on graph $G$.
Since Vertex Cover (even in weighted setting) 
on bipartial graphs can be solved in polynomial time.

Short note for myself just to remember how to this in polynomial time:

Non-weighted setting - Konig theorem + max matching

Weighted setting - Min cut in graph of $\neg A$ or $\neg B$
(edges directed from $V$ to $H$)

\section{FPT for arbitrary lines}
You can find this is Platypus book.
We will show FPT kernel of size at most $k^2$.

(Maybe we need to reduce lines with one point/points with one line).

For every line if there is more than $k$ points on it,
you have to take it. At the end, if there is more than $k^2$ points,
return NO.
Otherwise there is no more than $k^4$ lines.

In weighted settings among the same lines with different weights
you leave the cheapest one and use the same algorithm.

\section{APX-completeness for arbitrary lines}
We will show a reduction from Vertex Cover problem.
Let's take an instance of the Vertex Cover problem for graph $G$.
We will create a set of $|V(G)|$ pairwise non-pararell lines,
such that no three of them share a common point.

Then for every edge in $(v, w) \in E(G)$
we put a point on crossing of lines for vertices $v$ and $w$.
They are not pararell, so there exists exactly one such point
and any other line do not cover this point (any three of them do not
cross in the same point).

Solution of Geometric Set Cover for this instance would yield
a sound solution of Vertex Cover for graph $G$.
For every point (edge) we need to choose at least one of
lines (vertices) $v$ or $w$ to cover this point.

Vertex Cover for arbitrary graph is APX-complete,
so this problem in also APX-complete.

\section{2-approximation for arbitrary lines}
Vertex Cover has an easy 2-approximation algorithm,
but here very many lines can cross through
the same point, so we can do $d$-approximation,
where $d$ is the biggest number of lines crossing through the same point.
So for set where any 3 lines do not cross in the same point
it yields 2-approximation.

The problematic cases are where through all points
cross at least $k$ points and all lines have at least $k$ points on them.
It can be created by casting $k$-grid in $k$-D space on 2D space.

Greedy algorithm yields $\log |\mathcal{R}|$-approximation,
but I have example for this for bipartial graph and
reduction with taking all lines crossing through some point
(if there are no more than $k$) would solve this case.
So maybe it works.

Unfortunaly I have not done this :(

I can link some papers telling it's hard to do.

\section{Connection with general set cover}
Problem with finite set of lines with more dimensions
is equivalent
to problem in 2D, because we can project
lines on the plane which is not perpendicular
to any plane created by pairs of
(point from $\mathcal{C}$, line from $\mathcal{P}$).

Of course every two lines have at most one common point,
so is every family of sets that have at most one point
in common equivalent to some geometric set cover with lines?

No, because of Desargues's theorem.
Have to write down exactly what configuration is banned.


\chapter{Geometric Set Cover with polygons}
\section{State of the art}

Covering points with weighted discs admits PTAS \cite{li}
and with fat polygons with $\delta$-extensions with unit weights
admits EPTAS \cite{harpeled12}.

Although with thin objects, even if we allow $\delta$-expansion,
the Set Cover with rectangles
is APX-complete (for~$\delta = 1/2$),
it follows from APX-completeness for segments with $\delta$-expansion
in Section \ref{section:segment_apx}.

Covering points with squares is W[1]-hard \cite{marx05}.
It can be proven that assuming $SETH$,
there is no $f(k)\cdot(|\points|+|\sets|)^{k-\epsilon}$ time algorithm
for any computable function $f$ and $\epsilon >0$ that decides if there
are $k$ polygons in $\sets$ that together cover $\points$,
\textit{Theorem 1.9} in \cite{voronoi}.








\chapter{Conclusions}
We do not know FPT for axis-parallel segments without $\delta$-extensions.

\bibliographystyle{apalike}
\bibliography{bibl}

\end{document}


%%% Local Variables:
%%% mode: latex
%%% TeX-master: t
%%% coding: latin-2
%%% End:
