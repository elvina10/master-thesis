\section{W[1]-completeness for weighted segments in 3 directions}

\begin{tw}
	\textbf{W[1]-completeness for weighted segments in 3 directions}.
	Consider the problem of covering a set $\points$ of points
	by selecting $k$ axis-pararell or right-diagonal weighted segments
	with weights
	from a set $\sets$ with minimal weight.
	Assuming ETH, there is no algorithm for this
	problem with running time
	$f(k)\cdot(|\points| + |\sets|)^{o(\sqrt(k))}$
	for any computable function $f$.
\end{tw}

We will show reduction from grid tiling problem.


Let's have an instance of grid tiling problem -- size of the
gird $k$, number of elements available $n$
and $k^2$ sets of available pairs in every tile
$S_{i, j} \subseteq \{1,n\} \times \{1,n\}$.

\paragraph{Construction.}
We construct a set $\sets$ of segments and a set $\points$
of points.

First let's choose any ordering of $n^2$ elements
$\{1,n\} \times \{1,n\}$ and name this sequence $a_1 \ldots a_{n^2}$.

$$match_v(i, j) \iff
a_i = \{x_i, y_i\} \land a_j = \{x_j, y_j\} \land x_i = x_j$$

$$match_h(i, j) \iff
a_i = \{x_i, y_i\} \land a_j = \{x_j, y_j\} \land y_i = y_j$$


\subparagraph{Points.}

Define points:
	$$h_{i, j, t} = (j \cdot (n^2+1) + t, (n^2+1) \cdot i)$$
	$$v_{i, j, t} = ((n^2+1) \cdot i, j \cdot (n^2+1) + t)$$
	
Let's define sets $H$ and $V$ as:
$$H = \{h_{i, j, t} : 1 \le i, j, \le k, 1 \le t \le n^2\}$$
$$V = \{v_{i, j, t} : 1 \le i, j, \le k, 1 \le t \le n^2\}$$
	
Let's define $\epsilon = 0.1$.
For a point $\{x, y\} = p$ we define points
$p^{L} = \{x - \epsilon, y\}$,
$p^{R} = \{x + \epsilon, y\}$,
$p^{U} = \{x, y - \epsilon\}$,
and $p^{D} = \{x, y + \epsilon\}$.

Then we define:
$$\points := H \cup \{p^L : p \in H\} \cup \{p^R : p \in H\}
\cup V \cup \{p^U : p \in V\} \cup \{p^D : p \in V\} $$


\subparagraph{Segments.}
Define horizontal segments.

$$hor_{i, j, t_1, t_2} = (h^R_{i,j,t_1}, h^L_{i, j+1, t_2})$$
$$ver_{i, j, t_1, t_2} = (v^D_{i,j,t_1}, v^U_{i, j+1, t_2})$$

$$horbeg_{i, t} = (h^L_{i, 1, 1}, h^L_{i, 1, t})$$
$$horend_{i, t} = (h^R_{i, n, t}, h^R_{i, n, n^2})$$


$$verbeg_{i, t} = (v^U_{i, 1, 1}, v^U_{i, 1, t})$$
$$verend_{i, t} = (v^D_{i, n, t}, v^D_{i, n, n^2})$$

\begin{eqnarray*}
HOR &= &\{hor_{i, j, t_1, t_2} : 1 \le i \le k, 1 \le j < k,
1 \le t_1, t_2 \le n^2, match_h(t_1, t_2)\} \\
&\cup &\{horbeg_{i,t} : 1 \le i \le k, 1 \le t \le n^2\}
\\
&\cup &\{horend_{i,t} : 1 \le i \le k, 1 \le t \le n^2\}
\end{eqnarray*}

\begin{eqnarray*}
VER &= &\{ver_{i, j, t_1, t_2} : 1 \le i \le k, 1 \le j < k,
1 \le t_1, t_2 \le n^2, match_v(t_1, t_2)\} \\
&\cup &\{verbeg_{i,t} : 1 \le i \le k, 1 \le t \le n^2\}
\\
&\cup &\{verend_{i,t} : 1 \le i \le k, 1 \le t \le n^2\}
\end{eqnarray*}

$$DIAG := \{ (h_{i, j, t}, v_{j, i, t}) :
	1 \le i, j \le k, 1 \le t \le n^2, a_t \in S_{i, j}\}$$

TODO: explain that these segments are in fact diagonal

$$\sets := HOR \cup VER \cup DIAG$$



\begin{lemma}
	If there exists solution for grid tiling,
	then there exists solution for our construction
	using $2(k+1)k + k^2$ segments
	with weight exactly $2k \cdot (k(n^2+1) - 2 - 2\epsilon(k-1))$.
\end{lemma}

\begin{claim}
	If there exists a solution to the grid tiling
	$c_1 \ldots c_k$ and $r_1 \ldots r_k$,
	then there exists a solution covering
	all points
	$$\{h_{i, j, t} : 1 \le i, j \le k, t=(c_i, r_j)\}
	\cup \{v_{i, j, t} : 1 \le i, j \le k, t=(c_j, r_i)\}$$
	
	with segments in $DIAG$
	and the rest in $VER$ or $HOR$
	and has weight $2k \cdot (k(n^2+1) - 2 - 2\epsilon(k-1))$.
\end{claim}

\paragraph{Proof.}
TODO: jakiś prosty z definicji

\begin{lemma}
	If there exists solution for our construction
	using $2(k+1)k + k^2$ segments
	with weight exactly $2k \cdot (k(n^2+1) - 2 - 2\epsilon(k-1))$,
	then there exists a solution for grid tiling
\end{lemma}

\paragraph{Proof.}
This follows from Lemma $\ref{main_soundness_construction}$,
because we just take which points are covered with $DIAG$.

\begin{claim}
\label{guards}
Points $p^{L}, p^R, p^U, p^D$ cannot be covered with $DIAG$.
\end{claim}

\begin{claim}
\label{hor_ver_sound}
Points in $H \cup \{p^L : p \in H\} \cup \{p^R : p \in H\}$
cannot be covered with $VER$.

Points in $V \cup \{p^U : p \in V\} \cup \{p^D : p \in V\} $
cannot be covered with $HOR$.
\end{claim}

\begin{claim}
For given $i, j$ if none of the points $h_{i, j, t}$ ($v_{i, j, t}$)
for $1 \le t \le n^2$ are covered with $DIAG$,
then some spaces between neighbouring points were covered twice.
\end{claim}

\begin{claim}
For given $i, j$ two points $h_{i, j, t_1}, h_{i, j, t_2}$
($v_{i, j, t_1}, v_{i, j, t_2}$)
for $1 \le t_1 < t_2 \le n^2$ are covered with $DIAG$,
then one of them had to be also covered with a segment from $HOR$
($VER$).
\end{claim}
\paragraph{Proof.}
Point $v^L_{i, j, t_2}$ had to be covered with $VER$
from Claims $\ref{guards}$ and $\ref{hor_ver_sound}$.
And every segment in $VER$ covering $v^L_{i, j, t_2}$,
covers also $v^L_{i, j, t_1}$.

\begin{lemma}
	\label{main_soundness_construction}
	If there exists solution for our construction
	with weight at most (exactly)
	$2k \cdot (k(n^2+1) - 2 - 2\epsilon(k-1))$,
	then for every $i, j$
	there must be exactly one $t$ such that $h_{i, j, t}$
	($v_{i, j, t}$) 
	is covered with $DIAG$
	and moreover if $h_{i, j, t_1}$ and $h_{i, j+1, t_2}$
	are uncovered, then $math_h(t_1, t_2)$.
	Analogically for $v$.
\end{lemma}
\paragraph{Proof.}
Only $k^2$ points can be covered only in $DIAG$, the rest
has to be covered with $VER \cup HOR$.
Therefore every result must be at least $ALL\_LINES$ - $2k^2\epsilon$,
because only $2k^2$ spaces of length $\epsilon$
can be uncovered in this axis.

Of course if $h_{i, j, t_1}$ and $h_{i, j+1, t_2}$
are uncovered, then there must exist
a segment in $HOR$ between $h^R_{i, j, t_1}$ and $h^L_{i, j+1, t_2}$,
so $math_h(t_1, t_2)$ must be true.



\section{What is missing}
We don't know FPT for axis-pararell segments without $\delta$-extensions.
