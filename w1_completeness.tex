In this chapter we consider geometric set cover problem with weighted segments.
Theorem \ref{w1_hard} proves that this problem is 
W[1]-hard when parametrized by the size of the solution.
We additionally restrict the problem to only use segments
in three directions to achieve a stronger result.
W[1]-hardness is proved by reduction to a grid tiling problem,
which was introduced in \cite{marx_grid_tiling}.

\begin{defi}
	Line is \textbf{right-diagonal} if it is
	described by linear function $y = x + d$ for any $d \in \mathbb{R}$.
	Segment is \textbf{right-diagonal} if its
	direction is a right-diagonal line.
\end{defi}

\begin{tw}
\label{w1_hard}
	Consider the problem of covering a set $\points$ of points
	by selecting at most $k$ segments
	from a set of segments $\sets$ 
	with non-negative weights $w : \sets \rightarrow \mathbb{R^+}$
	so that the weight of the cover is minimal.
	Then this problem is W[1]-hard parametrized by $k$ and
	assuming ETH, there is no algorithm for this
	problem with running time
	$f(k)\cdot(|\points| + |\sets|)^{o(\sqrt{k})}$
	for any computable function $f$.
	Moreover, this holds even if all segments in $\sets$
	are axis-parallel or right-diagonal.
\end{tw}

Theorem \ref{w1_hard} is also true for less
restricted problem where segments have any direction.
We prove more tight setting in this section.

In order to prove Theorem \ref{w1_hard}
we will show reduction from a W[1]-hard problem.
We introduce the grid tiling problem, which is proven
to be W[1]-complete in literature.

\newcommand{\pow}{\mathsf{Pow}}

\begin{defi}
We define \textbf{powerset} of some set $A$, denoted as $\pow(A)$,
as a set of all subsets of $A$, ie. $\pow(A) = \{B : B \subseteq A\}$.
\end{defi}

\begin{defi}
In the \textbf{grid tiling} problem we are given integers $n$ and $k$,
and a function
$f : \{1 \ldots k\} \times \{1 \ldots k\} \rightarrow \pow(\{1 \ldots n\} \times \{1 \ldots n\})$
specifying the set of allowed tiles for each cell of a $k \times k$ grid.
The task is to find functions
$x,y : \{1 \ldots k\} \rightarrow \{1 \ldots n\}$
that assign numbers from $\{1 \ldots n\}$
to respectively columns and rows of the grid,
so that $(x(i), y(j)) \in f(i, j)$ for all valid $i$ and $j$,
or conclude that such an assignment does not exist.
\end{defi}

In short, in grid tiling problem you need to assign numbers
to rows and columns in such a way,
that for every pair of a row and a column,
the pair of numbers assigned
to the row and column 
belongs to the allowed set corresponding to the intersection
of the row and column in question.
The next theorem describes the complexity of this problem,
which is W[1]-hard when parametrized by the size of the grid.

\begin{tw}
\label{grid_tiling_w1_hard}
\textbf{\cite{marx_grid_tiling}}
Grid tiling is W[1]-hard parametrized by $k$ and
assuming ETH, there is no $f(k)\cdot n^{o(\sqrt{k})}$-time
algorithm solving the grid tiling problem
for any computable function $f$.
\end{tw}

The reminder of this section is proving Theorem \ref{w1_hard}
by reduction of grid tiling problem to geometric set cover.
That proves the W[1]-hardness of geometric set cover,
because if we could solve it with an FPT algorithm,
then we could also solve grid tiling problem
(which we reduced to geometric set cover).
Therefore geometric set cover with setting
described in Theorem \ref{w1_hard}
is at least as hard as the grid tiling problem.

We start with an instance of the grid tiling problem $(n, k, f)$.
The instance consists of:
\begin{itemize}
\item size of the grid $k$,
\item number of colors $n$,
\item function of allowed tiles
$f : \{1, \ldots, k\} \times \{1, \ldots, k\} \rightarrow \pow(\{1, \ldots, n\} \times \{1, \ldots, n\})$.
\end{itemize}

TODO: nice picture of instance of grid tiling with solution

\paragraph{Construction.}
\newcommand{\order}{\mathsf{order}}
\newcommand{\matchv}{\mathsf{match}_v}
\newcommand{\matchh}{\mathsf{match}_h}
\newcommand{\instanceSetCover}{(\points, \sets, w, 3k^2+2k)}
We construct an instance $\instanceSetCover$ of geometric set cover as follows.

First let us choose any bijection
$\order : \{1, \ldots, n^2\} \rightarrow \{1, \ldots, n\} \times \{1, \ldots, n\}$.


Define $\matchv(i, j)$ and $\matchh(i, j)$
as boolean functions denoting whether two points share x or y coordinate:
$$\matchv(i, j) \text{ is } \texttt{true} \iff
\order(i) \text{ and } \order(j) \text{ have the same x coordinate,}$$
$$\matchh(i, j) \text{ is } \texttt{true} \iff
\order(i) \text{ and } \order(j) \text{ have the same y coordinate.}$$


\subparagraph{Points.}

For $1 \le i,j \le k$ and $1 \le t \le n^2$ define points:
	$$h_{i, j, t} := (i \cdot (n^2+1) + t, j \cdot (n^2+1)),$$
	$$v_{i, j, t} := (i \cdot (n^2+1), j \cdot (n^2+1) + t).$$
	
Let us define sets $H$ and $V$ as:
$$H := \{h_{i, j, t} : 1 \le i, j, \le k, 1 \le t \le n^2\},$$
$$V := \{v_{i, j, t} : 1 \le i, j, \le k, 1 \le t \le n^2\}.$$
	
Let $\epsilon = \frac{1}{2k^2}$.
For a point $p = (x, y)$ we define points:
$$p^{L} := (x - \epsilon, y),$$
$$p^{R} := (x + \epsilon, y),$$
$$p^{U} := (x, y + \epsilon),$$
$$p^{D} := (x, y - \epsilon).$$

Then we define the point set as follows:
$$\points := H \cup \{p^L : p \in H\} \cup \{p^R : p \in H\}
\cup V \cup \{p^U : p \in V\} \cup \{p^D : p \in V\}.$$

\begin{defi}
	\label{guard_def}
	For every point $p \in H$, we name point $p^L$ its \textbf{left guard}
	and point $p^R$ its \textbf{right guard}.
	
	Similarily for every points $p \in V$, we name point $p^D$ its \textbf{lower guard}
	and point $p^U$ its \textbf{upper guard}.
\end{defi}

\subparagraph{Segments.}
\newcommand{\hor}[4]{\mathsf{hor}_{#1,#2,#3,#4}}
\newcommand{\ver}[4]{\mathsf{ver}_{#1,#2,#3,#4}}
\newcommand{\horbeg}[2]{\mathsf{horBeg}_{#1,#2}}
\newcommand{\verbeg}[2]{\mathsf{verBeg}_{#1,#2}}
\newcommand{\horend}[2]{\mathsf{horEnd}_{#1,#2}}
\newcommand{\verend}[2]{\mathsf{verEnd}_{#1,#2}}

For $1 \le i,j \le k$ and $1 \le t_1, t_2 \le n^2$ define segments:

$$\hor{i}{j}{t_1}{t_2} := (h^R_{i,j,t_1}, h^L_{i+1, j, t_2}),$$
$$\ver{i}{j}{t_1}{t_2} := (v^U_{i,j,t_1}, v^D_{i, j+1, t_2}),$$

$$\horbeg{i}{t} := (h^L_{1, i, 1}, h^L_{1, i, t}),$$
$$\horend{i}{t} := (h^R_{k, i, t}, h^R_{k, i, n^2}),$$

$$\verbeg{i}{t} := (v^D_{i, 1, 1}, v^D_{i, 1, t}),$$
$$\verend{i}{t} := (v^U_{i, k, t}, v^U_{i, k, n^2}).$$

\newcommand{\allhor}{\mathsf{HOR}}
\newcommand{\allver}{\mathsf{VER}}
\newcommand{\alldiag}{\mathsf{DIAG}}

Next, we define sets of vertical and horizontal segments:
\begin{eqnarray*}
\allhor &:= &\{\hor{i}{j}{t_1}{t_2} : 1 \le i < k, 1 \le j \le k,
1 \le t_1, t_2 \le n^2, \matchh(t_1, t_2) \text{ holds}\} \\
&\cup &\{\horbeg{i}{t} : 1 \le i \le k, 1 \le t \le n^2\}
\\
&\cup &\{\horend{i}{t} : 1 \le i \le k, 1 \le t \le n^2\},
\end{eqnarray*}

\begin{eqnarray*}
\allver &:= &\{\ver{i}{j}{t_1}{t_2} : 1 \le i \le k, 1 \le j < k,
1 \le t_1, t_2 \le n^2, \matchv(t_1, t_2) \text{ holds}\} \\
&\cup &\{\verbeg{i}{t} : 1 \le i \le k, 1 \le t \le n^2\}
\\
&\cup &\{\verend{i}{t} : 1 \le i \le k, 1 \le t \le n^2\}.
\end{eqnarray*}

Finally, we also define a set of right-diagonal segments:
$$\alldiag := \{ (h_{i, j, t}, v_{i, j, t}) :
	1 \le i, j \le k, 1 \le t \le n^2, \order(t) \in f(i, j)\}$$

TODO: explain that these segments are in fact diagonal

The constructed segment set is:

$$\sets := \allhor \cup \allver \cup \alldiag.$$

The weight of each segment in $\allhor \cup \allver$
is equal to the segment,
while every segment in $\alldiag$ has weight
$\delta = \frac{1}{4k^4}$.

TODO: Put a picture of small instance like 3x3 with n=2

\begin{equation}
w(s) =
	\begin{cases*}
	  length(s) 			& if $s \in \allhor \cup \allver$ \\
	  \delta        & if $s \in \alldiag$
	\end{cases*}
\end{equation}

\newcommand{\solWeight}{2k^2(n^2+1) - 4k^2\epsilon -4k(1-\epsilon) +k^2\delta }

Now, we prove that the constructed instance of geometric set cover
with weighted segments is indeed a correct and sound reduction
of the grid tiling problem. Lemma \ref{set_cover_solution_exists}
proves that if the solution of the grid tiling instance exists,
then there exists a solution with bounded size and weight
of constructed geometric set cover instance exists.

Then Lemma \ref{grid_tiling_exists} proves that if the solution
of the geometric set cover instance with bounded weight exists,
then there exists a solution to the original grid tiling instance.

\begin{lemma}
\label{set_cover_solution_exists}
	If there exists a solution of the grid tiling instance $(f_{i,j})$,
	then there exists a solution for the instance $\instanceSetCover$
	of geometric set cover of size at most $3k^2+2k$
	with weight $\solWeight$.
\end{lemma}

\begin{proof}
Suppose there exists a solution $x,y$ to the grid tiling problem.
	
We define subset of $\sets$ -- a proposed solution $\sol$
in three parts $D \subset \alldiag$, $A \subset \allhor$ and $B \subset \allver$:
\begin{eqnarray*}
	D & := & \{(v_{i, j, t}, h_{i, j, t}) : 1 \le i, j \le k, t = \order^{-1}(x(i), y(j))\}, \\
	A & := & \{\horbeg{i}{\order^{-1}(x(1), y(i))} : 1 \le i \le k\} \\
	& \cup & \{\horend{i}{\order^{-1}(x(k), y(i))} : 1 \le i \le k\} \\
	& \cup & \{\hor{i}{j}{\order^{-1}(x(i), y(j))}{\order^{-1}(x(i+1), y(j))} : 1 \le i < k, 1 \le j \le k\}, \\
	B & := & \{\verbeg{i}{\order^{-1}(x(i), y(1))} : 1 \le i \le k\} \\
	& \cup & \{\verend{i}{\order^{-1}(x(i), y(k))} : 1 \le i \le k\} \\
	& \cup & \{\ver{i}{j}{\order^{-1}(x(i), y(j))}{\order^{-1}(x(i), y(j+1))} : 1 \le i \le k, 1 \le j < k\},
\end{eqnarray*}
	$$\sol := D \cup A \cup B.$$

Since $\points = H \cup V$, we show that this covers the whole set $H$,
proof for $V$ is analogical.

Take any $1 \le j \le k$ and define $t_{i,j} := \order^{-1}(x(i), y(j)$:
$\horbeg{j}{t_1} = (h_{1,j,1}^L, h_{1, j, t_1}^L)$ and next segment
$\hor{1}{j}{t_1}{t_2} = (h_{1,j, t_1}^R, h_{2,j,t_2}^L)$.
Therefore points $h_{1,j,x}, h_{1,j,x}^L$ and $h_{1,j,x}^R$
for all $1 \le x \le n^2$ ale covered by $\horbeg{j}{t_1} and \hor{1}{j}{t_1}{t_2}$,
excluding point $h_{1,j,t_i}$.

$D$ covers all points $h_{i,j, t_{i,j}}$ and $v_{i,j, t_{i,j}}$, therefore
all points are covered.

Size of this proposed solution is:
$$|\sol| = |D| + |A| + |B| = k^2 + (k+1)k + (k+1)k = 3k^+2k.$$

TODO: Calculate weight of $\sol$
Whatevs, przeciez widac
	
Then consider a solution that covers
all points
$$\{h_{i, j, t} : 1 \le i, j \le k, \order(t)=(x(i), y(j))\}
\cup \{v_{i, j, t} : 1 \le i, j \le k, \order(t)=(x(i), y(j))\}$$

with $k^2$ segments from $\alldiag$
and the rest in $\allver$ or $\allhor$.
This solution has weight $\solWeight$.
\end{proof}


\begin{claim}
\label{guards}
In any solution of the instance $\instanceSetCover$:
\begin{itemize}
\item left and right guards of points in $H$
(points in $\{p^L : p \in H\} \cup \{p^R : p \in H\}$)
have to be covered with sgements from $\allhor$,
\item lower and upper guard of points in $V$
(points in $\{p^D : p \in V\} \cup \{p^U : p \in V\}$)
have to be covered with segments from $\allver$.
\end{itemize}
\end{claim}

\begin{proof}
We prove the claim for the points from $H$
as the proof for points from $V$ is analogical.

Every segment in $\allver$ is vertical and 
their x-coordinate is equal to $i(n^2+1)$ for some $1\le i \le k$,
so they all have different x-coordinate
than any left or right guard of points in $H$.

Every point $x$, which is a left or right guard of points in $H$
have $kn^2$ segments from $\alldiag$ that intersect with the horizontal
line that goes through $x$. All of these segments intersect with
this line in points from set $H$, therefore none of them
cover any of the guards.

Therefore none of the segments from $\allver$ or $\alldiag$ cover any
of the guards of the points in $H$.
\end{proof}

Now we present a few additional properties of the constructed instance
of the geometric set cover that help us to prove
the Lemma \ref{grid_tiling_exists}.

\begin{claim}
\label{one_diag_in_square}
For any $1 \le i, j \le n$
and any solution of the instance $\instanceSetCover$
all but at most one points $h_{i, j, t_1}, h_{i, j, t_2}$
($v_{i, j, t_1}, v_{i, j, t_2}$)
for $1 \le t_1 < t_2 \le n^2$
must be
covered with segments from $\allhor$ ($\allver$).
\end{claim}

\begin{proof}
We prove the claim for horizontal segments,
as the proof for vertical segments is analoguous.

Assume point $h_{i, j, t_1}$ is not covered with
segments from $\allhor$.
Point $h^R_{i, j, t_1}$ has to be covered with $\allhor$
by Claim $\ref{guards}$.
Every segment in $\allhor$ covering $h^R_{i, j, t_1}$,
but not $h_{i,j,t_1}$ covers also $h_{i, j, t_2}$.
\end{proof}

\begin{lemma}
\label{vertical_horizontal_sum}
For every solution of the instance $\instanceSetCover$,
the sum of weights of segments chosen
from sets $\allhor$ and $\allver$ at least
$W_{hv} = 2k^2(n^2+1) -4k^2\epsilon -4k(1-\epsilon)$.
\end{lemma}

\begin{proof}
We know that for all $1 \le i,j \le k$
there exists at most one $t_x$ and at most one $t_y$ such that
$v_{i,j,t_x}$ and $h_{i,j,t_y}$
are not covered by
segments from $\allhor$ and $\allver$ (Claim \ref{one_diag_in_square}),

TODO: Rephrase the sentence below

We sum the lower bound for sum of length for horizontal/vertical
lines for a single vertical line
(the bound is the same for every horizontal line).

Let us fix $1 \le i \le k$.

\begin{enumerate}[label={(\arabic*)}]
\item The total length between $v^D_{i, 1, 1}$ and $v^U_{i, k, n^2}$ is:
$$(k(n^2+1) + n^2 +\epsilon) - ((n^2+1)+1 -\epsilon) = k(n^2+1) - 2(1 - \epsilon).$$

\item For every $1 \le j \le k$ there exists at most one $1 \le t \le n^2$
such that $v_{i,j,t}$ is not covered by segments from $\allver$
(Claim \ref{one_diag_in_square}).
Its guards (see Definition \ref{guard_def}) $v^U_{i,j,t}$ and $v^D_{i,j,t}$
have to be covered in $\allver$ (Claim \ref{guards}).
Therefore at most $k$ spaces of length $2\epsilon$ can be left
not covered by segments from $\allver$ between $v_{i,1,1}^D$ and $v_{i,k,n^2}^U$.

Proof for horizontal segments is analogous.

\end{enumerate}
The sum of these lower bounds for vertical and horizontal lines is:
$$2k(k(n^2+1) -2k\epsilon -2(1-\epsilon)) = 2k^2(n^2+1) -4k^2\epsilon -4k(1-\epsilon)$$
\end{proof}

\begin{lemma}
\label{diag_correct}
For a constructed instance $\instanceSetCover$
for any solution $\sol$ with weight equal to $\solWeight$,
for every $1 \le i,j \le k$
there exists such $1 \le t \le n^2$ that:
\begin{enumerate}[label={(\arabic*)}]
\item $v_{i,j,t}, h_{i,j,t}$ are not covered by segments from $\allver$ or $\allhor$;
\item segment $(v_{i,j,t}, h_{i,j,t})$ is in solution $\sol$;
\item $\order(t) \in f(i,j)$, ie. it is an allowed tile for $(i,j)$;
\item for every $1 \le s\le n^2$, $s \neq t$, $v_{i,j,s}$ is covered in $\allver$;
\item for every $1 \le s\le n^2$, $s \neq t$, $h_{i,j,s}$ is covered in $\allhor$.
\end{enumerate}
Name the function of this $t$ as
$diagonal : \{1 \ldots k\} \times \{1 \ldots k\} \rightarrow \{1 \ldots n^2\}$.
\end{lemma}

\begin{proof}
At most one $h_{i,j,t_x}$ and $v_{i,j,t_y}$
points are covered with $\alldiag$
(Claim \ref{one_diag_in_square}).
	
Exactly one $h_{i,j,t_x}$ and $v_{i,j,t_y}$
points are covered with $\alldiag$,
because if one of them were not, then we would use too much weight
$$W_{hv} + 2\epsilon > \solWeight$$

This points are covered with the same segment from $\alldiag$,
because we need to use at least $k^2$ of them to use
exactly one DIAG segment for every pair of $1 \le i,j \le k$,
if we used 2 segments from $\alldiag$
for one pair $(i,j)$,
then we would have used too much weight by $\delta$.
Since these points $h_{i,j,t_x}$ and $v_{i,j,t_y}$ are covered by
segment from $\alldiag$, therefore $t_x = t_y$.

Therefore $diagonal(i,j) = t_x = t_y$
and $\order(t_x)$ is an allowed tile for $(i,j)$
because the respective segment is in $\alldiag$.

\end{proof}

\begin{lemma}
\label{vertical_horizontal_synchronized}
For a constructed instance $\instanceSetCover$
for any solution of weight $\solWeight$ it holds that $diagonal$ function
from Lemma \ref{diag_correct}:
\begin{enumerate}
\item 
for any $1 \le i < k, 1 \le j \le k$,
$\matchh(diagonal(i, j),diagonal(i+1, j))$ must be true;
\item 
for any $1 \le i \le k, 1 \le j < k$,
$\matchv(diagonal(i, j),diagonal(i, j+1))$ must be true.
\end{enumerate}
\end{lemma}

\begin{proof}
Every space between points in $H$ and $V$ covered by $\allhor$/$\allver$
has to be covered by only one segment, because otherwise it would
use $W_{hv} + \epsilon > \solWeight$,
therefore such solution would be too costly.

Proof for vertical (2), proof for horizontal is analogous.

Let us take any $1 \le i < k, 1 \le j \le k$
and name $t_1 = diagonal(i, j)$ and $t_2 = diagonal(i+1, j)$.
Therefore $h_{i,j,t_1}$ and $h_{i+1,j,t_2}$
are not covered by segments from $\allhor$,
$h^R_{i,j,t_1}$ and $h^L_{i+1,j,t_2}$
have to be covered by segments from $\allhor$ (Claim \ref{guards}).
Every segment from $\allhor$ starts at $h^R_{x,y,z_1}$
segment and finishes at $h^L_{x,y+1,z_2}$ segment for some
$1 \le x \le k$,$1 \le z < k$ and $1 \le z_1, z_2 \le n^2$.
Since all of the points between $h^R_{i,j,t_1}$ and $h^L_{i+1,j,t_2}$
are covered by segments in $\allhor$,
and there are two different segments covering
these points, one of them must begin
at $h^R_{i,j,t_1}$ and end at $h^L_{i,j+1,z_2}$
and the other one begin at $h^R_{i,j,z_1}$
and end at $h^L_{i+1,j,t_2}$
for some $1 \le z_1, z_2 \le n^2$.
Therefore space between $h^R_{i,j,z_1}$ and $h^L_{i,j+1,z_2}$
is covered twice and is longer than $\epsilon$.
By Lemma \ref{vertical_horizontal_sum}
lower bound for weight of such solution is $W_{hv} + \epsilon$
which contradicts that the given solution has size $\solWeight < W_{hv} + \epsilon$
Therefore $h^R_{i,j,t_1}$ and $h^L_{i+1,j,t_2}$ must be covered
by one segment from $\allhor$ and these points are ends of this segment.

$h^R_{i,j,t_1}$ and $h^L_{i+1,j,t_2}$
are ends of a segment from $\allhor$,
therefore $\matchh(t_1,t_2)$ must be true.
\end{proof}

\begin{corollary}
\label{vertical_horizontal_synchronized_inductive}
For a constructed instance $\instanceSetCover$
for any solution of weight $\solWeight$ it holds that $diagonal$ function
from Lemma \ref{diag_correct}:
\begin{enumerate}
\item 
for any $1 \le i, j \le k$,
$\matchh(diagonal(1, j),diagonal(i, j))$ must be true;
\item 
for any $1 \le i, j \le k$,
$\matchv(diagonal(i, 1),diagonal(i, j))$ must be true.
\end{enumerate}
\end{corollary}
\begin{proof}
Simple inductive proof based on Lemma \ref{vertical_horizontal_synchronized}.
\end{proof}

\begin{lemma}
\label{grid_tiling_exists}
	If there exists solution of instance $\instanceSetCover$
	with weight at most $\solWeight$,
	then there exists a solution for grid tiling instance.
\end{lemma}

\begin{proof}
Take $diagonal$ function from Lemma \ref{diag_correct}.

To define $x$ funtion 
for every $1 \le i \le k$ as $x(i) = x_i$
where $(x_i, a) = \order(v_{i,1})$
and $y$ function 
for every $1 \le i \le k$ as $y(i) = y_i$
where $(b, y_i) = \order(h_{1,i})$

To prove that it is a correct solution for grid tiling,
we need to prove that 
for every $1 \le i,j \le k$ $(x(i), y(j))$ is in
allowed tiles set $f(i,j)$.

Let us take any $1 \le i,j \le k$,
By Corollary \ref{vertical_horizontal_synchronized_inductive}
we know that $\matchh(diagonal(1, j),diagonal(i, j))$ and
$\matchv(diagonal(i, 1),diagonal(i, j))$ are true.
Therefore $\order(diagonal(i,j)) = (x(i), y(i))$.
By Lemma \ref{diag_correct} we know that 
$\order(diagonal(i,j))$ is in $f(i,j)$.
Therefore 
$(x(i), y(i))$
is in $f(i,j)$.

\end{proof}

\begin{proof}[Proof of Theorem \ref{w1_hard}]
Based on Lemmas \ref{set_cover_solution_exists} and \ref{grid_tiling_exists}
this is true.
\end{proof}
