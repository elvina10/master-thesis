\chapter{Preliminaries}

In this chapter we present some basic definitions that
will be used later.

\section{Geometric set cover}
\label{section:def:geometric__set_cover}
Whenever speaking about geometric set cover,
we consider it in the 2-dimensional plane.

In the geometric set cover problem we are are given
$\sets$ --- a set of objects, which are connected
subsets of the plane and $\points$ --- a set of points in the plane.
The task is to choose $\sol \subseteq \sets$ such that
every point in $\points$ is inside some object from $\sol$
and $|\sol|$ is minimized. We will mostly consider the case where
$\sets$ consists of segments in the plane.

In the weighted setting, there is some given weight function
$f : \sets \rightarrow \mathbb{R^+}$
and we would like to find a solution $\sol$
that minimizes $\sum_{R \in \sol} f(R)$.

\begin{defi}
Segment is \textbf{axis-parallel} if it lies on line that is
either horizontal $x = c$ or vertical $y = c$.
\end{defi}

\begin{defi}
	A line is \textbf{right-diagonal} if it is
	described by linear function $x + y = d$ for some $d \in \mathbb{R}$.
	Segment is \textbf{right-diagonal} if its
	direction is a right-diagonal line.
\end{defi}

\begin{defi}
\textbf{Segment Set Cover} is a Geometric Set Cover, where
objects that we cover the points with are segments.
\end{defi}

\section{Parameterization}

In the parameterized setting of the Geometric Set Cover
for a given $k$,
our task is to either find a solution $\sol$ such that $|\sol| \le k$
or decide that there is no such solution.

\begin{defi}
A \textbf{Fixed-parameter Tractable (FPT)} algorithm 
for a problem with parameter $k$ and instance size $n$
is an algorithm running in time $f(k) \cdot n^c$
for some constant $c$ and some computable function $f$.
\end{defi}

\begin{defi}
\label{definition:cnf}
Boolean formula is in \textbf{conjunctive normal form (CNF)} if
it is a conjunction of one or more formulas,
which are disjunction of literals.
\textbf{$k$-CNF} formula is a CNF formula, where
every disjunction consists of at most $k$ literals.
\end{defi}

\begin{defi}
\textbf{$k$-SAT} problem is 
a boolean satisfiability problem of $k$-CNF formulas.
Given $k$-CNF formula, one must answer if there
exists any variables assignment that satisfies the formula.
\end{defi}

\begin{defi}
For $k \ge 3$ set us define $S_k$ as a set of constants $\sigma$
such that there exists an algorithm solving $k$-SAT running in time
$\mathcal{O}^{*}(2^{\sigma n})$.
Set us define $s_k$ as the infimum  of the set $S_k$.

\textbf{Exponential Time Hypothesis (ETH)} is a conjecture
that $s_3 > 0$. This conjecture implies that
there does not exist an algorithm solving 3-SAT
running in time $2^{o(n)}$.
\end{defi}

We provide the main theorem that we use in this thesis for W[1]-hard
problems. To see the definition of a W[1]-hard problem,
see Chapter 13.3 of \cite{platypus_book}.

\begin{tw}
Problem parameterized by $k$ is \textbf{W[1]-hard} if assuming ETH there
does no algorithm solving this problem running in time
$f(k)\cdot n^{o(k)}$.
\end{tw}

\section{Approximation}

Let us recall some definitions related to optimization problems.

\begin{defi}
A \textbf{polynomial-time approximation scheme (PTAS)}
for a minimization problem $\Pi$
is a family of algorithms $\cal{A}_\epsilon$ for
every $\epsilon > 0$
such that $\cal{A}_\epsilon$ takes an instance $I$ of~$\Pi$
and in polynomial time
finds a solution that is within a factor
of ($1+\epsilon$) of being optimal.
This means that the reported solution has weight at most
$(1+\epsilon)opt(I)$, where $opt(I)$ is the weight
of an optimal solution to $I$.
\end{defi}

\begin{defi}
A problem $\Pi$ is \textbf{APX-hard} if assuming P $\neq$ NP,
there exists $\epsilon > 0$
such that there is no polynomial-time $(1+\epsilon)$-approximation algorithm
for $\Pi$.
\end{defi}

\section{$\delta$-extension}
\label{section:def:delta_extension}

Another idea presented here, which can be utilized only when considering
the problems with geometric objects,
is $\delta$-extension.
We define it specifically for the geometric set cover problem
with convex centre-symmetric objects.

Intuitively, we consider a problem with slightly larger objects,
which makes the instance more permissive.
However, we aim to find a solution that
is not larger than the
optimum solution to the original problem,
so this is substantially easier than just
solving the problem for the larger objects.
It may even be the case
that we are able to find a solution
of size smaller than the optimum solution
to the original problem.

Formal definition of $\delta$-extended objects.
is present in Definition
\ref{definition:delta_extension}.

The geometric set cover problem with $\delta$-extension
is a version of geometric set cover with
the following modifications.
\begin{itemize}
\item We need to cover all the points in $\points$
by selecting objects from $\{P^{+\delta} : P \in \sets\}$ (which always 
include no fewer points than the objects
before $\delta$-extension).
\item We look for a solution that is not larger than the optimum
solution to the original problem.
Note that it does not need to be an optimal solution in
the modified problem.
\end{itemize}
Formally, we have the following.

\begin{defi}
The \textbf{geometric set cover problem
with $\delta$-extension} is the problem where for an input instance
$I=(\sets, \points)$ of geometric set cover,
the task is to output a solution $\mathcal{R} \subseteq \sets$
such that the~$\delta$-extended set
$\{ R^{+\delta} :  R \in \mathcal{R} \}$ covers $\points$
and is not larger than the optimal solution to the~problem without
extension, i.e.~$|\mathcal{R}| \le |opt(I)|$.
\end{defi}

At last, we formulate a definition of the
polynomial-time approximation scheme (PTAS)
for a problem with $\delta$-extension.

\begin{defi}
A \textbf{PTAS for geometric set cover 
with $\delta$-extension} is a family of algorithms
$\{\mathcal{A}_{\delta, \epsilon}\}_{\delta, \epsilon > 0}$ that
each takes as an input instance $I=(\sets, \points)$
of geometric set cover where objects are centre-symmetric and strongly convex,
and in polynomial-time outputs a solution $\mathcal{R} \subseteq \sets$
such that the $\delta$-extended set
$\{ R^{+\delta} :  R \in \mathcal{R} \}$ covers $\points$
and is within a $(1+\epsilon)$ factor of the optimal
solution to this problem without
extension, i.e.~$(1+\epsilon)|\mathcal{R}| \le |opt(I)|$.
\end{defi}

\section{Weighted Geometric Set Cover}

In this thesis we also consider a weighted Geometric Set Cover problem,
which is a combination
of the weighted and parameterized setting described in 
\ref{section:def:geometric__set_cover}.
We already argued in the introduction
that there is no consensus of how it is defined, but when we discuss the
weighted parameterized setting we will consider the following
definition. There is a given weight function
$f : \sets \rightarrow \mathbb{R^+}$
and we would like to find a solution $\sol$,
such that $|\sol| \le k$
that minimizes $\sum_{R \in \sol} f(R)$ among such sets $\sol$.

\begin{defi}
The \textbf{weighted geometric set cover problem
with $\delta$-extension} is the problem where for an input instance
$I=(\sets, \points, f)$ of weighted geometric set cover,
the task is to output a solution $\mathcal{R} \subseteq \sets$
such that the~$\delta$-extended set
$\{ R^{+\delta} :  R \in \mathcal{R} \}$ covers $\points$
and it has weight not larger than the optimal solution to the~problem without
extension, i.e.~$\sum_{R \in \mathcal{R}} f(R) \le |opt(I)|$.
\end{defi}

We also consider weighted parameterized setting with $\delta$-extension,
which we formally define below.

\begin{defi}
The \textbf{weighted geometric set cover problem
with $\delta$-extension parameterized by the size of a solution}
is a problem where for an input instance
${I=(\sets, \points, f, k)}$ of weighted geometric set cover
parameterized by the size of a solution $k$,
the task is to output a solution $\mathcal{R} \subseteq \sets$
such that the~$\delta$-extended set
$\{ R^{+\delta} :  R \in \mathcal{R} \}$ covers $\points$,
uses no more than $k$ sets, i.e. $|\sol| \le k$
and it has weight not larger than the optimal solution to the~problem without
extension, i.e.~$\sum_{R \in \mathcal{R}} f(R) \le |opt(I)|$.
\end{defi}
