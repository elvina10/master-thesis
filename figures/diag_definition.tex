	
\definecolor{guards_color}{RGB}{80, 120, 255}

{\tikzset{guard_h/.style={
    circle, draw=guards_color, fill, fill=guards_color, minimum size=2pt,inner sep=0pt, outer sep=0pt,
    prefix after command= {\pgfextra{\tikzset{every
    label/.style={label distance=0.1cm,rotate=90,text=guards_color,font=\footnotesize}}}}
    }
}
{\tikzset{node_h/.style={
    circle, draw=black, fill, fill=black, minimum size=4pt,inner sep=0pt, outer sep=0pt,
    prefix after command= {\pgfextra{\tikzset{every
    label/.style={label distance=0.1cm,rotate=90,text=black}}}}
    }
}
{\tikzset{guard_v/.style={
    circle, draw=guards_color, fill, fill=guards_color, minimum size=2pt,inner sep=0pt, outer sep=0pt,
    prefix after command= {\pgfextra{\tikzset{every
    label/.style={label distance=0.1cm,text=guards_color,font=\footnotesize}}}}
    }
}
{\tikzset{node_v/.style={
    circle, draw=black, fill, fill=black, minimum size=4pt,inner sep=0pt, outer sep=0pt,
    prefix after command= {\pgfextra{\tikzset{every
    label/.style={label distance=0.1cm,text=black}}}}
    }
}

\newcommand{\addNodeV}[2]{
	\node[guard_v, label={left:$v_{i,j,#2}^D$}] at (0, \l#1) {};
	\node[node_v, label={left:$v_{i,j,#2}$}] at (0,\x#1) {};
	\node[guard_v, label={left:$v_{i,j,#2}^U$}] at (0, \r#1) {};
}

\newcommand{\addNodeH}[2]{
	\node[guard_h, label={left:$h_{i,j,#2}^L$}] at (\l#1,0) {};
	\node[node_h, label={left:$h_{i,j,#2}$}] at (\x#1,0) {};
	\node[guard_h, label={left:$h_{i,j,#2}^R$}] at (\r#1,0) {};
}

\begin{figure}
\begin{center}
\begin{tikzpicture}[main/.style = {draw, circle}]
\tikzmath{
	\step=2;
	\eps=0.5;
	\x1=\step;
	\x2=\x1+\step;
	\x3=\x2+\step;
	\x4=\x3+\step;
	\x5=\x4+\step;
	\x6=\x5+\step;
	\l1=\x1-\eps;
	\r1=\x1+\eps;
	\l2=\x2-\eps;
	\r2=\x2+\eps;
	\l3=\x3-\eps;
	\r3=\x3+\eps;
	\l4=\x4-\eps;
	\r4=\x4+\eps;
	\l5=\x5-\eps;
	\r5=\x5+\eps;
	\l6=\x6-\eps;
	\r6=\x6+\eps;
}

\addNodeV{1}{1}
\addNodeV{2}{2}
\addNodeV{3}{3}
\filldraw[black] (0,\x4) circle (0pt) node[anchor=east] {$\ldots$};
\addNodeV{5}{n^2-1}
\addNodeV{6}{n^2}

\addNodeH{1}{1}
\addNodeH{2}{2}
\addNodeH{3}{3}
\filldraw[black] (\x4,0) circle (0pt) node[anchor=north] {$\ldots$};
\addNodeH{5}{n^2-1}
\addNodeH{6}{n^2}
\draw (0,\x1) -- (\x1,0) node[midway, above] {$\delta$};
\draw (0,\x2) -- (\x2,0) node[midway, above] {$\delta$};
\draw (0,\x3) -- (\x3,0) node[midway, above] {$\delta$};
\draw (0,\x5) -- (\x5,0) node[midway, above] {$\delta$};
\draw (0,\x6) -- (\x6,0) node[midway, above] {$\delta$};
\end{tikzpicture} 
\end{center}
\caption{\textbf{Vertices and segments in $\alldiag$.}}
This is an example of constructed points any $1 \le i,j \le k$.
Points from $H$ and $V$ are marked in black,
their guards are marked in \textcolor{guards_color}{blue}.
You can also see segments from $\alldiag$ with their weights (equal to $\delta$).
\label{fig:diag_def}
\end{figure}
