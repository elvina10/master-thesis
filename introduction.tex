\chapter{Introduction}

\section{Background}
\begin{enumerate}
\item Some problems are hard, like Set Cover. It is NP-complete,
APX-hard and W[2]-hard \cite{platypus_book}
\item Therefore we restrict a problem and look at Geometric Set Cover,
      what can yield more interesting results.
\item Approximation of geometric set cover
	\begin{enumerate}
		\item with fat polygons with $\delta$-extensions admits EPTAS \cite{harpeled12}
	\end{enumerate}
\item Geometric set cover parameterized by size of solution
	\begin{enumerate}
		\item with squares is W[1]-hard
	\end{enumerate}
\item Explain what weighted FPT is. Similar weighted settings
were described there TODO
\end{enumerate}

In this paper, we focus on Geometric Set Cover wih segments
in various settings , ie. $\delta$-extensions, weighted segments
and investigate whether approximation and FPT algorithms exist.


\section{Our contribution}
\begin{enumerate}
\item Approximation of uweighted geometric set cover (even if we relax it with 
$\frac{1}{2}$-extensions) is APX-hard (Theorem \ref{segment_cover_apx_hard})
	\begin{enumerate}
	\item Stronger result than cover with unweighted axis-parallel
	rectangles being APX-hard in \cite{settling_apx_hardness}
	\item This proves that assumption about polgons being fat for
	the EPTAS in \cite{harpeled12} is a necessary condition
	\end{enumerate}
\item unweighted segments admit FPT algorithm (Theorem \ref{segment_cover_fpt})
\item weighted segments relaxed with $\delta$-extensions admit FPT (Theorem \ref{fpt_weighted_segment}),
    see Figure \ref{tab:weighted_fpt}
	\begin{enumerate}
	\item Permissive fpt, similar idea in \cite{permissive_problem1}, \cite{permissive_problem2}.
	\end{enumerate}
\item weighted segments in 2 directions is W[1]-hard (Theorem \ref{w1_hard}),
    see Figure \ref{tab:weighted_fpt}

\end{enumerate}


\begin{figure}[h]
\begin{center}
\begin{tabular}{ | c | c | c | }
\hline
                & exact     & $\delta$-extensions \\ 
\hline                
 2 directions   & W[1]-hard & FPT* \\  
\hline                
 any directions & W[1]-hard* & FPT \\
\hline                
\end{tabular}
\caption{\textbf{Our results for Geometric Set Cover problem with weighted segments 
parametrized by the size of solution.}}

Results marked with * directly follow from more or less restricted settings
respectively.
\label{tab:weighted_fpt}
\end{center}
\end{figure}


\iffalse

The Set Cover problem is one of the most common NP-complete problems.
[tutaj referencja]
We are given a family of sets and have to choose the smallest
subfamily of these sets that cover all their elements.
This problem naturally extends to settings
were we put different weights on the sets
and look for the subfamily of the minimal weight.
This problem is NP-complete even 
without weights and if we put
restrictions on what the sets can be.
One of such variants is Vertex Cover problem,
where sets have size 2 (they are edges in a graph).

In this work we focus on another such variant where the sets correspond
to some geometric shapes and
only some points of the plane have to be covered.
When these shapes are rectangles with edges parallel
to the axis, the problem can be proven to
be W[1]-complete (solution of size $k$ cannot be found
in $n^o(k)$ time),
APX-complete (for suffciently small $\epsilon > 0$, the problem
does not admit $1+\epsilon$-approximation scheme)
[refrencje].

Some of these settings are very easy.
Set cover with lines parallel to one of the axis
can be solved in polynomial time.

There is a notion of $\delta$-expansions,
which loosen the restrictions on geometric set cover.
We allow the objects to cover the points
after $\delta$-expansion and compare
the result to the original setting.
This way we can produce both FPT and EPTAS
for the rectangle set cover with $\delta$-extensions
[referencje].



\paragraph{Our contribution.}
In this work, we prove that unweighted geometric set cover
with segments is fixed parameter tractable (FPT).

Moreover, we show that geometric set cover with segments
is APX-complete for unweighted axis-parallel segments,
even with 1/2-extensions.
So the problem for very thin rectangles
also cannot admit PTAS.
Therefore, in the efficient polynomial-time approximation scheme (EPTAS)
for \textit{fat polygons} by \cite{harpeled12},
the assumption about polygons being fat is necessary. 

Finally, we show that geometric set cover with weighted segments in
3 directions is W[1]-complete.
However, geometric set cover with weighted segments is FPT if we allow
$\delta$-extension.

This result is especially interesting,
since it's counter-intuitive that
the unweighed setting is FPT and the weighted
setting is W[1]-complete.
Most of such problems (like vertex cover or [wiecej przykladow])
are equally hard in both weighted and unweighted settings.

\fi
