In this chapter we show fixed-parameter tractable algorithms
for the geometric set cover problem in~two different settings.
Section \ref{section:fpt_unweighted} shows 
a fixed-parameter tractable algorithm for geometric set cover with unweighted segments.
The reminder of the chapter presents
a fixed-parameter tractable algorithm for geometric set cover with weighted segments
with $\delta$-extensions.
We show an algorithm for the setting with $\delta$-extensions,
because the original problem with weights is W[1]-hard,
as we show in Chapter $\ref{chapter:w1_hard}$.

We start with a shared definition for this problem.
We define \textit{extreme points} for a set of~collinear points.

\begin{defi}
	For a set of collinear points $C$ in the plane,
	\textbf{extreme points} of $C$ are the endpoints
	of the smallest segment that covers all points from set $C$.
	
	If $C$ consists of one point or is empty, then
	there are 1 or 0 extreme points respectively.
\end{defi}

\section{Fixed-parameter tractable algorithm for unweighted segments}
\label{section:fpt_unweighted}
In this section we consider fixed-parameter tractable
algorithms for unweighted geometric set cover with segments.
The setting where segments are required to be axis-parallel
(or limited to a constant number of directions) has an FPT
algorithm already present in literature in
the Parametrized Algorithms book \cite{platypus_book}.
We present an FPT algorithm for geometric set cover
with unweighted segments, where segments are in arbitrary directions.

\subsection{Axis-parallel segments}
\begin{tw}
	\textbf{(FPT for segment cover with axis-parallel segments)}.
	There exists an algorithm that given a family $\sets$ of
	axis-parallel segments,
	a set of points $\points$
	and a parameter $k$,
	runs in time $O(2^k)$,
	and outputs a solution $\sol \subseteq \sets$
	such that $|\sol| \le k$ and $\sol$ covers all points in~$\points$,
	or determines that such a set $\sol$ does not exist.
\end{tw}

We present here a simple algorithm from \cite{platypus_book} for completeness.

We show an $\mathcal{O}(2^k)$-time branching algorithm.
In each step, the algorithm selects a point $a$ which is not yet covered,
branches to choose one of the two directions, and greedily chooses
a segment $a$ in that direction to cover.
This proceeds until either all points are covered or $k$~segments are chosen.

Let us take
the point $a=(x_a,y_a)$ which is the smallest 
among points that are not yet covered
in the lexicographic ordering
of points in $\mathbb{R}^2$.
We need to cover $a$ with some of~the~remaining segments.

Branch over the choice of one of the coordinates ($x$ or $y$);
without loss of generality, let us assume we chose $x$.
Among the segments lying on line $x = x_a$,
we greedily add to~the~solution the~one that covers the most points.
As $a$ was the smallest in the lexicographical order,
all points on the line $x = x_a$ have the $y$-coordinate larger than $y_a$.
Therefore, if we denote the~greedily chosen segment as $s$,
then any other segment on the line $x = x_a$ that covers $a$ can only
cover a (possibly strict) subset of points covered by $s$.
Thus, greedily choosing $s$ is optimal.

In each step of the algorithm we add one segment to the solution,
thus the recursion can be stopped at depth $k$.
If no branch finds a solution, then this means
that a solution of size at most $k$ does not exist.

Note that, the same algorithm can be used for segments in $d$ directions,
where we branch over $d$ choices of directions, and it runs in complexity $\mathcal{O}(d^k)$.

\subsection{Segments in arbitrary directions}
\label{segments_in_arbitrary_direction}
In this section we consider the setting where segments are not constrained
to a constant number of directions. 
We present a fixed-parameter tractable algorithm,
parameterized by the size of the solution.

\begin{tw}{
	\label{segment_cover_fpt}
	\textbf{(FPT for segment cover)}.
	There exists an algorithm that given a family $\sets$ of
	segments (in any direction),
	a set of points $\points$
	and a parameter $k$,
	runs in time $k^{O(k)} \cdot (|\points|\cdot|\sets|)^2$,
	and outputs a solution $\sol \subseteq \sets$
	such that $|\sol| \le k$ and $\sol$ covers all points in~$\points$,
	or determines that such a set $\sol$ does not exist.
}\end{tw}

We will need the following lemmas proving properties of any
instance of the problem.

\begin{lemma}
   \label{fpt_reduction}
   Given an instance $(\sets, \points)$ of the segment cover problem,
   without a loss of generality we can assume that
   no segment covers a superset of what another segment covers.
   That~is, for any distinct $A, B \in \sets$, we have
   $A \cap \points \not \subseteq B \cap \points$ and $A \cap \points \not \supseteq B \cap \points$.
\end{lemma}   
   
\begin{proof} 
If there are two distinct subsets of $\sets$,
$A, B$, such that $A \cap \points \subseteq B \cap \points$.

We construct a set $\sets' := \sets - \{A\}$.
We prove that for any solution $\sol$ of $(\sets, \points)$,
we can construct a solution $\sol' \subseteq \sets'$,
such that $|\sol'| \le |\sol|$.
Let us take any solution $\sol$ of $(\sets, \points)$.
If $A \in \sol$, then $\sol' := \sol \cup \{B\} - \{A\}$,
otherwise $\sol' := \sol$.
Let us consider the case when $A \in \sol$,
because the other case is trivial.
Since $A \cap \points \subseteq B \cap \points$,
then $\sol \cup \{B\} - \{A\}$
covers the same points from $\points$ as $\sol$.
Also $|\sol \cup \{B\} - \{A\}| \le |\sol|$.
\end{proof}

\begin{lemma}
	\label{fpt_long_lines}
	Given an instance $(\sets, \points)$
	of the segment cover problem 
	transformed by Lemma~\ref{fpt_reduction},
	if there exists a line $L$ with at least
	$k+1$ points on it, then there exists a subset $A \subseteq \sets$,
	of size at most $k$,
	such that every solution $\sol$ with $|\sol| \le k$
	satisfies $|A \cap \sol| \ge 1$.
	Moreover, such a subset can be found in~polynomial time.
\end{lemma}

\begin{proof}

Let us enumerate the points from $\points$ that lie on $L$ as $x_1, x_2, \ldots x_t$
in the order in which they appear on $L$.
Our proposed set is defined as:
$$A := \{ \text{segment collinear with } L \text{ that covers } x_i
\text{ and does not cover } x_{i-1} : i \in 1, \ldots k\}.$$
Where for $i = 1$ we just take a segment that covers $x_1$.

If such a segment does not exist for any point $x$
as the above, then $x$ does not give rise to any segment in $A$.
We prove the lemma by contradition. Let us assume that there
exists a solution $\sol$ of size at most $k$ such that $\sol \cap A = \emptyset$.

Every segment that is not collinear with $L$ can cover at~most one of
the points that lie on this line.
Hence if all segments from $\sol$ were not collinear with $L$, then
$\sol$ would cover at most $k$ points on line $L$,
but $L$ had at least $k+1$ different points from $\points$ on it.

Therefore we know that one of the segments from $\sol$ must be collinear with $L$
and at most $k-1$ segments can be not collinear with $L$.
Segments from $\sol$, that are not collinear with $L$ can cover at most $k-1$
points among $\{x_1, x_2, \ldots x_k\}$, therefore at least
one of these points must be covered by segments from $\sol$.
We take leftmost point from $\{x_1, x_2, \ldots x_k\}$ that is
covered in $\sol$ by segment collinear with $L$ and name it $a$.
After transformation from Lemma \ref{fpt_reduction}
there is only one segment that starts in $a$, therefore
this segment must be in both $\sol$ and $A$.

This contradiction concludes the proof that $|A \cap \sol| \ge 1$
for any solution $\sol$ of size at most $k$.
\end{proof}

We are now ready to prove Theorem \ref{segment_cover_fpt}.

\begin{proof}[Proof of Theorem \ref{segment_cover_fpt}.]\leavevmode

We will prove this theorem by presenting a branching algorithm that
works in desired complexity. It branches over the
choice of segments to cover the lines with \textit{a lot} of points,
then solves a small instance (where every line has at most $k$ points)
by checking all possible solutions.

\subparagraph{Algorithm.}
First we use Lemma \ref{fpt_reduction}.

Next, we present a recursive algorithm. Given an instance of the problem:

\begin{enumerate}[label={(\arabic*)}]
\item If there exist a line with at least $k+1$ points from $\points$,
we branch over choice of adding to~the~solution one of~the~at~most $k$ possible segments
provided by Lemma \ref{fpt_long_lines}; name this segment $s$
and name set of points from $\points$ that lie on $s$ as $S$.
Then we find a solution $\sol$
for the instance $(\points - S, \sets - \{s\})$,
and parameter $k-1$. We return $\sol \cup \{s\}$.
\item If every line has at most $k$ points on it and $|\points| > k^2$,
then answer \texttt{NO}.
\item If $|\points| \le k^2$, solve the problem by brute force:
check all subsets of $\sets$ of size at most $k$.
\end{enumerate}

\subparagraph{Correctness.}

Lemma \ref{fpt_long_lines} proves that at least one segment that we
branch over in (1) must be present in every solution $\sol$ with $|\sol| \le k$.
Therefore, the recursive call can find a~solution, provided there exists one.

In (2) the answer is no, because every line covers no more than $k$ points
from $\points$, which implies the same about every segment from $\sets$.
Under this assumption
we can cover only $k^2$ points with a solution of size $k$, which is less
than $|\points|$.

Checking all possible solutions in (3) is trivially correct.


\subparagraph{Complexity.}

In the leaves of the recursion we have $|\points| \le k^2$, so $|\sets| \le k^4$,
because every segment can be uniquely identified by the two extreme points it covers
(by Lemma \ref{fpt_reduction}). Therefore, there are $\binom{k^4}{k}$
possible solutions to check, each can be checked in time $O(k|\points|)$.
Thus, (3) takes time $k^{O(k)}$.


In this branching algorithm our parameter $k$ is decreased with every
recursive call, so we have at most $k$ levels of recursion with
branching over $k$ possibilites. Candidates to branch over
can be found on each level in time $O((|\points|\cdot|\sets|)^2)$.

Reduction from Lemma \ref{fpt_reduction} can be implemented in time $O(|\points|^2|\sets|)$.

It follows that the overall complexity is $O((|\points|\cdot|\sets|)^2 \cdot k^{O(k)})$
\end{proof}

