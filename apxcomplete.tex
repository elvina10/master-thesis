\newcommand{\setCoverInstance}{(\points, \sets)}
\newcommand{\true}{\texttt{true}}
\newcommand{\false}{\texttt{false}}

\label{chapter:segment_apx}

In this section we analyze whether there exists 
a PTAS for geometric set cover for rectangles.
We show that we can restrict this problem
to a very simple setting:
segments parallel to axes and allow (1/2)-extension,
and the problem is still APX-hard.
Note that segments are just degenerated rectangles
with one side being very narrow.


Our results can be summarized in the following
theorem and this section aims to prove it.

\begin{tw}{
\label{segment_cover_apx_hard}
	\textbf{(axis-parallel segment set cover with 1/2-extension is APX-hard)}.	
	Unweighted geometric set cover
	with axis-parallel segments in 2D (even with 1/2-extension)
	is APX-hard.
	That is, assuming $P\neq NP$, there does not exist a PTAS
	for this problem.
}\end{tw}
 
Theorem \ref{segment_cover_apx_hard} implies the following.

\begin{corollary}{
\label{rectangle_cover_apx_hard}
	\textbf{(rectangle set cover is APX-hard)}.	
	Unweighted geometric set cover
	with axis-parallel rectangles (even with 1/2-extension) is APX-hard.
}\end{corollary}


We prove Theorem \ref{segment_cover_apx_hard}
by taking a problem that is APX-hard
and showing a reduction.
For this problem we choose
MAX-(3,3)-SAT which we define below.


\section{MAX-(3,3)-SAT and statement of reduction}
\begin{defi}
\textbf{MAX-3SAT} is the following maximization problem. We are given a 3-CNF
formula, and need to find an assignment of variables
that satisfies the most clauses.
\end{defi}

\begin{defi}
\textbf{MAX-(3,3)-SAT} is a variant of MAX-3SAT with an additional
restriction that every variable appears in exactly 3 clauses
and every clause contains exactly 3 literals of 3 different variables.
Note that thus, the number of clauses is equal to the number of variables.
\end{defi}

In our proof of Theorem \ref{segment_cover_apx_hard} we use
hardness of approximation of MAX-(3,3)-SAT proved
in \cite{hastad} and described in
Theorem \ref{hastadtheorem} below.

\begin{defi}[$\alpha$-satisfiable MAX-3SAT formula]
MAX-3SAT formula with $m$ clauses is at most $\alpha$-satisfiable, if
every assignment of variables satisfies no more than $\alpha m$
clauses. 
\end{defi}

\begin{tw}{
	\label{hastadtheorem}
	\textbf{\cite{hastad}}
	For any $\epsilon > 0$, it is NP-hard to distinguish satisfiable
	\linebreak
	(3,3)-SAT formulas from
	at most
	\mbox{$(7/8 + \epsilon)$-satisfiable}
	(3,3)-SAT formulas.
}\end{tw}


Given an instance $I$ of MAX-(3,3)-SAT,
we construct an instance $J$ of 
axis-parallel segment set cover problem
such that for a sufficiently small $\epsilon > 0$,
a polynomial time $(1+\epsilon)$-approximation algorithm for $J$
would be able to distinguish  whether an instance $I$ of MAX-(3,3)-SAT
is fully satisfiable
or is at most $(7/8 + \epsilon)$-satisfiable.
However, according to Theorem \ref{hastadtheorem} the latter problem
is NP-hard.
This would imply P = NP, contradicting the assumption.

The following lemma encapsulates the properties
of the reduction described in this section,
and it allows us to prove Theorem \ref{segment_cover_apx_hard}.

\begin{lemma}{
	\label{apxconstruction}
	Given an instance $S$ of  MAX-(3,3)-SAT 
	with $n$ variables and optimum value $opt(S)$,
	we can construct an instance $I$ of geometric set cover with
	axis-parallel segments in 2D such that:
	\begin{enumerate}[label={(\arabic*)}]
	\item For every solution $X$ of instance $I$,
	there exists a solution of $S$ that satisfies at least  $15n - |X|$
	clauses.
	
	\item For every solution of instance $S$ that satisfies $w$ clauses,
	there exists a solution of $I$ of size $15n - w$.
	
	\item \label{lemma:apxconstruction:enumerate:extensions}
	Every solution with $1/2$-extensions of $I$
	is also a solution to the original instance $I$.
\end{enumerate}
Therefore, the optimum size of a solution of $I$
is $opt(I) = 15n - opt(S)$. 
	
}\end{lemma}

We prove Lemma \ref{apxconstruction} in
subsequent sections, but meanwhile let us prove
Theorem \ref{segment_cover_apx_hard} using Lemma \ref{apxconstruction}
and Theorem \ref{hastadtheorem}.

\begin{proof}[Proof of Theorem \ref{segment_cover_apx_hard}]
Consider any $0 < \epsilon < 1/(15 \cdot 8)$.

Let us assume that there exists a polynomial-time
$(1+\epsilon)$-approximation algorithm
for unweighted geometric set cover with axis-parallel segments in 2D
with (1/2)-extensions.
We construct an algorithm that solves the problem stated in 
Theorem \ref{hastadtheorem}, thereby proving that P~=~NP.

Take an instance~$S$ of MAX-(3,3)-SAT to be distinguished
and construct an instance of geometric set cover $I$
using Lemma \ref{apxconstruction}.
We now use the $(1+\epsilon)$-approximation algorithm
for geometric set cover on $I$.
Denote the size of the solution returned by this algorithm as $approx(I)$.
We prove that 
if in $S$
one can satisfy at most $(\frac{7}{8}+\epsilon)n$ clauses,
then $approx(I) \ge 15n - (\frac{7}{8} + \epsilon)n$
and if $S$ is
satisfiable, then $approx(I) < 15n - (\frac{7}{8} + \epsilon)n$.


\textbf{Assume $S$ satisfiable.}
From the definition of $S$ being satisfiable, we have:
$$opt(S) = n.$$

From Lemma \ref{apxconstruction} we have:

$$opt(I) = 14n.$$

Therefore,
$$approx(I) \le (1+\epsilon)opt(I) = 14n(1+\epsilon)
	= 14n + 14\epsilon\cdot n =$$ 
	$$= 14n + (15\epsilon - \epsilon)n < 
  14n + \left(\frac{1}{8} - \epsilon\right)n 
= 15n - \left(\frac{7}{8} + \epsilon\right)n.$$
\textbf{Assume $S$ is at most 
$\left(\frac{7}{8} + \epsilon\right)$ satisfiable.}
From the defintion of $S$ being at most 
$\left(\frac{7}{8} + \epsilon\right)n$ satisfiable, we have:
$$opt(S) \le \left(\frac{7}{8} + \epsilon\right)n$$

From Lemma \ref{apxconstruction} we have:
$$opt(I) \ge 15n - \left(\frac{7}{8} + \epsilon\right)n$$

Since a solution to $I$ with $\frac{1}{2}$-extension is
also a solution without any extention, by 
Lemma \ref{apxconstruction} \ref{lemma:apxconstruction:enumerate:extensions}, we have:

$$approx(I) \ge opt(I) = 15n - \left(\frac{7}{8} + \epsilon\right)n$$


Therefore, by using the assumed $(1+\epsilon)$-approximation
algorithm,
it is possible to distinguish the case when
$S$ is satisfiable: from the case when it is
at most $(\frac{7}{8} + \epsilon)n$ satisfiable,
it suffices to compare $approx(I)$ with $15n - (\frac{7}{8}+\epsilon)n$.
Hence, the assumed approximation algorithm cannot exist, unless P = NP.
\end{proof}

\section{Reduction}
\label{construction_description}
We proceed to the proof of Lemma \ref{apxconstruction}.
That is, we show a reduction from the MAX-(3,3)-SAT problem
to geometric set cover with segments
parallel to axis. Moreover, the obtained instance
of geometric set cover will be robust
to 1/2-extensions (have the same optimal solution
after 1/2-extension).

The construction will be composed of 2 types of gadgets:
\textbf{VARIABLE-gadgets} and \textbf{CLAUSE-gadgets}.
CLAUSE-gadgets will be constructed using two \textbf{OR-gadgets}
connected together.

\subsection{VARIABLE-gadget}

VARIABLE-gadget is responsible for choosing the value of a variable
in a CNF formula. It allows two minimum solutions of size 3 each.
These two choices correspond to the two Boolean values of the variable
corresponding to this gadget.

\paragraph{Points.}

Define points $a,b,c,d,e,f,g,h$ as follows, where $L = 12n$:


\newcommand{\pointsVarNoArg}{\mathsf{pointsVariable} }
\newcommand{\pointsVar}[1]{\mathsf{pointsVariable}_{#1} }
\newcommand{\chooseVar}[2]{\mathsf{chooseVariable}^{#1}_{#2} }
\newcommand{\segmentsVar}[1]{\mathsf{segmentsVariable}_{#1} }

\definecolor{x_true_colour}{RGB}{40, 40, 255}
\definecolor{x_false_colour}{RGB}{255, 40, 40}

{\tikzset{point/.style={
    circle, draw=black, fill, fill=black, minimum size=4pt,inner sep=0pt, outer sep=0pt,
    prefix after command= {\pgfextra{\tikzset{every
    label/.style={label distance=0.05cm,text=black}}}}
    }
}

{\tikzset{point_not_cover/.style={
    circle, draw=black, fill, fill=white, minimum size=4pt,inner sep=0pt, outer sep=0pt,
    prefix after command= {\pgfextra{\tikzset{every
    label/.style={label distance=0.05cm,text=black}}}}
    }
}

\begin{figure}[h]
\centering
\begin{tikzpicture}
\tikzmath{
\stepx=1.5;
\stepy=1;
\y1=0;
\y2=\y1+\stepy;
\y3=\y2+\stepy;
\x1=0;
\x2=\x1+\stepx;
\x3=\x2+\stepx;
\xend=4*\stepx;
}


\draw[x_false_colour,very thick] (\x1,\y1) -- (\x3,\y1);
\draw[x_false_colour,very thick] (\x1,\y2) -- (\x2,\y2);
\draw[x_false_colour,very thick] (\x2,\y3) -- (\xend,\y3);
\draw[x_true_colour,very thick] (\x1,\y1) -- (\x1,\y2);
\draw[x_true_colour,very thick] (\x2,\y1) -- (\x2,\y3);
\draw[x_true_colour,very thick] (\x3,\y1) -- (\xend,\y1);

\node[point,label={below:$a_i$}] at (\x1,\y1) {};
\node[point,label={below:$b_i$}] at (\x2,\y1) {};
\node[point,label={below:$c_i$}] at (\x3,\y1) {};
\node[point,label={left:$d_i$}] at (\x1,\y2) {};
\node[point,label={above left:$e_i$}] at (\x2,\y2) {};
\node[point,label={above left:$f_i$}] at (\x2,\y3) {};
\node[point_not_cover,label={right:$g_i$}] at (\xend,\y1) {};
\node[point_not_cover,label={right:$h_i$}] at (\xend,\y3) {};


\end{tikzpicture}
\caption{\textbf{VARIABLE-gadget.}
We denote the set of points marked with black circles as $\pointsVar{i}$,
and they need to be covered (are part of the set $\points$).
Note that some of the points are not marked as black dots
and exists only to name segments for further reference.
We denote the set of \textcolor{x_false_colour}{red} segments as $\chooseVar{false}{i}$
and the set of \textcolor{x_true_colour}{blue} segments as $\chooseVar{true}{i}$.}
\label{fig:apx_choose_variable}
\end{figure}


\begin{center}
\begin{tabular}{ l l l l}
	$a = (-L, 0)$ &
	$b = (-\frac{2}{3}L, 0)$ & 
	$c = (-\frac{1}{3}L, 0)$ & 
	$d = (-L, 1)$ \\  
	$e = (-\frac{2}{3}L, 1)$ & 
	$f = (-\frac{2}{3}L, 2)$ &
	$g = (L, 0)$ &
	$h = (L, 2)$
\end{tabular}
\end{center}


Let us define:
$$\pointsVarNoArg =  \{a, b, c, d, e, f\}$$
and, for any $1 \le i \le n$,
$$\pointsVar{i} = \pointsVarNoArg + (0, 4i).$$

We denote $a_i = a + (0,4i)$ etc.

\paragraph{Segments.}

\newcommand{\xTrueSegment}[1]{(c_{#1}, g_{#1})}
\newcommand{\xFalseSegment}[1]{(f_{#1}, h_{#1})}
\newcommand{\orTrueSegment}[2]{(t_{#1, #2}, v_{#1, #2})}

Let us define:

$$\chooseVar{true}{i} =\{ (a_i, d_i), (b_i, f_i), (c_i, g_i)\}$$
$$\chooseVar{false}{i} = \{(a_i, c_i), (d_i, e_i), (f_i, h_i)\}$$

$$\segmentsVar{i} = \chooseVar{true}{i} \cup \chooseVar{false}{i}$$


\begin{lemma}
\label{choose_variables_solution}
For any $1 \le i \le n$, points in $\pointsVar{i}$
can be covered using 3 segments from $\segmentsVar{i}$.
\end{lemma}

\begin{proof}
We can use either set $\chooseVar{true}{i}$ or $\chooseVar{false}{i}$.
\end{proof}

\begin{lemma}
\label{choose_variables_no_less}
For any $1 \le i \le n$, points in $\pointsVar{i}$
can not be covered with fewer than 3 segments from $\segmentsVar{i}$.
\end{lemma}

\begin{proof}
No segment of $\segmentsVar{i}$ covers more than one point from
$\{d_i, f_i, c_i\}$, therefore $\pointsVar{i}$ can
not be covered with fewer than 3 segments.
\end{proof}

\begin{lemma}
\label{choose_variables_both}
For every set $A \subseteq \segmentsVar{i}$ such that $A$ covers $\pointsVar{i}$
and $\xTrueSegment{i}, \xFalseSegment{i} \in A$,
it holds that $|A| \ge 4$.
\end{lemma}
\begin{proof}
No segment from $\segmentsVar{i}$ covers more than one point from
$\{a_i, e_i\}$,
therefore 
$\pointsVar{i} - \{c_i, f_i, g_i, h_i\}$
can not be covered with fewer than 2 segments.
\end{proof}


\subsection{OR-gadget}

OR-gadget connects input and output segments (see Figure \ref{fig:apx_or_gadget})
in a way, that is supposed to simulate a binary $or$ function.

Input segments are the only segments that cover points
that do not belong to the gadget.
Point $v_{i,j}$ is the only one that can be covered
by segments that do not belong to the gadget.

Also this gadget has a property that every set of segments
that cover all the points in this gadget, it uses
at least 3 segments from the gadget (excluding input segments).
Moreover output segment can belong to the solution of size 3
only if at least one of the input segments belong to the solution.
Therefore this gadget behaves in optimum solution
for this problem like a binary $or$ function
for the input segments.

\paragraph{Points.}

\newcommand{\chooseOr}[3]{\mathsf{chooseOr}^{#1}_{#2,#3}}
\newcommand{\orMoveVariable}[2]{\mathsf{orMoveVariable}_{#1,#2}}
\newcommand{\pointsOr}[2]{\mathsf{pointsOr}_{#1,#2}}
\newcommand{\segmentsOr}[2]{\mathsf{segmentsOr}_{#1,#2}}

%https://davidmathlogic.com/colorblind/#%23646464-%2389DA6D-%23FFCF3C-%2338D1F1-%23FF0909-%23238CD2
\definecolor{environment}{RGB}{100,100,100}
\definecolor{move_variable1}{RGB}{137, 218, 109}
\definecolor{move_variable2}{RGB}{255, 207, 60}
\definecolor{choose_true1}{RGB}{56, 209, 241}
\definecolor{choose_true2}{RGB}{35, 140, 210}
\definecolor{choose_false}{RGB}{255, 9, 9}

{\tikzset{point/.style={
    circle, draw=black, fill, fill=black, minimum size=4pt,inner sep=0pt, outer sep=0pt,
    prefix after command= {\pgfextra{\tikzset{every
    label/.style={label distance=0.05cm,text=black}}}}
    }
}

\begin{figure}[h]
\centering
\def\svgwidth{0.5\columnwidth}
\begin{tikzpicture}
\tikzmath{
\stepx=1.5;
\stepy=1.5;
\xbeg=0;
\x1=\xbeg+3*\stepx;
\x2=\x1+\stepx;
\x3=\x2+\stepx;
\xend=\x3+\stepx;
\y1=0;
\y2=\y1+\stepy;
\y3=\y2+\stepy;
\y4=\y3+\stepy;
\y5=\y4+\stepy;
}

\draw[environment,ultra thick] (\xbeg,\y1) -- (\x1,\y1) node[black,pos=0.15, above] {$input_x$};
\draw[environment,ultra thick] (\xbeg,\y5) -- (\x1,\y5) node[black,pos=0.15, above] {$input_y$};
\draw[move_variable1, ultra thick] (\x1,\y1) -- (\x1,\y3);
\draw[move_variable2, ultra thick] (\x1,\y5) -- (\x1,\y3);
\draw[choose_true1, ultra thick] (\x1,\y2) -- (\x3,\y2);
\draw[choose_true1, ultra thick] (\x1,\y4) -- (\x3,\y4);
\draw[choose_true2, ultra thick] (\x3,\y3) -- (\xend,\y3) node [black,pos=0.5, below] {$output$};
\draw[choose_false, ultra thick] (\x2,\y2) -- (\x2,\y4);
\draw[choose_false, ultra thick] (\x3,\y2) -- (\x3,\y4);

\node[point,label={above left:$l_{i,j}$}] at (\x1,\y1) {};
\node[point,label={left:$m_{i,j}$}] at (\x1,\y2) {};
\node[point,label={left:$n_{i,j}$}] at (\x1,\y3) {};
\node[point,label={left:$o_{i,j}$}] at (\x1,\y4) {};
\node[point,label={above left:$p_{i,j}$}] at (\x1,\y5) {};

\node[point,label={below:$q_{i,j}$}] at (\x2,\y2) {};
\node[point,label={above:$r_{i,j}$}] at (\x2,\y4) {};
\node[point,label={below:$s_{i,j}$}] at (\x3,\y2) {};
\node[point,label={left:$t_{i,j}$}] at (\x3,\y3) {};
\node[point,label={above:$u_{i,j}$}] at (\x3,\y4) {};
\node[point,label={above:$v_{i,j}$}] at (\xend,\y3) {};

\end{tikzpicture}
\caption{
	\textbf{OR-gadget.} Segments from $\chooseOr{\false}{i}{j}$ are \textcolor{choose_false}{red},
	segments from $\chooseOr{\true}{i}{j}$ are blue
	(both \textcolor{choose_true1}{light blue} and \textcolor{choose_true2}{dark blue}),
	segments from $\orMoveVariable{i}{j}$ are \textcolor{move_variable1}{green} and \textcolor{move_variable2}{yellow}.
	\textcolor{choose_true2}{Dark blue} segment is the $output$ segment.
	\textcolor{environment}{Grey segments} $input_x$ and $input_y$ are input segments that
	are not part of $\segmentsOr{i}{j}$.
}
\label{fig:apx_or_gadget}
\end{figure}


\begin{center}
	\begin{tabular}{ l l l l}
		$l_0 := (0, 0)$ &
		$m_0 := (0, 1)$ &
		$n_0 := (0, 2)$ &
		$o_0 := (0, 3)$ \\
		$p_0 := (0, 4)$ &
		$q_0 := (1, 1)$ &
		$r_0 := (1, 3)$ &
		$s_0 := (2, 1)$ \\
		$t_0 := (2, 2)$ &
		$u_0 := (2, 3)$ &
		$v_0 := (3, 2)$ &
	\end{tabular}
\end{center}


$$x_{i, j} := (10i + 3 + 3j, 4n + 2j)$$

For integers $i,j$,
define 
$\{ l_{i, j}, m_{i, j} \ldots v_{i, j} \}$
as $\{l_0, m_0 \ldots v_0\}$ shifted by $x_{i, j}$,
ie. $l_{i,j} = l_0 + x_{i,j}$ etc.

Note that $v_{i, 0} = l_{i, 1}$ (see Figure~\ref{fig:apx_clause})
 
$$\pointsOr{i}{j} := 
 \{l_{i, j}, m_{i, j}, n_{i, j}, o_{i, j},
 p_{i, j}, q_{i, j}, r_{i, j}, s_{i, j}, t_{i, j}, u_{i, j} \}
 $$
 
Note that $\pointsOr{i}{j}$ does not include the point $v_{i,j}$
 
\paragraph{Segments.}

We define set of segments in several parts:
 
$$\chooseOr{false}{i}{j} :=
\{ (q_{i, j}, r_{i, j}), (s_{i, j}, u_{i, j})\},$$
$$\chooseOr{true}{i}{j} :=
\{ (m_{i, j}, s_{i, j}), (o_{i, j}, u_{i, j}),
(t_{i, j}, v_{i, j}) \},$$

$$\orMoveVariable{i}{j} :=
\{ (l_{i, j}, n_{i, j}), (n_{i, j}, p_{i, j})\}.$$

Finally all segments in OR-gadget are defined as:

$$\segmentsOr{i}{j} := 
  \chooseOr{false}{i}{j} \cup \chooseOr{true}{i}{j} \cup \orMoveVariable{i}{j}
$$


\begin{lemma}
\label{cover_or_true}
For any $1 \le i \le n, j \in \{0, 1\}$ and 
 $x \in \{l_{i, j}, p_{i, j}\}$, points in
$\pointsOr_{i, j} - \{ x\} \cup \{v_{i, j}\}$
can be covered
with 4 segments from $\segmentsOr{i}{j}$.
\end{lemma}

\begin{proof}
We can do that using one segment from
$\orMoveVariable{i}{j}$, the one that does not cover $x$,
and all segments from $\chooseOr{true}{i}{j}$.
\end{proof}

\begin{lemma}
\label{cover_or_false}
For any $1 \le i \le n, j \in \{0, 1\}$, points in
$\pointsOr{i}{j}$ can be covered
with 4 segments from $\segmentsOr{i}{j}$.
\end{lemma}

\begin{proof}
We can do that using segments from $\orMoveVariable{i}{j} \cup \chooseOr{false}{i}{j}$.
\end{proof}


\subsection{CLAUSE-gadget}
A CLAUSE-gadget is responsible for determining whether
variable values assigned in variable gadgets
satisfy the corresponding clause in the input formula $\phi$.
It has a minimum solution of weight $w$
if and only if the clause is satisfied, i.e. at least one
of the respective variables is assigned a correct value.
Otherwise, a minimum solution of weight $w+1$.
This way, by analyzing the minimum solution for the whole problem,
we can tell how many clauses it was possible to satisfy
in the optimum solution of $\phi$.

The CLAUSE-gadgets consist of two OR-gadgets.
It would be inconvenient to position the CLAUSE-gadents
in between the very long variable segments.
Instead, we use a simple auxiliary gadget to
\textit{transfer} whether the segment
is in a solution, i.e. segments
$(x_{i, 0}, x_{i, 1}), (y_{i, 0}, y_{i, 1}), (z_{i, 0}, z_{i, 1})$.
Each gadget consists of two segments $(x_{i, 0}, x_{i, 1}), (x_{i, 1}, a)$.
These are the only segments that can cover $x_{i,1}$.
If $x_{i,0}$ is already covered by some other gadget,
we can cover $x_{i,1}$ by the other segment covering another point
from the gadget, say $a$.
If $x_{i,0}$ is not covered, then the only way to cover $x_{i,0}$
is to use segment $(x_{i, 0}, x_{i, 1})$.
Intuitively, the two segments \textit{transfer} the state of $x_{i,0}$
onto $a$, but there are less restrictions on where a can be placed,
simplifying the construction.

\paragraph{Points.}

\newcommand{\pointsClause}{\mathsf{pointsClause}}

\begin{figure}[h]
\centering
\def\svgwidth{0.8\columnwidth}
\input{apx_clause.pdf_tex}
\caption{\textbf{CLAUSE-gadget for a clause $a \lor b \lor \neg c$.}
Every green rectangle is an OR-gadget.
$y$-coordinates of $x_{i, 0}$, $y_{i, 0}$ and $z_{i,0}$
depend on the variables in the $i$-th clause.
Grey segments corresponds to the values of variables
satistying the $i$-th clause.
}
\label{fig:apx_clause}
\end{figure}

First, we define auxiliary functions for literals. 
For a literal $w$, let $idx(w)$ be the index of the variable in $w$,
and $neg(w)$ be the Boolean value whether the variable is negated in $w$
or not.

Let us assume that clause $C_i = a \lor b \lor c$
for any literals $a,b,c$. Then, we define points in the gadget as:

\begin{center}
\begin{tabular}{ l l }
	$x_{i, 0} = (10i+1, 4\cdot idx(a) + 2\cdot neg(c)),$ &
	$x_{i, 1} = (10i+1, 4(n+1)),$ \\
	$y_{i, 0} = (10i+2, 4\cdot idx(b) + 2\cdot neg(b)),$ &
	$y_{i, 1} = (10i+2, 4(n+1) + 4),$ \\
	$z_{i, 0} = (10i+3, 4\cdot idx(c) + 2\cdot neg(c)),$ &
	$z_{i, 1} = (10i+3, 4(n+1) + 6).$
\end{tabular}
\end{center}

\newcommand{\segmentsClause}{\mathsf{segmentsClause}}	
 
We are now ready to define set of points:
 
 $$\mathsf{moveVariable}_i = 
 \{x_{i, j} : j \in \{0, 1\}\} \cup
 \{y_{i, j} : j \in \{0, 1\}\} \cup
 \{z_{i, j} : j \in \{0, 1\}\},
 $$
 
 $$\pointsClause_i = 
 \mathsf{moveVariable}_i \cup \pointsOr{i}{0}
 \cup \pointsOr{i}{1} \cup \{v_{i, 1} \}.
 $$
 
Note that $v_{i,0} = l_{i,1}$.

\paragraph{Segments.}
We also define segments for the clause gadget as below:

\begin{eqnarray*}
\segmentsClause_i & = & \{ (x_{i, 0}, x_{i, 1}),
(y_{i, 0}, y_{i, 1}),
(z_{i, 0}, z_{i, 1}),
(x_{i, 1}, l_{i, 0}),
(y_{i, 1}, p_{i, 0}),
(z_{i, 1}, p_{i, 1}),
\} \cup \\
& & \cup \ \segmentsOr{i}{0} \cup \segmentsOr{i}{1}.
\end{eqnarray*}

\newcommand{\segmentsClauseSolTrue}[1]{\mathsf{solClause}^{true,#1}}
\newcommand{\segmentsClauseSolFalse}{\mathsf{solClause}^{false}}

\begin{lemma}
\label{cover_clauses_solution_true}
For any $1 \le i \le n$ and $a \in \{ x_{i, 0}, y_{i, 0}, z_{i, 0}\}$,
there is a set $\segmentsClauseSolTrue{a}_i \subseteq \segmentsClause_i$
with $|\segmentsClauseSolTrue{a}_i| = 11$
that covers all points in $\pointsClause_i - \{a\}$.
\end{lemma}

\begin{proof}
For $a = x_{i, 0}$ (analogous proof for $y_{i, 0}$):
First we use Lemma~\ref{cover_or_true} twice with excluded $x = l_{i, 0}$ and
$x = l_{i, 1} = v_{i, 0}$,
resulting with 8 segments in $\chooseOr{true}{i}{0} \cup \chooseOr{true}{i}{1}$
which cover all required points apart from
$x_{i, 1}, y_{i, 0}, y_{i, 1}, z_{i, 0}, z_{i, 1}, l_{i, 0}$.
We cover those using additional 3 segments:
$\{ (x_{i, 1}, l_{i, 0}), (y_{i, 0}, y_{i, 1}),
(z_{i, 0}, z_{i, 1}) \}$

For $a = z_{0, i}$:
Using Lemma~\ref{cover_or_false} and Lemma~\ref{cover_or_true} with
$x = p_{i, 1}$,
we obtain 8 segments in $\chooseOr{false}{i}{0} \cup \chooseOr{true}{i}{1}$
which cover all required points apart from
$x_{i, 0}, x_{i, 1}, y_{i, 0}, y_{i, 1}, z_{i, 1}, p_{i, 1}$.
We cover those using additional 3 segments:
$\{ (x_{i, 0}, x_{i, 1}), (y_{i, 0}, y_{i, 1}),
(z_{i, 1}, p_{i, 1}) \}$.
\end{proof}

\begin{lemma}
\label{cover_clauses_solution_false}
For any $1 \le i \le n$ there is
a set $\segmentsClauseSolFalse_i \subseteq \segmentsClause_i$
with $|\segmentsClauseSolFalse_i| = 12$
that covers all points in $\pointsClause_i$.
\end{lemma}

\begin{proof}
Using Lemma \ref{cover_or_false} twice we can
cover $\pointsOr{i}{0}$ and  $\pointsOr{i}{1}$
with 8 segments.
To cover the remaining points we additionally use:
$\{ (x_{i, 0}, x_{i, 1}), (y_{i, 0}, y_{i, 1}),
(z_{i, 0}, z_{i, 1}), (t_{i, 1}, v_{i, 1}) \}$
\end{proof}

\begin{lemma}
\label{cover_clauses_segments_no_less}
For any $1 \le i \le n$:
\begin{enumerate}[label={(\arabic*)}]
	\item points in $\pointsClause_i$ can not be covered 
	using any subset of segments
	from $\segmentsClause_i$ of size smaller than 12;
	\item points in $\pointsClause_i - \{ x_{i, 0}, y_{i, 0}, z_{i, 0}\}$
	can not be covered using any subset of segments
	from $\segmentsClause_i$ of size smaller than 11.
\end{enumerate}
\end{lemma}


\begin{proof}[Proof of (1).]
No segment in $\segmentsClause_i$ covers more than 1 point from
$$\{ x_{i, 0}, y_{i, 0}, z_{i, 0}, l_{i, 0}, p_{i, 0}, q_{i, 0},
u_{i, 0}, v_{i, 0} = l_{i, 1}, p_{i, 1}, q_{i, 1}, u_{i, 1}, v_{i, 1} \}.$$

Therefore we need to use at least 12 segments.
\end{proof}

\begin{proof}[Proof of (2).]

We can define disjoint sets $X, Y, Z$ such that
$X \cup Y \cup Z \subseteq \pointsClause_i - \{x_{i, 0}, y_{i, 0}, z_{i, 0}\}$
such that there are no segments in $\segmentsClause_i$ covering points from different sets.
And we prove a lower bound for each of these sets.
First, let:

$$X := \{x_{i, 1}, y_{i, 1}, z_{i, 1}\}.$$

No two points in $X$ can be covered with one segment
of $\segmentsClause_i$, so it must be covered with 3 different segments.

$$Y = \pointsOr{i}{0} - \{l_{i, 0}, p_{i, 0}\}$$
$$Z = \pointsOr{i}{1} - \{l_{i, 1}, p_{i, 1}\}$$


For both $Y$ and $Z$ we can check all of the subsets of 3 segments
of $\segmentsClause_i$
to conclude that none of them cover the considered,
so both $Y$ and $Z$ have to be covered with 
disjoint sets of 4 segments each.

Therefore, $\pointsClause_i - \{x_{i, 0}, y_{i, 0}, z_{i, 0}\}$
must be covered with at least 3 + 4 + 4 = 11 segments from $\segmentsClause_i$.
\end{proof}

\subsection{Summary}

Add some smart lemmas that sets will be exclusive to each other.

\begin{lemma}
\textbf{Robustness to 1/2-extensions}. For every segment $s \in \sets$,
$s$ and $s^{+1/2}$ cover the same points from $\points$.
\end{lemma}

\begin{proof}
We can just check every segment. Most of the segments $s$
are collinear only with points that lay on $s$,
so trivially $s^{+\frac{1}{2}}$ cannot cover more points than $s$ does.

TODO: list problematic segments here

In the same gadget:
$(n_{i,j}, p_{i,j})$ does not cover $m_{i,j}$ and symmetrically.
$(t_{i,j}, v_{i,j})$ does not cover $n_{i,j}$.
$(o_{i,0}, u_{i,0})$ does not cover $m_{i,1}$ and symmetrically.
$(y_{i,1}, p_{i,0})$ does not cover $n_{i,j}$.



From different gadgets:
$(b_i, f_i)$ after $\frac{1}{2}$-extensions does not
cover $b_{i+1}$ point.

VARIABLE-gadget's $(a_i, c_i)$ after $\frac{1}{2}$-extensions does not
cover any points $x_{i,0}, y_{i,0}$ or $z_{i,0}$ from CLAUSE-gadget.


\end{proof}


\subsection{Summary of construction}
{

{\tikzset{node/.style={
    prefix after command= {\pgfextra{\tikzset{every
    label/.style={font=\footnotesize}}}}
    }
}
\begin{figure}
\centering
\begin{tikzpicture}
\tikzmath{
\x0=0;
\y0=0;
\mepsx0=-0.2;
\yalabel=1.5;
\xalabel=12.5;
\xblabel=2.5;
\xclabel=5.0;
\yblabel=0.75;
\mepsy0=-0.2;
\epsxclabel=5.2;
\epsyalabel=1.7;
\yclabel=1.5;
\ydlabel=2.5;
\yelabel=1.5;
\yflabel=2.5;
\yglabel=4.0;
\xdlabel=12.5;
\xelabel=2.5;
\xflabel=5.0;
\yhlabel=3.25;
\mepsydlabel=2.3;
\epsxflabel=5.2;
\epsyglabel=4.2;
\yilabel=1.5;
\yajlabel=2.5;
\yaalabel=4.0;
\yablabel=5.0;
\yaclabel=1.5;
\yadlabel=2.5;
\yaelabel=4.0;
\yaflabel=5.0;
\yaglabel=6.5;
\xglabel=12.5;
\xhlabel=2.5;
\xilabel=5.0;
\yahlabel=5.75;
\mepsyablabel=4.8;
\epsxilabel=5.2;
\epsyaglabel=6.7;
\xajlabel=7.5;
\xaalabel=7.5;
\xablabel=9.5;
\yailabel=1.5;
\ybjlabel=2.5;
\ybalabel=4.0;
\ybblabel=5.0;
\ybclabel=6.5;
\ybdlabel=8.5;
\ybelabel=1.5;
\ybflabel=2.5;
\ybglabel=4.0;
\ybhlabel=5.0;
\ybilabel=6.5;
\xaclabel=7.7;
\ycjlabel=7.166666666666667;
\xadlabel=7.9;
\ycalabel=7.833333333333333;
\xaelabel=8.1;
\ycblabel=8.166666666666666;
\xaflabel=8.3;
\xaglabel=8.9;
\mepsxajlabel=7.3;
\epsxablabel=9.7;
\epsybdlabel=8.7;
\ycclabel=1.5;
\ycdlabel=2.5;
\ycelabel=1.5;
\ycflabel=2.5;
\ycglabel=4.0;
\ychlabel=5.0;
\ycilabel=1.5;
\ydjlabel=7.833333333333334;
\ydalabel=7.333333333333334;
\ydblabel=7.5;
\ydclabel=7.666666666666667;
\xahlabel=8.5;
\xailabel=8.7;
\xbjlabel=8.899999999999999;
\yddlabel=7.499999999999999;
\ydelabel=8.166666666666666;
\ydflabel=7.666666666666666;
\ydglabel=7.833333333333332;
\ydhlabel=7.999999999999999;
\xbalabel=9.1;
\xbblabel=9.299999999999999;
\xbclabel=9.499999999999998;
\xbdlabel=7.5;
\xbelabel=9.5;
\xbflabel=10.5;
\xbglabel=7.5;
\xbhlabel=9.5;
\xbilabel=10.5;
\xcjlabel=12.5;
\ydilabel=1.5;
\yejlabel=2.5;
\yealabel=4.0;
\yeblabel=5.0;
\yeclabel=6.5;
\yedlabel=8.5;
\yeelabel=1.5;
\yeflabel=2.5;
\yeglabel=4.0;
\yehlabel=5.0;
\yeilabel=6.5;
\xcalabel=10.7;
\yfjlabel=7.166666666666667;
\xcblabel=10.9;
\yfalabel=7.833333333333333;
\xcclabel=11.1;
\yfblabel=8.166666666666666;
\xcdlabel=11.3;
\xcelabel=11.9;
\mepsxbflabel=10.3;
\epsxcjlabel=12.7;
\epsyedlabel=8.7;
\yfclabel=1.5;
\yfdlabel=2.5;
\yfelabel=4.0;
\yfflabel=5.0;
\yfglabel=6.5;
\yfhlabel=1.5;
\yfilabel=2.5;
\ygjlabel=7.833333333333334;
\ygalabel=7.333333333333334;
\ygblabel=7.5;
\ygclabel=7.666666666666667;
\xcflabel=11.5;
\xcglabel=11.7;
\xchlabel=11.899999999999999;
\ygdlabel=7.499999999999999;
\ygelabel=8.166666666666666;
\ygflabel=7.666666666666666;
\ygglabel=7.833333333333332;
\yghlabel=7.999999999999999;
\xcilabel=12.1;
\xdjlabel=12.299999999999999;
\xdalabel=12.499999999999998;
}

\filldraw [fill=cyan!30, draw=black] (\mepsxajlabel,\mepsy0) rectangle (\epsxablabel, \epsybdlabel);
\draw (\xaclabel,\ycdlabel) -- (\xaclabel,\ycjlabel);
\draw (\xaclabel,\ycjlabel) -- (\xaflabel,\ycjlabel);
\draw (\xadlabel,\ychlabel) -- (\xadlabel,\ycalabel);
\draw (\xadlabel,\ycalabel) -- (\xaflabel,\ycalabel);
\draw (\xaelabel,\ycilabel) -- (\xaelabel,\ycblabel);
\draw (\xaelabel,\ycblabel) -- (\xaglabel,\ycblabel);
\node[above right] at (\mepsxajlabel,\epsybdlabel) {CLAUSE-gadget$_1$};
\draw (\xaflabel,\ycjlabel) -- (\xaflabel,\ydjlabel);
\draw (\xaflabel,\ydalabel) -- (\xailabel,\ydalabel);
\draw (\xaflabel,\ydclabel) -- (\xailabel,\ydclabel);
\draw (\xahlabel,\ydalabel) -- (\xahlabel,\ydclabel);
\draw (\xailabel,\ydalabel) -- (\xailabel,\ydclabel);
\draw (\xailabel,\ydblabel) -- (\xbjlabel,\ydblabel);
\draw (\xaglabel,\yddlabel) -- (\xaglabel,\ydelabel);
\draw (\xaglabel,\ydflabel) -- (\xbblabel,\ydflabel);
\draw (\xaglabel,\ydhlabel) -- (\xbblabel,\ydhlabel);
\draw (\xbalabel,\ydflabel) -- (\xbalabel,\ydhlabel);
\draw (\xbblabel,\ydflabel) -- (\xbblabel,\ydhlabel);
\draw (\xbblabel,\ydglabel) -- (\xbclabel,\ydglabel);

\filldraw [fill=cyan!30, draw=black] (\mepsxbflabel,\mepsy0) rectangle (\epsxcjlabel, \epsyedlabel);
\draw (\xcalabel,\yfglabel) -- (\xcalabel,\yfjlabel);
\draw (\xcalabel,\yfjlabel) -- (\xcdlabel,\yfjlabel);
\draw (\xcblabel,\y0) -- (\xcblabel,\yfalabel);
\draw (\xcblabel,\yfalabel) -- (\xcdlabel,\yfalabel);
\draw (\xcclabel,\yfilabel) -- (\xcclabel,\yfblabel);
\draw (\xcclabel,\yfblabel) -- (\xcelabel,\yfblabel);
\node[above right] at (\mepsxbflabel,\epsyedlabel) {CLAUSE-gadget$_2$};
\draw (\xcdlabel,\yfjlabel) -- (\xcdlabel,\ygjlabel);
\draw (\xcdlabel,\ygalabel) -- (\xcglabel,\ygalabel);
\draw (\xcdlabel,\ygclabel) -- (\xcglabel,\ygclabel);
\draw (\xcflabel,\ygalabel) -- (\xcflabel,\ygclabel);
\draw (\xcglabel,\ygalabel) -- (\xcglabel,\ygclabel);
\draw (\xcglabel,\ygblabel) -- (\xchlabel,\ygblabel);
\draw (\xcelabel,\ygdlabel) -- (\xcelabel,\ygelabel);
\draw (\xcelabel,\ygflabel) -- (\xdjlabel,\ygflabel);
\draw (\xcelabel,\yghlabel) -- (\xdjlabel,\yghlabel);
\draw (\xcilabel,\ygflabel) -- (\xcilabel,\yghlabel);
\draw (\xdjlabel,\ygflabel) -- (\xdjlabel,\yghlabel);
\draw (\xdjlabel,\ygglabel) -- (\xdalabel,\ygglabel);
\draw (\xclabel,\y0) -- (\xalabel,\y0) node[pos=0.17, above] {$x_1 = \true$};
\draw (\xclabel,\yalabel) -- (\xalabel,\yalabel) node[pos=0.17, above] {$x_1 = \false$};
\filldraw [fill=lime!30, draw=black] (\mepsx0,\mepsy0) rectangle (\epsxclabel, \epsyalabel);
\draw (\x0,\y0) -- (\xalabel,\y0);
\draw (\x0,\y0) -- (\x0,\yblabel);
\draw (\xblabel,\y0) -- (\xblabel,\yalabel);
\draw (\xblabel,\yalabel) -- (\xalabel,\yalabel);
\draw (\x0,\yblabel) -- (\xblabel,\yblabel);
\node[above right] at (\mepsx0,\epsyalabel) {VARIABLE-gadget$_1$};
\draw (\xflabel,\ydlabel) -- (\xdlabel,\ydlabel) node[pos=0.17, above] {$x_2 = \true$};
\draw (\xflabel,\yglabel) -- (\xdlabel,\yglabel) node[pos=0.17, above] {$x_2 = \false$};
\filldraw [fill=lime!30, draw=black] (\mepsx0,\mepsydlabel) rectangle (\epsxflabel, \epsyglabel);
\draw (\x0,\ydlabel) -- (\xdlabel,\ydlabel);
\draw (\x0,\ydlabel) -- (\x0,\yhlabel);
\draw (\xelabel,\ydlabel) -- (\xelabel,\yglabel);
\draw (\xelabel,\yglabel) -- (\xdlabel,\yglabel);
\draw (\x0,\yhlabel) -- (\xelabel,\yhlabel);
\node[above right] at (\mepsx0,\epsyglabel) {VARIABLE-gadget$_2$};
\draw (\xilabel,\yablabel) -- (\xglabel,\yablabel) node[pos=0.17, above] {$x_3 = \true$};
\draw (\xilabel,\yaglabel) -- (\xglabel,\yaglabel) node[pos=0.17, above] {$x_3 = \false$};
\filldraw [fill=lime!30, draw=black] (\mepsx0,\mepsyablabel) rectangle (\epsxilabel, \epsyaglabel);
\draw (\x0,\yablabel) -- (\xglabel,\yablabel);
\draw (\x0,\yablabel) -- (\x0,\yahlabel);
\draw (\xhlabel,\yablabel) -- (\xhlabel,\yaglabel);
\draw (\xhlabel,\yaglabel) -- (\xglabel,\yaglabel);
\draw (\x0,\yahlabel) -- (\xhlabel,\yahlabel);
\node[above right] at (\mepsx0,\epsyaglabel) {VARIABLE-gadget$_3$};
\end{tikzpicture}
\caption{\textbf{Scheme of the whole construction.}}
General layout of VARIABLE-gadgets and CLAUSE-gadgets and how they
interact with each other.
\label{fig:segment_apx_whole}
\end{figure}



Finally we define set of points and segments for the constructed instance:
$$\points := \bigcup_{1 \le i \le n} \pointsVar{i} \cup \pointsClause_i $$
$$\sets := \bigcup_{1 \le i \le n} \segmentsVar{i} \cup \segmentsClause_i $$

\section{Construction lemmas and proof of Lemma \ref{apxconstruction}}

In order to prove Lemma \ref{apxconstruction} we introduce several
auxiliary lemmas proving properties of the construction
described in the previous section.

Consider an instance $S$ of MAX-(3,3)-SAT of size $n$
with optimum solution satisfying $k$ clauses.
Let us construct an instance $\setCoverInstance$ of geometric set cover
as described in Section~\ref{construction_description}
for the instance $S$ of MAX-(3,3)-SAT.

\begin{lemma}
	\label{construction_correctness}
	Instance $\setCoverInstance$ of geometric set cover
	admits a solution of size $15n - k$.
\end{lemma}

\begin{proof}
Let the clauses in $S$ be $c_1$,~$c_2$~$\ldots$~$c_n$
and the variables be $x_1$,~$x_2$~$\ldots$~$x_n$.
Let the variable assignment in
the optimum solution to $S$ be
$\phi : \{ x_1, x_2 \ldots x_n\} \rightarrow \{\true, \false\}$.


We cover every VARIABLE-gadget with solution described in
Lemma~\ref{choose_variables_solution}, where
in the $i$-th gadget we choose the set of segments corresponding to the
value of $\phi(x_i)$.

For every clause that is satisfied, say $c_i$, 
let us name the variable that is $\true$ in it as $x_i$
and point corresponding to $x_i$ in $\pointsClause_i$ as $a$.
Points in $\pointsClause_i$ 
are covered with set $\segmentsClauseSolTrue{a}_i$ described in
Lemma~\ref{cover_clauses_solution_true}.
For every clause that is not satisfied, say $c_j$,
points in $\pointsClause_j$ are covered
with set $\segmentsClauseSolFalse_i$ described in
Lemma~\ref{cover_clauses_solution_false}.

Formally we define 
sets responsible for choosing variable assignment and satisfing clauses,
$R_i$ and $C_i$ respectively, as following:

\begin{align}
	\begin{split}
	& R_i = \begin{cases}
		\chooseVar{true}{i} & \text{if}\ \phi(x_i) = \true \\
		\chooseVar{false}{i} & \text{if}\ \phi(x_i) = \false \\
		\end{cases} \\
	& C_i = \begin{cases}
		\segmentsClauseSolTrue{a}_i & \text{if}\ c_i \text{ satisfied by literal corresponding to point } a \\
		\segmentsClauseSolFalse_i & \text{if}\ c_i \text{ not satisfied}
		\end{cases} \\
	& \sol = \bigcup\limits_{i=1}^{n} \{R_i \cup C_i : 1 \le i \le n\}.
    \end{split}
\end{align}


This set covers all the points from $\points$, because
the sets $R_i$, $C_i$ individually cover their corresponding gadgets,
as proved in the respective lemmas.

All of these sets are disjoint, so the size of the obtained solution is:

$$|\sol| = \sum_{i=1}^{n} R_i + \sum_{i=1}^n C_i = 3n + 11k + 12(n-k) = 15n - k.\qedhere$$
\end{proof}

\begin{lemma}
	\label{at_most_one_var_segment}
	Suppose we have a solution $\sol$ of the instance $\setCoverInstance$
	of geometric set cover.
	Then there exists a solution $\sol'$, such that $|\sol'| \le |\sol|$, and for each VARIABLE-gadget $\sol'$ contains at most one of the segments $\xTrueSegment{i}$ and $\xFalseSegment{i}$.
\end{lemma}
\begin{proof}\leavevmode
Assume that we have $\{\xTrueSegment{i}, \xFalseSegment{i}\} \subseteq \sol$ for some $i$. We will show how to modify $\sol$ into $\sol'$, such that the number of such $i$ decreases, while $\sol'$ is still a valid solution of $\setCoverInstance$, and $|\sol'| \le |\sol|$. Then, by repeating this procedure, we can eventually construct a solution satisfying the property from the Lemma.

To construct $\sol'$, we remove either $\xTrueSegment{i}$ or $\xFalseSegment{i}$ from $\sol$, and then add one extra segment to make $\sol'$ valid. Recall that the $i$-th VARIABLE-gadget corresponds to variable $x_i$ in $S$. As every variable in $S$ is used in exactly 3 clauses, one of the ways of setting $x_i$ (to either $\true$ or $\false$) must satisfy at least 2 clauses. If that setting is $x_i = \true$, then we remove $\xFalseSegment{i}$, otherwise we remove $\xTrueSegment{i}$. Now, there exists at most one CLAUSE-gadget which needs adjustment to make $\sol'$ valid; we do that by adding $\orTrueSegment{j}{1}$ to $\sol'$.

TODO: Can we really just remove one segment and add another one? I'd think we need to "restructure" $\sol$ around $\pointsVar{i}$ (saving one segment due to Lemma~\ref{choose_variables_no_less} and Lemma~\ref{choose_variables_both}) and then again restructure $\sol$ around the clause that we need to fix?
\end{proof}

\begin{lemma}
	\label{construction_completness}
	Suppose we have a solution $\sol$ of the instance $\setCoverInstance$
	of geometric set cover that is of size $w$.
	Then there exists a solution of $S$
	that satisfies at least $15n - w$ clauses.
\end{lemma}


\begin{proof}\leavevmode
Let the clauses in $S$ be $c_1$,~$c_2$~$\ldots$~$c_n$
and the variables be $x_1$,~$x_2$~$\ldots$~$x_n$.
Given a solution $\sol$
of the instance $\setCoverInstance$ of geometric set cover, we use Lemma~\ref{at_most_one_var_segment} to modify $\sol$ such that for any $i$ it contains at most one of $\xTrueSegment{i}$ and $\xFalseSegment{i}$; this may decrease the cost of $\sol$, but that does not matter in the subsequent construction. To simplify notation, in the remainder of this proof we use $\sol$ to refer to the modified solution.

Given $\sol$, we construct a solution of $S$ by defining an
assignment of variables 
$\phi : \{ x_1, x_2 \ldots x_n\} \rightarrow \{\true, \false\}$
that satisfies at least $15n-w$ clauses in $S$.

\subparagraph{Variables.}
Recall that due to Lemma~\ref{at_most_one_var_segment}, $\sol$ contains at most one of $\xTrueSegment{i}$ and $\xFalseSegment{i}$.

We define the value $\phi(x_i)$ for the variable $x_i$ as follows:
\begin{align}
	\begin{split}
	\label{eqn:variable_assignment}
	& \begin{cases}
	\phi(x_i) = \true & \text{if}\ \xTrueSegment{i} \in \sol \\
	\phi(x_i) = \false & \text{otherwise}
	\end{cases}
	\end{split}
\end{align}

Moreover, from Lemma~\ref{choose_variables_no_less} we get $|\pointsVar{i} \cap \sol| \ge 3$ for every $i$.

\subparagraph{Clauses}

For a clause $c_i = x \lor y \lor z$,
$\sol$ needs to use at least 11 segments to cover $\pointsClause_i - \{x_{i,0}, y_{i,0}, z_{i,0}\}$
in CLAUSE-gadget (Lemma~\ref{cover_clauses_segments_no_less}).

TODO: maybe put something with cases and names of sets as above

Moreover, if all of the points $\{x_{i,0}, y_{i,0}, z_{i,0}\}$
are not covered by the segments from $\sol~\cap~\pointsVar{i}$,
then $\sol$ needs to cover $\pointsClause_i$
with at least 12 segments
by Lemma~\ref{cover_clauses_segments_no_less}.


TODO: Maybe remove section below, because we do this calculation at the end anyway
We covered CLAUSE-gadget with at least 11 or at least 12 segments:
$$\left|\bigcup\limits_{i=1}^n \segmentsClause_i \cap \sol\right| \ge 11n + a$$
where $a$ is the number of clauses
where none of the points $x_{i,0}, y_{i,0}, z_{i,0}$
were covered by $\sol \cap \segmentsVar{j}$ for their respective
variable $x_j$.

\subparagraph{Satisfied clauses with chosen variable assignment.}

Consider a clause, say $c_i$. If none of
the points $x_{i,0}, y_{i,0}, z_{i,0}$ in $\pointsClause_i$ were covered by
segments from $\sol~\cap~\segmentsVar{j}$,
this clause is not satisfied by assignment $\phi$.

If one of these points is covered by 
segments from VARIABLE-gadget (TODO better this or $\sol \cap \segmentsVar{j}$),
then denote this point as $t$ and say it corresponds to variable $x_j$.
Consider the cases of choosing value of $\phi(x_j)$
in equation \eqref{eqn:variable_assignment}.

If $\sol$ contains exactly one of the segments $\xTrueSegment{j}$ and $\xFalseSegment{j}$,
then the value $\phi(x_j)$ satisfies $c_i$.

If $\sol$ contains neither $\xTrueSegment{j}$ nor $\xFalseSegment{j}$,
then it is impossible that $t$ is covered by segments in $\sol \cap \segmentsVar{j}$.

This means that $\phi$ satisfies all but at most $a$ clauses in $S$.


To conclude, we proved that given a solution of $\setCoverInstance$ of size $w$,
we have constructed a variables assignment $\phi$
that satisfies at least $n-a$ clauses of $S$.
Finally, note that

$$w \ge 3n + 11(n-a) + 12a = 3n + 11n + a = 14n + a,$$
hence
$$15n - w  \le 15n - 14n - a = n - a.$$

So $\phi$ satisfies at least $15n-w$ clauses of $S$.
\end{proof}

We are ready to conclude the proof of Lemma $\ref{apxconstruction}$.

\begin{proof}[Proof of Lemma \ref{apxconstruction}]
By Lemma~\ref{construction_correctness}, we know
that there exists a solution to $\setCoverInstance$ of size $15n-k$, so: 
$$opt(\setCoverInstance) \le 15n - k.$$
Since the optimum solution of $S$ satisfies $k$ clauses,
then according to Lemma~\ref{construction_completness}:
$$opt(\setCoverInstance) \ge 15n -k.$$
Therefore, the solution given by Lemma~\ref{construction_correctness} 
of size $15n - k$ is an optimum solution to the instance $\setCoverInstance$.
\end{proof}
