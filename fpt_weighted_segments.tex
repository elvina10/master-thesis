\section{Fixed-parameter tractable algorithm for weighted segments with $\delta$-extension}
\label{section:fpt_weighted}

In this section we consider the geometric set cover problem
for weighted segments relaxed with $\delta$-extension.
We show that this problem
admits an FPT algorithm when parameterized by the size
of the solution and $\delta$.
In the next chapter we show that the assumption
about the problem being relaxed with $\delta$-extension is necessary:
we prove that geometric set cover problem
for weighted segments (without extension) is W[1]-hard, which means
there does not exist any FPT algorithm parameterized by solution size for it,
assuming FPT $\neq$ W[1].

\fptWeightedSegment*

To solve this problem we will introduce a lemma about choosing
a \textit{dense} subset of points. A dense subset of points
for a set of collinear points $C$ and parameters $k$ and $\delta$
is a subset of $C$ such that
if we cover it with at most $k$ segments,
these segments after $\delta$-extension will cover all of the points from $C$.
We will prove that such set 
of size bounded by some function $f(k, \delta)$
always exists (Lemma \ref{dense_set_exists}).
Later, Lemma \ref{dense_set_exists} will allow us to find a kernel
for our original problem.

\begin{defi}
	For a set of collinear points $C$,
	a subset $A \subseteq C$ is \textbf{$(k,\delta)$-dense} 
	if for any set of segments $R$ that covers $A$ and
	such that $|R| \le k$, it holds that $R^{+\delta}$ covers $C$.
\end{defi}

\begin{lemma}
	\label{dense_set_exists}
	For any set of collinear points $C$, $\delta > 0$ and $k \ge 1$,
	there exists a $(k,\delta)$-dense set $A \subseteq C$ of size
	at most $(2+\frac{2}{\delta})^k$.
	Moreover, there exists an algorithm that computes the $(k,\delta)$-dense set
	in time $\mathcal{O}(|C| \cdot (2+\frac{2}{\delta})^k)$.
\end{lemma}

\begin{proof}
We prove this for a fixed $\delta$ by induction on $k$.

\subparagraph{Inductive hypothesis.}
For any set of collinear points $C$, there exists a set $A$ such that:
\begin{itemize}
\item $A$ is subset of $C$,
\item $A$ is $(\ell, \delta)$-dense for every $1 \le \ell \le k$,
\item $|A| \le (2+\frac{2}{\delta})^k$,
\item the extreme points of $C$ are in $A$.
\end{itemize}

\subparagraph{Base case for $k = 1$.}
It is sufficient that $A$ consists of the extreme points of $C$.

If they are covered with one segment, it must be a segment 
that includes the extreme points from $C$, so it covers the whole set $C$.

There are at most 2 extreme points in $C$ and $2 < 2+\frac{2}{\delta}$.

\subparagraph{Inductive step.}
Assuming inductive hypothesis for any set of collinear points $C$
and for parameter $k$, we will prove it for $k+1$.

Let $s$ be the minimal segment that includes all points from $C$.
That is, the extreme points of $C$ are endpoints of $s$.

We define $M = \lceil1+\frac{2}{\delta}\rceil$ subsegments of $s$
by splitting $s$ into $M$ closed segments of equal length.
We name these segments $v_i$, note that
$|v_i| = \frac{|s|}{M}$ for each $1 \le i \le M$.

Let $C_i$ be the subset of $C$ consisting of points lying on $v_i$.

Let $t_i$ be the segment with endpoints being the extreme points of $C_i$.
It might be a degenerate segment if $C_i$ consists of one point,
or $t_i$ might be empty if $C_i$ is empty.

Figure $\ref{fig:fpt_v_f_def}$ presents an example
of such segments $v_i$ and $t_i$.

\begin{figure}[h]
\begin{center}
\centering
\hspace*{0.6cm}
\def\svgwidth{\textwidth}
\input{fpt_v_t_def.pdf_tex}
\end{center}
\caption{\textbf{Example of segments $v_i$ and $t_i$.}}
Example for $M = 7$ and some set of points (marked with black circles).
The top panel shows segments $v_i$ and the bottom panel shows segments $t_i$
on the same set of points.
$a$ and $b$ are the extreme points and therefore segment $s$
ends at $a$ and $b$.
Red segments depict the split into $M$ segments of equal length $v_i$.
Blue segments depict the segments $t_i$. $t_5$ is an empty segment,
because there are no points that lie on segment $v_5$.
Segments $t_3$ and $t_7$ are degenerated to one point --
$c$ and $d$, respectively.
Segments $t_1$ and $t_2$ share one point $b$.
\label{fig:fpt_v_f_def}
\end{figure}

We use the inductive hypothesis to choose $(k, \delta)$-dense sets $A_i$
for sets $C_i$. Note that if $|C_i| \le 1$, then $A_i = C_i$
and it is still a $(k, \delta)$-dense set for $C_i$.

Then we define $A = \bigcup_{i=1}^{M} A_i$.
Thus $A$ includes the extreme points of $C$,
because they are included in the sets $A_1$ and $A_M$.

The size of each $A_i$ is at most $(2+\frac{2}{\delta})^{k}$
from the inductive hypothesis, therefore size of $A$ is at most:
$$M\left(2+\frac{2}{\delta}\right)^{k} =
\left\lceil1+\frac{2}{\delta}\right\rceil\cdot\left(2+\frac{2}{\delta}\right)^{k}
\le \left(2+\frac{2}{\delta}\right)^{k+1}.$$


\subparagraph{Proof that $A$ is $(k+1, \delta)$-dense for $C$.}
Let us take any cover of $A$ with $k+1$ segments and call it $\sol$.

For every segment $t_i$, if there exists a segment $x$ in $\sol$ 
that is disjoint with $t_i$,
then we have a cover of $A_i$ with at most $k$
segments using $\sol - \{x\}$.
Since $A_i$ is $(k, \delta)$-dense for $t_i$ and $C_i$,
$(\sol - \{x\})^{+\delta}$ covers $C_i$.
So $\sol^{+\delta}$ covers $C_i$ as well.

If there exists a segment $t_i$ for which a segment $x$ as defined above
does not exist, then all $k+1$ segments that cover
$A_i$ intersect $t_i$.
An example of such segments is depicted in Figure~\ref{fig:fpt_tricky_case}.
Let us consider any such $t_i$.
By the inductive hypothesis, the endpoints of $s$ are
in $A_1$ and $A_M$ respectively, so $\sol$ must cover them.
For each endpoint of $s$, there exists
a segment that contains this endpoint and intersects $t_i$.
Let us call these two segments $y$ and $z$. It follows that:
$|y| + |z| + |t_i| \ge |s|$.
Since $|t_i| \le |v_i| = \frac{|s|}{M} \le \frac{|s|}{1+\frac{2}{\delta}} = \frac{|s|\delta}{\delta+2}$,
we have $\max(|y|, |z|) \ge |s|(1-\frac{\delta}{\delta+2})/2 = \frac{|s|}{\delta+2}$.

\begin{figure}[h]
\begin{center}
\def\svgwidth{\columnwidth}
\input{fpt_tricky_case.pdf_tex}
\end{center}
\caption{\textbf{Example of all $k+1$ segments intersecting one segment $t_i$.}}
Both panels show the same set $\points$ (black circles),
the same as in Figure $\ref{fig:fpt_v_f_def}$.
The top panel shows blue segments $t_i$ for $M=7$.
The bottom panel shows green segments -- solution $\sol$ of size 4.
All segments from $\sol$ intersect $t_4$.
Segments $z$ and $y$ are named in the figure.
\label{fig:fpt_tricky_case}
\end{figure}

After $\delta$-extension, the longer of these segments will
expand at both ends by at least:
$$\max(|y|, |z|)\delta \ge \frac{|s|\delta}{\delta+2} =
\frac{|s|}{1+\frac{2}{\delta}} \ge \frac{|s|}{M} = |v_i| \ge |t_i|.$$

Therefore, the longer of segments $y$ and $z$ will cover the whole segment $t_i$
after $\delta$-extension. We conclude that $\sol^{+\delta}$ covers $C_i$.

Since $C = \bigcup_{i=1}^M C_i$,
it follows that $\sol^{+\delta}$ covers $C$.


\subparagraph{Algorithm.}

We can simulate the inductive proof presented above by a recursive algorithm with
the following complexity:
$$O\left(|C|+\frac{1}{\delta}\right) + O\left(|C|\cdot\left(2+\frac{2}{\delta}\right)^k\right).$$

\end{proof}

Let us now formulate some claims about the
properties for the problem parameterized by the solution size.
These properties provide bounds for different
objects in the problem instance,
which help us to find a small kernel for the problem
or conclude that the optimum
solution to this instance must be, in terms of size, above some treshold.

\begin{defi}
A line in the plane is \textbf{long}
if there are at least $k+1$ points from $\points$ on it.
\end{defi}

\begin{claim}
\label{few_long_lines}
If there are more than $k$ different long lines, then 
$\points$ can not be covered with $k$ segments.
\end{claim}

\begin{proof}
We prove the claim by contradiction.
Let us assume that we have at least $k+1$ different
long lines in our instance of the problem
and there is a solution $\sol$ of size at most $k$
covering points $\points$.

Choose any long line $L$.
Every segment from $\sol$ which is not collinear with $L$,
covers at most one point that lies on $L$.
$L$ is long, so there are at least $k+1$ points from $\points$ that lie on $L$.
This implies that there must be a segment in $\sol$ that is
collinear with $L$.

Since we have at least $k+1$ different long lines,
there are at least $k+1$
segments in $\sol$ collinear with different lines.
This contradicts with the assumption that $|\sol| \le k$.
\end{proof}

\begin{claim}
\label{few_points}
If there are more than $k^2$ points from $\points$
that do not lie on any long line,
then $\points$ can not be covered with $k$ segments.
\end{claim}

\begin{proof}
We prove the claim by contradiction.
Let us assume that we have at least $k^2+1$ points
from $\points$ that do not lie on any long line, call this set $A$,
and a solution $\sol$ of size at most $k$
covering all points in $\points$.

Every segment $s$ from $\sol$ covers at most $k$
points from $A$.
This is because if $s$ covered at least $k+1$ points from $A$,
then the line in the direction of $s$ would be a long line
and that contradicts the definiton of $A$.

If every segment from $\sol$ covers at most $k$ points from $A$
and $|\sol| \le k$, then at most $k^2$ points from $A$ are covered by $\sol$
and that contradicts the fact that $\sol$ is a solution to the given
geometric set cover instance.
\end{proof}

We are now ready to give a proof of Theorem \ref{fpt_weighted_segment}.

\newcommand{\instance}{(\points,\sets)}
\newcommand{\instancePrim}{(\points',\sets')}

\begin{proof}[Proof of Theorem \ref{fpt_weighted_segment}]
Our goal is to either answer \texttt{NO} or to find a kernel
$\instancePrim$ of size bounded by $f(k)$ for some function $f$, such that:
\begin{itemize}
\item \textit{(Property 1)} for every solution
$\sol$ to $\instance$ of size at most $k$,
there exists a set $\sol_1 \subseteq \sets'$ such that
$|\sol_1| \le k$, the weight of $\sol_1$ is not greater than the weight of $\sol$
and $\sol_1$ covers $\points'$;
\item \textit{(Property 2)}
for every set $\sol_2 \subseteq \sets'$ such that $|\sol_2| \le k$
and $\sol_2$ covers all points in $\points'$, $\sol_2^{+\delta}$
covers all points in the original set $\points$.
\end{itemize}

If we found such sets $\instancePrim$,
using \textit{Property 1} we know that an optimum solution 
of size at most $k$ to $\instancePrim$
has no greater weight than an optimum solution
of size at most $k$ to $\instance$.
Using \textit{Property 2} we know that
any solution to $\instancePrim$
after $\delta$-extension covers $\points$.

Therefore, finding such sets 
and solving the instance $\instancePrim$
by iterating over all of the subsets of $\sets'$
of size at most $k$
in desired complexity
is sufficient to prove Theorem \ref{fpt_weighted_segment}.

\paragraph{Definition of $\points'$ and $\sets'$.}
Let us name the number of different long lines as $l$.
Applying Claims \ref{few_long_lines} and \ref{few_points},
if we have more than $k$ different long lines
or more than $k^2$ points from $\points$
that do not lie on any long line, then we answer \texttt{NO},
becase these lemmas prove that there is no solution of size at most $k$
to this instance.

Otherwise, we can split $\points$ into at most $k+1$ sets:
\begin{itemize}
\item $D$: points that do not lie on any long line, $|D| \le k^2$;
\item $C_i$ for $1 \le i \le l$: points that lie on the $i$-th long line, $|C_i| > k$.
\end{itemize}
Note that sets $C_i$ do not need to be disjoint.

Then, for every set $C_i$ we can use Lemma \ref{dense_set_exists}
to obtain a $(k,\delta)$-dense set $A_i$
for $C_i$ with $|A_i| \le (2+\frac{2}{\delta})^k$.

We define $\points':= D \cup (\bigcup A_i)$. $\points'$ has size at most
$k^2 + k(2+\frac{2}{\delta})^k$.
We define $\sets'$ as follows: for every pair of points $\points'$,
we choose one segment from
$\sets$ that has the lowest weight
among segments that cover these points 
or decide that there is no segment that covers them.
There are at most $|\points'|^2$ different segments in $\sets'$,
therefore both $\sets'$ and $\points'$ have size bounded
by~$\mathcal{O}\left((k^2 + k(2+\frac{2}{\delta}))^2\right)$.

\paragraph{Proof of Property 2.}
Firstly, we prove that
for every set $\sol_2 \subseteq \sets'$ such that $|\sol_2| \le k$
and $\sol_2$ covers points in $\points'$, $\sol_2^{+\delta}$
covers points in the original instance $\points$.

Let us take such a set $\sol_2$.

$\points$ is partitioned into several parts -- sets $D$ and $C_i$.
Points from $D$ are covered by $\sol_2$, because $D$ is part of $\points'$.
Each point from any $A_i$ is covered, because $A_i$ is a part of $\points'$;
$A_i$ is a $(k,\delta)$-dense set for $C_i$, therefore $\sol_2^{+\delta}$
covers all points in $C_i$. Therefore, $\sol_2^{+\delta}$ covers
all points in $\points$.

\paragraph{Proof of Property 1.}
Secondly, we prove that for every solution
$\sol$ to $\instance$ of size at most $k$,
there exists a set $\sol_1 \subseteq \sets'$ such that
$|\sol_1| \le k$, the weight of $\sol_1$ is not greater than
the weight of $\sol$ and $\sol_1$ covers $\points'$.

For every segment in $\sol$, say $s$,
let us look at the points from $\points'$ that lie on $s$
and call this set of points $F$.
$F$ is of course a set of collinear points.
We can cover $F$ with any segment that covers extreme points of $F$,
because all other points lie on the segment between these points.
Therefore, we can replace $s$ with a segment $s'$
that has lowest weight among the points that cover the extreme points of $F$.
Such a segment belongs to $\sets'$, because this is how it was defined.
Segment $s'$ has weight no greater than the weight of $s$,
because $s$ also covers $F$.

Therefore, we produced the set $\sol_1$ that has size not greater
than the size of $\sol$ (because some segments $s$ can map
to the same segment $s'$),
weight not greater than $\sol$, and it covers $\points'$.

\paragraph{Complexity}
We find a solution of $\instancePrim$ by iterating
over all the possible subsets of $\sets'$.
Finding sets $\sets'$ and $\points'$ and then solving 
problem for kernel has overall complexity
${(|\sets| + |\points|)^{\mathcal{O}(1)}
\mathcal{O}((2 + \frac{2}{\delta})^k) + \mathcal{O}((k^2 + k(2 + \frac{2}{\delta})^k)^k)}$.
\end{proof}

