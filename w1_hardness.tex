\chapter{W[1]-hardness for axis-parallel weighted segments}
\label{chapter:w1_hard}

In this chapter we consider the geometric set cover problem with 
axis-parallel or right-diagonal weighted segments.
In Theorem~\ref{w1_hard} below, we prove that this problem is 
W[1]-hard when parameterized by the size of the solution.

We believe that the below construction can be improved to only
utilize the axis-parallel segments.

\wOneHard*

In~order to prove Theorem \ref{w1_hard}
we will show a reduction from a W[1]-hard problem: grid tiling.
This problem was introduced in \cite{marx_grid_tiling}
(the author called it matrix tiling instead).
It was originally described as an approximation problem,
but W[1]-hardness follows directly from the theorems stated there.
For a more contemporary description of this problem
and a proof of W[1]-hardness see Chapter 14 of \cite{platypus_book}.

\newcommand{\pow}{\mathsf{Pow}}

\begin{defi}
We define the \textbf{powerset} of a set $A$, denoted as $\pow(A)$,
as the set of all subsets of $A$, i.e. $\pow(A) = \{B : B \subseteq A\}$.
\end{defi}

\begin{defi}
In the \textbf{grid tiling} problem we are given integers $n$ and $k$,
and a function
$f : \{1 \ldots k\} \times \{1 \ldots k\} \rightarrow \pow(\{1 \ldots n\} \times \{1 \ldots n\})$
specifying the set of allowed tiles for each cell of a $k \times k$ grid.
The task is to decide whether there exist functions
$x,y : \{1 \ldots k\} \rightarrow \{1 \ldots n\}$
that assign colors from $\{1 \ldots n\}$
to respectively columns and rows of the grid,
so that $(x(i), y(j)) \in f(i, j)$ for all $i,j \in \{1 \ldots k\}$.
\end{defi}

In short, in the grid tiling problem one needs to assign numbers
to rows and columns in such a way
that for every pair of a row and a column,
the pair of colors assigned
to the row and column 
belongs to the allowed set of tiles for this pair.
The next theorem describes the complexity of this problem,
which is W[1]-hard when parameterized by the size of the grid.

\definecolor{alternative_sol}{RGB}{48, 48, 255}
\definecolor{bad_sol}{RGB}{255, 48, 48}

\begin{figure}[h]
\begin{center}
\begin{tabular}{ c|c|c|c|c| } 
          & $x(1)=3$ & $x(2)=1$ & $x(3)=3$ & $x(4) = 7$\\ 
 \hline
 $y(4)=1$
	& \makecell{$\textcolor{alternative_sol}{(2,1)};(2,2);$\\$\textbf{(3,1)};(3,9)$}
	& $\textbf{(1,1)}; (3,1)$
	& $\textbf{(3,1)}; (7,2)$
	& $\textcolor{bad_sol}{(2,1)}; \textbf{(7,1)}$\\ 
 \hline
 $y(3)=1$
	& \makecell{$\textcolor{alternative_sol}{(2,1)};\textbf{(3,1)};$\\$(4,2);(8,2)$}
	& $\textbf{(1,1)}; (1,3)$
	& $\textbf{(3,1)}; (4,3)$
	& $\textcolor{bad_sol}{\textbf{(2,2)}}; \textbf{(7,1)}$\\ 
 \hline
 $y(2)=6$
	& $\textcolor{alternative_sol}{(2,6)};\textbf{(3,6)}$
	& \makecell{$(1,2); \textbf{(1,6)};$\\$(2,6)$}
	& $(2,6); \textbf{(3,6)}$
	& $\textcolor{bad_sol}{(2,6)};\textbf{(7,6)}$\\ 
 \hline
 $y(1)=4$
	& \makecell{$\textcolor{alternative_sol}{(2,4)};(2,6);$\\$\textbf{(3,4)};\textcolor{bad_sol}{(3,9)}$}
	& $\textbf{(1,4)}; \textcolor{bad_sol}{(1,9)}$
	& $\textbf{(3,4)}; \textcolor{bad_sol}{(3,9)}$
	& $\textcolor{bad_sol}{(2,9)}; \textbf{(7,4)}$\\ 
 \hline
\end{tabular}
\end{center}
\caption{\textbf{Example of a $\GridTiling$ instance and its solution.}}
In the first row and column of the table you can see the solution:
functions $x$ and $y$.
The~tiles used in this solution are marked in \textbf{bold}.
If we instead chose the tiles marked in \textcolor{alternative_sol}{blue}
(whenever there is one, taking the tile marked in \textbf{bold} otherwise),
then that corresponds to setting $x(1)=2$, and would also form a correct solution.
On the other hand, if we instead chose the tiles marked in \textcolor{bad_sol}{red}
(as before), then this corresponds to setting ${y(1)=9}$ and $x(4)=2$
and that would $\textbf{not}$ form a correct solution.
Even though the first row is correct,
the cell with coordinates (3,4) requires tile (2,1), not (2,2)
(marked in \textbf{\textcolor{bad_sol}{bold red}}).
\label{fig:grid_tiling_exmample}
\end{figure}


\begin{tw}
\label{grid_tiling_w1_hard}
\textbf{\cite{marx_grid_tiling}}
Grid tiling is W[1]-hard when parameterized by $k$ and
assuming ETH, there is no $f(k)\cdot n^{o(k)}$-time
algorithm solving the grid tiling problem
for any computable function $f$.
\end{tw}

The remainder of this section is devoted to proving Theorem \ref{w1_hard}
by a reduction from a~grid tiling problem instance
with parameter $k$ (number of rows in the grid)
to a geometric set cover instance with parameter $k^2$ (size of solution).
This reduction is described in Lemma~\ref{w1_construction}.
This proves the W[1]-hardness of the geometric set cover problem,
because if we could solve it with an FPT algorithm,
then we could also solve the grid tiling problem
(which we reduced to the geometric set cover).
Therefore, geometric set cover with setting
described in Theorem \ref{w1_hard}
is at least as hard as the grid tiling problem.

\newcommand{\hvWeight}{W_{\mathsf{hv}}}
\newcommand{\solWeight}{\hvWeight+k^2\delta }
\newcommand{\instanceSetCover}{(\points, \sets, w, 3k^2+2k)}
\newcommand{\instanceGridTiling}{(n,k,f)}
\newcommand{\yes}{\texttt{YES}}
\newcommand{\no}{\texttt{NO}}

Let us denote an instance of grid tiling problem as $\instanceGridTiling$ consisting of:
\begin{itemize}
\item the number of colors $n$,
\item the size of the grid $k$,
\item the function specifying the allowed tiles
$f : \{1, \ldots, k\} \times \{1, \ldots, k\} \rightarrow \pow(\{1, \ldots, n\} \times \{1, \ldots, n\})$.
\end{itemize}

Let us also define constants: 
\begin{eqnarray*}
\epsilon & := & \frac{1}{2k^2} \\
\delta & := & \frac{1}{4k^4} \\
\hvWeight & := & 2k^2(n^2+1) -4k^2\epsilon -4k(1-\epsilon)
\end{eqnarray*}
which are going to be used when defining the weight of the constructed
instance of geometric set cover with weighted segments.


\begin{lemma}
\label{w1_construction}
Given an instance $\instanceGridTiling$ of the grid tiling problem,
we can construct an instance $\instanceSetCover$ of geometric set cover
with weighted segments such that:
\begin{enumerate}[label={(\arabic*)}]
\item \label{part1} if the answer to $\instanceGridTiling$ is $\yes$, then there exists a solution
to $\instanceSetCover$ of weight at most $\solWeight$;
\item \label{part2} if there exists a solution to $\instanceSetCover$ of weight at most $\solWeight$,
then the answer to $\instanceGridTiling$ is $\yes$.
\end{enumerate}
\end{lemma}


First, let us prove Theorem \ref{w1_hard} using Lemma \ref{w1_construction}.

\begin{proof}[Proof of Theorem \ref{w1_hard}]
Let us take any instance $(n,l,f)$ of the grid tiling problem.
We prove the theorem by contradiction, therefore we assume
that geometric set cover with weighted segments
parameterized by solution size $k$ admits a
$g(k)\cdot n^{o(\sqrt{k})}$-time algorithm for some computable function $g$.

Using Lemma \ref{w1_construction} let us construct an instance $I$
for $(n,l,f)$.
Let us assume that the optimum solution of size at most $k$
to the instance $I$ has weight $u$.
Using \ref{part2} we know that if $u \le \solWeight$,
then the answer to $(n,l,f)$ is $\yes$.
If $u > \solWeight$, then using \ref{part1}
we know that the answer to $(n,l,f)$ must be $\no$.

Therefore if we could find the solution in time $g(k) \cdot n^{o(\sqrt{k})}$,
then we could solve the grid tiling problem
in time $g(l)\cdot n^{o(l)}$ by constructing an instance of the set cover with
weighted segments, solving it 
for parameter $k = 3l^2+2l$ in time $n^{o(\sqrt{3l^2+2l})}$
and then answering based on the weight
of the optimum solution.
As $\mathcal{O}(n^{o(l)}) \subseteq \mathcal{O}(n^{o(\sqrt{3l^2+2l})})$,
the existence of this algorithm contradicts Theorem
\ref{grid_tiling_w1_hard}.
Hence such an algorithm can not exist.
\end{proof}

We prove Lemma \ref{w1_construction} in subsequent sections.
First, we define a constructed instance $I$, later property
\ref{part1} is proved by Lemma \ref{set_cover_solution_exists}
and property \ref{part2} is proved by Lemma \ref{grid_tiling_exists}.

Permissive FPT is a relaxed FPT problem, where 
we need to find solution of any size in FPT-time,
but we compare it to the optimum solution of size at most $k$.
Idea for permissive FPT in local search was presented in \cite{permissive_problem1}, \cite{permissive_problem2}.

In the proof of Lemma \ref{grid_tiling_exists}
we do not use the assumption that
the solution is bounded by the size,
which the problem is parametrized by, $3k^2+2k$.
If we had a permissive FPT algorithm
that finds a solution of any size that still
has weight no more than $\solWeight$,
then we still would have a contradiction with grid tiling being W[1]-hard
in proof of Theorem \ref{w1_hard}.
Thus this reduction
proves that the problem is not only W[1]-hard, but assuming ETH 
there also does not exist permissive FPT algorithm for this problem.
Formally we state this in the Theorem $\ref{permissive_w1_hard}$.


\begin{restatable}{tw}{permissiveWOneHard}
\label{permissive_w1_hard}
\textbf{(Permissive FPT does not exist).}
	Consider the problem of covering a set $\points$ of points
	using segments from a set $\sets$ 
	with non-negative weights $w : \sets \rightarrow \mathbb{R^+}$
	so that the weight of the cover is minimal.
	Let $\sol^k$ be the
	optimum solution to this problem of size at most $k$.
	The task is to find a solution $\sol$ of any size
	such that weight of $\sol$ is not greater than the weight of $\sol^k$.
	
	Assuming ETH, there is no algorithm for this
	problem with running time
	$f(k)\cdot(|\points| + |\sets|)^{o(\sqrt{k})}$
	for any computable function $f$.
	Moreover, this holds even if all segments in $\sets$
	are axis-parallel or right-diagonal.
\end{restatable}

\paragraph{Construction.}
\newcommand{\order}{\mathsf{order}}
\newcommand{\matchv}{\mathsf{match}_v}
\newcommand{\matchh}{\mathsf{match}_h}

We construct an instance $\instanceSetCover$ of geometric set cover as follows.

First, let us choose any bijection
$\order : \{1, \ldots, n^2\} \rightarrow \{1, \ldots, n\} \times \{1, \ldots, n\}$.


Define $\matchv(i, j)$ and $\matchh(i, j)$
as boolean functions denoting whether two points share x or y coordinate:
$$\matchv(i, j) \text{ is } \true \iff
\order(i) \text{ and } \order(j) \text{ have the same x coordinate,}$$
$$\matchh(i, j) \text{ is } \true \iff
\order(i) \text{ and } \order(j) \text{ have the same y coordinate.}$$


\subparagraph{Points.}

For $1 \le i,j \le k$ and $1 \le t \le n^2$ define points:
	$$h_{i, j, t} := (i \cdot (n^2+1) + t, j \cdot (n^2+1)),$$
	$$v_{i, j, t} := (i \cdot (n^2+1), j \cdot (n^2+1) + t).$$
	
Let us define sets $H$ and $V$ as:
$$H := \{h_{i, j, t} : 1 \le i, j, \le k, 1 \le t \le n^2\},$$
$$V := \{v_{i, j, t} : 1 \le i, j, \le k, 1 \le t \le n^2\}.$$
	
Let us recall that $\epsilon = \frac{1}{2k^2}$.
For a point $p = (x, y)$ we define points:
$$p^{L} := (x - \epsilon, y),$$
$$p^{R} := (x + \epsilon, y),$$
$$p^{U} := (x, y + \epsilon),$$
$$p^{D} := (x, y - \epsilon).$$

Then we define the point set as follows:
$$\points := H \cup \{p^L : p \in H\} \cup \{p^R : p \in H\}
\cup V \cup \{p^U : p \in V\} \cup \{p^D : p \in V\}.$$

\begin{defi}
	\label{guard_def}
	For every point $p \in H$, we name point $p^L$ its \textbf{left guard}
	and point $p^R$ its \textbf{right guard}.
	
	Similarily for every points $p \in V$, we name point $p^D$ its \textbf{lower guard}
	and point $p^U$ its \textbf{upper guard}.
\end{defi}

\subparagraph{Segments.}
\newcommand{\hor}[4]{\mathsf{hor}_{#1,#2,#3,#4}}
\newcommand{\ver}[4]{\mathsf{ver}_{#1,#2,#3,#4}}
\newcommand{\horbeg}[2]{\mathsf{horBeg}_{#1,#2}}
\newcommand{\verbeg}[2]{\mathsf{verBeg}_{#1,#2}}
\newcommand{\horend}[2]{\mathsf{horEnd}_{#1,#2}}
\newcommand{\verend}[2]{\mathsf{verEnd}_{#1,#2}}

For $1 \le i,j \le k$ and $1 \le t, t_1, t_2 \le n^2$ define segments:
\begin{eqnarray*}
\hor{i}{j}{t_1}{t_2} & := & (h^R_{i,j,t_1}, h^L_{i+1, j, t_2}), \\
\ver{i}{j}{t_1}{t_2} & := & (v^U_{i,j,t_1}, v^D_{i, j+1, t_2}), \\
\horbeg{i}{t} & := & (h^L_{1, i, 1}, h^L_{1, i, t}), \\
\horend{i}{t} & := & (h^R_{k, i, t}, h^R_{k, i, n^2}), \\
\verbeg{i}{t} & := & (v^D_{i, 1, 1}, v^D_{i, 1, t}), \\
\verend{i}{t} & := & (v^U_{i, k, t}, v^U_{i, k, n^2}). \\
\end{eqnarray*}

\newcommand{\allhor}{\mathsf{HOR}}
\newcommand{\allver}{\mathsf{VER}}
\newcommand{\alldiag}{\mathsf{DIAG}}

Next, we define sets of vertical and horizontal segments:
\begin{eqnarray*}
\allhor &:= &\{\hor{i}{j}{t_1}{t_2} : 1 \le i < k, 1 \le j \le k,
1 \le t_1, t_2 \le n^2, \matchh(t_1, t_2) \text{ holds}\} \\
&\cup &\{\horbeg{i}{t} : 1 \le i \le k, 1 \le t \le n^2\}
\\
&\cup &\{\horend{i}{t} : 1 \le i \le k, 1 \le t \le n^2\},
\end{eqnarray*}
\begin{eqnarray*}
\allver &:= &\{\ver{i}{j}{t_1}{t_2} : 1 \le i \le k, 1 \le j < k,
1 \le t_1, t_2 \le n^2, \matchv(t_1, t_2) \text{ holds}\} \\
&\cup &\{\verbeg{i}{t} : 1 \le i \le k, 1 \le t \le n^2\}
\\
&\cup &\{\verend{i}{t} : 1 \le i \le k, 1 \le t \le n^2\}.
\end{eqnarray*}
An example is depicted in Figure \ref{fig:segments_def}.

Finally, we also define a set of right-diagonal segments:
$$\alldiag := \{ (h_{i, j, t}, v_{i, j, t}) :
	1 \le i, j \le k, 1 \le t \le n^2, \order(t) \in f(i, j)\}.$$
An example of such segments is depicted in Figure \ref{fig:diag_def}.

	
\definecolor{guards_color}{RGB}{80, 120, 255}

{\tikzset{guard_h/.style={
    circle, draw=guards_color, fill, fill=guards_color, minimum size=2pt,inner sep=0pt, outer sep=0pt,
    prefix after command= {\pgfextra{\tikzset{every
    label/.style={label distance=0.1cm,rotate=90,text=guards_color,font=\footnotesize}}}}
    }
}
{\tikzset{node_h/.style={
    circle, draw=black, fill, fill=black, minimum size=4pt,inner sep=0pt, outer sep=0pt,
    prefix after command= {\pgfextra{\tikzset{every
    label/.style={label distance=0.1cm,rotate=90,text=black}}}}
    }
}
{\tikzset{guard_v/.style={
    circle, draw=guards_color, fill, fill=guards_color, minimum size=2pt,inner sep=0pt, outer sep=0pt,
    prefix after command= {\pgfextra{\tikzset{every
    label/.style={label distance=0.1cm,text=guards_color,font=\footnotesize}}}}
    }
}
{\tikzset{node_v/.style={
    circle, draw=black, fill, fill=black, minimum size=4pt,inner sep=0pt, outer sep=0pt,
    prefix after command= {\pgfextra{\tikzset{every
    label/.style={label distance=0.1cm,text=black}}}}
    }
}

\newcommand{\addNodeV}[2]{
	\node[guard_v, label={left:$v_{i,j,#2}^D$}] at (0, \l#1) {};
	\node[node_v, label={left:$v_{i,j,#2}$}] at (0,\x#1) {};
	\node[guard_v, label={left:$v_{i,j,#2}^U$}] at (0, \r#1) {};
}

\newcommand{\addNodeH}[2]{
	\node[guard_h, label={left:$h_{i,j,#2}^L$}] at (\l#1,0) {};
	\node[node_h, label={left:$h_{i,j,#2}$}] at (\x#1,0) {};
	\node[guard_h, label={left:$h_{i,j,#2}^R$}] at (\r#1,0) {};
}

\begin{figure}
\begin{center}
\begin{tikzpicture}[main/.style = {draw, circle}]
\tikzmath{
	\step=2;
	\eps=0.5;
	\x1=\step;
	\x2=\x1+\step;
	\x3=\x2+\step;
	\x4=\x3+\step;
	\x5=\x4+\step;
	\x6=\x5+\step;
	\l1=\x1-\eps;
	\r1=\x1+\eps;
	\l2=\x2-\eps;
	\r2=\x2+\eps;
	\l3=\x3-\eps;
	\r3=\x3+\eps;
	\l4=\x4-\eps;
	\r4=\x4+\eps;
	\l5=\x5-\eps;
	\r5=\x5+\eps;
	\l6=\x6-\eps;
	\r6=\x6+\eps;
}

\draw (0,\x1) -- (\x1,0) node[midway, above] {$\delta$};
\draw (0,\x2) -- (\x2,0) node[midway, above] {$\delta$};
\draw (0,\x3) -- (\x3,0) node[midway, above] {$\delta$};
\draw (0,\x5) -- (\x5,0) node[midway, above] {$\delta$};
\draw (0,\x6) -- (\x6,0) node[midway, above] {$\delta$};

\addNodeV{1}{1}
\addNodeV{2}{2}
\addNodeV{3}{3}
\filldraw[black] (0,\x4) circle (0pt) node[anchor=east] {$\ldots$};
\addNodeV{5}{n^2-1}
\addNodeV{6}{n^2}

\addNodeH{1}{1}
\addNodeH{2}{2}
\addNodeH{3}{3}
\filldraw[black] (\x4,0) circle (0pt) node[anchor=north] {$\ldots$};
\addNodeH{5}{n^2-1}
\addNodeH{6}{n^2}
\end{tikzpicture} 
\end{center}
\caption{\textbf{Vertices and segments in $\alldiag$.}}
This is an example of constructed points any $1 \le i,j \le k$.
Points from $H$ and $V$ are marked in black,
their guards are marked in \textcolor{guards_color}{blue}.
You can also see segments from $\alldiag$ with their weights (equal to $\delta$).
\label{fig:diag_def}
\end{figure}


Every segment in $\alldiag$
connects points $(i(n^2+1) + t, j \cdot (n^2+1))$
and ${(i \cdot (n^2+1), j(n^2+1) + t)}$
for some $1 \le i,j \le k, 1 \le t \le n^2$.
The line on which it lies can be described
by linear equation ${x+y=t+(i+j)(n^2+1)}$,
thus these segments are in fact right-diagonal.

The constructed segment set is defined as:

$$\sets := \allhor \cup \allver \cup \alldiag.$$

The weight of each segment in $\allhor \cup \allver$
is equal to its length,
while every segment in $\alldiag$ has weight
$\delta$.


\definecolor{beg_color}{RGB}{255, 40, 40}
\definecolor{seg_color1}{RGB}{40, 40, 255}
\definecolor{seg_color2}{RGB}{40, 150, 40}

\newcommand{\addNode}[4]{
	\node[guard_h, label={left:$h_{#1,j,t_{#2,#3}}^L$}] at (\l#4,0) {};
	\node[node_h, label={left:$h_{#1,j,t_{#2,#3}}$}] at (\x#4,0) {};
	\node[guard_h, label={left:$h_{#1,j,t_{#2,#3}}^R$}] at (\r#4,0) {};
}

\begin{figure}
\hspace*{-1.5cm}
\begin{tikzpicture}[main/.style = {draw, circle}]
\tikzmath{
\step=2;
\eps=0.5;
%genereted by gen_math.py
\x1=\step;
\x2=\x1+\step;
\x3=\x2+\step;
\x4=\x3+\step;
\x5=\x4+\step;
\x6=\x5+\step;
\x7=\x6+\step;
\x8=\x7+\step;
\x9=\x8+\step;
\l1=\x1-\eps;
\r1=\x1+\eps;
\l2=\x2-\eps;
\r2=\x2+\eps;
\l3=\x3-\eps;
\r3=\x3+\eps;
\l4=\x4-\eps;
\r4=\x4+\eps;
\l5=\x5-\eps;
\r5=\x5+\eps;
\l6=\x6-\eps;
\r6=\x6+\eps;
\l7=\x7-\eps;
\r7=\x7+\eps;
\l8=\x8-\eps;
\r8=\x8+\eps;
\l9=\x9-\eps;
\r9=\x9+\eps;
}

\draw [beg_color] (\l1,0) to[out=150,in=30, looseness=200] (1.49,0);
\draw [beg_color] (\l1,0) to[out=40,in=140] (\l2,0);
\draw [beg_color] (\l1,0) to[out=40,in=140] (\l3,0);
\draw [beg_color] (\l1,0) to[out=40,in=140] (\l4,0);


\draw [seg_color1] (\r1,0) to[out=20,in=160] (\l6,0);
\draw [seg_color1] (\r1,0) to[out=20,in=160] (\l8,0);
\draw [seg_color2] (\r2,0) to[out=20,in=160] (\l7,0);
\draw [seg_color2] (\r2,0) to[out=20,in=160] (\l9,0);
\draw [seg_color1] (\r3,0) to[out=20,in=160] (\l6,0);
\draw [seg_color1] (\r3,0) to[out=20,in=160] (\l8,0);
\draw [seg_color2] (\r4,0) to[out=20,in=160] (\l7,0);
\draw [seg_color2] (\r4,0) to[out=20,in=160] (\l9,0);

\addNode{1}{1}{1}{1}
\addNode{1}{1}{2}{2}
\addNode{1}{2}{1}{3}
\addNode{1}{2}{2}{4}
\addNode{2}{1}{1}{6}
\addNode{2}{1}{2}{7}
\addNode{2}{2}{1}{8}
\addNode{2}{2}{2}{9}

\end{tikzpicture} 
\caption{\textbf{Vertices and segments in $HOR$.}}
This is an example for $n=2$ and any $1 \le j \le k$.
Points from $H$ are marked in black, their guards are marked in \textcolor{guards_color}{light blue}.
$t_{i,j}$ is a notation that we use for $\order^{-1}(i,j)$.
Segments are represented as arcs between endpoints.
You can see $\horbeg{j}{t}$ segments in \textcolor{beg_color}{red}.
$\horbeg{j}{1}$ is degenerated to a single point at $h_{1,1,t_{1,1}}^L$.
Segments $\hor{i}{j}{t_{x_1,y}}{t_{x_2,y}}$
are marked in \textcolor{seg_color1}{blue} and \textcolor{seg_color2}{green}.
\textcolor{seg_color1}{Blue} segments connect $t_{x_1,y}$ and $t_{x_2,y}$
such that they share y-coordinate equal to 1,
for~\textcolor{seg_color2}{green} segments it is equal to 2.

\label{fig:segments_def}
\end{figure}


\begin{equation}
w(s) =
	\begin{cases*}
	  length(s) 			& if $s \in \allhor \cup \allver$ \\
	  \delta        & if $s \in \alldiag$
	\end{cases*}
\end{equation}

Now, we prove that the constructed instance of geometric set cover
with weighted segments indeed gives a correct and sound reduction
of the grid tiling problem. Lemma \ref{set_cover_solution_exists}
proves that if a solution to the instance of the grid tiling instance exists,
then there exists a solution with suitably bounded size and weight
of the constructed instance of geometric set cover.
Then Lemma \ref{grid_tiling_exists} proves that if
there is a solution to the geometric set cover instance with bounded weight,
then there exists a solution to the original grid tiling instance.

\begin{lemma}
\label{set_cover_solution_exists}
	If there exists a~solution to the grid tiling instance $(f_{i,j})$,
	then there exists a~solution to the instance $\instanceSetCover$
	of geometric set cover with weight $\solWeight$.
\end{lemma}

\begin{proof}
Suppose there exists a solution $x,y$ of the instance $(f_{i,j})$
of the grid tiling problem.
	
We define the proposed solution $\sol \subseteq \sets$ of the instance
of geometric set cover
in three parts: $D \subseteq \alldiag$, $A \subseteq \allhor$ and $B \subseteq \allver$:
\begin{eqnarray*}
	D & := & \{(v_{i, j, t}, h_{i, j, t}) : 1 \le i, j \le k, t = \order^{-1}(x(i), y(j))\}, \\
	A & := & \{\horbeg{i}{\order^{-1}(x(1), y(i))} : 1 \le i \le k\} \\
	& \cup & \{\horend{i}{\order^{-1}(x(k), y(i))} : 1 \le i \le k\} \\
	& \cup & \{\hor{i}{j}{\order^{-1}(x(i), y(j))}{\order^{-1}(x(i+1), y(j))} : 1 \le i < k, 1 \le j \le k\}, \\
	B & := & \{\verbeg{i}{\order^{-1}(x(i), y(1))} : 1 \le i \le k\} \\
	& \cup & \{\verend{i}{\order^{-1}(x(i), y(k))} : 1 \le i \le k\} \\
	& \cup & \{\ver{i}{j}{\order^{-1}(x(i), y(j))}{\order^{-1}(x(i), y(j+1))} : 1 \le i \le k, 1 \le j < k\},
\end{eqnarray*}
	$$\sol := D \cup A \cup B.$$

Since $\points = H \cup V$, we show that $\sol$ covers the whole set $H$;
the proof for $V$ is analogous.

Fix any $1 \le j \le k$ and define $t_{i} := \order^{-1}(x(i), y(j))$.
The two leftmost segments in~$A$ for this $j$ are
$\horbeg{j}{t_1} = (h_{1,j,1}^L, h_{1, j, t_1}^L)$ and
$\hor{1}{j}{t_1}{t_2} = (h_{1,j, t_1}^R, h_{2,j,t_2}^L)$.
Therefore, points $h_{1,j,x}, h_{1,j,x}^L$ and $h_{1,j,x}^R$
for all $1 \le x \le n^2$ ale covered by $\horbeg{j}{t_1}$ and $\hor{1}{j}{t_1}{t_2}$,
excluding point $h_{1,j,t_1}$.

Analogously for $2 \le i \le k-1$,
the two consecutive segments $\hor{i-1}{j}{t_{i-1}}{t_i}$
and $\hor{i}{j}{t_i}{t_{i+1}}$ cover points $h_{i,j,x}, h_{i,j,x}^L$ and $h_{i,j,x}^R$
for all $1 \le x \le n^2$,
excluding point $h_{i,j,t_i}$.

Finally $\hor{k-1}{j}{t_{k-1}}{t_k}$ and $\horend{j}{t_k}$
cover all points $h_{k,j,x}, h_{k,j,x}^L$ and $h_{k,j,x}^R$
for ${1 \le x \le n^2}$, excluding point $h_{k,j,t_k}$.

$D$ covers all points $h_{i,j,t_i}$ and $v_{i,j,t_i}$.
As $j$ was chosen arbitrarily, all points in $H$ are covered.

The size of this proposed solution is:
$$|\sol| = |D| + |A| + |B| = k^2 + (k+1)k + (k+1)k = 3k^2+2k.$$

Then, we need to compute the total weight of the solution $\sol$.
First, we compute the sum of weights of segments in $A$.
Fix $1 \le j \le k$ and consider segments collinear
with the $j$-th horizontal line.
All points $h_{i,j,t}$, $h_{i,j,t}^L$ and $h_{i,j,t}^R$
for every $1 \le i \le k$ and $1 \le t \le n^2$ are covered by $A$
excluding points $h_{i,j,\order^{-1}(x(i),y(j))}$.
Every such point leaves a gap of length $2\epsilon$ between
$h_{i,j,\order^{-1}(x(i),y(j))}^L$ and $h_{i,j,\order^{-1}(x(i),y(j))}^R$.
Therefore, the total weight of segments in $A$
that lie on the line in question equals the length of the segment
$(h_{i,1,1}^L, h_{i,k,n^2}^R)$
minus $2\epsilon k$, which is $k(n^2+1) -2(1-\epsilon)-2k\epsilon$.
We need to multiply that by $k$, as we consider all possible values of $j$.

Computation for vertical segments is analogous and yields the same result.
Every segment in~$D$ has weight $\delta$, therefore the sum of all weights
is equal to:

$$2k(k(n^2+1) -2(1-\epsilon)-2k\epsilon) + k^2\delta= \solWeight.$$
\end{proof}

Now we present a few additional properties of the constructed instance
of the geometric set cover that help us to prove
Lemma \ref{grid_tiling_exists}.

\begin{claim}
\label{guards}
In any solution to the instance $\instanceSetCover$:
\begin{itemize}
\item the left and right guards of points in $H$
(points in $\{p^L : p \in H\} \cup \{p^R : p \in H\}$)
have to~be~covered with segments from $\allhor$,
\item the lower and upper guards of points in $V$
(points in $\{p^D : p \in V\} \cup \{p^U : p \in V\}$)
have to~be~covered with segments from $\allver$.
\end{itemize}
\end{claim}

\begin{proof}
We prove the claim for the points from $H$
as the proof for points from $V$ is analogous.

Every segment in $\allver$ is vertical and 
has x-coordinate equal to $i(n^2+1)$ for some $1\le i \le k$,
so they all have different x-coordinate
than any left or right guard of points in $H$.

For every point $x$ which is a left or right guard of a point in $H$,
there are $kn^2$ segments from $\alldiag$ that intersect with the horizontal
line that goes through $x$. All of these segments intersect with
this line in points from set $H$, therefore none of them
covers any of the guards.

Therefore none of the segments from $\allver$ or $\alldiag$ covers any
of the guards of the points in $H$.
\end{proof}

\begin{claim}
\label{one_diag_in_square}
For any $1 \le i, j \le n$
and any solution to the instance $\instanceSetCover$,
all but at most one point $h_{i, j, t}$
and at most one point $v_{i, j, t}$
for $1 \le t \le n^2$
must be
covered with segments from $\allhor$ or $\allver$.
\end{claim}

\begin{proof}
We prove the claim for horizontal segments,
as the proof for vertical segments is analoguous.

We prove this by contradiction. Assume that we
have two points $h_{i,j,t_1}, h_{i,j,t_2},1 \le t_1 <  t_2 \le n^2$,
such that they are not covered with segments from $\allhor$.

Point $h^R_{i, j, t_1}$ has to be covered with a segment from $\allhor$
by Claim $\ref{guards}$.
Every segment in $\allhor$ covering $h^R_{i, j, t_1}$,
but not $h_{i,j,t_1}$ must start at $h^R_{i, j, t_1}$
and all such segments cover also $h_{i, j, t_2}$.
This contradicts the assumption, which concludes the proof.
\end{proof}

\begin{lemma}
\label{vertical_horizontal_sum}
For every solution $\sol$ to the instance $\instanceSetCover$,
the sum of weights of segments chosen
from sets $\allhor$ and $\allver$ is at least
$\hvWeight$.
\end{lemma}

\begin{proof}
Let us fix $1 \le i \le k$.

We provide a lower bound for the sum of lengths
of vertical segments from $\sol \cap \allver$.
This bound is the same for each $i$ and is the same
for horizontal lines, thus we need to multiply such a bound by $2k$.

\begin{enumerate}[label={(\arabic*)}]
\item The total length between $v^D_{i, 1, 1}$ and $v^U_{i, k, n^2}$ is:
$$(k(n^2+1) + n^2 +\epsilon) - ((n^2+1)+1 -\epsilon) = k(n^2+1) - 2(1 - \epsilon).$$

\item For every $1 \le j \le k$ there exists at most one $1 \le t \le n^2$
such that $v_{i,j,t}$ is not covered by segments from $\allver$
(Claim \ref{one_diag_in_square}).
Its guards (see Definition \ref{guard_def}) $v^U_{i,j,t}$ and $v^D_{i,j,t}$
have to be covered in $\allver$ (Claim \ref{guards}).
Therefore, at most $k$ spaces of length $2\epsilon$ can be left
not covered by segments from $\allver$ between $v_{i,1,1}^D$ and $v_{i,k,n^2}^U$.

\end{enumerate}
The sum of these lower bounds for vertical and horizontal lines is:
$$2k(k(n^2+1) -2k\epsilon -2(1-\epsilon)) = 2k^2(n^2+1) -4k^2\epsilon -4k(1-\epsilon) = \hvWeight.$$
\end{proof}

\begin{lemma}
\label{diag_correct}
Let $\sol$ be a solution to a constructed instance $\instanceSetCover$
with weight at most  $\solWeight$.
Then for every $1 \le i,j \le k$
there exists $1 \le t \le n^2$ such that:
\begin{enumerate}[label={(\arabic*)}]
\item $v_{i,j,t}, h_{i,j,t}$ are not covered by segments from $\allver$ or $\allhor$;
\item segment $(v_{i,j,t}, h_{i,j,t})$ is in solution $\sol$;
\item $\order(t) \in f(i,j)$, that is, $\order(t)$ is an allowed tile for $(i,j)$;
\item for every $1 \le s\le n^2$, $s \neq t$, $v_{i,j,s}$ is covered in $\allver$;
\item for every $1 \le s\le n^2$, $s \neq t$, $h_{i,j,s}$ is covered in $\allhor$.
\end{enumerate}
\end{lemma}

\begin{proof}
At most one of the points $\{h_{i,j,t_x} : 1 \le t_x \le n^2\}$
and one of the points $\{v_{i,j,t_y} : 1 \le t_y \le n^2\}$
is covered with $\alldiag$
(Claim \ref{one_diag_in_square}).
	
Moreover, exactly one such point $h_{i,j,t_x}$ and one such point $v_{i,j,t_y}$
is covered with $\alldiag$,
because if none of them were covered, then the solution would have to
have weight at least $\hvWeight + 2\epsilon$ (see the proof of Lemma \ref{vertical_horizontal_sum}),
which is more than $\solWeight$.

We observe that points $h_{i,j,t_x}$ and $v_{i,j,t_y}$
have to be covered with the same segment from $\alldiag$.
Indeed we need to use at least $k^2$ of them to use
exactly one DIAG segment for every pair of $1 \le i,j \le k$,
if we used 2 segments from $\alldiag$
for one pair $(i,j)$,
then we would have used total weight at least
$\hvWeight + k^2\delta + \delta$ (Lemma \ref{vertical_horizontal_sum}),
which if more than $\solWeight$.
Since points $h_{i,j,t_x}$ and $v_{i,j,t_y}$ are covered by
a single segment from $\alldiag$, we have $t_x = t_y$.

Therefore $t_x = t_y$
and $\order(t_x)$ is an allowed tile for $(i,j)$
because the corresponding segment is in $\alldiag$.
\end{proof}

\newcommand{\diagonal}{\mathsf{diagonal}}
We refer to the function mapping $1 \le x \le k$ to $t_x$ from Lemma \ref{diag_correct}
as $\diagonal : \{1 \ldots k\} \times \{1 \ldots k\} \rightarrow \{1 \ldots n^2\}$.

\begin{lemma}
\label{vertical_horizontal_synchronized}
Let $\sol$ be any solution
of a constructed instance $\instanceSetCover$
with weight at most $\solWeight$. Then:
\begin{enumerate}
\item 
for any $1 \le i < k, 1 \le j \le k$,
$\matchh(\diagonal(i, j),\diagonal(i+1, j))$ is $\true$;
\item 
for any $1 \le i \le k, 1 \le j < k$,
$\matchv(\diagonal(i, j),\diagonal(i, j+1))$ is $\true$.
\end{enumerate}
\end{lemma}

\begin{proof}
We prove (1) by contradiction, the proof of (2) is analogous.

Let us take any $1 \le i < k, 1 \le j \le k$
and name $t_1 = \diagonal(i, j)$ and $t_2 = \diagonal(i+1, j)$.
We also assume that $\matchh(t_1,t_2)$ is \false,
which is equivalent to the fact that
segment $(h_{i,j,t_1}^R, h_{i+1,j,t_2}^L)$
is not in set $\allhor$.

Therefore $h_{i,j,t_1}$ and $h_{i+1,j,t_2}$
are not covered by segments from $\allhor$ (Lemma \ref{diag_correct}),
while $h^R_{i,j,t_1}$ and $h^L_{i+1,j,t_2}$
have to be covered by segments from $\allhor$ (Claim \ref{guards}).


Every segment from $\allhor$ either:
\begin{itemize}
\item starts at point $h^R_{x,y,z_1}$
and ends at point $h^L_{x+1,y,z_2}$ for some
$1 \le x < k$,$1 \le y \le k$ and $1 \le z_1, z_2 \le n^2$;
\item is $\horbeg{y}{z}$ 
and starts at $h^L_{1,y,1}$ and ends at $h^L_{1,y,z}$ for some $1 \le y \le k$ and $1 \le z \le n^2$;
\item is $\horend{y}{z}$
and starts at $h^R_{k,y,z}$ and ends at $h^R_{k,y,n^2}$ for some $1 \le y \le k$ and $1 \le z \le n^2$.
\end{itemize}
All of the points between $h^R_{i,j,t_1}$ and $h^L_{i+1,j,t_2}$
are covered by segments in $\allhor$ 
and there is no segment $(h^R_{i,j,t_1}, h^L_{i+1,j,t_2})$ in $\allhor$.
Hence, there are at least two different segments covering them.
If both of these segments are neither $\horbeg{y}{z}$ nor $\horend{y}{z}$,
then one of them must begin
at $h^R_{i,j,t_1}$ and end at $h^L_{i+1,j,z_2}$
and there must be other one that begins at $h^R_{i,j,z_1}$
and ends at $h^L_{i+1,j,t_2}$
for some $1 \le z_1, z_2 \le n^2$.

Thus, the space between $h^R_{i,j,z_1}$ and $h^L_{i,j+1,z_2}$
would be covered twice and is longer than $\epsilon$.
The case when one of them is $\horbeg{y}{z}$ or $\horend{y}{z}$ is analogous.
They can't be both $\horbeg{y}{z}$ or $\horend{y}{z}$.

By the proof of Lemma \ref{vertical_horizontal_sum},
the lower bound for weight of such a solution is $\hvWeight + \epsilon$
which is more than $\solWeight$.

Therefore $h^R_{i,j,t_1}$ and $h^L_{i+1,j,t_2}$ must be covered
by one segment from $\allhor$, namely $(h^R_{i,j,t_1}, h^L_{i+1,j,t_2})$.
Hence $(h^R_{i,j,t_1}, h^L_{i+1,j,t_2})$ is a segment in $\allhor$
and $\matchh(t_1,t_2)$ is $\true$.
\end{proof}


\begin{lemma}
\label{grid_tiling_exists}
	If there exists a solution to instance $\instanceSetCover$
	with weight at most $\solWeight$,
	then there exists a solution to the grid tiling instance $(f_{i,j})$.
\end{lemma}

\begin{proof}
Take $\diagonal$ function from Lemma \ref{diag_correct}.

To define the $x$ funtion 
for every $1 \le i \le k$ set $x(i) := x_i$
where $(x_i, a) = \order(v_{i,1})$.
Similarily, to define the $y$ function,
for every $1 \le i \le k$ set $y(i) := y_i$
where $(b, y_i) = \order(h_{1,i})$

To prove that this is a correct solution to grid tiling,
we need to prove that 
for every $1 \le i,j \le k$, $(x(i), y(j))$ is in
the allowed tiles set $f(i,j)$.

Let us take any $1 \le i,j \le k$.
By Lemma \ref{vertical_horizontal_synchronized}
and simple induction,
we know that $\matchh(\diagonal(1, j),\diagonal(i, j))$ and
$\matchv(\diagonal(i, 1),\diagonal(i, j))$ are $\true$.
Therefore $\order(\diagonal(i,j)) = (x(i), y(j))$.
By Lemma \ref{diag_correct} we know that 
$\order(\diagonal(i,j))$ is in $f(i,j)$.
Therefore 
$(x(i), y(j))$
is in $f(i,j)$.
\end{proof}

