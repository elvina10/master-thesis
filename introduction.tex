\chapter{Introduction}

Some problems in Computer Science are known to be NP-complete,
meaning that assuming P$\neq$NP there is no polynomial time
algorithm that can solve these problems.
Even so, they still can be amendable to different approaches,
such as approximation or parametrization.

\begin{defi}
In \textbf{Set Cover} problem we are given a set of elements (universe)
$\points$ and a family of sets $\sets$, that are subsets of unvierse $\points$
and they sum up to the whole $\points$ set.
Our task is to find such a set $\sol \subseteq \sets$,
that $\bigcup \sol = \points$ and size of $\sol$ is minimal possible.
\end{defi}

Set Cover is one classical example of an NP-complete problem,
which has been proven
in literature to be
inapproximable with factor $(1-o(1))\ln n$ unless $P = NP$
(which is a stronger result than APX-hardness),
and W[2]-hard with natural parametrization,
but restricting the problem to various specialized settings
can lead to more tractable special cases.
In this thesis we take a closer look at the Geometric Set Cover problem
in the plane, where points to cover are points in the plane
and sets to cover them with are geometric objects.

\paragraph{Approximation}
Over the years there has been a lot of work related to approximation
of Geometric Set Cover. Notably,
Geometric Set Cover with unweighted unit disks admits a PTAS (see
Corollary 1.1 in \cite{unit_disks}). When we consider the same problem
with weighted unit disks (or unit squares), the problem admits a QPTAS
\cite{settling_apx_hardness}, see also \cite{voronoi_true}.
On the other hand, \cite{rectangles_apx_hard} 
proves that Geometric Set Cover with unweighted axis-parallel rectangles
is APX-hard; they also show similar hardness
for Geometric Set Cover with many other standard geometric objects.

\paragraph{Parametrization}
We consider Geometric Set Cover 
parameterized by the size of solution.
Geometric Set Cover with unit squares was first proven to be W[1]-hard 
in \cite{marx05} (Theorem 5), later follow-up \cite{voronoi}
shows that there is an algorithm running in time $\mathcal{O}(n^{\sqrt{k}})$
that solves Geometric Set Cover with unit squares or disks
and that there is no algorithm running in time $f(k) \cdot o(n^{\sqrt{k}})$
for any computable $f$, so this is a tight bound for this problem.

We also consider parametrization of weighted problems.
There does not seem to be a consensus of what parametrization
in the weighted setting is exactly; there
was an attempt to introduce a quite complicated general
framework of weighted parameterized setting in \cite{weighted_framework}.
Kernels for several well-known weighted problems
such as Subset Sum or Knapsack are presented in \cite{kernel_weighted}.
Another work \cite{weighted_flow} considers weighted
parametrization of Weighted Directed Feedback Set and Weighted $st$-Cut.

\paragraph{$\delta$-extensions}
In this paper, we focus on Geometric Set Cover with segments with $\delta$-extension.
$\delta$-extension is a problem relaxation method based on the
$\delta$-shrinking model which was introduced in \cite{shrinking_original}
and later used in \cite{shrinking2} and \cite{shrinking1}.

\begin{defi}
\label{definition:delta_extension}
For any $\delta > 0$ and a center-symmetric object $L$ with
centre of symmetry $S = (x_s, y_s)$,
the \textbf{$\delta$-extension} of $L$ is the object $L^{+\delta} =
\{(1 + \delta)\cdot(x - x_s, y - y_s) + (x_s, y_s) : (x, y) \in L\}$,
that is, $L^{+\delta}$ is the image of $L$ under homothety centered
at $S$ with scale $(1+\delta)$.
\end{defi}

Similar model is used to prove that Geometric Set Cover with fat polygons
relaxed with $\delta$-extension admits EPTAS \cite{harpeled12}.

\section*{Our contribution}
In this paper we make the following contributions.

We show that approximation of uweighted Geometric Set Cover with axis-parallel segments
(even if we relax it with  $\frac{1}{2}$-extensions) is APX-hard
(Theorem \ref{segment_cover_apx_hard}).
This expands the previous result of Geometric Set Cover
with unweighted axis-parallel rectangles being APX-hard in \cite{rectangles_apx_hard}.
That also proves that the assumption in \cite{harpeled12}
for EPTAS about polygons being fat is necessary, because
cover with arbitrary polygons with $\delta$-extensions is APX-hard.

We also provide two FPT algorithms for parameterized Geometric Set Cover 	
with unweighted segments (Theorem~\ref{segment_cover_fpt})
and weighted segments relaxed with $\delta$-extensions
(Theorem~\ref{fpt_weighted_segment}).
But Geometric Set Cover with weighted
axis-parallel segments is W[1]-hard (Theorem~\ref{w1_hard})
and assiming ETH there does not exist algorithm for this problem
that runs in time $f(k)\cdot(|\points| + |\sets|)^{o(\sqrt{k})}$.
See Figure \ref{tab:weighted_fpt} for a summary of parameterized
results for the weighted setting.

\begin{figure}[h]
\begin{center}
\begin{tabular}{ | c | c | c | }
\hline
                & exact     & $\delta$-extensions \\ 
\hline                
 axis-parallel   & W[1]-hard & FPT* \\  
\hline                
 any direction   & W[1]-hard* & FPT \\
\hline                
\end{tabular}
\caption{\textbf{Our results for Geometric Set Cover problem with weighted segments 
parameterized by the size of solution.}}

Results marked with * directly follow from more or less restricted settings.
\label{tab:weighted_fpt}
\end{center}
\end{figure}


\iffalse

The Set Cover problem is one of the most common NP-complete problems.
[tutaj referencja]
We are given a family of sets and have to choose the smallest
subfamily of these sets that cover all their elements.
This problem naturally extends to settings
were we put different weights on the sets
and look for the subfamily of the minimal weight.
This problem is NP-complete even 
without weights and if we put
restrictions on what the sets can be.
One of such variants is Vertex Cover problem,
where sets have size 2 (they are edges in a graph).

In this work we focus on another such variant where the sets correspond
to some geometric shapes and
only some points of the plane have to be covered.
When these shapes are rectangles with edges parallel
to the axis, the problem can be proven to
be W[1]-complete (solution of size $k$ cannot be found
in $n^o(k)$ time),
APX-complete (for suffciently small $\epsilon > 0$, the problem
does not admit $1+\epsilon$-approximation scheme)
[refrencje].

Some of these settings are very easy.
Set cover with lines parallel to one of the axis
can be solved in polynomial time.

There is a notion of $\delta$-expansions,
which loosen the restrictions on geometric set cover.
We allow the objects to cover the points
after $\delta$-expansion and compare
the result to the original setting.
This way we can produce both FPT and EPTAS
for the rectangle set cover with $\delta$-extensions
[referencje].



\paragraph{Our contribution.}
In this work, we prove that unweighted geometric set cover
with segments is fixed parameter tractable (FPT).

Moreover, we show that geometric set cover with segments
is APX-complete for unweighted axis-parallel segments,
even with 1/2-extensions.
So the problem for very thin rectangles
also cannot admit PTAS.
Therefore, in the efficient polynomial-time approximation scheme (EPTAS)
for \textit{fat polygons} by \cite{harpeled12},
the assumption about polygons being fat is necessary. 

Finally, we show that geometric set cover with weighted segments in
3 directions is W[1]-complete.
However, geometric set cover with weighted segments is FPT if we allow
$\delta$-extension.

This result is especially interesting,
since it's counter-intuitive that
the unweighed setting is FPT and the weighted
setting is W[1]-complete.
Most of such problems (like vertex cover or [wiecej przykladow])
are equally hard in both weighted and unweighted settings.

\fi
