\chapter{APX-hardness of geometric set cover}
\newcommand{\setCoverInstance}{(\points, \sets)}
\newcommand{\true}{\texttt{true}}
\newcommand{\false}{\texttt{false}}
\newcommand{\opt}{\mathsf{opt}}
\newcommand{\approximate}{\mathsf{approx}^{*}}

\label{chapter:segment_apx}

In this section we analyze whether there exists 
a PTAS for geometric set cover for rectangles.
We show that Segment Set Cover
is APX-hard even if we can restrict this problem
to a very simple setting:
segments parallel to axes and allow $\frac{1}{2}$-extension.

Our result can be summarized in the following
theorem and this section aims to prove it.

\segmentCoverApxHard*

We prove Theorem \ref{segment_cover_apx_hard}
by taking a problem that is APX-hard
and showing a reduction.
For this problem we choose
MAX-(3,3)-SAT which we define below.

\section{MAX-(3,3)-SAT}

See Definition \ref{definition:cnf}
for the definition of the $k$-CNF formula.

\begin{defi}
\textbf{MAX-3SAT} is the following maximization problem. We are given a 3-CNF
formula, and we need to find a boolean assignment of variables
that satisfies the most clauses.
\end{defi}

\begin{defi}
\textbf{MAX-(3,3)-SAT} is a variant of MAX-3SAT with an additional
restriction that every variable appears in exactly 3 clauses
and every clause contains exactly 3 literals of 3 different variables.
Note that thus, the number of clauses is equal to the number of variables.
\end{defi}

In our proof of Theorem \ref{segment_cover_apx_hard} we use
hardness of approximation of MAX-(3,3)-SAT proved
in \cite{hastad} and described in
Theorem \ref{hastadtheorem} below.

\begin{defi}
MAX-3SAT formula with $m$ clauses is \textbf{at most $\alpha$-satisfiable}, if
every assignment of variables satisfies no more than $\alpha m$
clauses. 
\end{defi}

\begin{tw}[\cite{hastad}]{
	\label{hastadtheorem}
	For any $\epsilon > 0$, it is NP-hard to distinguish satisfiable
	\linebreak
	(3,3)-SAT formulas from
	at most
	\mbox{$(\frac{7}{8} + \epsilon)$-satisfiable}
	(3,3)-SAT formulas.
}\end{tw}

\section{Reduction}

The following lemma encapsulates the properties
of the reduction described in this section,
and it allows us to prove Theorem \ref{segment_cover_apx_hard}.

\begin{lemma}{
	\label{apxconstruction}
	Given an instance $S$ of MAX-(3,3)-SAT 
	with $n$ variables and optimum value $\opt(S)$,
	we can construct an instance $\setCoverInstance$ of geometric set cover with
	axis-parallel segments in 2D such that:
	\begin{enumerate}[label={(\arabic*)}]
	\item \label{item:apxconstruction_correctness}
	For every solution to instance $S$ that satisfies $k$ clauses,
	there exists a solution to $\setCoverInstance$ of size $15n - k$.
	
	\item \label{item:apxconstruction_completness}
	For every solution $\sol$ to instance $\setCoverInstance$,
	there exists a solution to $S$ that satisfies at least  $15n - |\sol|$
	clauses.
	
	\item \label{lemma:apxconstruction:enumerate:extension}
	For every $\sol \subseteq \sets$, if $\sol^{+\frac{1}{2}}$
	is a solution to $\setCoverInstance$, then $\sol$
	is also a solution to $\setCoverInstance$.
\end{enumerate}
Therefore, the optimum size of a solution to $\setCoverInstance$
is $\opt(\setCoverInstance) = 15n - \opt(S)$. 
	
}\end{lemma}

We prove Lemma \ref{apxconstruction} in
subsequent sections. Section \ref{construction_description}
describes the proposed instance $\setCoverInstance$.
Property \ref{item:apxconstruction_correctness} is proved by Lemma~\ref{construction_correctness},
\ref{item:apxconstruction_completness} by Lemma~\ref{construction_completness},
and finally \ref{lemma:apxconstruction:enumerate:extension} trivially
follows from Lemma~\ref{lemma:exntension_robust}.
Firstly let us prove
Theorem \ref{segment_cover_apx_hard} using Lemma \ref{apxconstruction}
and Theorem \ref{hastadtheorem}.

\begin{proof}[Proof of Theorem \ref{segment_cover_apx_hard}]
Consider any $0 < \epsilon < \frac{1}{15 \cdot 8}$.

Let us assume that there exists a polynomial-time
$(1+\epsilon)$-approximation algorithm
for unweighted geometric set cover with axis-parallel segments in 2D
with $\frac{1}{2}$-extension.
We construct an algorithm that solves the problem stated in 
Theorem \ref{hastadtheorem}, thereby proving that P~=~NP.

Take an instance~$S$ of MAX-(3,3)-SAT to be distinguished
and construct an instance of geometric set cover $\setCoverInstance$
using Lemma \ref{apxconstruction}.
We now use the $(1+\epsilon)$-approximation algorithm
for geometric set cover relaxed with $\frac{1}{2}$-extensions on $\setCoverInstance$.
Denote the size of the solution returned by this algorithm as $\approximate(\setCoverInstance)$.
We prove that 
if in $S$
one can satisfy at most $(\frac{7}{8}+\epsilon)n$ clauses,
then $\approximate(\setCoverInstance) \ge 15n - (\frac{7}{8} + \epsilon)n$,
and if $S$ is
satisfiable, then $\approximate(\setCoverInstance) < 15n - (\frac{7}{8} + \epsilon)n$.


\textbf{Assume $S$ satisfiable.}
From the definition of $S$ being satisfiable, we have:
$$\opt(S) = n.$$

From Lemma \ref{apxconstruction} we have:

$$\opt(\setCoverInstance) = 14n.$$

Therefore,
$$\approximate(\setCoverInstance) \le (1+\epsilon)\opt(\setCoverInstance) = 14n(1+\epsilon)
	= 14n + 14\epsilon\cdot n =$$ 
	$$= 14n + (15\epsilon - \epsilon)n < 
  14n + \left(\frac{1}{8} - \epsilon\right)n 
= 15n - \left(\frac{7}{8} + \epsilon\right)n.$$
\textbf{Assume $S$ is at most 
$\left(\frac{7}{8} + \epsilon\right)$ satisfiable.}
From the definition of $S$ being at most 
$\left(\frac{7}{8} + \epsilon\right)n$ satisfiable, we have:
$$\opt(S) \le \left(\frac{7}{8} + \epsilon\right)n$$

From Lemma \ref{apxconstruction} we have:
$$\opt(\setCoverInstance) \ge 15n - \left(\frac{7}{8} + \epsilon\right)n$$

Since a solution to $\setCoverInstance$ with $\frac{1}{2}$-extension is
also a solution without any extension, by 
Lemma \ref{apxconstruction} \ref{lemma:apxconstruction:enumerate:extension}, we have:

$$\approximate(\setCoverInstance) \ge \opt(\setCoverInstance) = 15n - \left(\frac{7}{8} + \epsilon\right)n$$


Therefore, by using the assumed $(1+\epsilon)$-approximation
algorithm,
it is possible to distinguish the case when
$S$ is satisfiable from the case when it is
at most $(\frac{7}{8} + \epsilon)n$ satisfiable:
it suffices to compare $\approximate(\setCoverInstance)$ with $15n - (\frac{7}{8}+\epsilon)n$.
Hence, the assumed approximation algorithm cannot exist, unless P = NP.
\end{proof}

\section{Construction of the Geometric Set Cover instance}
\label{construction_description}
We proceed to the proof of Lemma \ref{apxconstruction}.
That is, we show a reduction from the MAX-(3,3)-SAT problem
to geometric set cover with segments
parallel to axes. Moreover, the obtained instance
of geometric set cover will be robust
to $\frac{1}{2}$-extension (have the same optimal solution
after $\frac{1}{2}$-extension).

The construction will be composed of 2 types of gadgets:
\textbf{VARIABLE-gadgets} and \textbf{CLAUSE-gadgets}.
CLAUSE-gadgets will be constructed using two \textbf{OR-gadgets}
connected together.

\subsection{VARIABLE-gadget}

VARIABLE-gadget is responsible for choosing the value of a variable
in a CNF formula. It allows two minimum solutions of size 3 each.
These two choices correspond to the two Boolean values of the variable
corresponding to this gadget.

\paragraph{Points.}

Define points $a,b,c,d,e,f,g,h$ as follows, where $L = 22n$:


\newcommand{\pointsVarNoArg}{\mathsf{pointsVariable} }
\newcommand{\pointsVar}[1]{\mathsf{pointsVariable}_{#1} }
\newcommand{\chooseVar}[2]{\mathsf{chooseVariable}^{#1}_{#2} }
\newcommand{\segmentsVar}[1]{\mathsf{segmentsVariable}_{#1} }

\definecolor{x_true_colour}{RGB}{40, 40, 255}
\definecolor{x_false_colour}{RGB}{255, 40, 40}

{\tikzset{point/.style={
    circle, draw=black, fill, fill=black, minimum size=4pt,inner sep=0pt, outer sep=0pt,
    prefix after command= {\pgfextra{\tikzset{every
    label/.style={label distance=0.05cm,text=black}}}}
    }
}

{\tikzset{point_not_cover/.style={
    circle, draw=black, fill, fill=white, minimum size=4pt,inner sep=0pt, outer sep=0pt,
    prefix after command= {\pgfextra{\tikzset{every
    label/.style={label distance=0.05cm,text=black}}}}
    }
}

\begin{figure}[h]
\centering
\begin{tikzpicture}
\tikzmath{
\stepx=1.5;
\stepy=1;
\y1=0;
\y2=\y1+\stepy;
\y3=\y2+\stepy;
\x1=0;
\x2=\x1+\stepx;
\x3=\x2+\stepx;
\xend=\x3+2*\stepx;
}


\draw[x_false_colour,very thick] (\x1,\y1) -- (\x3,\y1);
\draw[x_false_colour,very thick] (\x1,\y2) -- (\x2,\y2);
\draw[x_false_colour,very thick] (\x2,\y3) -- (\xend,\y3);
\draw[x_true_colour,very thick] (\x1,\y1) -- (\x1,\y2);
\draw[x_true_colour,very thick] (\x2,\y1) -- (\x2,\y3);
\draw[x_true_colour,very thick] (\x3,\y1) -- (\xend,\y1);

\node[point,label={below:$a_i$}] at (\x1,\y1) {};
\node[point,label={below:$b_i$}] at (\x2,\y1) {};
\node[point,label={below:$c_i$}] at (\x3,\y1) {};
\node[point,label={left:$d_i$}] at (\x1,\y2) {};
\node[point,label={above left:$e_i$}] at (\x2,\y2) {};
\node[point,label={above left:$f_i$}] at (\x2,\y3) {};
\node[point_not_cover,label={right:$g_i$}] at (\xend,\y1) {};
\node[point_not_cover,label={right:$h_i$}] at (\xend,\y3) {};


\end{tikzpicture}
\caption{\textbf{VARIABLE-gadget.}
We denote the set of points marked with black circles as $\pointsVar{i}$,
and they need to be covered (are part of the set $\points$).
Note that some of the points are not marked as black dots
and exists only to name segments for further reference.
We denote the set of \textcolor{x_false_colour}{red} segments as $\chooseVar{\false}{i}$
and the set of \textcolor{x_true_colour}{blue} segments as $\chooseVar{\true}{i}$.}
\label{fig:apx_choose_variable}
\end{figure}


\begin{center}
\begin{tabular}{ l l l l}
	$a := (-3L, 0)$ &
	$b := (-2L, 0)$ & 
	$c := (-L, 0)$ & 
	$d := (-3L, 1)$ \\  
	$e := (-2L, 1)$ & 
	$f := (-2L, 2)$ &
	$g := (L, 0)$ &
	$h := (L, 2)$
\end{tabular}
\end{center}


Let us define:
$$\pointsVarNoArg :=  \{a, b, c, d, e, f\}$$
and, for any $1 \le i \le n$,
$$\pointsVar{i} := \pointsVarNoArg + (0, 4i).$$

We denote $a_i := a + (0,4i)$ etc.

\paragraph{Segments.}

\newcommand{\xTrueSegmentDef}[1]{(c_{#1}, g_{#1})}
\newcommand{\xFalseSegmentDef}[1]{(f_{#1}, h_{#1})}
\newcommand{\xTrueSegment}[1]{\mathsf{xTrueSegment}_{#1}}
\newcommand{\xFalseSegment}[1]{\mathsf{xFalseSegment}_{#1}}
\newcommand{\orTrueSegment}[2]{(t_{#1, #2}, v_{#1, #2})}

Let us define:

$$\chooseVar{\true}{i} :=\{ (a_i, d_i), (b_i, f_i), (c_i, g_i)\},$$
$$\chooseVar{\false}{i} := \{(a_i, c_i), (d_i, e_i), (f_i, h_i)\},$$
$$\segmentsVar{i} := \chooseVar{\true}{i} \cup \chooseVar{\false}{i}.$$

We also name two of these segments for future reference:
$\xTrueSegment{i} := \xTrueSegmentDef{i},$
$\xFalseSegment{i} := \xFalseSegmentDef{i}.$

\begin{lemma}
\label{choose_variables_solution}
For any $1 \le i \le n$, points in $\pointsVar{i}$
can be covered using 3 segments from $\segmentsVar{i}$.
\end{lemma}

\begin{proof}
We can use either set $\chooseVar{\true}{i}$ or $\chooseVar{\false}{i}$.
\end{proof}

\begin{lemma}
\label{choose_variables_no_less}
For any $1 \le i \le n$, points in $\pointsVar{i}$
can not be covered with fewer than 3 segments from $\segmentsVar{i}$.
\end{lemma}

\begin{proof}
No segment of $\segmentsVar{i}$ covers more than one point from
$\{d_i, f_i, c_i\}$, therefore $\pointsVar{i}$ can
not be covered with fewer than 3 segments.
\end{proof}

\begin{lemma}
\label{choose_variables_both}
For every set $A \subseteq \segmentsVar{i}$ such that $A$ covers $\pointsVar{i}$
and $\xTrueSegment{i}, \xFalseSegment{i} \in A$,
it holds that $|A| \ge 4$.
\end{lemma}
\begin{proof}
No segment from $\segmentsVar{i}$ covers more than one point from
$\{a_i, e_i\}$,
therefore 
$\pointsVar{i} - \{c_i, f_i\}$
can not be covered with fewer than 2 segments.
\end{proof}


\subsection{OR-gadget}

An OR-gadget connects input and output segments (see Figure \ref{fig:apx_or_gadget})
in a way that is supposed to simulate the binary disjunction.

Input segments are the only segments that cover points outside of the gadget,
as their left ends lie outside of it.
Point $v_{i,j}$ is the only one that can be covered
by segments that do not belong to the gadget.

The OR-gadget has the property that every set of segments
that covers all the points in the gadget uses at least 3 segments from it.
Moreover, the output segment belongs to the solution of size 3
only if at least one of the input segments belongs to the solution.
Therefore, optimum solutions restricted to the OR-gadget behave
like a binary disjunction for the input segments.


\paragraph{Points.}
We define

\newcommand{\chooseOr}[3]{\mathsf{chooseOr}^{#1}_{#2,#3}}
\newcommand{\orMoveVariable}[2]{\mathsf{orMoveVariable}_{#1,#2}}
\newcommand{\pointsOr}[2]{\mathsf{pointsOr}_{#1,#2}}
\newcommand{\segmentsOr}[2]{\mathsf{segmentsOr}_{#1,#2}}

%https://davidmathlogic.com/colorblind/#%23646464-%2389DA6D-%23FFCF3C-%2338D1F1-%23FF0909-%23238CD2
\definecolor{environment}{RGB}{100,100,100}
\definecolor{move_variable1}{RGB}{137, 218, 109}
\definecolor{move_variable2}{RGB}{255, 207, 60}
\definecolor{choose_true1}{RGB}{56, 209, 241}
\definecolor{choose_true2}{RGB}{35, 140, 210}
\definecolor{choose_false}{RGB}{255, 9, 9}

{\tikzset{point/.style={
    circle, draw=black, fill, fill=black, minimum size=4pt,inner sep=0pt, outer sep=0pt,
    prefix after command= {\pgfextra{\tikzset{every
    label/.style={label distance=0.05cm,text=black}}}}
    }
}

\begin{figure}[h]
\centering
\def\svgwidth{0.5\columnwidth}
\begin{tikzpicture}
\tikzmath{
\stepx=1.5;
\stepy=1.5;
\xbeg=0;
\x1=\xbeg+3*\stepx;
\x2=\x1+\stepx;
\x3=\x2+\stepx;
\xend=\x3+\stepx;
\y1=0;
\y2=\y1+\stepy;
\y3=\y2+\stepy;
\y4=\y3+\stepy;
\y5=\y4+\stepy;
}

\draw[environment,ultra thick] (\xbeg,\y1) -- (\x1,\y1) node[black,pos=0.15, above] {$input_x$};
\draw[environment,ultra thick] (\xbeg,\y5) -- (\x1,\y5) node[black,pos=0.15, above] {$input_y$};
\draw[move_variable1, ultra thick] (\x1,\y1) -- (\x1,\y3);
\draw[move_variable2, ultra thick] (\x1,\y5) -- (\x1,\y3);
\draw[choose_true1, ultra thick] (\x1,\y2) -- (\x3,\y2);
\draw[choose_true1, ultra thick] (\x1,\y4) -- (\x3,\y4);
\draw[choose_true2, ultra thick] (\x3,\y3) -- (\xend,\y3) node [black,pos=0.5, below] {$output$};
\draw[choose_false, ultra thick] (\x2,\y2) -- (\x2,\y4);
\draw[choose_false, ultra thick] (\x3,\y2) -- (\x3,\y4);

\node[point,label={above left:$l_{i,j}$}] at (\x1,\y1) {};
\node[point,label={left:$m_{i,j}$}] at (\x1,\y2) {};
\node[point,label={left:$n_{i,j}$}] at (\x1,\y3) {};
\node[point,label={left:$o_{i,j}$}] at (\x1,\y4) {};
\node[point,label={above left:$p_{i,j}$}] at (\x1,\y5) {};

\node[point,label={below:$q_{i,j}$}] at (\x2,\y2) {};
\node[point,label={above:$r_{i,j}$}] at (\x2,\y4) {};
\node[point,label={below:$s_{i,j}$}] at (\x3,\y2) {};
\node[point,label={left:$t_{i,j}$}] at (\x3,\y3) {};
\node[point,label={above:$u_{i,j}$}] at (\x3,\y4) {};
\node[point,label={above:$v_{i,j}$}] at (\xend,\y3) {};

\end{tikzpicture}
\caption{
	\textbf{OR-gadget.} Segments from $\chooseOr{\false}{i}{j}$ are \textcolor{choose_false}{red},
	segments from $\chooseOr{\true}{i}{j}$ are blue
	(both \textcolor{choose_true1}{light blue} and \textcolor{choose_true2}{dark blue}),
	segments from $\orMoveVariable{i}{j}$ are \textcolor{move_variable1}{green} and \textcolor{move_variable2}{yellow}.
	\textcolor{choose_true2}{Dark blue} segment is the $output$ segment.
	\textcolor{environment}{Grey segments} $input_x$ and $input_y$ are input segments that
	are not part of $\segmentsOr{i}{j}$.
}
\label{fig:apx_or_gadget}
\end{figure}


\newcommand{\clauseXFactor}{20}

\begin{center}
	\begin{tabular}{ l l l l}
		$l_0 := (0, 0)$ &
		$m_0 := (0, 1)$ &
		$n_0 := (0, 2)$ &
		$o_0 := (0, 3)$ \\
		$p_0 := (0, 4)$ &
		$q_0 := (1, 1)$ &
		$r_0 := (1, 3)$ &
		$s_0 := (2, 1)$ \\
		$t_0 := (2, 2)$ &
		$u_0 := (2, 3)$ &
		$v_0 := (3, 2)$ &
	\end{tabular}
\end{center}


$$vec_{i, j} := (\clauseXFactor i + 3 + 3j, 4(n+1) + 2j)$$

For integers $i,j$,
define 
$\{ l_{i, j}, m_{i, j}, \ldots, v_{i, j} \}$
as $\{l_0, m_0, \ldots, v_0\}$ shifted by $vec_{i, j}$,
i.e. ${l_{i,j} = l_0 + vec_{i,j}}$ etc.

Note that $v_{i, 0} = l_{i, 1}$ (see Figure~\ref{fig:apx_clause}).
Next, let

$$\pointsOr{i}{j} := 
 \{l_{i, j}, m_{i, j}, n_{i, j}, o_{i, j},
 p_{i, j}, q_{i, j}, r_{i, j}, s_{i, j}, t_{i, j}, u_{i, j} \}
 $$
Note that $\pointsOr{i}{j}$ does not include the point $v_{i,j}$.
 
\paragraph{Segments.}

We define the set of segments in several parts:
 
 
\begin{eqnarray*}
\chooseOr{\false}{i}{j} &:=& \{ (q_{i, j}, r_{i, j}), (s_{i, j}, u_{i, j})\}, \\
\chooseOr{\true}{i}{j} &:=& \{ (m_{i, j}, s_{i, j}), (o_{i, j}, u_{i, j}), (t_{i, j}, v_{i, j}) \}, \\
\orMoveVariable{i}{j} &:=& \{ (l_{i, j}, n_{i, j}), (n_{i, j}, p_{i, j})\}.
\end{eqnarray*}
Finally all segments on OR-gadget are defined as:

$$\segmentsOr{i}{j} := 
  \chooseOr{\false}{i}{j} \cup \chooseOr{\true}{i}{j} \cup \orMoveVariable{i}{j}
$$


\begin{lemma}
\label{cover_or_true}
For any $1 \le i \le n, j \in \{0, 1\}$ and 
 $x \in \{l_{i, j}, p_{i, j}\}$, points in
${\pointsOr_{i, j} - \{ x\} \cup \{v_{i, j}\}}$
can be covered
with 4 segments from $\segmentsOr{i}{j}$.
\end{lemma}

\begin{proof}
We can do this using one segment from
$\orMoveVariable{i}{j}$, the one that does not cover $x$,
and all segments from $\chooseOr{\true}{i}{j}$.
\end{proof}

\begin{lemma}
\label{cover_or_false}
For any $1 \le i \le n, j \in \{0, 1\}$, points in
$\pointsOr{i}{j}$ can be covered
with 4 segments from $\segmentsOr{i}{j}$.
\end{lemma}

\begin{proof}
We can do this using segments from $\orMoveVariable{i}{j} \cup \chooseOr{\false}{i}{j}$.
\end{proof}


\subsection{CLAUSE-gadget}
A CLAUSE-gadget is responsible for determining whether
variable values assigned in variable gadgets
satisfy the corresponding clause in the input formula $\phi$.
It has a minimum solution of size $w$
if and only if the clause is satisfied, i.e. at least one
of the respective variables is assigned the correct value.
Otherwise, its minimum solution has size $w+1$.
In this way, by analyzing the size of the minimum
solution to the entire constructed instance, we will be able to tell
how many clauses it is possible to satisfy
in an optimum solution to $\phi$.


\paragraph{Points.}

\newcommand{\pointsClause}[1]{\mathsf{pointsClause}_{#1}}
\newcommand{\negate}{\mathsf{neg}}
\newcommand{\idx}{\mathsf{idx}}

\definecolor{environment}{RGB}{150,150,150}

{\tikzset{point/.style={
    circle, draw=black, fill, fill=black, minimum size=4pt,inner sep=0pt, outer sep=0pt,
    prefix after command= {\pgfextra{\tikzset{every
    label/.style={label distance=0.05cm,text=black,font=\footnotesize}}}}
    }
}

\begin{figure}[h]
\centering
\begin{tikzpicture}
\tikzmath{
\stepx=1.7;
\stepy=1.5;
\boxsize=1.2*\stepx;
\xbeg=0;
\x1=\xbeg+2*\stepx;
\x2=\x1+\stepx;
\x3=\x2+\stepx;
\x4=\x3+\stepx;
\x5=\x4+\boxsize;
\x6=\x5+\stepx;
\x7=\x6+\boxsize;
\x8=\x7+\stepx;
\y1=0;
\y2=\y1+\stepy;
\y3=\y2+\stepy;
\y4=\y3+2*\stepy;
\y5=\y4+0.5*\stepy;
\y6=\y5+0.5*\stepy;
\y7=\y6+0.5*\stepy;
}

\draw[thick] (\x1,\y3) -- (\x1,\y4);
\draw[thick] (\x2,\y1) -- (\x2,\y6);
\draw[thick] (\x3,\y2) -- (\x3,\y7);
\draw[thick] (\x1,\y4) -- (\x4,\y4);
\draw[thick] (\x2,\y6) -- (\x4,\y6);
\draw[thick] (\x3,\y7) -- (\x6,\y7);
\draw[thick] (\x5,\y5) -- (\x6,\y5);
\draw[thick] (\x7,\y6) -- (\x8,\y6);
\draw [fill=green!50, draw=green!50] (\x4,\y6) rectangle (\x5,\y4) node[pos=0.5] {\scriptsize OR-gadget};
\draw [fill=green!50, draw=green!50] (\x6,\y5) rectangle (\x7,\y7) node[pos=0.5] {\scriptsize OR-gadget};

\draw[environment,thick] (\xbeg, \y1) -- (\x8, \y1) node[black,pos=0.7, below] {$variable_b^{true}$};
\draw[environment,thick] (\xbeg, \y2) -- (\x8, \y2) node[black,pos=0.7, below] {$variable_c^{false}$};
\draw[environment,thick] (\xbeg, \y3) -- (\x8, \y3) node[black,pos=0.7, above] {$variable_a^{true}$};

\node[point,label={above left:$x_{i,0}$}] at (\x1,\y3) {};
\node[point,label={above left:$y_{i,0}$}] at (\x2,\y1) {};
\node[point,label={above left:$z_{i,0}$}] at (\x3,\y2) {};
\node[point,label={above left:$x_{i,1}$}] at (\x1,\y4) {};
\node[point,label={above left:$y_{i,1}$}] at (\x2,\y6) {};
\node[point,label={above left:$z_{i,1}$}] at (\x3,\y7) {};
\node[point,label={above left:$l_{i,0}$}] at (\x4,\y4) {};
\node[point,label={above left:$p_{i,0}$}] at (\x4,\y6) {};
\node[point,label={above right:$t_{i,0}$}] at (\x5,\y5) {};
\node[point,label={below:$v_{i,0}=l_{i,1}$}] at (\x6,\y5) {};
\node[point,label={above:$p_{i,1}$}] at (\x6,\y7) {};
\node[point,label={above right:$t_{i,1}$}] at (\x7,\y6) {};
\node[point,label={above right:$v_{i,1}$}] at (\x8,\y6) {};
\end{tikzpicture}

\caption{\textbf{CLAUSE-gadget for a clause $a \lor b \lor \neg c$.}
Every green rectangle is an OR-gadget.
$y$-coordinates of $x_{i, 0}$, $y_{i, 0}$ and $z_{i,0}$
depend on the variables in the $i$-th clause.
Grey segments corresponds to the values of variables
satistying the $i$-th clause.
}
\label{fig:apx_clause}
\end{figure}

First, we define auxiliary functions for literals. 
For a literal $w$, let $\idx(w)$ be the index of the variable in $w$,
and $\negate(w)$ be the Boolean value (0 or 1) whether the variable is negated in $w$
or not.

\begin{eqnarray*}
\idx(w) & := &  i \text{ when } w = x_i \\
\negate(w) & := & 
\begin{cases}
 0 & \text{if } w = x_i \\
 1 & \text{if } w = \neg x_i
\end{cases}
\end{eqnarray*}

Let us assume that clause $C_i = a \lor b \lor c$
for any literals $a,b,c$. Then, we define points in the gadget as:

\begin{center}
\begin{tabular}{ l l }
	$x_{i, 0} := (\clauseXFactor i, 4\cdot \idx(a) + 2\cdot \negate(c)),$ &
	$x_{i, 1} := (\clauseXFactor i, 4(n+1)),$ \\
	$y_{i, 0} := (\clauseXFactor i+1, 4\cdot \idx(b) + 2\cdot \negate(b)),$ &
	$y_{i, 1} := (\clauseXFactor i+1, 4(n+1) + 4),$ \\
	$z_{i, 0} := (\clauseXFactor i+2, 4\cdot \idx(c) + 2\cdot \negate(c)),$ &
	$z_{i, 1} := (\clauseXFactor i+2, 4(n+1) + 6).$
\end{tabular}
\end{center}

\newcommand{\segmentsClause}{\mathsf{segmentsClause}}	
 
We are now ready to define the set of points in a clause gadget:
 
 $$\mathsf{moveVariable}_i := 
 \{x_{i, j} : j \in \{0, 1\}\} \cup
 \{y_{i, j} : j \in \{0, 1\}\} \cup
 \{z_{i, j} : j \in \{0, 1\}\},
 $$
 $$\pointsClause{i} := 
 \mathsf{moveVariable}_i \cup \pointsOr{i}{0}
 \cup \pointsOr{i}{1} \cup \{v_{i, 1} \}.
 $$
 
Note that these two points are equal: $v_{i,0} = l_{i,1}$.
This translates to the fact that the output of the first OR-gadget
is an input to the second OR-gadget.
This creates an $or$ of 3 boolean values.

\paragraph{Segments.}
We also define segments for the clause gadget as below:
\begin{eqnarray*}
\segmentsClause_i & := & \{ (x_{i, 0}, x_{i, 1}),
(y_{i, 0}, y_{i, 1}),
(z_{i, 0}, z_{i, 1}),
(x_{i, 1}, l_{i, 0}),
(y_{i, 1}, p_{i, 0}),
(z_{i, 1}, p_{i, 1}),
\} \cup \\
& & \cup \ \segmentsOr{i}{0} \cup \segmentsOr{i}{1}.
\end{eqnarray*}

\newcommand{\segmentsClauseSolTrue}[1]{\mathsf{solClause}^{\true,#1}}
\newcommand{\segmentsClauseSolFalse}{\mathsf{solClause}^{\false}}

The CLAUSE-gadgets consist of two OR-gadgets.
Ideally, we would place the $i$-th CLAUSE-gadget close to the
$\xTrueSegment{j_1}$ or $\xFalseSegment{j_1}$ segments
corresponding to the literals that occur in the $i$-th clause.
It would be inconvenient to position them there,
because between these segments there may be additional
$\xTrueSegment{j_2}$ or $\xFalseSegment{j_2}$
segments corresponding to the other literals.

Instead, we use simple auxiliary gadgets to
\textit{transfer} whether the segment
is in a solution, i.e. segments
$(x_{i, 0}, x_{i, 1}), (y_{i, 0}, y_{i, 1}), (z_{i, 0}, z_{i, 1})$.
Each transfer gadget consists of two segments $(x_{i, 0}, x_{i, 1}), (x_{i, 1}, a)$.
These are the only segments that can cover $x_{i,1}$.
We place $x_{i,0}$ on a segment that we want to transfer (i.e.
segment responsible for choosing the variable value satisfying the
corresponding literal).
If in some solution $x_{i,0}$ is already covered by this segment, then
we can cover $x_{i,1}$ by $(x_{i,1}, a)$, thus also covering $a$.
If $x_{i,0}$ is not covered by this segment,
then the only way to cover $x_{i,0}$ is to use segment $(x_{i, 0}, x_{i, 1})$.
Intuitively,
in any optimal solution the two segments \textit{transfer} the state of whether $x_{i,0}$
is covered onto whether $a$ is covered.
Therefore, the number of segments in the optimal solution is increased by one,
and we get a point $a$ that was effectively placed
on some segment $s$, but it can be placed anywhere in the plane instead,
consequently simplifying the construction.

\begin{lemma}
\label{cover_clauses_solution_true}
For any $1 \le i \le n$ and $a \in \{ x_{i, 0}, y_{i, 0}, z_{i, 0}\}$,
there is a set $\segmentsClauseSolTrue{a}_i \subseteq \segmentsClause_i$
with $|\segmentsClauseSolTrue{a}_i| = 11$
that covers all points in $\pointsClause{i} - \{a\}$.
\end{lemma}

\begin{proof}
For $a = x_{i, 0}$ (analogous proof for $y_{i, 0}$):
First we use Lemma~\ref{cover_or_true} twice with excluded $x = l_{i, 0}$ and
$x = l_{i, 1} = v_{i, 0}$,
resulting with 8 segments in $\chooseOr{\true}{i}{0} \cup \chooseOr{\true}{i}{1}$
which cover all required points apart from
$x_{i, 1}, y_{i, 0}, y_{i, 1}, z_{i, 0}, z_{i, 1}, l_{i, 0}$.
We cover those using additional 3 segments:
$\{ (x_{i, 1}, l_{i, 0}), (y_{i, 0}, y_{i, 1}),
(z_{i, 0}, z_{i, 1}) \}$.

For $a = z_{0, i}$:
Using Lemma~\ref{cover_or_false} and Lemma~\ref{cover_or_true} with
$x = p_{i, 1}$,
we obtain 8 segments in $\chooseOr{\false}{i}{0} \cup \chooseOr{\true}{i}{1}$
which cover all required points apart from
$x_{i, 0}, x_{i, 1}, y_{i, 0}, y_{i, 1}, z_{i, 1}, p_{i, 1}$.
We cover those using additional 3 segments:
$\{ (x_{i, 0}, x_{i, 1}), (y_{i, 0}, y_{i, 1}),
(z_{i, 1}, p_{i, 1}) \}$.
\end{proof}

\begin{lemma}
\label{cover_clauses_solution_false}
For any $1 \le i \le n$ there is
a set $\segmentsClauseSolFalse_i \subseteq \segmentsClause_i$
with $|\segmentsClauseSolFalse_i| = 12$
that covers all points in $\pointsClause{i}$.
\end{lemma}

\begin{proof}
Using Lemma \ref{cover_or_false} twice we can
cover $\pointsOr{i}{0}$ and  $\pointsOr{i}{1}$
with 8 segments.
To cover the remaining points we additionally use:
$\{ (x_{i, 0}, x_{i, 1}), (y_{i, 0}, y_{i, 1}),
(z_{i, 0}, z_{i, 1}), (t_{i, 1}, v_{i, 1}) \}.$
\end{proof}

\begin{lemma}
\label{cover_clauses_segments_no_less}
For any $1 \le i \le n$:
\begin{enumerate}[label={(\arabic*)}]
	\item points in $\pointsClause{i}$ can not be covered 
	using any subset of segments
	from $\segmentsClause_i$ of size smaller than 12;
	\item points in $\pointsClause{i} - \{ x_{i, 0}, y_{i, 0}, z_{i, 0}\}$
	can not be covered using any subset of segments
	from $\segmentsClause_i$ of size smaller than 11.
\end{enumerate}
\end{lemma}


\begin{proof}[Proof of (1).]
No segment in $\segmentsClause_i$ covers more than 1 point from
$$\{ x_{i, 0}, y_{i, 0}, z_{i, 0}, l_{i, 0}, p_{i, 0}, q_{i, 0},
u_{i, 0}, v_{i, 0} = l_{i, 1}, p_{i, 1}, q_{i, 1}, u_{i, 1}, v_{i, 1} \}.$$
Therefore we need to use at least 12 segments.
\end{proof}

\begin{proof}[Proof of (2).]

We can define disjoint sets $X, Y, Z$ such that
$$X \cup Y \cup Z \subseteq \pointsClause{i} - \{x_{i, 0}, y_{i, 0}, z_{i, 0}\}$$
and there are no segments in $\segmentsClause_i$ covering points from different sets.
And we prove a lower bound for each of these sets.
First, let:

$$X := \{x_{i, 1}, y_{i, 1}, z_{i, 1}\}.$$

No two points in $X$ can be covered with one segment
of $\segmentsClause_i$, so it must be covered with 3 different segments.
Next we define other sets:

$$Y := \pointsOr{i}{0} - \{l_{i, 0}, p_{i, 0}\},$$
$$Z := \pointsOr{i}{1} - \{l_{i, 1}, p_{i, 1}\}.$$


For both $Y$ and $Z$ we can check all of the subsets of 3 segments
of $\segmentsClause_i$
to conclude that none of them cover the considered,
so both $Y$ and $Z$ have to be covered with 
disjoint sets of 4 segments each.

Therefore, $\pointsClause{i} - \{x_{i, 0}, y_{i, 0}, z_{i, 0}\}$
must be covered with at least 3 + 4 + 4 = 11 segments from $\segmentsClause_i$.
\end{proof}

\subsection{Summary}

{

{\tikzset{node/.style={
    prefix after command= {\pgfextra{\tikzset{every
    label/.style={font=\footnotesize}}}}
    }
}
\begin{figure}
\centering
\begin{tikzpicture}
\tikzmath{
\x0=0;
\y0=0;
\mepsx0=-0.2;
\yalabel=1.5;
\xalabel=12.5;
\xblabel=2.5;
\xclabel=5.0;
\yblabel=0.75;
\mepsy0=-0.2;
\epsxclabel=5.2;
\epsyalabel=1.7;
\yclabel=1.5;
\ydlabel=2.5;
\yelabel=1.5;
\yflabel=2.5;
\yglabel=4.0;
\xdlabel=12.5;
\xelabel=2.5;
\xflabel=5.0;
\yhlabel=3.25;
\mepsydlabel=2.3;
\epsxflabel=5.2;
\epsyglabel=4.2;
\yilabel=1.5;
\yajlabel=2.5;
\yaalabel=4.0;
\yablabel=5.0;
\yaclabel=1.5;
\yadlabel=2.5;
\yaelabel=4.0;
\yaflabel=5.0;
\yaglabel=6.5;
\xglabel=12.5;
\xhlabel=2.5;
\xilabel=5.0;
\yahlabel=5.75;
\mepsyablabel=4.8;
\epsxilabel=5.2;
\epsyaglabel=6.7;
\xajlabel=7.5;
\xaalabel=7.5;
\xablabel=9.5;
\yailabel=1.5;
\ybjlabel=2.5;
\ybalabel=4.0;
\ybblabel=5.0;
\ybclabel=6.5;
\ybdlabel=8.5;
\ybelabel=1.5;
\ybflabel=2.5;
\ybglabel=4.0;
\ybhlabel=5.0;
\ybilabel=6.5;
\xaclabel=7.7;
\ycjlabel=7.166666666666667;
\xadlabel=7.9;
\ycalabel=7.833333333333333;
\xaelabel=8.1;
\ycblabel=8.166666666666666;
\xaflabel=8.3;
\xaglabel=8.9;
\mepsxajlabel=7.3;
\epsxablabel=9.7;
\epsybdlabel=8.7;
\ycclabel=1.5;
\ycdlabel=2.5;
\ycelabel=1.5;
\ycflabel=2.5;
\ycglabel=4.0;
\ychlabel=5.0;
\ycilabel=1.5;
\ydjlabel=7.833333333333334;
\ydalabel=7.333333333333334;
\ydblabel=7.5;
\ydclabel=7.666666666666667;
\xahlabel=8.5;
\xailabel=8.7;
\xbjlabel=8.899999999999999;
\yddlabel=7.499999999999999;
\ydelabel=8.166666666666666;
\ydflabel=7.666666666666666;
\ydglabel=7.833333333333332;
\ydhlabel=7.999999999999999;
\xbalabel=9.1;
\xbblabel=9.299999999999999;
\xbclabel=9.499999999999998;
\xbdlabel=7.5;
\xbelabel=9.5;
\xbflabel=10.5;
\xbglabel=7.5;
\xbhlabel=9.5;
\xbilabel=10.5;
\xcjlabel=12.5;
\ydilabel=1.5;
\yejlabel=2.5;
\yealabel=4.0;
\yeblabel=5.0;
\yeclabel=6.5;
\yedlabel=8.5;
\yeelabel=1.5;
\yeflabel=2.5;
\yeglabel=4.0;
\yehlabel=5.0;
\yeilabel=6.5;
\xcalabel=10.7;
\yfjlabel=7.166666666666667;
\xcblabel=10.9;
\yfalabel=7.833333333333333;
\xcclabel=11.1;
\yfblabel=8.166666666666666;
\xcdlabel=11.3;
\xcelabel=11.9;
\mepsxbflabel=10.3;
\epsxcjlabel=12.7;
\epsyedlabel=8.7;
\yfclabel=1.5;
\yfdlabel=2.5;
\yfelabel=4.0;
\yfflabel=5.0;
\yfglabel=6.5;
\yfhlabel=1.5;
\yfilabel=2.5;
\ygjlabel=7.833333333333334;
\ygalabel=7.333333333333334;
\ygblabel=7.5;
\ygclabel=7.666666666666667;
\xcflabel=11.5;
\xcglabel=11.7;
\xchlabel=11.899999999999999;
\ygdlabel=7.499999999999999;
\ygelabel=8.166666666666666;
\ygflabel=7.666666666666666;
\ygglabel=7.833333333333332;
\yghlabel=7.999999999999999;
\xcilabel=12.1;
\xdjlabel=12.299999999999999;
\xdalabel=12.499999999999998;
}

\filldraw [fill=cyan!30, draw=black] (\mepsxajlabel,\mepsy0) rectangle (\epsxablabel, \epsybdlabel);
\draw (\xaclabel,\ycdlabel) -- (\xaclabel,\ycjlabel);
\draw (\xaclabel,\ycjlabel) -- (\xaflabel,\ycjlabel);
\draw (\xadlabel,\ychlabel) -- (\xadlabel,\ycalabel);
\draw (\xadlabel,\ycalabel) -- (\xaflabel,\ycalabel);
\draw (\xaelabel,\ycilabel) -- (\xaelabel,\ycblabel);
\draw (\xaelabel,\ycblabel) -- (\xaglabel,\ycblabel);
\node[above right] at (\mepsxajlabel,\epsybdlabel) {CLAUSE-gadget$_1$};
\draw (\xaflabel,\ycjlabel) -- (\xaflabel,\ydjlabel);
\draw (\xaflabel,\ydalabel) -- (\xailabel,\ydalabel);
\draw (\xaflabel,\ydclabel) -- (\xailabel,\ydclabel);
\draw (\xahlabel,\ydalabel) -- (\xahlabel,\ydclabel);
\draw (\xailabel,\ydalabel) -- (\xailabel,\ydclabel);
\draw (\xailabel,\ydblabel) -- (\xbjlabel,\ydblabel);
\draw (\xaglabel,\yddlabel) -- (\xaglabel,\ydelabel);
\draw (\xaglabel,\ydflabel) -- (\xbblabel,\ydflabel);
\draw (\xaglabel,\ydhlabel) -- (\xbblabel,\ydhlabel);
\draw (\xbalabel,\ydflabel) -- (\xbalabel,\ydhlabel);
\draw (\xbblabel,\ydflabel) -- (\xbblabel,\ydhlabel);
\draw (\xbblabel,\ydglabel) -- (\xbclabel,\ydglabel);

\filldraw [fill=cyan!30, draw=black] (\mepsxbflabel,\mepsy0) rectangle (\epsxcjlabel, \epsyedlabel);
\draw (\xcalabel,\yfglabel) -- (\xcalabel,\yfjlabel);
\draw (\xcalabel,\yfjlabel) -- (\xcdlabel,\yfjlabel);
\draw (\xcblabel,\y0) -- (\xcblabel,\yfalabel);
\draw (\xcblabel,\yfalabel) -- (\xcdlabel,\yfalabel);
\draw (\xcclabel,\yfilabel) -- (\xcclabel,\yfblabel);
\draw (\xcclabel,\yfblabel) -- (\xcelabel,\yfblabel);
\node[above right] at (\mepsxbflabel,\epsyedlabel) {CLAUSE-gadget$_2$};
\draw (\xcdlabel,\yfjlabel) -- (\xcdlabel,\ygjlabel);
\draw (\xcdlabel,\ygalabel) -- (\xcglabel,\ygalabel);
\draw (\xcdlabel,\ygclabel) -- (\xcglabel,\ygclabel);
\draw (\xcflabel,\ygalabel) -- (\xcflabel,\ygclabel);
\draw (\xcglabel,\ygalabel) -- (\xcglabel,\ygclabel);
\draw (\xcglabel,\ygblabel) -- (\xchlabel,\ygblabel);
\draw (\xcelabel,\ygdlabel) -- (\xcelabel,\ygelabel);
\draw (\xcelabel,\ygflabel) -- (\xdjlabel,\ygflabel);
\draw (\xcelabel,\yghlabel) -- (\xdjlabel,\yghlabel);
\draw (\xcilabel,\ygflabel) -- (\xcilabel,\yghlabel);
\draw (\xdjlabel,\ygflabel) -- (\xdjlabel,\yghlabel);
\draw (\xdjlabel,\ygglabel) -- (\xdalabel,\ygglabel);
\draw (\xclabel,\y0) -- (\xalabel,\y0) node[pos=0.17, above] {$x_1 = \true$};
\draw (\xclabel,\yalabel) -- (\xalabel,\yalabel) node[pos=0.17, above] {$x_1 = \false$};
\filldraw [fill=lime!30, draw=black] (\mepsx0,\mepsy0) rectangle (\epsxclabel, \epsyalabel);
\draw (\x0,\y0) -- (\xalabel,\y0);
\draw (\x0,\y0) -- (\x0,\yblabel);
\draw (\xblabel,\y0) -- (\xblabel,\yalabel);
\draw (\xblabel,\yalabel) -- (\xalabel,\yalabel);
\draw (\x0,\yblabel) -- (\xblabel,\yblabel);
\node[above right] at (\mepsx0,\epsyalabel) {VARIABLE-gadget$_1$};
\draw (\xflabel,\ydlabel) -- (\xdlabel,\ydlabel) node[pos=0.17, above] {$x_2 = \true$};
\draw (\xflabel,\yglabel) -- (\xdlabel,\yglabel) node[pos=0.17, above] {$x_2 = \false$};
\filldraw [fill=lime!30, draw=black] (\mepsx0,\mepsydlabel) rectangle (\epsxflabel, \epsyglabel);
\draw (\x0,\ydlabel) -- (\xdlabel,\ydlabel);
\draw (\x0,\ydlabel) -- (\x0,\yhlabel);
\draw (\xelabel,\ydlabel) -- (\xelabel,\yglabel);
\draw (\xelabel,\yglabel) -- (\xdlabel,\yglabel);
\draw (\x0,\yhlabel) -- (\xelabel,\yhlabel);
\node[above right] at (\mepsx0,\epsyglabel) {VARIABLE-gadget$_2$};
\draw (\xilabel,\yablabel) -- (\xglabel,\yablabel) node[pos=0.17, above] {$x_3 = \true$};
\draw (\xilabel,\yaglabel) -- (\xglabel,\yaglabel) node[pos=0.17, above] {$x_3 = \false$};
\filldraw [fill=lime!30, draw=black] (\mepsx0,\mepsyablabel) rectangle (\epsxilabel, \epsyaglabel);
\draw (\x0,\yablabel) -- (\xglabel,\yablabel);
\draw (\x0,\yablabel) -- (\x0,\yahlabel);
\draw (\xhlabel,\yablabel) -- (\xhlabel,\yaglabel);
\draw (\xhlabel,\yaglabel) -- (\xglabel,\yaglabel);
\draw (\x0,\yahlabel) -- (\xhlabel,\yahlabel);
\node[above right] at (\mepsx0,\epsyaglabel) {VARIABLE-gadget$_3$};
\end{tikzpicture}
\caption{\textbf{Scheme of the whole construction.}}
General layout of VARIABLE-gadgets and CLAUSE-gadgets and how they
interact with each other.
\label{fig:segment_apx_whole}
\end{figure}



Finally we define the set of points and segments for the constructed instance:
$$\points := \bigcup_{1 \le i \le n} \pointsVar{i} \cup \pointsClause{i},$$
$$\sets := \bigcup_{1 \le i \le n} \segmentsVar{i} \cup \segmentsClause_i.$$


TODO: Add some smart lemmas that sets will be exclusive to each other.

\begin{lemma}[Robustness to $\frac{1}{2}$-extension]
\label{lemma:exntension_robust}
For every segment $s \in \sets$,
$s$ and $s^{+\frac{1}{2}}$ cover the same points from $\points$.
\end{lemma}

TODO: We can rewrite the proof below with rectangles
that should take care of the smart lemmas above and make the proof easier.

\begin{proof}
We can just check every segment. Most of the segments $s$
are collinear only with points that lie on $s$,
so trivially $s^{+\frac{1}{2}}$ cannot cover more points than $s$ does.
We discuss only the segments that are collinear with
a point from outside of its gadget
or with a point that belongs to another copy of the gadget.

Within VARIABLE-gadget for any $1 \le i \le n$ after $\frac{1}{2}$-extension:
$(c_i,g_i)$ does not cover $b_i$.

Within OR-gadget some of the segments are collinear and share one point;
specifically, for any $1 \le i \le n$ and $j \in \{0,1\}$, after $\frac{1}{2}$-extension:
\begin{itemize}
\item $(l_{i,j}, n_{i,j})$ does not cover $o_{i,j}$,
\item $(n_{i,j}, p_{i,j})$ does not cover $m_{i,j}$,
\item $(t_{i,j}, v_{i,j})$ does not cover $n_{i,j}$.
\end{itemize}
Within CLAUSE-gadget, for any $1 \le i \le n$ after $\frac{1}{2}$-extension:
\begin{itemize}
\item $(o_{i,0}, u_{i,0})$ does not cover $m_{i,1}$,
\item $(m_{i,1}, s_{i,1})$ does not cover $u_{i,0}$,
\item $(y_{i,1}, p_{i,0})$ does not cover $n_{i,1}$.
\end{itemize}
For two consecutive VARIABLE-gadgets, for any $1 \le i < n$ after $\frac{1}{2}$-extension:
$(b_i, f_i)$ does not cover $b_{i+1}$ (nor $f_{i-1}$ for $i>1$).
Similarly $(a_i,d_i)$ does not cover $a_{i+1}$ (nor $d_{i-1}$ for $i>1$),
because this segment is shorter than the previous one and $a_i$ and $b_i$
share y-coordinate.

For two consecutive CLAUSE-gadgets,
segments from one do not cover anything from the other,
as the gadgets have width 9 and
every leftmost x-coordinate is divisible by $\clauseXFactor$.
Hence two different gadgets do not interact with each other
after $\frac{1}{2}$-extension.

Next we need to check whether VARIABLE-gadget's segments
do not cover any points $x_{i,0}, y_{i,0}$ or $z_{i,0}$ from CLAUSE-gadget.
For any $1 \le i \le n$ and $1 \le j \le n$, all points $x_{j,0}, y_{j,0}$ and $z_{j,0}$
have x-coordinate strictly positive. Segment $(a_i, c_i)$ have length $2L$
and $c_i$ has x-coordinate equal to $-L$, so after $\frac{1}{2}$-extension
this segment does not cover any points with a positive x-coordinate.
\end{proof}

\section{Proof that the construction is correct and sound}

In order to prove Lemma \ref{apxconstruction} we introduce several
auxiliary lemmas proving properties of the construction
described in the previous section.

Consider an instance $S$ of MAX-(3,3)-SAT of size $n$
with optimum solution satisfying $k$ clauses.
Let us construct an instance $\setCoverInstance$ of geometric set cover
as described in Section~\ref{construction_description}
for the instance $S$ of MAX-(3,3)-SAT.

\begin{lemma}
	\label{construction_correctness}
	The instance $\setCoverInstance$ of geometric set cover
	admits a solution of size $15n - k$.
\end{lemma}

\begin{proof}
Let the clauses in $S$ be $c_1,c_2,\ldots,c_n$
and the variables be $x_1,x_2,\ldots,x_n$.
Let the variable assignment in
the optimum solution to $S$ be
$\phi : \{ x_1, x_2, \ldots, x_n\} \rightarrow \{\true, \false\}$.


We cover every VARIABLE-gadget with solution described in
Lemma~\ref{choose_variables_solution}, where
in the $i$-th gadget we choose the set of segments corresponding to the
value of $\phi(x_i)$.

For every clause that is satisfied, say $c_i$, 
let us name the variable that is $\true$ in it as $x_i$
and the point corresponding to $x_i$ in $\pointsClause{i}$ as $a$.
Points in $\pointsClause{i}$ 
are covered with set $\segmentsClauseSolTrue{a}_i$ described in
Lemma~\ref{cover_clauses_solution_true}.
For every clause that is not satisfied, say $c_j$,
points in $\pointsClause{j}$ are covered
with set $\segmentsClauseSolFalse_i$ described in
Lemma~\ref{cover_clauses_solution_false}.

Formally, we define 
sets responsible for choosing variable assignment and satisfying clauses,
$R_i$ and $C_i$ respectively, as following:

\begin{align}
	\begin{split}
	& R_i := \begin{cases}
		\chooseVar{\true}{i} & \text{if}\ \phi(x_i) = \true \\
		\chooseVar{\false}{i} & \text{if}\ \phi(x_i) = \false \\
		\end{cases} \\
	& C_i := \begin{cases}
		\segmentsClauseSolTrue{a}_i & \text{if}\ c_i \text{ satisfied by the literal corresponding to point } a \\
		\segmentsClauseSolFalse_i & \text{if}\ c_i \text{ not satisfied}
		\end{cases} \\
	& \sol := \bigcup\limits_{i=1}^{n} \{R_i \cup C_i : 1 \le i \le n\}.
    \end{split}
\end{align}


This set covers all the points from $\points$, because
the sets $R_i$, $C_i$ individually cover their corresponding gadgets,
as proved in the respective lemmas.

All of these sets are disjoint, so the size of the obtained solution is:

$$|\sol| = \sum_{i=1}^{n} R_i + \sum_{i=1}^n C_i = 3n + 11k + 12(n-k) = 15n - k.\qedhere$$
\end{proof}

\begin{lemma}
	\label{at_most_one_var_segment}
	Suppose we have a solution $\sol$ of the instance $\setCoverInstance$
	of geometric set cover.
	Then there exists a solution $\sol'$ such that $|\sol'| \le |\sol|$
	and $\sol'$ contains
	at most one of the segments $\xTrueSegment{i}$ and $\xFalseSegment{i}$
	from each VARIABLE-gadget.
\end{lemma}
\begin{proof}\leavevmode
Assume that we have $\{\xTrueSegment{i}, \xFalseSegment{i}\} \subseteq \sol$ for some $i$.
We will show how to modify $\sol$ into $\sol'$,
such that the number of such $i$ decreases,
while $\sol'$ is still a valid solution to $\setCoverInstance$,
and $|\sol'| \le |\sol|$. Then, by repeating this procedure,
we can eventually construct a solution satisfying the property from the Lemma.

To construct $\sol'$, 
we first remove from $\sol$ all segments belonging to $\segmentsVar{i}$.
Recall that the $i$-th VARIABLE-gadget corresponds to variable $x_i$ in $S$.
As every variable in $S$ is used in exactly 3 clauses,
then one literal $x_i$ or $\neg x_i$ must appear in at least
2 clauses.
If that literal is $x_i$, then we add to the constructed solution all segments from $\chooseVar{\true}{i}$,
otherwise we add all segments from $\chooseVar{\false}{i}$.

Now, there exists at most one CLAUSE-gadget which needs adjustment to make $\sol'$ valid;
assuming it is the $j$-th clause, then one of the points $x_{j,0}, y_{j,0}$ or $z_{j,0}$ for this
CLAUSE-gadget might be not covered, say $y_{j,0}$.
We amend the solution by adding $(y_{j,0}, y_{j,1})$ to $\sol'$.

By Lemma \ref{choose_variables_both} we know 
that $\sol$ used at least 4 segments from $\segmentsVar{i}$.
Therefore, we removed at least 4 segments and added at most 4 segments,
so $|\sol'| \le |\sol|$.
\end{proof}

\begin{lemma}
	\label{construction_completness}
	Suppose we have a solution $\sol$ of the instance $\setCoverInstance$
	of geometric set cover.
	Then there exists a solution to $S$
	that satisfies at least $15n - |\sol|$ clauses.
\end{lemma}


\begin{proof}\leavevmode
Let the clauses in $S$ be $c_1,c_2,\ldots,c_n$
and the variables be $x_1,x_2,\ldots,x_n$.
Given a solution $\sol$
of the instance $\setCoverInstance$ of geometric set cover,
we use Lemma~\ref{at_most_one_var_segment} to modify $\sol$
so that for any $i$, $\sol$ contains at most
one of $\xTrueSegment{i}$ and $\xFalseSegment{i}$;
this may decrease the size of $\sol$,
but that does not matter in the subsequent construction.
To simplify notation,
in the remainder of this proof we use $\sol$ to refer to the modified solution.

Given $\sol$, we construct a solution to $S$ by defining an
assignment of variables:
$$\phi : \{ x_1, x_2, \ldots, x_n\} \rightarrow \{\true, \false\}$$
that satisfies at least $15n-|\sol|$ clauses in $S$.

\subparagraph{Definition of $\phi$.}
Recall that due to Lemma~\ref{at_most_one_var_segment},
$\sol$ contains at most one of $\xTrueSegment{i}$ and $\xFalseSegment{i}$.

We define the value $\phi(x_i)$ for the variable $x_i$ as follows:
\begin{align}
	\begin{split}
	\label{eqn:variable_assignment}
	\phi(x_i) := & \begin{cases}
	\true & \text{if}\ \xTrueSegment{i} \in \sol, \\
	\false & \text{otherwise}
	\end{cases}
	\end{split}
\end{align}
Moreover, from Lemma~\ref{choose_variables_no_less} we get $|\segmentsVar{i} \cap \sol| \ge 3$ for every $i$.

\subparagraph{Clauses satisfied with the chosen variable assignment.}

For a clause $c_i$,
$\sol$ needs to use at least 11 segments to cover $\pointsClause{i} - \{x_{i,0}, y_{i,0}, z_{i,0}\}$
in the $i$-th CLAUSE-gadget (Lemma~\ref{cover_clauses_segments_no_less}).

Moreover, if none of the points $\{x_{i,0}, y_{i,0}, z_{i,0}\}$
are covered by the segments from $\sol \cap \segmentsVar{i}$,
then $\sol$ needs to cover $\pointsClause{i}$
with at least 12 segments
by Lemma~\ref{cover_clauses_segments_no_less}.

Let $a$ be the number of clauses $c_i$ for which none of
the points $x_{i,0}, y_{i,0}, z_{i,0}$ in $\pointsClause{i}$ are covered by
segments from $\sol~\cap~\segmentsVar{j}$ for any $1 \le j \le n$.

Consider a clause $c_i$ for which at least one of the points
$x_{i,0}, y_{i,0}, z_{i,0}$ in $\pointsClause{i}$ is covered by
segments from $\sol~\cap~\segmentsVar{j}$ for some $1 \le j \le n$.
Denote this point as $t$ and say it corresponds to
literal $q$ and variable $x_j$.
Point $t$ can be only covered in $\segmentsVar{j}$
by a corresponding segment $\xTrueSegment{j}$ or $\xFalseSegment{j}$
(depending on whether the literal $q$ is negated or not).
From the definition of $\phi$ and the fact that one of this segment is
in $\sol$, we know that
$\phi(j)$ has the value that evaluates $q$ to be $\true$.
Therefore, clause $c_i$ is satisfied.

Consequently, $\phi$ satisfies all but at most $a$ clauses in $S$.

To conclude,
given a solution $\sol$ to $\setCoverInstance$ we constructed
a variable assignment $\phi$
that satisfies at least $n-a$ clauses of $S$.
Finally, note that

$$|\sol| \ge 3n + 11(n-a) + 12a = 3n + 11n + a = 14n + a,$$
hence
$$15n - |\sol|  \le 15n - 14n - a = n - a.$$
Therefore, $\phi$ satisfies at least $15n-|\sol|$ clauses of $S$.
\end{proof}
