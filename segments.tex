\chapter{Geometric Set Cover with segments}

\section{FPT for segments}
\subsection{Segments parallel to one of the axis}
You can find this in Platypus book.

We'll show $\mathcal{O}(2^k)$ branching algorithm.
Let's take point $K$ that hasn't been covered yet with
the smallest coordinate in lexicograpical order.
We need to cover $K$ with some of the remaining segments.

We choose one of the 2 directions on which we cover this point.
In this direction we take greedly the segment that will cover
the most points (there are points in $\points$ only on
one side of $K$ in this direction, so all
segments covering $K$ in this direction create monotone sequence
of sets -- zbiory zstępujące).

\subsection{Segments in $d$ directions}
The same algorithm as before but in complexity $\mathcal{O}(d^k)$.

\subsection{Segments in arbitrary direction}
\begin{tw}{
	\label{segment_cover_fpt}
	\textbf{(FPT for segment cover)}.
	There exists an algorithm that given a family $\sets$ of
	$n$ segments (in any direction),
	a set of $m$ points $\points$
	and a parameter $k$,
	runs in time $f(k) \cdot (nm)^c$ for some computable function f and constant c,
	and outputs a subfamily $\sol \subseteq \sets$
	such that $|\sol| \le k$ and $\sol$ covers all points in $\points$
	or determins that the solution of size at most $k$ doesn't exist.
}\end{tw}

This theorem is proved by following lemmas.

\begin{lemma}
   \label{fpt_reduction}
   \textbf{(Reduction)}.
   Given a family $\sets$ of
	$n$ segments (in any direction),
	a set of $m$ points $\points$
	for segment cover problem,
   without a loss of generality we can assume that
   no two segments cover the same set of points.
\end{lemma}   
   
\paragraph{Proof.} Trivial.

\begin{lemma}
	\label{kernel_definition}
	\textbf{(Kernel for segment cover)}.
	For a problem of segment cover that given a family $\sets$ of
	$n$ segments (in any direction),
	a set of $m$ points $\points$
	and a parameter $k$, there exists a family of
	kernels $K \subset \sets$,
	where \textit{any line} contains no more than $k$ points
	and there exists optimal solution, that is subset of $K$.
	We can find all such kernels in complexity $O(k^k)$ using branching technique
	and there are $O(k^k)$ of them.
\end{lemma}

\paragraph{Proof.}

%As long as there is a line with more than $k$ points, do branching.
%Let's name points on this line $x_1, x_2, \ldots x_t$
%in order they appear on the line.

%So we choose on which point the chosen segment on this line
%will start. Of course we have to take at least one segment
%covering at least one point among first $k+1$ points,
%because covering all of them with only segments
%on different lines we would use
%exactly $k+1$ segments (any of them can't contain more than
%one point from this line).

First we use Lemma \ref{fpt_reduction}.

Assume there exists a line $l$ containing points $x_1$, \ldots $x_t$,
where $t \geq k + 1$.
Note that a segment that does not lie on $l$ can cover only at most one
of the points $x_i$.
Therefore, out of points $x_1$, ..., $x_{k+1}$,
at least one has to be covered by a segment that lies on $l$,
let us fix $x_i$ to be the first such point (We branch over $i \in \{1 \dots k+1\}$.
Then, we can greedily choose a segment that lies on $l$,
covers $x_i$,
and also covers the largest number of points $x_j$ for $j > i$.

Since we have at most $k + 1$ choices to branch over
and each choice adds a segment to the constructed solution,
we obtain an algorithm with complexity $O(k^k)$.

\paragraph{Proof of theorem \ref{segment_cover_fpt}.} Assuming Lemma \ref{kernel_definition}.

First we use Lemma \ref{fpt_reduction}.

Since any segment covers a set of colinear points,
for such a kernel $k$ segments can cover only at most $k^2$ points.
Therefore, for the answer to be positive,
the number of points has to be at most $k^2$.
The number of segments is now bounded by $k^4$,
since if we consider two \textit{extreme} points
covered by a given segment,
then these pairs must be distinct,
otherwise two segments would contain the same set of points.
Since both the number of points and the number of segments
is bounded by a function of $k$,
this instance can be easily solved in time $O(f(k))$.
Since there are $O(k^k)$ possible kernels, the final algotihm
work in $f(k) \cdot (nm)$.
