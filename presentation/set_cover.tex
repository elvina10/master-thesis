%%%%%%%%%%%%%%%%%%%%%%%%%%%%%%%%%%%%%%%%%
% Beamer Presentation
% LaTeX Template
% Version 1.0 (10/11/12)
%
% This template has been downloaded from:
% http://www.LaTeXTemplates.com
%
% License:
% CC BY-NC-SA 3.0 (http://creativecommons.org/licenses/by-nc-sa/3.0/)
%
%%%%%%%%%%%%%%%%%%%%%%%%%%%%%%%%%%%%%%%%%

%----------------------------------------------------------------------------------------
%	PACKAGES AND THEMES
%----------------------------------------------------------------------------------------

\documentclass{beamer}
\usepackage{eucal}
\usepackage{lmodern}
\usepackage{color, colortbl, linegoal}
\usepackage{setspace}
\usepackage{svg}
\usepackage{pgfplots}

\mode<presentation> {

% The Beamer class comes with a number of default slide themes
% which change the colors and layouts of slides. Below this is a list
% of all the themes, uncomment each in turn to see what they look like.

%\usetheme{default}
%\usetheme{AnnArbor}
%\usetheme{Antibes}
%\usetheme{Bergen}
%\usetheme{Berkeley}
%\usetheme{Berlin}
%\usetheme{Boadilla}
%\usetheme{CambridgeUS}
%\usetheme{Copenhagen}
%\usetheme{Darmstadt}
%\usetheme{Dresden}
%\usetheme{Frankfurt}
%\usetheme{Goettingen}
%\usetheme{Hannover}
%\usetheme{Ilmenau}
%\usetheme{JuanLesPins}
%\usetheme{Luebeck}
\usetheme{Madrid}
%\usetheme{Malmoe}
%\usetheme{Marburg}
%\usetheme{Montpellier}
%\usetheme{PaloAlto}
%\usetheme{Pittsburgh}
%\usetheme{Rochester}
%\usetheme{Singapore}
%\usetheme{Szeged}
%\usetheme{Warsaw}

% As well as themes, the Beamer class has a number of color themes
% for any slide theme. Uncomment each of these in turn to see how it
% changes the colors of your current slide theme.

%\usecolortheme{albatross}
%\usecolortheme{beaver}
%\usecolortheme{beetle}
%\usecolortheme{crane}
%\usecolortheme{dolphin}
%\usecolortheme{dove}
%\usecolortheme{fly}
%\usecolortheme{lily}
%\usecolortheme{orchid}
%\usecolortheme{rose}
%\usecolortheme{seagull}
%\usecolortheme{seahorse}
%\usecolortheme{whale}
%\usecolortheme{wolverine}

%\setbeamertemplate{footline} % To remove the footer line in all slides uncomment this line
%\setbeamertemplate{footline}[page number] % To replace the footer line in all slides with a simple slide count uncomment this line

%\setbeamertemplate{navigation symbols}{} % To remove the navigation symbols from the bottom of all slides uncomment this line
}

\usepackage{graphicx} % Allows including images
\usepackage{booktabs} % Allows the use of \toprule, \midrule and \bottomrule in tables


\usepackage{hyperref}
\hypersetup{
    colorlinks=true,
    linkcolor=blue,
    filecolor=magenta,      
    urlcolor=cyan,
}

\urlstyle{same}

\usepackage{polski}
\usepackage[utf8]{inputenc}

\usepackage{amsmath}
\DeclareMathOperator*{\argmax}{arg\,max}
\DeclareMathOperator*{\argmin}{arg\,min}

%----------------------------------------------------------------------------------------
%	TITLE PAGE
%----------------------------------------------------------------------------------------

\title[Segment Set Cover]{Approximation and parametrization\\
of Segment Set Cover}
% The short title appears at the bottom of every slide, the full title is only on the title page

\author{Katarzyna Kowalska, Michał Pilipczuk} % Your name
\institute[UW] % Your institution as it will appear on the bottom of every slide, may be shorthand to save space
{
University of Warsaw, MIMUW \\ % Your institution for the title page
\medskip
\textit{kk371053@students.mimuw.edu.pl} % Your email address
}
\date{21.06.2022} % Date, can be changed to a custom date

\begin{document}

\begin{frame}
\titlepage % Print the title page as the first slide
\end{frame}

\begin{frame}
\frametitle{Overview} % Table of contents slide, comment this block out to remove it
\tableofcontents % Throughout your presentation, if you choose to use \section{} and \subsection{} commands, these will automatically be printed on this slide as an overview of your presentation
\end{frame}

\definecolor{olivegreen}{HTML}{2a9d8f}
\definecolor{shary}{HTML}{8d99ae}

%----------------------------------------------------------------------------------------
%	PRESENTATION SLIDES
%----------------------------------------------------------------------------------------

\section{\textsc{Set Cover}}
\begin{frame}
\frametitle{Problem Statement: \textsc{Set Cover}}
\textbf{Input:} universe $\mathcal{U}$,
family $S$ of subsets of $\mathcal{U}$
\newline
\textbf{Output:} $C \subseteq S$ such that $|C|$ is minimal and
$\bigcup C = \mathcal{U}$
\newline
Minimal subfamily that covers all elements in $\mathcal{U}$.

% TODO: Better picture
\includesvg[width=0.5\textwidth]{set_cover.svg}

\end{frame}
\subsection{\textsc{Geometric Set Cover}}

\begin{frame}
\frametitle{Problem Statement: \textsc{Geometric Set Cover}}
\begin{itemize}
\item Universe $\mathcal{U}$ is set of points in the plane
\item Sets $S$ are some geometric shapes
\item Formally, each set in $S$ is intersection of $\mathcal{U}$
with some geometric shape.
\end{itemize}

Example for rectangles:

\includesvg[width=0.5\textwidth]{geometric_set_cover.svg}


\end{frame}
\subsection{\textsc{Segment Set Cover}}

\begin{frame}
\frametitle{Problem Statement: \textsc{Segment Set Cover}}
\begin{itemize}
\item Universe $\mathcal{U}$ is set of points in the plane
\item Sets $S$ are segments.
\end{itemize}

Example (TODO):

\includesvg[width=0.5\textwidth]{geometric_set_cover.svg}


\end{frame}

\section{$\delta$-extension}
\begin{frame}
\frametitle{$\delta$-extension for \textsc{Segment Set Cover}}
\begin{itemize}
\item $\delta$-extension for a segment is a segment which is longer by
$\delta$ fraction at both ends
\item We accept solution in which segments cover solution after
$\delta$-extension
\item The solution is compared to optimal solution without extension.
\end{itemize}

Example:

\includesvg[width=0.5\textwidth]{delta_extension.svg}
\end{frame}

\begin{frame}
\frametitle{$\delta$-extensions for \textsc{Segment Set Cover}}
\begin{itemize}
\item We are given universe $\mathcal{U}$ and set of segments $S$
\item We construct set $S^{+\delta}$ consisting of  
segments longer by $\delta$ fraction at both ends
\item We look for $C \subseteq S^{+\delta}$ that $\bigcup C = \mathcal{U}$
\item solution $C$ is compared with optimal solution for $S$ without extensions
for both parametrized and approximation setting
\end{itemize}

Example:

\includesvg[width=0.5\textwidth]{delta_extension.svg}
\end{frame}

\section{Approximation}
\begin{frame}
\frametitle{Preliminaries: Approximation}
Given:
\begin{itemize}
\item instance $I$ of the optimization problem (looking for minimal solution)
\item weight of the optimal solution $\mathsf{opt}(I)$
\end{itemize}
$p$-approximation is algorithm that yields solution of weight not larger than
$p\cdot \mathsf{opt}(I)$
\begin{block}{PTAS -- polynomial time approximation scheme}
For every $\epsilon > 0$, there exists $(1+\epsilon)$-approximation
algorithm running in time $n^{f(\epsilon)}$ for any computable function $f$.

Example: \textsc{Knapsack}.

\end{block}

\begin{block}{APX-hardness}
For sufficiently small $\epsilon > 0$,
if $(1+\epsilon)$-approximation exists, then P = NP.

Example: \textsc{MAX-3-SAT} (how many clauses in 3-SAT can be solved)
\end{block}


\end{frame}
\begin{frame}
\frametitle{Approximation results for \textsc{Set Cover} from literature}
\textsc{Set Cover}
\begin{itemize}
\item $\log n$-approximation, no $o(\log n)$ approximation assuing P $\neq$ NP
\end{itemize}

\textsc{Geometric Set Cover}
\begin{itemize}
\item with fat rectangles is APX-hard
\item (E?Q?)PTAS for FAT polygons with $\delta$-ext (TODO reference)
\end{itemize}


\begin{definition}{
Fatness of polygons $\tau$ is a ratio between a radius of circle
circumscribed on this polygon and inscribed in this polygon.
}\end{definition}
\includesvg[width=0.2\textwidth]{fat_polygon.svg}

\end{frame}

\begin{frame}
\frametitle{Parametrization of \textsc{Segment Set Cover}}
\begin{block}{Theorem 1}
	\textsc{Segment Set Cover} parametrized by the solution size $k$
	can be solved in time $k^k \cdot n^{O(1)}$.
\end{block}

\textbf{Technique:} Branching over at most $k+1$
segments on the lines with more than $k$ points on them.
\end{frame}

\begin{frame}
\frametitle{Approxmiation of \textsc{Segment Set Cover}}
\begin{block}{Theorem 2}
	Set cover with segments is APX-hard, ie.\newline
	for small $\epsilon =0.001$,
	doesn't have $(1+\epsilon)$-approximation unless P = NP.
\end{block}

How can we make this problem \textit{easier}?
\begin{itemize}
\item Allow only segments in 2 directions (pararell to the two axes)
\item Allow segments to \textit{almost} cover the points, ie. be very
close to them
\end{itemize}

\end{frame}


\begin{frame}
\frametitle{Approxmiation of set cover with segments}
\begin{block}{Theorem 2 (rephrased)}
	Set cover with segments parallel to two axes with $\frac{1}{2}$-extension
	is APX-hard, ie.\newline
	for small $\epsilon =0.001$,
	doesn't have $(1+\epsilon)$-approximation unless P = NP.
\end{block}

\end{frame}


\begin{frame}
\frametitle{Parametrized algorithms vs. W[1]-hardness}
Given:
\begin{itemize}
\item parameter $k$ (usually size of the solution)
\item size of the instance $n$
\end{itemize}
\bigskip

\begin{tabular}{|c|c|c|}
\hline
\textbf{Class} & \textbf{Complexity} & \textbf{Example}\\
\hline
W[1]-hard & $n^{O(k)}$ & \textsc{$k$-Clique}, \textsc{Grid Tiling}\\
\hline
FPT & $f(k) \cdot n^{O(1)}$ & \textsc{Vertex Cover}\\
\hline
\end{tabular}

\end{frame}

\begin{frame}
\frametitle{Parametrization of \textsc{Set Cover}}
W[2]-complete (no algorithm with running time $n^{k-\epsilon}$)
\end{frame}

\begin{frame}
\frametitle{Approxmiation of set cover with segments}
How do we prove that there doesn't exist $(1+\epsilon)$-approximation?

\begin{block}{Max-(3,3)-SAT problem}
Given 3-SAT instance with $n$ variables, $n$ clauses
and every variable appears in exactly 3 clauses,
find an assignment
that satisfies the maximum number of clauses
(not necessarily all of them).
\end{block}

\end{frame}


\begin{frame}{PCP theorem}
How do we prove that there doesn't exist $(1+\epsilon)$-approximation?
\begin{block}{NP-completeness}
Distinguishing if result of MAX-(3,3)-SAT is
$n$ or less than $n$ is NP-complete.
\end{block}

\begin{block}{Remark}
Result of MAX-(3,3)-SAT is not less than $\frac{7}{8}n$
\end{block}

\begin{block}{PCP theorem}
Distinguishing if result of MAX-(3,3)-SAT is $n$
or at most $(\frac{7}{8} + \epsilon)n$
cannot be done in polynomial time assuming $NP \nsubseteq DTIME(2^{O(\log n \log \log n)})$
\end{block}

Therefore MAX-(3,3)-SAT does not have
$(\frac{7}{8}+\epsilon)$-approximation


\end{frame}

\begin{frame}
\frametitle{Approxmiation of set cover with segments}
I initially started with rectangles problem\ldots

\begin{block}{Theorem 2.5 (Har-Peled also did that in 2012)}
	Set cover with polygons of bounded fatness (at least $\tau$)
	 with $\delta$-extension has $(1+\epsilon)$-approximation
	 for any $\epsilon > 0$.
	
\end{block}



\end{frame}


\section{Results for set cover with weighted segments} 

\begin{frame}
\frametitle{Definition of set cover with weighted segments}

\textbf{Input:} universe $\mathcal{U}$,
subfamily $S$ of subsets of $\mathcal{U}$
and function of weights assigned to sets $w : S \rightarrow \mathbb{N}$
\newline
\textbf{Output:} $C \subseteq S$ such that $\sum_{c \in C} w(c)$ is minimal and
$\bigcup C = \mathcal{U}$
\newline
Minimal subfamily that covers all elements in $\mathcal{U}$.

\bigskip
\textbf{In parametrized setting:}
We look for the \textit{best} solution with restricted size $|C| \le k$.

\textbf{In approximation setting:}
It is still APX-hard, nothing interesting


\end{frame}

\begin{frame}
\frametitle{Parametrization for set cover with weighted segments}
\begin{block}{Theorem 3}
	Weighted set cover with segments is W[1]-hard.
	
	Doesn't have FPT algorithm
\end{block}

How can we make this problem easier?
\begin{itemize}
\item Allow less directions (maybe just two parallel to axes)
\item Allow $\delta$-extension
\end{itemize}
\end{frame}

\begin{frame}
\frametitle{Parametrization for set cover with weighted segments}
\begin{block}{Theorem 4}
	Weighted set cover with segments 
	with $\delta$-extensions is in FPT.
	
	It can be solved with complexity $O((k/\delta)^k \cdot n^{O(1)})$.
\end{block}

\textbf{Technique intuition:} Provide kernel in problem where you need to
cover both points and some segments (these are required
by high density of points to cover on them)
\end{frame}

\begin{frame}
\frametitle{Parametrization for set cover with weighted segments}
\begin{block}{Theorem 3 (rephrased)}
	Weighted set cover with segments in 3 directions is W[1]-hard.
	
	
These directions are: parallel to axis and diagonal 45 degrees.
	
	Doesn't have FPT algorithm
\end{block}

Nobody\footnote{that I know of} knows if this still holds with
2 directions or in 3D with segments parallel to axis.

This is particalarily interesting, because unweighted problem is in FPT.

\end{frame}

\begin{frame}

\frametitle{Summary of results for parametrization of set cover with segments}

\textbf{Unweighted}

\begin{tabular}{|c|c|c|}
\hline
 &2 directions & arbitrarily many (at least 3)\\
 \hline
 exact& FPT & \textcolor{olivegreen}{FPT (Theorem 1)}\\
 
 \hline
 $\delta$-extensions & FPT & \textcolor{olivegreen}{FPT}\\
\hline
\end{tabular}
\bigskip

\textbf{Weighted}

\begin{tabular}{|c|c|c|}
\hline
 &2 directions & arbitrarily many (at least 3)\\
 \hline
 exact& \textbf{???} & \textcolor{olivegreen}{W[1]-hard (Theorem 3)}\\
 
 \hline
 $\delta$-extensions & \textcolor{olivegreen}{FPT} & \textcolor{olivegreen}{FPT (Theorem 4)}\\
\hline
\end{tabular}

\end{frame}

\begin{frame}
\frametitle{Summary of results for approximation of set cover with segments}

\textbf{Unweighted}

\begin{tabular}{|c|c|c|c|}
\hline
 &2 directions & any directions & Fat Polygons\\
 \hline
 exact& APX-hard & \textcolor{olivegreen}{APX-hard} & APX-hard \\
 
 \hline
 $\delta$-extensions & \textcolor{olivegreen}{APX-hard (Th. 2)} & \textcolor{olivegreen}{APX-hard} & PTAS (Har-Peled)\\
\hline
\end{tabular}


\end{frame}

\begin{frame}
\frametitle{Approxmiation of set cover with lines}
We can reduce set cover with lines to Vertex Cover, so it's
NP-hard and APX-hard.

Nobody\footnote{that I know of} knows if there exist 2-approximation.

\end{frame}


%----------------------------------------------------------------------------------------

\end{document} 
