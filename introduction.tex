\chapter{Introduction}
\section{Background}
Some problems in Computer Science are known to be NP-complete,
meaning that assuming P$\neq$NP there is no polynomial time
algorithm that can solve these problems.
Even so, they still can be amenable to different approaches,
such as approximation or parameterization.

\begin{defi}
In the \textbf{$\SetCover$} problem we are given a set of elements (universe)
$\points$ and~a~family of sets $\sets$ that are subsets of the universe $\points$
and sum up to the whole $\points$.
Our~task is to find a set $\sol \subseteq \sets$
such that $\bigcup \sol = \points$ and the size of $\sol$ is minimum possible.
\end{defi}

\SetCover is a classical example of an NP-complete problem,
which has been proven
in \cite{set_cover_inapproximation} to be
inapproximable with factor $(1-o(1))\ln n$ assuming P~$\neq$~NP
(which is a stronger result than APX-hardness),
and W[2]-complete with the natural parameterization,
see Theorem 13.21 in \cite{platypus_book}.
However restricting the problem to various specialized settings
can lead to more tractable special cases.
In this thesis we take a closer look at the $\GeometricSetCover$ problem
in the plane, where elements to cover are~points in the plane
and sets to cover them with are geometric objects.

\begin{defi}
\textbf{$\SegmentSetCover$} is $\GeometricSetCover$ where
objects that we cover the points with are segments in the plane.
\end{defi}

\paragraph{Approximation}
Over the years there has been a lot of work related to approximation
algorithms for $\GeometricSetCover$. Notably,
$\GeometricSetCover$ with unweighted unit disks admits a PTAS (see
Corollary 1.1 in \cite{unit_disks}). When we consider the same problem
with weighted unit disks (or unit squares), the problem admits a QPTAS
\cite{settling_apx_hardness}, see also \cite{voronoi_true}.
On the other hand, \cite{rectangles_apx_hard} 
proves that $\GeometricSetCover$ with unweighted axis-parallel fat rectangles
is APX-hard; they also show similar hardness
for $\GeometricSetCover$ with many other standard geometric objects.

\paragraph{Parameterization}
We consider $\GeometricSetCover$ 
parameterized by the size of solution.
$\GeometricSetCover$ with unit squares was first proven to be W[1]-hard 
in \cite{marx05} (Theorem 5). A later follow-up work \cite{voronoi}
shows that there is an~algorithm running in time $n^{\mathcal{O}(\sqrt{k})}$
that solves $\GeometricSetCover$ with unit squares or disks
and that there is no algorithm running in time $f(k) \cdot n^{o(\sqrt{k})}$
for any computable $f$ under the~Exponential-Time Hypothesis,
so this is a tight bound for this problem.

We also consider parameterization of weighted problems.
There does not seem to be a~consensus of what parameterization
in the weighted setting is exactly; there
was an attempt to introduce a quite complicated general
framework of weighted parameterized setting in \cite{weighted_framework}.
Kernels for several well-known weighted problems
such as \textsc{Weighted Subset Sum} or~\textsc{Weighted Knapsack} are presented in \cite{kernel_weighted}.
Another work \cite{weighted_flow} considers weighted
parameterization of \textsc{Weighted Directed Feedback Set} and \textsc{Weighted $st$-Cut}.

\paragraph{$\delta$-extension}
In this paper, we focus on $\SegmentSetCover$ with $\delta$-extension.
$\delta$-extension is a problem relaxation method based on the
$\delta$-shrinking model which was introduced in \cite{shrinking_original}
to provide an interesting result for
the \textsc{Maximum Weight Independent Set of Rectangles} problem.
In this problem one needs to find a set of non-overlapping
weighted rectangles with maximum sum of weight possible.
In the $\delta$-shrinking relaxed problem
the returned set of rectangles must be non-overlapping
after all the rectangles are shrunk by a tiny fraction $\delta$
towards the centre of symmetry.
This problem is easier, because we compare this result
to the optimum result before the shrinking. It might
even lead to finding a set with result better than the optimum
for the original problem.
The author in~\cite{shrinking_original} presents a PTAS
for \textsc{Maximum Weight Independent Set of Rectangles} with $\delta$-shrinking,
which is later improved to EPTAS in \cite{shrinking1} alongside
presenting a new FPT result for this problem with the natural parameterization.
Later the~similar $\delta$-shrinking model was used in \cite{shrinking2}
to present a PTAS for
\textsc{Maximum Weight Independent Set of Polygons} with $\delta$-shrinking.

\newcommand{\Int}{\mathsf{Int}}
\begin{defi}
\label{definition:delta_extension}
For any $\delta > 0$ and a centre-symmetric convex object $L$ with
centre of symmetry $S = (x_s, y_s)$,
the \textbf{$\delta$-extension} of $L$ is the open set of points:
$$L^{+\delta} = \Int\{(1 + \delta)\cdot(x - x_s, y - y_s) + (x_s, y_s) : (x, y) \in L\},$$
where $\Int$ denotes the interior of a set of points.
That is, $L^{+\delta}$ is interior of the image of $L$ under homothety centred
at $S$ with scale $(1+\delta)$.
\end{defi}

Analogous to $\delta$-shrinking,
$\delta$-extension provides a framework for relaxing
\textsc{Geometric} \textsc{Set} \textsc{Cover} problems, where we allow the returned set of
objects $\sol$ to \textit{almost} cover the points in the universe
by requiring that they are covered by $\sol$ after $\delta$-extension,
i.e. by set $\sol^{+\delta}$.
The same concept could be used for \textsc{Geometric Set Hitting} problems.
 
For a longer discussion of this concept see Section
\ref{section:def:delta_extension}.

Similar model is used to prove that $\GeometricSetCover$ with fat polygons
relaxed with $\delta$-extension admits EPTAS \cite{harpeled12}.
The $\delta$-extension model presented there is well-defined only
for fat polygons. It extends the object by all the points that
have distance to the closest point in the object $P$
no larger than $\delta\cdot \mathsf{rad}(P)$, where $\mathsf{rad}(P)$
is a radius of a circle inscribed into $P$.
Since segments do not have any circle inscribed into them,
the definition presented there cannot be utilized for this problem.
Polygons extended by $\delta$-extension
defined in Definition \ref{definition:delta_extension}
covers a superset of set of points that object extended
by $\delta$-extension defined in \cite{harpeled12}.
Since our definition is more permissive for any polygon,
the EPTAS from \cite{harpeled12}
also works for polygons extended by our $\delta$-extension.

\section{Our contribution}
In this paper we make the following contributions.

We show that approximation of $\SegmentSetCover$,
even if segments are axis-parallel and we relax the problem with  $\frac{1}{2}$-extension,
is APX-hard (Theorem \ref{segment_cover_apx_hard}).

\begin{restatable}{tw}{segmentCoverApxHard}{
\label{segment_cover_apx_hard}
	\textbf{($\SegmentSetCover$ with axis-parallel segments and $\frac{1}{2}$-extension is APX-hard)}.	
	$\SegmentSetCover$
	with axis-parallel segments in the 2D plane
	is APX-hard (even with $\frac{1}{2}$-extension).
	That is, assuming $P\neq NP$, there does not exist a PTAS
	for this problem.
}\end{restatable}

Theorem \ref{segment_cover_apx_hard} implies the following.
Note that segments are just degenerated rectangles.

\begin{corollary}{
\label{rectangle_cover_apx_hard}
	\textbf{($\GeometricSetCover$ with rectangles is APX-hard)}.	
	\textsc{Geometric} \textsc{Set} \textsc{Cover}
	with axis-parallel rectangles is APX-hard (even with $\frac{1}{2}$-extension).
}\end{corollary}

This expands the previous result of \cite{rectangles_apx_hard} 
that $\GeometricSetCover$
with axis-parallel fat rectangles is APX-hard,
we improved the result that rectangles no longer
have to be fat (Corollary \ref{rectangle_cover_apx_hard}).
This also proves that the assumption in \cite{harpeled12}
for EPTAS about polygons being fat is necessary, because
cover with arbitrary polygons with $\delta$-extension is APX-hard.

We also provide two FPT algorithms for parameterized $\SegmentSetCover$ 	
(Theorem~\ref{segment_cover_fpt})
and $\WeightedSegmentSetCover$ relaxed with $\delta$-extension
(Theorem~\ref{fpt_weighted_segment}).

\begin{restatable}{tw}{segmentCoverFpt}{
	\label{segment_cover_fpt}
	\textbf{(FPT for $\SegmentSetCover$).}
	There exists an algorithm that given a family $\sets$ of
	segments (in any direction),
	a set of points $\points$
	and a parameter $k$,
	runs in time ${k^{\mathcal{O}(k)} (|\points|\cdot|\sets|)^2}$,
	and outputs a solution $\sol \subseteq \sets$
	such that $|\sol| \le k$ and $\sol$ covers all points in~$\points$,
	or determines that such a set $\sol$ does not exist.
}\end{restatable}

\begin{restatable}{tw}{fptWeightedSegment}{
	\label{fpt_weighted_segment}
	\textbf{(FPT for $\WeightedSegmentSetCover$ with $\delta$-extension).}
	There exists an algorithm that given a family $\sets$ of
	$n$ weighted segments (in any direction),
	a set of $m$ points $\points$, and parameters $k$ and $\delta > 0$,
	runs in time $f(k, \delta) \cdot (nm)^c$ for some computable function $f$ and a constant $c$ and
	outputs a set $\sol$ such that:
	\begin{itemize}
	\item $\sol \subseteq \sets$,
	\item $|\sol| \le k$,
	\item $\sol^{+\delta}$ covers all points in $\points$,
	\item the weight of $\sol$ is not greater than the weight
	of an optimum solution of size at most $k$
	for this problem without $\delta$-extension,
	\end{itemize}
	or determines that there is no set $\sol$ with $|\sol| \le k$
	such that $\sol$ covers all points in $\points$.
}\end{restatable}

On the other hand, we prove that $\WeightedSegmentSetCover$ 
is W[1]-hard even if segments are limited to 3 directions (Theorem~\ref{w1_hard})
and assuming ETH there does not exist algorithm for this problem
that runs in time ${f(k)(|\points| + |\sets|)^{o(\sqrt{k})}}$.
See Figure \ref{tab:weighted_fpt} for a summary of parameterized
results for $\WeightedSegmentSetCover$.
Similar table for unweighted problem is present in Figure \ref{tab:unweighted_fpt}.

\begin{restatable}{tw}{wOneHard}
\label{w1_hard}
	Consider the problem of covering a set $\points$ of points
	by selecting at most $k$ segments
	from a set of segments $\sets$ 
	with non-negative weights $w : \sets \rightarrow \mathbb{R^+}$
	so that the weight of the cover is minimal.
	Then this problem is W[1]-hard when parameterized by $k$ and
	assuming ETH, there is no algorithm for this
	problem with running time
	$f(k)\cdot(|\points| + |\sets|)^{o(\sqrt{k})}$
	for any computable function $f$.
	Moreover, this holds even if all segments in $\sets$
	are axis-parallel or right-diagonal.
\end{restatable}

See Section \ref{section:def:geometric__set_cover}
for exact definitions of axis-parallel and right-diagonal segments.

This result is particularly interesting,
because problem without weights is FPT and weighted problem is W[1]-hard.
Moreover $\delta$-extension allowed us to provide an FPT algorithm
for a problem, which is W[1]-hard otherwise.

Note that the result of Theorem \ref{w1_hard} is not tight:
there exists a simple algorithm 
running in time ${\mathcal{O}(f(k)(|\points| + |\sets|)^k)}$.
So the question whether there exists an algorithm
for this problem running in time ${f(k)\cdot(|\points| + |\sets|)^{o(k)}}$
is still open.

Permissive FPT is a relaxed FPT problem, where 
we need to find solution of \textit{any} size in FPT-time,
but we compare it to the optimum solution of size at most $k$.
Idea for permissive FPT in local search was presented
in \cite{permissive_problem1}, \cite{permissive_problem2}.

Theorem \ref{w1_hard} can be improved to show that a permissive FPT
algorithm does not exist.
This is formulated precisely in Theorem \ref{permissive_w1_hard}.

\begin{figure}[h]
\begin{center}
\begin{tabular}{ | c | c | c | }
\hline
                & exact     & $\delta$-extension \\ 
\hline                
 axis-parallel   & ? & FPT* \\  
\hline                
 3 directions    & W[1]-hard & FPT* \\  
\hline                
 any direction   & W[1]-hard* & FPT \\
\hline                
\end{tabular}
\caption{Our results for $\WeightedSegmentSetCover$
parameterized by the size of solution.
Results marked with * are not explicitly given in this thesis,
but they trivially follow from stronger results shown in the other cells of the table.}
\label{tab:weighted_fpt}
\end{center}
\end{figure}

\begin{figure}[h]
\begin{center}
\begin{tabular}{ | c | c | c | }
\hline
                & exact     & $\delta$-extension \\ 
\hline                
 axis-parallel   & FPT* & FPT* \\  
\hline                
 3 directions    & FPT* & FPT* \\  
\hline                
 any direction   & FPT  & FPT* \\
\hline                
\end{tabular}
\caption{Our results for unweighted $\SegmentSetCover$
parameterized by the size of solution.
Results marked with * are not explicitly given in this thesis,
but they trivially follow from stronger results shown in the other cells of the table.}
\label{tab:unweighted_fpt}
\end{center}
\end{figure}

\paragraph{Future work.} There are two aforementioned problems
that relate to Theorem \ref{w1_hard} and were not solved in this thesis.
We have not presented W[1]-hardness proof
of \textsc{Weighted} \textsc{Segment} \textsc{Set} \textsc{Cover} where segments are limited to 3 directions,
but it may be possible to improve this construction to use segments
in 2 directions instead of 3 directions. 
The other question is what is the tight bound for this problem.
The simple algorithm solving
this problem is running in time ${\mathcal{O}(f(k)(|\points| + |\sets|)^k)}$.

Another problem to consider is whether
\textsc{Geometric Set Hitting} relaxed with $\delta$-extension
can yield some better results.
